\documentclass[a4paper, 11pt]{report}

%%%
%%% Use a custom geometry for the page
%%%
\usepackage{geometry}
\geometry{
  a4paper,
  left=20mm,
  right=20mm,
  top=20mm,
  bottom=25mm
}

\usepackage{lipsum}

%%%
%%% File encoding
%%%
\usepackage[utf8]{inputenc}
\usepackage[italian]{babel}
\usepackage[T1]{fontenc}

%%%
%%% Font
%%%
\usepackage{lmodern}
\usepackage{textcomp}
\usepackage[font=itshape]{quoting}

%%%
%%% References
%%%
\usepackage[perpage]{footmisc}
\usepackage{hyperref}
\hypersetup{ 
  linkbordercolor={.77 .4 .20},
  citebordercolor={.31 .63 .31},
  pdftitle={Appunti}
}

%%%
%%% Pictures
%%%

% General
\usepackage{caption}
\usepackage{graphicx}

% TikZ
\usepackage{tikz}
\usetikzlibrary{cd}

% Inkscape figures
\usepackage{import}
\usepackage{xifthen}
\usepackage{pdfpages}
\usepackage{transparent}
\newcommand{\incfig}[1]{%
    \def\svgwidth{\columnwidth}
    \import{./figures}{#1.pdf_tex}
}

%%%
%%% Basic math and layout packages
%%%
\usepackage{amssymb}
\usepackage{amsmath}
\usepackage{amsthm}
\usepackage{mathrsfs}
\usepackage{mathtools}
% \usepackage{xfrac}
\usepackage{array}
\usepackage[all]{xy}
\usepackage{stmaryrd}
\usepackage{multicol}
\usepackage{centernot}

%%%
%%% Enumerations
%%%
\usepackage{enumitem}

% Custom settings for compact lists
\setitemize{topsep=5pt}
\setlist[itemize,1]{label=$\bullet$}
\setlist[itemize,2]{label=$\circ$}

\setlist[enumerate,1]{label=\roman*)}
\setenumerate{topsep=5pt}

%%%
%%% Sections and page borders styles
%%%

% \usepackage{titlesec}
% Adds horizontal rules under "\section" entries
% \titleformat{\section}{\Large\scshape\raggedright}{}{0em}{}[\titlerule]
% \usepackage{fancyhdr}
% \pagestyle{fancy}
% \pagestyle{fancyplain} % Makes all pages in the document conform to the custom headers and footers
% \fancyhead[L]{\leftmark}% Empty left header
% \fancyhead[C]{} % Page numbering for center header  
% \fancyhead[R]{\rightmark}% Empty right header
% \fancyfoot[L]{}% Empty left footer
% \fancyfoot[C]{\thepage}% Empty center footer
% \fancyfoot[R]{}% Empty left footer
% \fancyhf{}
% \chead{\footnotesize\rightmark}
% \rfoot{\thepage}

%%%
%%% Styles by @aziis98
%%%

% Better "\setminus"
\newcommand{\mysetminusD}{%
  \hbox{\tikz{\draw[line width=0.6pt,line cap=round] (3pt,0) -- (0,6pt);}}
}
\newcommand{\mysetminusS}{%
  \hbox{\tikz{\draw[line width=0.45pt,line cap=round] (2pt,0) -- (0,4pt);}}
}
\newcommand{\mysetminusSS}{%
  \hbox{\tikz{\draw[line width=0.4pt,line cap=round] (1.5pt,0) -- (0,3pt);}}
}
\newcommand{\mysetminusT}{\mysetminusD}
\newcommand{\mysetminus}{\mathbin{\mathchoice{\mysetminusD}{\mysetminusT}{\mysetminusS}{\mysetminusSS}}}
\renewcommand{\setminus}{\mysetminus}

% Generali
% \renewcommand{\implies}{\Rightarrow}
% \renewcommand{\iff}{\Leftrightarrow}
\newcommand{\longiff}{\Longleftrightarrow}
\newcommand{\norm}[1]{\left\lVert #1 \right\rVert}
\newcommand{\sfrac}[2]{
    \raisebox{+0.55ex}{$#1$}
    /
    \raisebox{-0.3ex}{$#2$}
}

\renewcommand{\epsilon}{\varepsilon}
\newcommand{\myphi}{\varphi}
\newcommand{\compose}{\circ}
\newcommand{\mquad}{\;\;}
\newcommand{\mymid}{\;\middle|\;}
\newcommand{\curry}{\,\cdot\,}
\newcommand{\C}{\mathbb C}
\newcommand{\R}{\mathbb R}
\newcommand{\Z}{\mathbb Z}
\newcommand{\N}{\mathbb N}
\newcommand{\ds}{\displaystyle}
\newcommand{\dd}{\,\mathrm{d}}
\newcommand{\pd}{\partial}

% Fulmine dell'assurdo
\usepackage{stmaryrd}
\let\stmaryrdLightning\lightning
\renewcommand{\lightning}{\stmaryrdLightning}
\newcommand{\absurd}{$\lightning$}

% Funzione indicatrice
\newcommand{\bbOne}{\text{\usefont{U}{bbold}{m}{n}1}}
\MakeRobust{\bbOne}
\newcommand{\One}{\bbOne}

% Ambienti temporanei
% [Convergenza Monotona (Beppo Levi)]

\newenvironment{theorem}{\textbf{Teorema.}}{}
\newenvironment{named-theorem}[1]{\textbf{Teorema.} (\textit{#1})}{}

% \newcommand{\TdF}{\mathcal F}
\newcommand{\foralmostall}{\widetilde\forall}
\newcommand{\almosteverywhere}{\;\;\text{q.o.}}

\let\oldtilde\tilde
\renewcommand{\tilde}{\widetilde}

%%%
%%% Styles by @<arianna>
%%%

%%
%% Nuovi comandi
%%

% Insiemi numerici
% \newcommand{\N}{\mathbb{N}}
% \newcommand{\Z}{\mathbb{Z}}
% \newcommand{\Q}{\mathbb{Q}}
% \newcommand{\R}{\mathbb{R}}
% \newcommand{\C}{\mathbb{C}}
\newcommand{\restr}[2]{\left.#1\right|_{#2}}

% Dichiarazione lettera Chi maiuscola -> \Chi
\DeclareRobustCommand{\rchi}{{\mathpalette\irchi\relax}}
\newcommand{\irchi}[2]{\raisebox{\depth}{\mbox{\Large$#1\chi$}}} % inner command, used by \rchi%
\newcommand{\Chi}{\rchi}

% Dichiarazioni nuovi ambienti
% \theoremstyle{definition}
% \newtheorem{definizione}{Definizione}[chapter]
% \newtheorem{teorema}{Teorema}[chapter]
% \newtheorem{proposizione}[definizione]{Proposizione}
% \newtheorem{lemma}[teorema]{Lemma}
% \newtheorem{corollario}[teorema]{Corollario}
% \newenvironment{dimostrazione}{\textit{Dimostrazione:}}{\hfill$\square$\break}
\newtheorem{osservazione}{Osservazione}
% \newtheorem{proprieta}[teorema]{Proprietà}
% \newtheorem{esempio}[teorema]{Esempio}
% \newtheorem{nota}[osservazione]{Nota}


\title{{\Huge Analisi 3}\\{\small Appunti di Analisi 3 del corso di Giovanni Alberti e Maria Stella Gelli}}
\author{Arianna Carelli e Antonio De Lucreziis}
\date{I Semestre 2021/2021}

\begin{document}

%
% Removes initial indentation from paragraphs.
%
\parskip 1ex
\setlength{\parindent}{0pt}

% Initial page
\maketitle

% Table of contents
\tableofcontents
\newpage

\chapter{Teoria della misura}

\section{Misure astratte}
Siano
\begin{itemize}[label={}]
	\item $X$ un insieme qualunque.
	\item $\mathcal{A}$ una $\sigma$-algebra di sottoinsiemi di $X$, ovvero una famiglia di sottoinsiemi di $X$ che rispetta le seguenti proprietà:
		\begin{itemize}[label={--}]
			\item $\emptyset, X \in \mathcal{A}$.
			\item $\mathcal{A}$ è chiusa per complementare, unione e intersezione numerabile.
			\end{itemize}
	\item $\mu$ una misura su $X$, ossia una funzione $\mu \colon A \to [0,+\infty]$ $\sigma$-addittiva, cioè tale che data una famiglia numerabile $\left\{ E_k \right\} \subset A$ disgiunta e posto $E \coloneqq \bigcup E_n $, allora
	\[
		\mu(E) = \sum_{n} \mu (E_n).
	\] 
\end{itemize}
%
Seguono le proprietà.
\begin{enumerate}[label=(\roman*)]
	\item $\mu(\emptyset) = 0$.
	\item \textit{Monotonia}. Dati $E,E' \in \mathcal{A}$ e $E \subset E'$, allora $\mu(E) \leq \mu(E')$.
	\item Data una successione crescente di insiemi, $E_n \uparrow E$, allora $\mu(E) = \lim_{n \to \infty} \mu(E_n) = \sup_{n} \mu(E_n)$.
	\item Se $E_n \uparrow E$ e $\mu (E_{\bar{n}}) < +\infty$ per qualche $\bar{n}$, allora $\mu(E) = \lim_{n \to + \infty} \mu(E_n) = \inf_{n} \mu(E_n)$.
	\item \textit{Subadditività}. Se $\bigcup E_n \supset E$, allora $\mu(E) \leq \sum_{n}^{} \mu(E_n)$.
\end{enumerate}
%
Dove una successione crescente di insiemi $E_n \uparrow E$ è tale che $E_1 \subset E_2 \subset \ldots E_n \subset \ldots $ e $\bigcup E_n = E$.
%
Notiamo infine che dato $X' \in \mathcal{A}$ si possono restringere $\mathcal{A}$ e $\mu$ a $X'$.
%

\textbf{Terminologia}.
\begin{itemize}[label={}]
	\item Sia $P(X)$ un'affermazione che dipende da $x \in X$. Si dice che $P(X)$ vale $\mu$-quasi ogni $x \in X$ se l'insieme $\left\{ x \colon P(x) \text{ non vale}  \right\}$ è (contenuto in) un insieme di misura $\mu$ nulla.
	\item $\mu$ si dice \textit{completa} se $F \subset E, E \in \mathcal{A}$ e $\mu(E) = 0$, allora $F \in \mathcal{A}$ (e di conseguenza $\mu(F) = 0$).
	\item $\mu$ si dice \textit{finita} se $\mu(X) < + \infty$.
\end{itemize}
%
D'ora in poi consideriamo solo misure complete.
%
\section{Esempi di misure}
\textit{Misura che conta i punti}. Siano
\begin{itemize}[label={}]
	\item $X$ qualunque.
	\item $\mathcal{A} \coloneqq \mathcal{P}(X)$.
	\item $\mu(E) \coloneqq \# E \in \N \cup \left\{ +\infty \right\}$.
\end{itemize}
%
\textit{Delta di Dirac in $x_0$}. Siano
\begin{itemize}[label={}]
	\item $X$ qualunque
	\item $\mathcal{A} \coloneqq \mathcal{P}(X)$.
	\item $x_0 \in X$ fissato, allora $\mu(E) \coloneqq \delta_{x_0}(E) = \One_E (x_0)$.
\end{itemize}
%
2. \textit{Misura di Lebesgue}
Siano
\begin{itemize}[label={}]
	\item $X = \R^n$.
	\item $\mathcal{M}^n$ la $\sigma$-algebra dei misurabili secondo Lebesgue.
	\item $\mathcal{L}^n$ la misura di Lebesgue.
\end{itemize}
%
Di seguito definiamo la misura di Lebesgue $\mathcal{L}^n$.

Dato $R$ parallelepipedo in $\R^n$, cioè $R = \prod_{k=1}^{n} I_k $ con $I_k$ intervalli in $\R$.
Si pone
\[
	\mathrm{vol}_n (R) \coloneqq \prod_{k=1}^{n}  \mathrm{lungh} (I_k)
\] 
per ogni $E \subset \R^n$. Si pone
\[
	\mathcal{L}^n(E) \coloneqq \inf \left\{ \sum_{i}^{} \mathrm{vol}_n (R_i) \mid \left\{ R_i \right\} \text{tale che} \cup_i R_i \supset E  \right\}.
\] 
\begin{osservazione}
Si hanno le seguenti.
\begin{itemize}[label={--}]
	\item $\mathcal{L}^n(R) = \mathrm{vol}_n (R)$.
	\item $\mathcal{L}^n$ è così definita se $\mathcal{P}(\R^n)$ ma non è $\sigma$-addittiva.
	\item $\mathcal{L}^n$ è $\sigma$-addittiva su $\mathcal{M}^n$ (è per questo che bisogna introdurre $\mathcal{M}^n$).
\end{itemize}
\end{osservazione}
%
Il terzo punto giustifica l'introduzione dei \textit{misurabili secondo Lebesgue}. Dunque definiamo $\mathcal{M}^n$.

Dato $E \subset \R^n$, si dice che $E$ è misurabile (secondo Lebesgue) se per ogni $\epsilon > 0$ esiste un aperto $A$ e un chiuso $C$ tale che
\begin{itemize}[label={--}]
	\item $C \subset E \subset A$,
	\item $\mathcal{L}^n (A \smallsetminus C) \leq \epsilon$.
\end{itemize}
%
\begin{osservazione}
Si hanno le seguenti.
\begin{itemize}[label={--}]
	\item Per ogni $E$ misurabile vale
	\[
	\mathcal{L}^n = \inf \left\{ \mathcal{L}^n \colon A \ \text{aperto}, A \supset E \right\} = \sup \left\{ \mathcal{L}^n \colon K \ \text{compatto}, K \subset E \right\}.
	\] 
	\item Notiamo che se $F \subset E$ con $E \subset \mathcal{M}^n$ e $\mathcal{L}^n(E) = 0$, allora $F \in \mathcal{M}^n$. Ovvero la misura di Lebesgue è completa!
\end{itemize}
%
\textit{Notazione.} $\left| E \right| \coloneqq \mathcal{L}^n (E).$
%
\end{osservazione}

\section{Funzioni misurabili}
%
\begin{definizione}
	Dati $X, \mathcal{A}, \mu$ e $f \colon X \to \R$ (o in $Y$ spazio topologico), diciamo che $f$ è \textit{misurabile} ($\mathcal{A}$-misurabile), se
	\[
		f^{-1} (A) \in \mathcal{A} \quad \text{per ogni} \ A \ \text{aperto}.
	\] 
\end{definizione}
%
\begin{osservazione}
Si hanno le seguenti.
\begin{itemize}[label={--}]
	\item Dato $E \subset X$, vale $E \in \mathcal{A}$ se solo se $\One_E$ è misurabile.
	\item La classe delle funzioni misurabili è chiusa rispetto a moolte operazioni:
	\begin{itemize}[label={$\ast$}]
		\item somma, prodotto ( se hanno senso nello spazio immagine della funzione).
		\item Composizione con funzioni continue. In particolare, se $f \colon X \to Y$ continua e $g \colon  Y \to Y'$ continua, allora $g \circ f$ è misurabile.
		\item Convergenza puntuale: data una successione di $f_n$ misurabili e $f_n \to f$ puntualmente, allora $f$ è misurabile.
		\item $\liminf$ e $\limsup$ (nel caso $Y = \R$).
	\end{itemize}
\end{itemize}
\end{osservazione}
%
\subsection{Funzioni semplici}
Indico con $\mathcal{S}$ la classe delle funzioni $f \colon  X \to \R$ \textit{semplici}, cioè della forma $f = \sum_{i}^{n} \alpha_i \One_{E_i}$ con $\{E_i \}_{1 \leq i \leq n}$ misurabili e $\alpha_i \in \R$. \\
\textit{Nota.} Se necessario posso supporre gli $E_i$ disgiunti.
%
\section{Integrale}
La definizione di $\int_X f \, \dd \mu$ è data per passi:
\begin{enumerate}
	\item \label{item:def_int_1} $f \in \mathcal{S}, f \geq 0$, cioè $f = \sum_i \alpha_i \One_{E_i}$, $\alpha_i \geq 0$, si pone
	\[
		\int\limits_{\mathclap{X}}^{} f \, \dd \mu \coloneqq \sum_{i}^{} \alpha_i \ \mu(E_i),
	\] 
	convenendo che $+ \infty \cdot 0 = 0$.
	\item \label{item:def_int_2} $f \colon  X \to [0,+\infty]$ misurabile si pone
	\[
		\int\limits_{\mathclap{X}}^{} f \, \dd \mu \coloneqq \sup_{\substack{g \in \mathcal{S} \\ 0 \leq g \leq f}} \int\limits_{\mathclap{X}}^{} g \, \dd \mu.
	\] 
	\item $f \colon X \to \overline{\R}$ misurabile si dice \textit{integrabile} se 
	\[
		\int\limits_{\mathclap{X}}^{} f^+ \, \dd \mu < + \infty \quad \text{oppure} \quad \int\limits_{\mathclap{X}}^{} f^- \, \dd \mu < +\infty.
	\] 
	Per tali $f$ si pone
	\[
		\int\limits_{\mathclap{X}}^{} f \, \dd \mu \coloneqq  \int\limits_{\mathclap{X}}^{} f^+ \, \dd \mu - \int\limits_{\mathclap{X}}^{} f^- \, \dd \mu.
	\] 
	\item $f \colon X \to \R^n$ si dice \textit{sommabile} (o di \textit{classe} $\mathcal{L}^1$) se $\int_X \left| f \right| \, \dd \mu < +\infty$.
	In tal caso, se $\int_X f_i^{\pm} \, \dd \mu < +\infty$ per ogni $f_i$ componente di $f$, allora $\int_X f \, \dd \mu$ esiste ed è finito.
\end{enumerate}
%
Per tali $f$ si pone
\[
	\int\limits_{\mathclap{X}}^{} f \, \dd \mu \coloneqq  \left( \int_X f_1 \, \dd \mu, \ldots , \int_X f_n \, \dd \mu \right).
\] 
%
\textit{Notazione.} Scriveremo spesso $\int_E f(x) \, \dd x$ invece di $\int_E f \, \dd \mathcal{L}^n$ .
%
\begin{osservazione}
Si hanno le seguenti.
\begin{itemize}[label={--}]
	\item L'integrale è lineare (sulle funzioni sommabili).
	\item Le definizioni in \ref{item:def_int_1} E \ref{item:def_int_2} Danno lo stesso risultato per $f$ semplice $\geq 0$.
	\item La definizione in \ref{item:def_int_2} Ha senso per ogni $f \colon X \to [0,+\infty]$ anche non misurabile. Ma in generale vale solo che
	\[
		\int\limits_{\mathclap{X}}^{} f_1 + f_2 \, \dd \mu \geq \int_X f_1 \, \dd \mu + \int_X f_2 \, \dd \mu.
	\] 
	\item Dato $E \in \mathcal{A}$, $f$ misurabile su $E$, notiamo che vale l'uguaglianza
	\[
		\int_E f \, \dd \mu \coloneqq \int_X f \cdot \One_E \, \dd \mu.
	\] 
	\item Si può definire $\int_X f \, \dd \mu$ anche per $f \colon X \to Y$ con $Y$ spazio vettoriale normato finito dimensionale e $f$ sommabile. (è necessario avere uno spazio vettoriale, perchè mi serve la linearità e la moltiplicazione per scalare).
	\item Se $f_1 = f_2$ $\mu$-q.o. allora $\int_X f_1 \, \dd \mu = \int_X f_2 \, \dd \mu$.
	\item Si definisce $\int_X f \, \dd \mu$ anche se  $f$ è misurabile e definita su $X \setminus N$ con $\mu(N) = 0$.
	\item se $f \colon [a,b] \to \R$ è integrabile secondo Rienmann allora è misurabile secondo Lebesgue e le due nozioni di integrale coincidono. 
	\textit{Nota.} Lo stesso vale per integrali impropri di funzioni positive. Ma nel caso più generale non vale: se $f \colon (0,+\infty) \to \R$ è data da $f(x) \coloneqq \dfrac{\sin x}{x}$, allora l'integrale di $f$ definito su $(0,+\infty)$ esiste come integrale improprio ma non secondo Lebesgue $\left( \int_0^{+\infty} f^+ \dd x = \int_0^{+\infty} f^- \dd x = +\infty \right)$.
	\item $\int_X f \, \dd \delta_{x_0} = f(x_0)$.
	\item se $X = \N$ e $\mu$ è la misura che conta i punti l'integrale è 
	\[
		\int_X f \, \dd \mu = \sum_{n = 0}^{\infty} f(n) 
	\] 
	per le $f$ positive o tali che $\sum_{}^{} f^+(n) < +\infty $ oppure $\sum_{}^{} f^+(n) < -\infty $. \\
	\textit{Nota.} $\sum_{1}^{\infty} \frac{(-1)^n}{n} $ esiste come serie ma non come integrale. Da questo si osserva che serie e integrale non coincidono.
	\item Dato $X$ qualunque, $\mu$ misura che conta i punti e $f \colon  X \to [0,+\infty] $ posso definire la somma di tutti i valori di $f$ 
	\[
		\sum_{x \in X}^{} f(x) \coloneqq \int\limits_{\mathclap{X}}^{} f \, \dd \mu.  
	\] 
\end{itemize}
\end{osservazione}
%
\section{Teoremi di convergenza}
Prendo $X, \mathcal{A}, \mu$ come al solito.
%
\begin{teorema}[Convergenza Monotona (Beppo Levi)]
Date $f_n \colon  X \to [0,+\infty]$ misurabili, tali che $f_n \uparrow f$ ovunque in $X$, allora
\[
	\lim_{n \to +\infty} \int\limits_{\mathclap{X}}^{} f_n \, \dd \mu = \int\limits_X f \, \dd \mu,
\] 
dove
\[
	\lim_{n \to +\infty} \int\limits_{\mathclap{X}}^{} f_n \, \dd \mu = \sup_n \int f_n \, \dd \mu.
\] 
\end{teorema}
%
\begin{teorema}[Lemma di Fatou]
Date $f_n \colon X \to [0,+\infty]$ misurabili, allora
\[
	\liminf_{n \to +\infty} \int\limits_X f \, \dd \mu \geq \int\limits_{\mathclap{X}}^{} \left( \liminf_{n \to +\infty} f_n \right) \, \dd \mu.
\] 
\end{teorema}
%
\begin{teorema}
Date $f_n \colon  X \to \R$ (o anche $\R^n$) tali che
%
\begin{itemize}
	\item[] \textit{convergenza puntuale}. $f_n (x) \to f(x)$ per ogni $x \in X$.
	\item[] \textit{dominazione}. Esiste $g \colon X \to [0,+\infty]$ sommabile tale che $\left| f_n (x) \right| \leq g(x)$ per ogni $x \in X$ e per ogni $n \in \N$.
\end{itemize}
Allora
\[
	\lim_{n \to \infty} \int\limits_{\mathclap{X}}^{} f_n \, \dd \mu = \int\limits_X f \, \dd \mu. 
\] 
\end{teorema}
%
\textit{Nota.}
la seconda proprietà è essenziale; sostituirla con $\int_X \left| f_n \right| \, \dd \mu \leq C < + \infty$ non basta!


\textit{Altro esempio di misura}. Data $\rho \colon  \R^n \to [0,+\infty]$ misurabile, la misura con densità $\rho$ è data da
\[
\mu(E) = \int\limits_{\mathclap{E}}^{} \rho \, \dd x \quad \text{per ogni} \ E \ \text{misurabile}.
\] 
%
\begin{osservazione}
Si hanno le seguenti.
\begin{itemize}[label={}]
	\item $\R^n$ e $\mathcal{L}^n$ possono essere sostituiti da $X$ e $\widetilde{\mu}$.
	\item il fatto che $\mu$ è una misura segue da Beppo Levi, in particolare serve per mostrare la subadditività.
\end{itemize}
\end{osservazione}
%
\begin{teorema}[Cambio di variabile]
Siano $\Omega, \Omega'$ aperti di $\R^n$, $\phi \colon \Omega \to \Omega' $ un diffeomorfismo di classe $C^1$ e $f \colon \Omega' \to [0,+\infty]$ misurabile. Allora
\[
\int\limits_{\mathclap{\Omega'}}^{} f(x') \, \dd x' = \int\limits_{\mathclap{\Omega}}^{} f(\phi(x)) \left| \det(\nabla \phi(x)) \right| \, \dd x.
\] 
\end{teorema}
%
La stessa formula vale per $f$ a valori in $\overline{\R}$ integrabile e per $f$ a valori in $\R^n$ sommabile.
\begin{osservazione}
Si hanno le seguenti.
\begin{itemize}[label={--}]
	\item Se $n = 1$, $\left| \det(\nabla \phi(x)) \right| = \left| \phi'(x) \right|$ e non $\phi'(x)$ come nella formula vista ad analisi 1 (l'informazione del segno viene data dall'inversione degli estremi).
	\item Indebolire le ipotesi su $\phi$ è delicato. Basta $\phi$ di classe $C^1$ e $\# \phi^{-1}(x') = 1$ per quasi ogni $x' \in \Omega'$ (supponendo $\phi$ iniettiva la proprietà precedente segue immediatamente).
	Se $\phi$ non è "quasi" iniettiva bisogna correggere la formula per tenere conto della molteplicità.
\end{itemize}
\end{osservazione}
%
Di seguito riportiamo il teorema di Fubini-Tonelli per la misura di Lebesgue.
%
\begin{teorema}[Fubini-Tonelli]
Sia $R^{n_1} \times \R^{n_2} \simeq \R^n$ con $n = n_1 + n_2$, $ E \coloneqq E_1 \times E_2 $ dove $E_1, E_2$ sono misurabili e $f$ è una funzione misurabile definita su $E$.
Se $f$ ha valori in $[0,+\infty]$ allora
\[
	\int\limits_X f \, \dd \mu = \int\limits_{\mathclap{E_2}}^{} \left( \int\limits_{\mathclap{E_1}}^{} f(x_1,x_2) \, \dd x_1  \right) \, \dd x_2 = \int\limits_{\mathclap{E_1}}^{} \left( \int\limits_{\mathclap{E_2}}^{} f(x_1,x_2) \, \dd x_2  \right) \, \dd x_1.
\] 
\end{teorema}
%
Vale lo stesso per $f$ a valori in $\R$ o in $\R^n$ sommabile.
%
\begin{osservazione}
Si hanno le seguenti.
\begin{itemize}[label={--}]
	\item Se $X_1, X_2$ sono spazi con misure $\mu_1,\mu_2$ (con opportune ipotesi) vale:
	\[
		\int\limits_{\mathclap{E_2}}^{} \Bigg( \int\limits_{\mathclap{E_1}}^{} f(x_1,x_2) \, \dd \mu_1(x_1)  \Bigg) \, \dd \mu_2(x_2) = \int\limits_{\mathclap{E_1}}^{} \Bigg( \int\limits_{\mathclap{E_2}}^{} f(x_1,x_2) \, \dd \mu_2(x_2)  \Bigg) \, \dd \mu_1(x_1).
	\] 
	se $f\geq 0$ oppure $\displaystyle \int\limits_{\mathclap{X_1}}^{} \Bigg( \int\limits_{\mathclap{x_2}}^{} \left| f \right| \, \dd \mu_2(x_2)  \Bigg) \, \dd \mu_1(x_1) < + \infty $.
	\item Se $X_1 \subset \R$ (oppure $X_1 \subset \R^n$), $\mu_1 = \mathcal{L}^n$ e $X_2 = \N$, $\mu_2$ è la misura che conta i punti, allora la formula sopra diventa
	\[
		\sum_{n=0}^{\infty} \Bigg( \int\limits_{\mathclap{X_1}}^{} f_n(x) \, \dd x  \Bigg) = \int\limits_{\mathclap{X_1}}^{} \Bigg( \sum_{n=0}^{\infty} f_n(x)  \Bigg) \, \dd x.
	\] 
	Se $f_i \geq 0$ oppure $\displaystyle \sum_{i} \Bigg( \int\limits_{\mathclap{X_1}}^{} \left| f_i(x) \right| \, \dd x  \Bigg) < + \infty $.
	\item se $X_1 = X_2 = \N$ e $\mu_1 = \mu_2$ è la misura che conta i punti la formula sopra diventa
	\[
		\sum_{j=0}^{\infty} \Bigg( \sum_{i=0}^{\infty} a_{i,j}  \Bigg) = \sum_{i=0}^{\infty} \Bigg( \sum_{j=0}^{\infty} a_{i,j}  \Bigg)
	\] 
	se $a_{i,j} \geq 0$ oppure $\sum_{i} \sum_{j} \left| a_{i,j} \right| < +\infty $.
\end{itemize}
\end{osservazione}
%






\chapter{Spazi $L^p$ e convoluzione}

% TODO

\chapter{Spazi di Hilbert}

% TODO

\chapter{Serie di Fourier}

% TODO

\chapter{Applicazioni della serie di Fourier}

% TODO

\chapter{Trasformata di Fourier}

% TODO

\chapter{Funzioni armoniche}

% TODO

\chapter{Integrazione di superfici}

% TODO

\newpage

\section{Indice Analitico}

Lorem ipsum dolor sit amet, consectetur adipisicing elit, sed do eiusmod
tempor incididunt ut labore et dolore magna aliqua. Ut enim ad minim veniam,
quis nostrud exercitation ullamco laboris nisi ut aliquip ex ea commodo
consequat. Duis aute irure dolor in reprehenderit in voluptate velit esse
cillum dolore eu fugiat nulla pariatur. Excepteur sint occaecat cupidatat non
proident, sunt in culpa qui officia deserunt mollit anim id est laborum.

\begin{multicols*}{2}

\makebox[3cm][l]{\absurd} Assurdo

\makebox[3cm][l]{\absurd} Assurdo

\makebox[3cm][l]{\absurd} Assurdo

\makebox[3cm][l]{\absurd} Assurdo

\vfill\null\columnbreak

\makebox[3cm][l]{\absurd} Assurdo

\makebox[3cm][l]{\absurd} Assurdo

\makebox[3cm][l]{\absurd} Assurdo

\makebox[3cm][l]{\absurd} Assurdo

\end{multicols*}

\end{document}
