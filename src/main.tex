\documentclass[a4paper, 11pt]{report}

%%%
%%% Use a custom geometry for the page
%%%
\usepackage{geometry}
\geometry{
  a4paper,
  left=20mm,
  right=20mm,
  top=20mm,
  bottom=25mm
}

\usepackage{lipsum}

%%%
%%% File encoding
%%%
\usepackage[utf8]{inputenc}
\usepackage[italian]{babel}
\usepackage[T1]{fontenc}

%%%
%%% Font
%%%
\usepackage{lmodern}
\usepackage{textcomp}
\usepackage[font=itshape]{quoting}

%%%
%%% References
%%%
\usepackage[perpage]{footmisc}
\usepackage{hyperref}
\hypersetup{ 
  linkbordercolor={.77 .4 .20},
  citebordercolor={.31 .63 .31},
  pdftitle={Appunti}
}

%%%
%%% Pictures
%%%

% General
\usepackage{caption}
\usepackage{wrapfig}

%%% 
%%% Graphics
%%% 
\usepackage{graphicx}

\newcommand{\inputfigure}[1]{\input{#1.pdf_tex}}

% Path per le figure
\graphicspath{{./figures/.pdf_tex}}

% Magia arcana equivalente a "\graphicspath" ma per "\input"
\makeatletter
\def\input@path{{./figures/.pdf_tex}}
\makeatother

% TikZ
\usepackage{tikz}
\usetikzlibrary{cd}

% Inkscape figures
\usepackage{import}
\usepackage{xifthen}
\usepackage{pdfpages}
\usepackage{transparent}
\newcommand{\incfig}[1]{%
    \def\svgwidth{\columnwidth}
    \import{./figures}{#1.pdf_tex}
}

%%%
%%% Basic math and layout packages
%%%
\usepackage{amssymb}
\usepackage{amsmath}
\usepackage{amsthm}
\usepackage{mathrsfs}
\usepackage{mathtools}
% \usepackage{xfrac}
\usepackage{array}
\usepackage[all]{xy}
\usepackage{stmaryrd}
\usepackage{multicol}
\usepackage{centernot}

%%%
%%% Enumerations
%%%
\usepackage{enumitem}

% Custom settings for compact lists
\setitemize{topsep=5pt}
\setlist[itemize,1]{label=$\bullet$}
\setlist[itemize,2]{label=$\circ$}

\setlist[enumerate,1]{label=\roman*)}
\setenumerate{topsep=5pt}

%%%
%%% Sections and page borders styles
%%%

% \usepackage{titlesec}
% Adds horizontal rules under "\section" entries
% \titleformat{\section}{\Large\scshape\raggedright}{}{0em}{}[\titlerule]
% \usepackage{fancyhdr}
% \pagestyle{fancy}
% \pagestyle{fancyplain} % Makes all pages in the document conform to the custom headers and footers
% \fancyhead[L]{\leftmark}% Empty left header
% \fancyhead[C]{} % Page numbering for center header  
% \fancyhead[R]{\rightmark}% Empty right header
% \fancyfoot[L]{}% Empty left footer
% \fancyfoot[C]{\thepage}% Empty center footer
% \fancyfoot[R]{}% Empty left footer
% \fancyhf{}
% \chead{\footnotesize\rightmark}
% \rfoot{\thepage}

%%%
%%% Styles by @aziis98
%%%

% Better "\setminus"
\newcommand{\mysetminusD}{%
  \hbox{\tikz{\draw[line width=0.6pt,line cap=round] (3pt,0) -- (0,6pt);}}
}
\newcommand{\mysetminusS}{%
  \hbox{\tikz{\draw[line width=0.45pt,line cap=round] (2pt,0) -- (0,4pt);}}
}
\newcommand{\mysetminusSS}{%
  \hbox{\tikz{\draw[line width=0.4pt,line cap=round] (1.5pt,0) -- (0,3pt);}}
}
\newcommand{\mysetminusT}{\mysetminusD}
\newcommand{\mysetminus}{\mathbin{\mathchoice{\mysetminusD}{\mysetminusT}{\mysetminusS}{\mysetminusSS}}}
\renewcommand{\setminus}{\mysetminus}

% Generali
% \renewcommand{\implies}{\Rightarrow}
% \renewcommand{\iff}{\Leftrightarrow}
\newcommand{\longiff}{\Longleftrightarrow}
\newcommand{\norm}[1]{\left\lVert #1 \right\rVert}
\newcommand{\sfrac}[2]{
    \raisebox{+0.55ex}{$#1$}
    /
    \raisebox{-0.3ex}{$#2$}
}

\renewcommand{\epsilon}{\varepsilon}
\newcommand{\myphi}{\varphi}
\newcommand{\compose}{\circ}
\newcommand{\mquad}{\;\;}
\newcommand{\mymid}{\;\middle|\;}
\newcommand{\curry}{\,\cdot\,}
\newcommand{\C}{\mathbb C}
\newcommand{\R}{\mathbb R}
\newcommand{\Z}{\mathbb Z}
\newcommand{\N}{\mathbb N}
\newcommand{\ds}{\displaystyle}
\newcommand{\dd}{\,\mathrm{d}}
\newcommand{\pd}{\partial}

% Fulmine dell'assurdo
\usepackage{stmaryrd}
\let\stmaryrdLightning\lightning
\renewcommand{\lightning}{\stmaryrdLightning}
\newcommand{\absurd}{$\lightning$}

% Funzione indicatrice
\newcommand{\bbOne}{\text{\usefont{U}{bbold}{m}{n}1}}
\MakeRobust{\bbOne}
\newcommand{\One}{\bbOne}

% Ambienti temporanei
% [Convergenza Monotona (Beppo Levi)]

\newenvironment{theorem}{\textbf{Teorema.}}{}
\newenvironment{named-theorem}[1]{\textbf{Teorema.} (\textit{#1})}{}

% \newcommand{\TdF}{\mathcal F}
\newcommand{\foralmostall}{\widetilde\forall}
\newcommand{\almosteverywhere}{\;\;\text{q.o.}}

\let\oldtilde\tilde
\renewcommand{\tilde}{\widetilde}

%%%
%%% Styles by @<arianna>
%%%

%%
%% Nuovi comandi
%%

% Insiemi numerici
% \newcommand{\N}{\mathbb{N}}
% \newcommand{\Z}{\mathbb{Z}}
% \newcommand{\Q}{\mathbb{Q}}
% \newcommand{\R}{\mathbb{R}}
% \newcommand{\C}{\mathbb{C}}
\newcommand{\restr}[2]{\left.#1\right|_{#2}}

% Dichiarazione lettera Chi maiuscola -> \Chi
\DeclareRobustCommand{\rchi}{{\mathpalette\irchi\relax}}
\newcommand{\irchi}[2]{\raisebox{\depth}{\mbox{\Large$#1\chi$}}} % inner command, used by \rchi%
\newcommand{\Chi}{\rchi}

% Dichiarazioni nuovi ambienti
% \theoremstyle{definition}
% \newtheorem{definizione}{Definizione}[chapter]
% \newtheorem{teorema}{Teorema}[chapter]
% \newtheorem{proposizione}[definizione]{Proposizione}
% \newtheorem{lemma}[teorema]{Lemma}
% \newtheorem{corollario}[teorema]{Corollario}
% \newenvironment{dimostrazione}{\textit{Dimostrazione:}}{\hfill$\square$\break}
\newtheorem{osservazione}{Osservazione}
% \newtheorem{proprieta}[teorema]{Proprietà}
% \newtheorem{esempio}[teorema]{Esempio}
% \newtheorem{nota}[osservazione]{Nota}


\title{{\Huge Analisi 3}\\{\small Appunti di Analisi 3 del corso di Giovanni Alberti e Maria Stella Gelli}}
\author{Arianna Carelli e Antonio De Lucreziis}
\date{I Semestre 2021/2021}

\begin{document}

%
% Removes initial indentation from paragraphs.
%
\parskip 1ex
\setlength{\parindent}{0pt}

% Initial page
\maketitle

% Table of contents
\tableofcontents
\newpage

\chapter{Teoria della misura}

\section{Misure astratte}
Siano
\begin{itemize}
	\item $X$ un insieme qualunque;
	\item $\mathcal{A}$ una $\sigma$-algebra di sottoinsiemi di $X$. Ovvero una famiglia di sottinsiemi di $X$ che rispetta le seguenti proprietà.
		\begin{itemize}
			\item $\emptyset, X \in \mathcal{A}$;
			\item $\mathcal{A}$ è chiusa per complementare, unione e intersezione numerabile.
			\end{itemize}
	\item $\mu$ una misura su $X$, cioè una funzione $\mu \colon A \to [0,+\infty]$ $\sigma$-addittiva, cioè tale che data una famiglia numerabile $\left\{ E_k \right\} \subset A$ disgiunta e posto $E \coloneqq \bigcup E_n $, allora
	\[
		\mu(E) = \sum_{n} \mu (E_n).
	\] 
\end{itemize}
%
Seguono le proprietà.
\begin{itemize}
	\item $\mu(\emptyset) = 0$;
	\item \textit{monotonia:} dati $E,E' \in \mathcal{A}$ e $E \subset E'$, allora $\mu(E) \leq \mu(E')$;
	\item data una successione crescente di insiemi, $E_n \uparrow E$, allora $\mu(E) = \lim_{n \to \infty} \mu(E_n) = \sup_{n} \mu(E_n)$;
	\item se $E_n \uparrow E$ e $\mu (E_{\overline{n}}) < +\infty$ per qualche $\overline{n}$, allora $\mu(E) = \lim_{n \to + \infty} \mu(E_n) = \inf_{n} \mu(E_n)$;
	\item \textit{subadditività} se $\bigcup E_n \supset E$, allora $\mu(E) \leq \sum_{n}^{} \mu(E_n)$.
\end{itemize}
%
Dove una successione crescente di insiemi $E_n \uparrow E$ è tale che $E_1 \subset E_2 \subset \ldots E_n \subset \ldots $ e $\bigcup E_n = E$.
%
\begin{osservazione}
Dato $X' \in \mathcal{A}$ si possono restringere $\mathcal{A}$ e $\mu$ a $X'$.
\end{osservazione}
%
\textbf{Terminologia}.
\begin{itemize}
	\item Sia $P(X)$ un'affermazione che dipende da $x \in X$. Si dice che $P(X)$ vale per quasi $\mu$-quasi ogni $x \in X$ se l'insieme $\left\{ x \colon P(x) \text{ non vale }  \right\}$ è (contenuto in) un insieme di misura $\mu$ nulla.
	\item $\mu$ si dice \textit{completa} se $F \subset E, E \in \mathcal{A}$ e $\mu(E) = 0$, allora $F \in \mathcal{A}$ (e di conseguenza $\mu(F) = 0$).
	\item $\mu$ si dice \textit{finita} se $\mu(X) < + \infty$.
\end{itemize}
%
D'ora in poi consideriamo solo misure complete.
%
\section{Esempi di misure}
\textit{Misura che conta i punti}. Siano
\begin{itemize}
	\item $X$ qualunque
	\item $\mathcal{A} \coloneqq \mathcal{P}(X) = \left\{ \text{sottoinsiemi di } X \right\}$;
	\item $\mu(E) \coloneqq \# E \in \N \cup \left\{ +\infty \right\}$.
\end{itemize}
%
\textit{Delta di Dirac in $x_0$}. Siano
\begin{itemize}
	\item $X$ qualunque
	\item $\mathcal{A} \coloneqq \mathcal{P}(X)$;
	\item $x_0 \in X$ fissato, allora $\mu(E) \coloneqq \delta_{x_0}(E) = \One_E (x_0)$.
\end{itemize}
%
2. \textit{Misura di Lebesgue}
Siano
\begin{itemize}
	\item $X = \R^n$;
	\item $\mathcal{M}^n$ la $\sigma$-algebra dei misurabili secondo Lebesgue;
	\item $\mathcal{L}^n$ la misura di Lebesgue.
\end{itemize}
%
Definiamo la misura di Lebesgue $\mathcal{L}^n$.

Dato $R$ parallelepipedo in $\R^n$, cioè $R = \prod_{k=1}^{n} I_k $ con $I_k$ intervalli in $\R$.
Si pone
\[
\mathrm{vol}_n (R) \coloneqq \prod_{k=1}^{n}  \mathrm{lungh} (I_k)
\] 
per ogni $E \subset \R^n$. Si pone
\[
\mathcal{L}^n(E) \coloneqq \inf \left\{ \sum_{i}^{} \mathrm{vol}_n (R_i) \mid \left\{ R_i \right\} \text{tale che} \cup_i R_i \supset E  \right\}.
\] 
\begin{osservazione}
Seguono le seguenti osservazioni.
\begin{itemize}
\item $\mathcal{L}^n(R) = \mathrm{vol}_n (R)$ (fatto non del tutto ovvio);
\item $\mathcal{L}^n$ è così definita se $\mathcal{P}(\R^n)$ ma non è $\sigma$-addittiva;
\item $\mathcal{L}^n$ è $\sigma$-addittiva su $\mathcal{M}^n$ (è per questo che bisogna introdurre $\mathcal{M}^n$).
\end{itemize}
\end{osservazione}
%
Definizione di $\mathcal{M}^n$. \\
Dato $E \subset \R^n$, si dice che $E$ è misurabile (secondo Lebesgue) se per ogni $\epsilon > 0$ eiste $A$ aperto, $C$ chiuso tale che
\begin{itemize}
\item $C \subset E \subset A$,
\item $\mathcal{L}^n (A \smallsetminus C) \leq \epsilon$.
\end{itemize}
%
\begin{osservazione}
Seguono le seguenti osservazioni.
\begin{itemize}
\item Per ogni $E$ misurabile vale
\[
\mathcal{L}^n = \inf \left\{ \mathcal{L}^n \colon A \ \text{aperto}, A \supset E \right\} = \sup \left\{ \mathcal{L}^n \colon K \ \text{compatto}, K \subset E \right\}.
\] 
\item $F \subset E$ con $E \subset \mathcal{M}^n$ e $\mathcal{L}^n(E) = 0$, allora $F \in \mathcal{M}^n$. Ovvero la misura di Lebesgue è completa!
\end{itemize}
%
Useremo spesso la notazione
\[
\left| E \right| \coloneqq \mathcal{L}^n (E).
\] 
\end{osservazione}

prova

\chapter{Spazi $L^p$ e convoluzione}

% TODO

\chapter{Spazi di Hilbert}

% TODO

\chapter{Serie di Fourier}

% TODO

\chapter{Applicazioni della serie di Fourier}

% TODO

\chapter{Trasformata di Fourier}

% TODO

\chapter{Funzioni armoniche}

% TODO

\chapter{Integrazione di superfici}

% TODO

\newpage

prova test prova 

\end{document}
