\documentclass[a4paper, 12pt]{report}

%%%
%%% Use a custom geometry for the page
%%%
\usepackage{geometry}
\geometry{
  a4paper,
  left=20mm,
  right=20mm,
  top=20mm,
  bottom=25mm
}

\usepackage{lipsum}

%%%
%%% File encoding
%%%
\usepackage[utf8]{inputenc}
\usepackage[italian]{babel}
\usepackage[T1]{fontenc}

%%%
%%% Font
%%%
\usepackage{lmodern}
\usepackage{textcomp}
\usepackage[font=itshape]{quoting}

%%%
%%% References
%%%
\usepackage[perpage]{footmisc}
\usepackage{hyperref}
\hypersetup{ 
  linkbordercolor={.77 .4 .20},
  citebordercolor={.31 .63 .31},
  pdftitle={Appunti}
}

%%%
%%% Pictures
%%%

% General
\usepackage{caption}
\usepackage{graphicx}

% TikZ
\usepackage{tikz}
\usetikzlibrary{cd}

% Inkscape figures
\usepackage{import}
\usepackage{xifthen}
\usepackage{pdfpages}
\usepackage{transparent}
\newcommand{\incfig}[1]{%
    \def\svgwidth{\columnwidth}
    \import{./figures}{#1.pdf_tex}
}

%%%
%%% Basic math and layout packages
%%%
\usepackage{amssymb}
\usepackage{amsmath}
\usepackage{amsthm}
\usepackage{mathrsfs}
\usepackage{mathtools}
% \usepackage{xfrac}
\usepackage{array}
\usepackage[all]{xy}
\usepackage{stmaryrd}
\usepackage{multicol}
\usepackage{centernot}

%%%
%%% Enumerations
%%%
\usepackage{enumitem}

% Custom settings for compact lists
\setitemize{topsep=5pt}
\setlist[itemize,1]{label=$\bullet$}
\setlist[itemize,2]{label=$\circ$}

\setlist[enumerate,1]{label=\roman*)}
\setenumerate{topsep=5pt}

%%%
%%% Sections and page borders styles
%%%

% \usepackage{titlesec}
% Adds horizontal rules under "\section" entries
% \titleformat{\section}{\Large\scshape\raggedright}{}{0em}{}[\titlerule]
% \usepackage{fancyhdr}
% \pagestyle{fancy}
% \pagestyle{fancyplain} % Makes all pages in the document conform to the custom headers and footers
% \fancyhead[L]{\leftmark}% Empty left header
% \fancyhead[C]{} % Page numbering for center header  
% \fancyhead[R]{\rightmark}% Empty right header
% \fancyfoot[L]{}% Empty left footer
% \fancyfoot[C]{\thepage}% Empty center footer
% \fancyfoot[R]{}% Empty left footer
% \fancyhf{}
% \chead{\footnotesize\rightmark}
% \rfoot{\thepage}

%%%
%%% Styles by @aziis98
%%%

% Better "\setminus"
\newcommand{\mysetminusD}{%
  \hbox{\tikz{\draw[line width=0.6pt,line cap=round] (3pt,0) -- (0,6pt);}}
}
\newcommand{\mysetminusS}{%
  \hbox{\tikz{\draw[line width=0.45pt,line cap=round] (2pt,0) -- (0,4pt);}}
}
\newcommand{\mysetminusSS}{%
  \hbox{\tikz{\draw[line width=0.4pt,line cap=round] (1.5pt,0) -- (0,3pt);}}
}
\newcommand{\mysetminusT}{\mysetminusD}
\newcommand{\mysetminus}{\mathbin{\mathchoice{\mysetminusD}{\mysetminusT}{\mysetminusS}{\mysetminusSS}}}
\renewcommand{\setminus}{\mysetminus}

% Generali
% \renewcommand{\implies}{\Rightarrow}
% \renewcommand{\iff}{\Leftrightarrow}
\newcommand{\longiff}{\Longleftrightarrow}
\newcommand{\norm}[1]{\left\lVert #1 \right\rVert}
\newcommand{\sfrac}[2]{
    \raisebox{+0.55ex}{$#1$}
    /
    \raisebox{-0.3ex}{$#2$}
}

\renewcommand{\epsilon}{\varepsilon}
\newcommand{\myphi}{\varphi}
\newcommand{\compose}{\circ}
\newcommand{\mquad}{\;\;}
\newcommand{\mymid}{\;\middle|\;}
\newcommand{\curry}{\,\cdot\,}
\newcommand{\C}{\mathbb C}
\newcommand{\R}{\mathbb R}
\newcommand{\Z}{\mathbb Z}
\newcommand{\N}{\mathbb N}
\newcommand{\ds}{\displaystyle}
\newcommand{\dd}{\,\mathrm{d}}
\newcommand{\pd}{\partial}

% Fulmine dell'assurdo
\usepackage{stmaryrd}
\let\stmaryrdLightning\lightning
\renewcommand{\lightning}{\stmaryrdLightning}
\newcommand{\absurd}{$\lightning$}

% Funzione indicatrice
\newcommand{\bbOne}{\text{\usefont{U}{bbold}{m}{n}1}}
\MakeRobust{\bbOne}
\newcommand{\One}{\bbOne}

% Ambienti temporanei
% [Convergenza Monotona (Beppo Levi)]

\newenvironment{theorem}{\textbf{Teorema.}}{}
\newenvironment{named-theorem}[1]{\textbf{Teorema.} (\textit{#1})}{}

% \newcommand{\TdF}{\mathcal F}
\newcommand{\foralmostall}{\widetilde\forall}
\newcommand{\almosteverywhere}{\;\;\text{q.o.}}

\let\oldtilde\tilde
\renewcommand{\tilde}{\widetilde}

%%%
%%% Styles by @<arianna>
%%%

%%
%% Nuovi comandi
%%

% Insiemi numerici
% \newcommand{\N}{\mathbb{N}}
% \newcommand{\Z}{\mathbb{Z}}
% \newcommand{\Q}{\mathbb{Q}}
% \newcommand{\R}{\mathbb{R}}
% \newcommand{\C}{\mathbb{C}}
\newcommand{\restr}[2]{\left.#1\right|_{#2}}

% Dichiarazione lettera Chi maiuscola -> \Chi
\DeclareRobustCommand{\rchi}{{\mathpalette\irchi\relax}}
\newcommand{\irchi}[2]{\raisebox{\depth}{\mbox{\Large$#1\chi$}}} % inner command, used by \rchi%
\newcommand{\Chi}{\rchi}

% Dichiarazioni nuovi ambienti
% \theoremstyle{definition}
% \newtheorem{definizione}{Definizione}[chapter]
% \newtheorem{teorema}{Teorema}[chapter]
% \newtheorem{proposizione}[definizione]{Proposizione}
% \newtheorem{lemma}[teorema]{Lemma}
% \newtheorem{corollario}[teorema]{Corollario}
% \newenvironment{dimostrazione}{\textit{Dimostrazione:}}{\hfill$\square$\break}
\newtheorem{osservazione}{Osservazione}
% \newtheorem{proprieta}[teorema]{Proprietà}
% \newtheorem{esempio}[teorema]{Esempio}
% \newtheorem{nota}[osservazione]{Nota}


\title{{\Huge Analisi 3}\\{\small Appunti di Analisi 3 del corso di Giovanni Alberti e Maria Stella Gelli}}
\author{Arianna Carelli e Antonio De Lucreziis}
\date{I Semestre 2021/2021}

\begin{document}

%
% Removes initial indentation from paragraphs.
%
\parskip 1ex
\setlength{\parindent}{0pt}

% Initial page
\maketitle

% Table of contents
\tableofcontents

\newpage

% Teoria della misura
%
% Lezione del 27 settembre
%

\chapter{Teoria della misura}

\section{Misure astratte}

\textbf{Definizione.} 
Uno spazio misurabile è una terna $(X, \mc A, \mu)$ tale che
\begin{itemize}
	
	\item $X$ è un insieme qualunque.

	\item $\mc{A}$ è una $\sigma$-algebra di sottoinsiemi di $X$ (chiamata $\sigma$-algebra dei misurabili) ovvero una famiglia di sottoinsiemi di $X$ che rispetta le seguenti proprietà:
		\begin{itemize}
			\item $\emptyset, X \in \mc{A}$.
			\item $\mc{A}$ è chiusa per complementare, unione e intersezione numerabile.
		\end{itemize}
	
	\item $\mu$ è una misura su $X$, ossia una funzione $\mu \colon \mc A \to [0, +\infty]$ $\sigma$-addittiva, cioè tale che data una famiglia numerabile $\left\{ E_k \right\} \subset \mc A$ disgiunta e posto $E \coloneqq \bigcup E_n $, allora
		$$
		\mu(E) = \sum_{n} \mu (E_n).
		$$

\end{itemize}


\textbf{Notazione.}
Data una successione crescente di insiemi $E_1 \subset E_2 \subset \cdots E_n \subset \cdots$ con $\bigcup E_n = E$, scriviamo $E_n \uparrow E$.

\textbf{Proprietà.}
\begin{itemize}
	\item $\mu(\emptyset) = 0$
	\item \textit{Monotonia}: Dati $E,E' \in \mc{A}$ e $E \subset E'$, allora $\mu(E) \leq \mu(E')$.
	\item Data $E_n \uparrow E$, vale $\ds \mu(E) = \lim_{n \to \infty} \mu(E_n) = \sup_{n} \mu(E_n)$.
	\item Se $E_n \uparrow E$ e $\mu (E_{\bar{n}}) < +\infty$ per qualche $\bar{n}$, allora $\ds \mu(E) = \lim_{n \to \infty} \mu(E_n) = \inf_{n} \mu(E_n)$.
	\item \textit{Subadditività}: Se $\ds E \subset \bigcup E_n $, allora $\ds \mu(E) \leq \sum_{n} \mu(E_n)$.
\end{itemize}

\textbf{Osservazione.} 
Dato $X' \in \mc A$ si possono restringere $\mc A$ e $\mu$ a $X'$ nel modo ovvio.

\textbf{Definizioni}.
\begin{itemize}
	\item $\mu$ si dice \textbf{completa} se $F \subset E, E \in \mc{A}$ e $\mu(E) = 0$, allora $F \in \mc{A}$ (e di conseguenza $\mu(F) = 0$).
	\item $\mu$ si dice \textbf{finita} se $\mu(X) < + \infty$.
	\item $\mu$ si dice \textbf{$\sigma$-finita} se esiste una successione $\{ E_n \}$ con $E_n \subset E_{n+1}$ tale che $\bigcup E_n = X$ con $\mu(E_n) < +\infty$ per ogni $n$.
\end{itemize}

\textbf{Notazione.}
Sia $P(X)$ un predicato che dipende da $x \in X$ allora si dice che \textbf{$P(X)$ vale $\mu$-quasi ogni $x \in X$} se l'insieme $\left\{ x \mid P(x) \text{ è falso}  \right\}$ è (contenuto in) un insieme di misura $\mu$ nulla.

D'ora in poi consideriamo solo misure complete.

\section{Esempi di misure}

\begin{itemize}
	
	\item \textbf{Misura che conta i punti.}
		$$
		X \text{ insieme}
		\qquad
		\mc A \coloneqq \mc P(X)
		\qquad
		\mu(E) \coloneqq \# E \in \N \cup \left\{ +\infty \right\}
		$$

	\item \textbf{Delta di Dirac in $x_0$.}
		$$
		X \text{ insieme, } x_0 \in X \text{ fissato}
		\qquad
		\mc A \coloneqq \mc P(X)
		\qquad
		\mu(E) \coloneqq \delta_{x_0}(E) = \One_E (x_0)
		$$

	\item \textbf{Misura di Lebesgue.}
		$$
		X = \R^n
		\qquad
		\mc{M}^n \text{ $\sigma$-algebra dei misurabili secondo Lebesgue}
		\qquad
		\mathscr L^n \text{ misura di Lebesgue}
		$$
		Dato $R$ parallelepipedo in $\R^n$, cioè $R = \prod_{k=1}^{n} I_k $ con $I_k$ intervalli in $\R$.
		Si pone
		$$
		\mathrm{vol}_n (R) \coloneqq \prod_{k=1}^{n}  \mathrm{lungh} (I_k)
		$$ 
		per ogni $E \subset \R^n$ (assumendo $\mathrm{lungh}([a, b]) = b - a$). Infine poniamo
		$$
		\mathscr L^n(E) \coloneqq \inf \left\{ \sum_{i} \mathrm{vol}_n (R_i) \mymid \left\{ R_i \right\} \text{tale che } E \subset \bigcup_i R_i  \right\}.
		$$
\end{itemize}
 
\textbf{Osservazioni.}
\begin{itemize}
	\item $\mathscr L^n(R) = \mathrm{vol}_n (R)$.

	\item $\mathscr L^n$ non è $\sigma $-addittiva su $\mc{P}(\R^n)$.
\end{itemize}

Il secondo punto giustifica l'introduzione della $\sigma $-algebra dei \textbf{misurabili secondo Lebesgue} che denotiamo con $\mc{M}^n$.

Dato $E \subset \R^n$ si dice che $E$ è misurabile (secondo Lebesgue) se
$$
\forall \epsilon > 0 \; \exists A \text{ aperto e } C \text{ chiuso, tali che }
C \subset E \subset A \; \text{e} \; \mathscr L^n (A \smallsetminus C) \leq \epsilon.
$$

\textbf{Osservazioni.}
\begin{itemize}
	\item Per ogni $E$ misurabile vale
$$
	\mathscr L^n(E) = \inf \left\{ \mathscr L^n \colon A \ \text{aperto}, A \supset E \right\} = \sup \left\{ \mathscr L^n \colon K \ \text{compatto}, K \subset E \right\}.
$$
	\item Notiamo che se $F \subset E$ con $E \subset \mc{M}^n$ e $\mathscr L^n(E) = 0$, allora $F \in \mc{M}^n$. Ovvero la misura di Lebesgue è completa!
\end{itemize}

\textbf{Notazione.} $\left| E \right| \coloneqq \mathscr L^n (E)$


% Teoria dell'integrazione, teoremi di convergenza, teorema di Fubini-Tonelli
%
% Lezione del 29 settembre
%

\section{Funzioni misurabili}

\textbf{Definizione.}
Dato $(X, \mc{A}, \mu)$ e $f \colon X \to \R$ (o al posto di $\R$ in $Y$ spazio topologico), diciamo che $f$ è \textbf{misurabile} (più precisamente $\mc{A}$-misurabile), se
$$
\forall A \text{ aperto} \; f^{-1} (A) \in \mc{A}
$$ 


\textbf{Osservazioni.}
\nopagebreak
\begin{itemize}
	\item Dato $E \subset X$, vale $E \in \mc{A}$ se solo se $\One_E$ è misurabile.
	\item La classe delle funzioni misurabili è chiusa rispetto a molte operazioni
	\begin{itemize}
		\item \textit{Somma}, \textit{prodotto} (se hanno senso nello spazio immagine della funzione).
		\item \textit{Composizione con funzione continua}: Se $f \colon X \to Y$ continua e $g \colon  Y \to Y'$ continua, allora $g \circ f$ è misurabile.
		\item \textit{Convergenza puntuale}: data una successione di $f_n$ misurabili e $f_n \to f$ puntualmente, allora $f$ è misurabile.
		\item $\liminf$ e $\limsup$ (almeno nel caso $Y = \R$).
	\end{itemize}
\end{itemize}


\subsection{Funzioni semplici}

% Indico con $\mc{S}$ la classe delle funzioni $f \colon  X \to \R$ \textit{semplici}, cioè della forma $f = \sum_{i}^{n} \alpha_i \One_{E_i}$ con $\{E_i \}_{1 \leq i \leq n}$ misurabili e $\alpha_i \in \R$.
\textbf{Definizione.}
Definiamo la classe delle \textbf{funzione semplici} come
$$
\mc S := \left\{ f \colon X \to \R \mymid f = \sum_i \alpha_i \One_{E_i} \text{ con $E_i$ misurabili e $\{\alpha_i\}$ finito} \right\}
$$

\textbf{Osservazione.} La rappresentazione di una funzione semplice come combinazione lineare di indicatrici di insiemi \textit{non è unica}, però se necessario possiamo prendere gli $E_i$ disgiunti.

\section{Integrale}

\textbf{Definizione.}
Diamo la definizione di $\ds\int_X f \dd \mu$ per passi
\begin{enumerate}
	\item \label{item:def_int_1} 
		Se $f \in \mc{S}$ e $f \geq 0$ cioè $f = \sum_i \alpha_i \One_{E_i}$ con $\alpha_i \geq 0$ allora poniamo
$$
			\int_{X} f \dd \mu \coloneqq \sum_{i} \alpha_i \mu(E_i),
$$
		convenendo che $0 \cdot +\infty = 0$ in quanto la misura di un insieme non è necessariamente finita.
	
	\item \label{item:def_int_2} 
		Se $f \colon  X \to [0,+\infty]$ misurabile si pone
$$
			\int_{X} f \dd \mu \coloneqq \sup_{\substack{g \in \mc{S} \\ 0 \leq g \leq f}} \int_{X} g \dd \mu.
$$
		
	\item 
		$f \colon X \to \overline{\R}$ misurabile si dice \textbf{integrabile} se 
$$
			\int_{X} f^+ \dd \mu < + \infty \quad \text{oppure} \quad \int_{X} f^- \dd \mu < +\infty.
$$
		e per tali $f$ si pone
$$
			\int_{X} f \dd \mu \coloneqq  \int_{X} f^+ \dd \mu - \int_{X} f^- \dd \mu.
$$
	
	\item 
		$f \colon X \to \R^n$ si dice \textbf{sommabile} (o di \textbf{classe} $\mathscr L^1$) se $\int_X \left| f \right| \dd \mu < +\infty$. In tal caso, se $\int_X f_i^{\pm} \dd \mu < +\infty$ per ogni $f_i$ componente di $f$, allora $\int_X f \dd \mu$ esiste ed è finito.
\end{enumerate}

Per tali $f$ si pone
$$
	\int_{X} f \dd \mu \coloneqq  \left( \int_X f_1 \dd \mu, \ldots , \int_X f_n \dd \mu \right).
$$

\textbf{Notazione.}
Scriveremo spesso $\ds \int_E f(x) \dd x$ invece di $\ds \int_E f \dd \mathscr L^n$ .

\textbf{Osservazioni.}
\begin{itemize}
	\item L'integrale è lineare (sulle funzioni sommabili).
	
	\item I passaggi \ref{item:def_int_1} e \ref{item:def_int_2} danno lo stesso risultato per $f$ semplice $\geq 0$.
	
	\item La definizione in \ref{item:def_int_2} ha senso per ogni $f \colon X \to [0,+\infty]$ anche non misurabile. Ma in generale vale solo che
		$$
		\int_{X} f_1 + f_2 \dd \mu \geq \int_X f_1 \dd \mu + \int_X f_2 \dd \mu.
		$$
	
	\item Dato $E \in \mc{A}$, $f$ misurabile su $E$, notiamo che vale l'uguaglianza
		$$
		\int_E f \dd \mu \coloneqq \int_X f \cdot \One_E \dd \mu.
		$$ 
	
	\item Si può definire l'integrale anche per $f \colon X \to Y$ con $Y$ \textit{spazio vettoriale normato finito dimensionale}\footnote{È necessario avere uno spazio vettoriale, perché serve la linearità e la moltiplicazione per scalare} ed $f$ sommabile.
	
	\item Se $f_1 = f_2$ $\mu$-q.o. allora $\ds \int_X f_1 \dd \mu = \int_X f_2 \dd \mu$.
	
	\item Si definisce $\ds \int_X f \dd \mu$ anche se  $f$ è misurabile e definita su $X \setminus N$ con $\mu(N) = 0$.
	
	\item Se $f \colon [a,b] \to \R$ è integrabile secondo Riemann allora è misurabile secondo Lebesgue e le due nozioni di integrale coincidono. 
		
		\textbf{Nota.} Lo stesso vale per integrali impropri di funzioni positive. Ma nel caso più generale non vale: se consideriamo la funzione
		$$
		f \colon (0,+\infty) \to \R 
		\qquad
		f(x) \coloneqq \dfrac{\sin x}{x}
		$$
		allora l'integrale di $f$ definito su $(0,+\infty)$ esiste come integrale improprio ma non secondo Lebesgue, infatti
		$$
		\int_0^{+\infty} f^+ \dd x = \int_0^{+\infty} f^- \dd x = +\infty
		$$
	
	\item $\ds \int_X f \dd \delta_{x_0} = f(x_0)$
	
	\item Se $X = \N$ e $\mu$ è la misura che conta i punti l'integrale è 
		$$
		\int_X f \dd \mu = \sum_{n = 0}^{\infty} f(n) 
		$$ 
		per le $f$ positive o tali che $\sum f^+(n) < +\infty $ oppure $\sum f^-(n) < +\infty $.
		
		\textbf{Nota.} Come prima nel caso di funzioni non sempre positive ci sono casi in cui la serie solita non è definita come integrale di una misura, ad esempio
		$$
		\sum_{n=1}^{\infty} \frac{(-1)^n}{n}
		$$
		esiste come serie ma non come integrale.
		
	\item Dato $X$ qualunque, $\mu$ misura che conta i punti e $f \colon  X \to [0,+\infty] $ possiamo definire la somma di tutti i valori di $f$ 
		$$
		\sum_{x \in X} f(x) \coloneqq \int_{X} f \dd \mu.  
		$$ 
\end{itemize}

\section{Teoremi di convergenza}

Sia $(X, \mc{A}, \mu)$ come in precedenza.

\textbf{Teorema} (di convergenza monotona o Beppo-Levi).
Date $f_n \colon  X \to [0,+\infty]$ misurabili, tali che $f_n \uparrow f$ ovunque in $X$, allora
$$
\lim_{n \to +\infty} \int_{X} f_n \dd \mu = \int_X f \dd \mu,
$$
dove
$$
\lim_{n \to +\infty} \int_{X} f_n \dd \mu = \sup_n \int f_n \dd \mu.
$$


\textbf{Teorema} (lemma di Fatou).
Date $f_n \colon X \to [0,+\infty]$ misurabili, allora
$$
\liminf_{n \to +\infty} \int_X f \dd \mu \geq \int_{X} \left( \liminf_{n \to +\infty} f_n \right) \dd \mu.
$$ 

\textbf{Teorema} (di convergenza dominata o di Lebesgue).
Date $f_n \colon  X \to \R$ (o anche $\R^n$) con le seguenti proprietà
\begin{itemize}
	\item \textit{Convergenza puntuale:} $f_n (x) \to f(x)$ per ogni $x \in X$.
	\item \textit{Dominazione:} Esiste $g \colon X \to [0,+\infty]$ sommabile tale che $\left| f_n (x) \right| \leq g(x)$ per ogni $x \in X$ e per ogni $n \in \N$.
\end{itemize}
allora
$$
\lim_{n \to \infty} \int_{X} f_n \dd \mu = \int_X f \dd \mu. 
$$ 

\textbf{Nota.}
La seconda proprietà è essenziale; sostituirla con $\ds \int_X \left| f_n \right| \dd \mu \leq C < + \infty$ non basta!

\textbf{Definizione.}
Data una \textit{densità} $\rho \colon  \R^n \to [0,+\infty]$ misurabile, la \textbf{misura $\mu$ con densità $\rho$} è data da
$$
\forall E \in \mc A \quad \mu(E) \coloneqq \int_{E} \rho \dd x
$$ 

\textbf{Osservazioni.}
\begin{itemize}
	\item $\R^n$ e $\mathscr L^n$ possono essere sostituiti da $X$ e $\widetilde{\mu}$.
	\item il fatto che $\mu$ è una misura segue da Beppo Levi, in particolare serve per mostrare la subadditività.
\end{itemize}


\textbf{Teorema} (di cambio di variabile).
Siano $\Omega$ e $\Omega'$ aperti di $\R^n$, $\Phi \colon \Omega \to \Omega' $ un diffeomorfismo \footnote{funzione differenziabile con inversa differenziabile.} di classe $C^1$ e $f \colon \Omega' \to [0,+\infty]$ misurabile. Allora
$$
\int_{\Omega'} f(x') \dd x' = \int_{\Omega} f(\Phi(x)) \left| \det(\nabla \Phi(x)) \right| \dd x.
$$

La stessa formula vale per $f$ a valori in $\overline{\R}$ integrabile e per $f$ a valori in $\R^n$ sommabile.

\textbf{Osservazioni.}
\begin{itemize}
	\item Se $n = 1$, $\left| \det(\Lambda \Phi(x)) \right| = \left| \Phi'(x) \right|$ e non $\Phi'(x)$ come nella formula vista ad Analisi 1 (l'informazione del segno viene data dall'inversione degli estremi).
	
	\item Indebolire le ipotesi su $\Phi$ è delicato. Basta $\Phi$ di classe $C^1$ e $\foralmostall x' \in \Omega' \; \# \Phi^{-1}(x') = 1$ (supponendo $\Phi$ iniettiva la proprietà precedente segue immediatamente).
	Se $\Phi$ non è "quasi" iniettiva bisogna correggere la formula per tenere conto della molteplicità.

	\item Quest'ultima osservazione serve giusto per far funzionare il cambio in coordinate polari che non è iniettivo solo nell'origine.
\end{itemize}

\subsection{Fubini-Tonelli}

Di seguito riportiamo il teorema di Fubini-Tonelli per la misura di Lebesgue.

\textbf{Teorema} (di Fubini-Tonelli).
Sia $\R^{n_1} \times \R^{n_2} \simeq \R^n$ con $n = n_1 + n_2$, $ E \coloneqq E_1 \times E_2 $ dove $E_1, E_2$ sono misurabili e $f$ è una funzione misurabile definita su $E$.
Se $f$ ha valori in $[0,+\infty]$ allora
$$
\int_X f \dd \mu 
= \int_{E_2} \int_{E_1} f(x_1,x_2) \dd x_1 \dd x_2 
= \int_{E_1} \int_{E_2} f(x_1,x_2) \dd x_2 \dd x_1
$$ 

Vale lo stesso per $f$ a valori in $\R$ o in $\R^n$ sommabile.

\textbf{Osservazioni.}
Possiamo generalizzare il teorema di Fubini-Tonelli a misure generiche ed ottenere alcuni risultati utili che useremo ogni tanto.
\begin{itemize}
	\item Se $X_1, X_2$ sono spazi con misure $\mu_1,\mu_2$ (con opportune ipotesi) vale:
		$$
		\int_{E_2} \int_{E_1} f(x_1,x_2) \dd \mu_1(x_1)  \dd \mu_2(x_2) 
		= \int_{E_1} \int_{E_2} f(x_1,x_2) \dd \mu_2(x_2)  \dd \mu_1(x_1).
		$$ 
		se $f\geq 0$ oppure $\ds \int_{X_1} \int_{X_2} \left| f \right| \dd \mu_2(x_2)  \dd \mu_1(x_1) < + \infty $.
	
	\item \textbf{Teorema} (di scambio serie-integrale). Se $X_1 \subset \R$ (oppure $X_1 \subset \R^n$), $\mu_1 = \mathscr L^n$ e $X_2 = \N$, $\mu_2$ è la misura che conta i punti, allora la formula sopra diventa
		$$
		\sum_{n=0}^{\infty} \, \int_{X_1} f_n(x) \dd x  
		= \int_{X_1} \sum_{n=0}^{\infty} f_n(x)  \dd x.
		$$ 
		se $f_i \geq 0$ oppure $\ds \sum_{i} \int_{X_1} \left| f_i(x) \right| \dd x  < + \infty $.
	
	\item \textbf{Teorema} (di scambio di serie). Se $X_1 = X_2 = \N$ e $\mu_1 = \mu_2$ è la misura che conta i punti la formula sopra diventa
		$$
		\sum_{j=0}^{\infty} \sum_{i=0}^{\infty} a_{i,j}  
		= \sum_{i=0}^{\infty} \sum_{j=0}^{\infty} a_{i,j} 
		$$ 
		se $a_{i,j} \geq 0$ oppure $\ds \sum_{i} \sum_{j} \left| a_{i,j} \right| < +\infty $.
\end{itemize}



% Introduzione spazi L^p. Disuguaglianze di Young, Holder, Minkowski
%
% Lezione del 30 settembre
%

\chapter{Spazi $L^p$ e convoluzione}

\section{Disuguaglianze}

\subsection{Disuguaglianza di Jensen}

Ricordiamo che una funzione $f \colon \R^d \to [-\infty, +\infty]$ è \textbf{convessa} se e solo se dati $x_1, \dots, x_n \in \R^d$ e $\lambda_1, \dots, \lambda_n \in [0, 1]$ con $\sum_i \lambda_i = 1$ abbiamo che
$$
	f \left(\sum_i \lambda_i x_i \right) \leq \sum_i \lambda_i f(x_i)
$$

\textbf{Definizione.} Una funzione $f \colon X \to \ol{\R}$ si dice \textbf{semicontinua inferiormente nel punto} $x_0 \in X$ se per ogni $t \in \R$ con $t < f(x_0)$ esiste $U \in \mc{I}(x_0)$ tale che $t < f(y)$ per ogni $y \in U$. Infine $f$ si dice \textbf{semicontinua inferiormente su} $X$ se lo è in ogni $x \in X$.

\textbf{Proprietà.} Dalla definizione segue che $f \colon X \to \ol{\R}$ è semicontinua inferiormente in $x_0 \in X$ se e solo se
$$
	f(x_0) \leq \sup_{U \in \mc{I}(x_0)} \inf f(y) = \liminf_{x \to x_0} f(x).
$$

\mybox{%
\textbf{Teorema} (Jensen).
Dato $(X, \mc A, \mu)$ con $\mu(X) = 1$ e $f \colon \R^d \to [-\infty, +\infty]$ convessa e semi-continua inferiormente (S.C.I.) e $u \colon X \to \R^d$ sommabile allora vale
$$
	f \left( \int_X u \dd \mu \right) \leq \int_X f \compose u \dd \mu
$$
e $f \compose u$ è integrabile.
}

\textbf{Osservazioni.}
\begin{itemize}
	\item $(f \compose u)^-$ ha integrale finito.

	\item Interpretando $\mu$ come probabilità si riscrive come $\mathbb E[f \compose \mu] \geq f(\mathbb{E}[u])$.

	\item Se $u$ è una funzione semplice, cioè $\ds u = \sum_i y_i \cdot \One_{E_i}$ con $E_i$ disgiunti e $\bigcup E_i = X$ allora posti $\lambda_i = \mu(E_i)$ abbiamo
		$$
		\int_X f \compose u \dd \mu = \int_X \sum_i f(y_i) \cdot \One_{E_i} \dd \mu = \sum_i \lambda_i f(y_i) \geq f \left( \sum_i \lambda_i y_i \right) = f \left( \int_X u \dd \mu \right)
		$$

		Questo ci darebbe una strada per dimostrare in generale per passi il teorema di Jensen ma in realtà si presentano vari problemi tecnici.

	\item Ogni funzione convessa e S.C.I su $\Omega$ convesso in $\R^d$ si estende a $\wtilde f \colon \R^d \to (-\infty, +\infty]$ convessa e S.C.I.; ad esempio se $\Omega = (0, +\infty)$
		$$
		f(y) = \frac{1}{y}
		\quad\rightsquigarrow\quad
		\wtilde f(y) = 
		\begin{cases}
			+\infty & y \leq 0 \\[5pt]
			\dfrac{1}{y} & y > 0
		\end{cases}
		$$

	\item L'ipotesi di semi-continuità è non banale perché le funzioni convesse sono continue solo se a valori in $\R$, ad esempio per $k$ costante la funzione
		$$
		f(y) := 
		\begin{cases}
			k & y < 0 \\
			+\infty & y \geq 0
		\end{cases}
		$$
		è convessa ma non semi-continua inferiormente (e neanche continua).

	\item Se $0 < \mu(X) < +\infty$ la disuguaglianza di Jensen va modificata come segue
	$$
		\fint_X f \circ u \dd \mu \leq f \left( \fint_X u \dd \mu \right).
	$$
	Per la dimostrazione ci si riconduce al caso precedente sostituendo $\mu$ con $\wtilde{\mu} \coloneqq m \cdot \mu$, con $m \coloneqq 1 / \mu(X)$.
	
\end{itemize}

\textbf{Dimostrazione.}
Poniamo $\ds y_0 \coloneqq \int_X u \dd \mu$, allora la tesi diventa
$$
	f \bigg( \int_X u \dd \mu \bigg) \leq \int_X f \compose u \dd \mu
	\quad
	\longiff
	\quad
	f(y_0) \leq \int_X f \compose u \dd \mu.
$$
Prendiamo $\phi \colon \R^d \to \R$ affine (ovvero $\phi(y) = a \cdot y + b$ con $a \in \R^d$ e $b \in \R$) tale che $\phi \leq f$, allora
$$
	\int_X f \compose u \dd \mu \underset{(\star)}{\geq} \int_X \phi \compose u \dd \mu = \int_X a \cdot u + b \dd \mu = a y_0 + b = \phi(y_0)
$$

Concludiamo usando il seguente lemma di caratterizzazione delle funzioni convesse ed S.C.I.

\textbf{Lemma.}
Ogni $f \colon \R^d \to (-\infty, +\infty]$ convessa e S.C.I è tale che
$$
\forall y_0 \in \R^d \quad \sup_{\substack{\phi \text{ affine} \\ \phi \leq f}} \phi(y_0) = f(y_0)
$$

Infine, si dimostra che $(f \compose u)^-$ è integrabile. 
Dal fatto che $f \circ u \geq \phi \circ u$ con $\phi \circ u$ sommabile\footnote{$u$ è sommabile e $\phi$ è affine.} si ha che
\begin{align*}
	& \Longrightarrow \left( f \circ u \right)^- \leq \left( \phi \circ u \right)^- \\
	& \Longrightarrow \left( f \circ u \right)^- \quad \text{è sommabile} \\
	& \Longrightarrow f \circ u \quad \text{è integrabile}
\end{align*}
\qed

\textbf{Nota.} Nel caso $d = 1$ e $f \colon \R \to \R$ possiamo usare il fatto che le funzioni convesse ammettono sempre derivata destra o sinistra, il $\sup$ diventa un massimo e ci basta prendere come $\phi$ la retta tangente in $(y_0, f(y_0))$ o una con pendenza compresa tra $f'(y_0^-)$ e $f'(y_0^+)$.

\textbf{Definizione.} Dati $p_1, p_2 \in [1, +\infty]$ diciamo che sono \textbf{coniugati} se
$$
\frac{1}{p_1} + \frac{1}{p_2} = 1
$$
convenendo che $1/\infty = 0$.

Fissiamo $p \in [1, +\infty]$ detto \textit{esponente di sommabilità} e sia $(X, \mc A, \mu)$ come sempre.

\textbf{Definizione.} Data $f \colon X \to \overline \R$ o $\R^d$ misurabile, la \textbf{norma} $p$ di $f$ è
$$
	\norm{f}_p \coloneqq \left( \int_X |f|^p \dd \mu \right)^{1/p} \quad p \in [1, +\infty)
$$
mentre per $p = +\infty$ poniamo
$$
	\norm{f}_\infty = \operatorname{supess}_{x \in X} f(x) \coloneqq \inf \{ m \in [0, +\infty] \mid |f(x)| \leq m \text{ per $\mu$-q.o. } x \}.
$$
\textbf{Nota.} In realtà queste sono solo delle semi-norme.

\begin{itemize}
	\item $\ds \norm{f}_\infty \leq \sup_{x \in X} |f(x)|$

	\item $\norm{f}_p = 0 \iff f = 0$ quasi ovunque

		\textbf{Dimostrazione.}
		\begin{itemize}
			\item[$\boxed{\Rightarrow}$] [TODO: Facile ma non ovvia]
			\item[$\boxed{\Leftarrow}$] Ovvio.
		\end{itemize}
		\qed

	\item Se $f_1 = f_2$ quasi ovunque $\implies \norm{f_1}_p = \norm{f_2}_p$.

		\textbf{Dimostrazione.} 
		$f_1 = f_2$ quasi ovunque $\implies \exists D \subset X$ con $\mu(D) = 0$ tale che $f_1(x) = f_2(x)$ su $X \setminus D$, usiamo il fatto che l'integrale non cambia se modifichiamo la funzione su un insieme di misura nulla
		$$
		\norm{f_1}_p^p
		= \int_X |f_1|^p \dd \mu 
		= \int_{X \setminus D} |f_1|^p \dd \mu 
		= \int_{X \setminus D} |f_2|^p \dd \mu 
		= \int_{X} |f_2|^p \dd \mu 
		= \norm{f_2}_p^p
		$$ 
		\qed
\end{itemize}

\subsection{Disuguaglianza di Young}

\mybox{%
\textbf{Proposizione.}
Per ogni $a_1, a_2 \geq 0$ e $\lambda_1, \lambda_2 > 0$ con $\lambda_1 + \lambda_2 = 1$ abbiamo che
$$
	a_1^{\lambda_1} a_2^{\lambda_2} \leq \lambda_1 a_1 + \lambda_2 a_2
$$
inoltre vale l'uguale se e solo se $a_1 = a_2$.
}

\textbf{Dimostrazione.}
Se $a_1$ o $a_2 = 0$ allora è ovvia. Supponiamo dunque $a_1, a_2 > 0$. Per la concavità del logaritmo abbiamo
% e passiamo $a_1^{\lambda_1} a_2^{\lambda_2}$ al logaritmo
$$
	\lambda_1 \log a_1 + \lambda_2 \log a_2 \leq \log(\lambda_1 a_2 + \lambda_2 a_2),
	\longiff \log \left( a_1^{\lambda_1} a_2^{\lambda_2} \right) \leq \log(\lambda_1 a_2 + \lambda_2 a_2)
$$
e dalla monotonia
%
$$
	 a_1^{\lambda_1} a_2^{\lambda_2} \leq \lambda_1 a_2 + \lambda_2 a_2.
$$
%
Infine, il se e solo se per l'uguale segue dal fatto che il logaritmo è \textit{strettamente concavo}. 
\qed

\subsection{Disuguaglianza di H\"older}

\mybox{%
\textbf{Proposizione.}
Date $f_1, f_2 \colon X \to \overline\R$ o $\R^d$ e $p_1, p_2$ esponenti coniugati allora
$$
	\int_X |f_1| \cdot |f_2| \dd \mu \leq \norm{f_1}_{p_1} \cdot \norm{f_2}_{p_2}
$$ 
vale anche per $p = +\infty$ convenendo che $+\infty \cdot 0 = 0$ nel membro di destra.
}

\textbf{Dimostrazione.}
Se $\norm{f_1}_{p_1} = 0,+\infty$ o $\norm{f_2}_{p_2} = 0,+\infty$ la dimostrazione è immediata, supponiamo dunque $\norm{f_1}_{p_1}, \norm{f_2}_{p_2} > 0$ e finiti.

\begin{itemize}
	\item \textit{Caso 1:} se $p_1 = 1, p_2 = +\infty$ allora
		$$
		\int_X |f_1| \cdot |f_2| \dd \mu 
		\leq \int_X |f_1| \cdot \norm{f_2}_{\infty} \dd \mu
		= \norm{f_2}_{\infty} \cdot \int_X |f_1| \dd \mu
		= \norm{f_2}_{\infty} \cdot \norm{f_1}_{1} 
		$$

	\item \textit{Caso 2:} se $1 < p_1, p_2 < +\infty$, introduciamo un parametro $\gamma > 0$ allora
		$$
		\int_X |f_1| \cdot |f_2| \dd \mu 
		= \int_X \left( \gamma^{p_1} \cdot |f_1|^{p_1} \right)^{1/p_1} \cdot \left( \gamma^{-p_2} \cdot |f_2|^{p_2} \right)^{1/p_2} \dd \mu
		$$
		a questo punto chiamiamo per comodità $g_1 := \gamma^{p_1} \cdot |f_1|^{p_1}$, $\lambda_1 := 1 / p_1$ e $g_2 := \gamma^{-p_2} \cdot |f_1|^{p_2}$, $\lambda_2 := 1 / p_2$ da cui
		$$
		= \int_X g_1^{\lambda_1} \cdot g_2^{\lambda_2} 
		\overset{\text{Young}}{\leq} \int_X \lambda_1 g_1 + \lambda_2 g_2 \dd \mu
		= \lambda_1 \gamma^{p_1} \int_X |f_1|^{p_1} + \lambda_2 \gamma^{-p_2} \int_X |f_1|^{p_2} \dd \mu
		$$
		$$
		= \lambda_1 \gamma^{p_1} \cdot \norm{f_1}_{p_1}^{p_1} + \lambda_2 \gamma^{-p_2} \cdot \norm{f_2}_{p_2}^{p_2}
		$$
		posti ora $a_1 := \gamma^{p_1} \norm{f_1}_{p_1}^{p_1}$ e $a_2 := \gamma^{-p_2} \norm{f_1}_{p_2}^{p_2}$, per $\gamma \to 0$ abbiamo che $a_1 \to 0, a_2 \to +\infty$ mentre per $\gamma \to +\infty$ abbiamo che $a_1 \to +\infty, a_2 \to 0$ dunque per il teorema del valor medio esisterà $\gamma$ tale che $a_1 = a_2$, ma allora siamo nel caso dell'uguaglianza per la disuguaglianza di Young dunque
		$$
		\lambda_1 \gamma^{p_1} \norm{f_1}_{p_1}^{p_1} + \lambda_2 \gamma^{-p_2} \norm{f_1}_{p_2}^{p_2} 
		= \lambda_1 a_1 + \lambda_2 a_2 = a_1^{\lambda_1} \cdot a_2^{\lambda_2} 
		= \norm{f_1}_{p_1} \cdot \norm{f_2}_{p_2}
		$$
\end{itemize}
% In particolare, vale l'uguaglianza se prendiamo un valore di $\gamma$ tale che $a_1 = a_2$. Resta da verificare che tale valore di $\gamma$ esista [TODO].
\qed

\textbf{Osservazione.}
La disuguaglianza di H\"older può essere generalizzata a $n$ funzioni, date $f_1, \dots, f_n$ e $p_1, \dots, p_n$ con $\frac{1}{p_1} + \dots + \frac{1}{p_2} = 1$ allora
$$
\int_X \prod_i^n |f_i| \dd \mu \leq \prod_i^n \norm{f_i}_{p_i} 
$$

\subsection{Disuguaglianza di Minkowski}

\mybox{%
\textbf{Proposizione.} 
Consideriamo sempre $(X, \mc A, \mu)$ e sia $p \in [1, +\infty]$ un esponente di sommabilità ed $f_1, f_2 \colon X \to \ol{\R}$ oppure $\R^d$. Allora vale la disuguaglianza triangolare
$$
	\norm{f_1 + f_2}_p \leq \norm{f_1}_p + \norm{f_2}_p.
$$
}
%
\textbf{Dimostrazione.}
\begin{itemize}
	\item \textit{Caso 1:} se $p = 1$ o $p = +\infty$, allora svolgiamo il calcolo diretto
		
		\begin{itemize}
			\item Se $p = 1$
				$$
				\norm{f_1 + f_2}_1 
				= \int_X |f_1 + f_2| \dd \mu 
				\leq \int_X |f_1| + |f_2| \dd \mu 
				= \int_X |f_1| \dd \mu + \int_X |f_2| \dd \mu
				= \norm{f_1}_1 + \norm{f_2}_1
				$$
			\item Se $p = +\infty$ allora poniamo $D$ l'insieme di \textit{misura nulla} che realizza su $X \setminus D$ il supess\footnote{Si usa questa strategia perché il $\mathrm{supess}$ non soddisfa le proprietà del modulo e della somma del $\sup$.} ovvero $\mathrm{supess}_X |f_1 + f_2| = \mathrm{sup}_{X \,\setminus\, D} |f_1 + f_2|$
				$$
				\norm{f_1 + f_2}_\infty
				= \mathrm{supess}_X |f_1 + f_2| 
				= \mathrm{sup}_{X \,\setminus\, D} |f_1 + f_2| 
				\leq \mathrm{sup}_{X \,\setminus\, D} (|f_1| + |f_2|)
				$$
				$$
				\leq \mathrm{sup}_{X \,\setminus\, D} |f_1| + \mathrm{sup}_{X \,\setminus\, D} |f_2|
				= \mathrm{supess}_X |f_1| + \mathrm{supess}_X |f_2|
				= \norm{f_1}_\infty + \norm{f_2}_\infty
				$$
		\end{itemize}

	\item \textit{Caso 2:} se $1 < p < +\infty$ e $0 < \norm{f_1 + f_2}_p < +\infty$
		$$
		\begin{aligned}
			\norm{f_1 + f_2}_p^p 
			&= \int_X |f_1 + f_2|^p 
			\leq \int_X (|f_1| + |f_2|) \cdot |f_1 + f_2|^{p-1} \dd \mu = \\
			&= \int_X |f_1| \cdot |f_1 + f_2|^{p-1} \dd \mu + \int_X |f_2| \cdot |f_1 + f_2|^{p-1} \dd \mu =
		\end{aligned}
		$$
		ora introduciamo $q$ esponente coniugato di $p$ e notiamo
		$$
		q = \frac{p}{p-1} 
		\quad
		\text{e}
		\quad
		\norm{|f|^{p-1}}_q = \norm{f}_p^{p-1}
		$$
		ora continuiamo a svolgere il conto di prima usando H\"older con esponenti $p$ e $q$
		$$
		\begin{aligned}
		&\overset{\text{H\"older}}{\leq} \norm{f_1}_p \cdot \norm{|f_1 + f_2|^{p-1}}_q + \norm{f_2}_p \cdot \norm{|f_1 + f_2|^{p-1}}_q = \\
			& = (\norm{f_1}_p + \norm{f_2}_p) \cdot \norm{|f_1 + f_2|^{p-1}}_q 
			= (\norm{f_1}_p + \norm{f_2}_p) \cdot \norm{f_1 + f_2}_p^{p-1}  \\
		\end{aligned}
		$$
		infine per l'ipotesi $\norm{f_1 + f_2}_p > 0$ possiamo dividere per l'ultimo fattore e ottenere la tesi.
		% $$
		% \implies \frac{\norm{f_1 + f_2}_p^p }{\norm{f_1 + f_2}_p^{p-1}} \leq \norm{f_1}_p + \norm{f_2}_p
		% \implies \norm{f_1 + f_2}_p \leq \norm{f_1}_p + \norm{f_2}_p
		% $$

	\item \textit{Caso 3:} se $1 < p < +\infty$ ma $\norm{f_1 + f_2} = 0$ o $+\infty$ allora se $\norm{f_1 + f_2} = 0$ la disuguaglianza è banale mentre se $\norm{f_1 + f_2} = +\infty$ si usa la seguente disuguaglianza
		$$
			\norm{f_1 + f_2}_p^p \leq 2^{p-1} (\norm{f_1}_p^p + \norm{f_2}_p^p),
		$$
		che si ottiene usando la convessità della funzione $x \mapsto x^p$ e la combinazione affine $\frac{1}{2}x_1 + \frac{1}{2}x_2$ infatti
		$$
			\left( \frac{x}{2} + \frac{y}{2} \right)^p \leq \frac{1}{2} x^p + \frac{1}{2} y^p 
			\implies \frac{1}{2^{p-1}} (x+y)^p \leq x^p + y^p 
			\implies (x+y)^p \leq 2^{p-1} (x^p + y^p).
		$$
		% ma sostituendo con $f_1$ e $f_2$, integrando e poi sostituendo le norme otteniamo
		% $$
		% 	\norm{f_1 + f_2}_p^p \leq 2^{p-1} (\norm{f_1}_p^p + \norm{f_2}_p^p).
		% $$
		% $$
		% \norm{f_1 + f_2}_p^p 
		% = \int_X |f_1 + f_2|^p \dd \mu 
		% = 2^p \int_X \left| \frac{f_1 + f_2}{2} \right|^p \dd \mu 
		% $$
		% $$
		% \leq 2^p \int_X \frac{1}{2} |f_1|^p + \frac{1}{2}|f_2|^p \dd \mu 
		% = 2^{p-1} (\norm{f_1}_p^p + \norm{f_2}_p^p)
		% $$
		% da cui possiamo ricavare subito che almeno uno dei due termini deve essere $+\infty$.
\qed
\end{itemize}



% Esercitazione
%
% Lezione del 4 ottobre
%

\section{Esercitazione del 4 ottobre}

\subsection{Esercizi di teoria della misura}

Di seguito riportiamo alcune proprietà di base di teoria della misura.

\textbf{Proprietà.}

\begin{enumerate}
\item Se $A \subset B$, allora $\mu(A) \leq \mu(B)$.

\textbf{Dimostrazione.}
Scomponiamo $B = (B \setminus A) \cup (A \cap B)$. Per ipotesi $A \cap B = A$ ed essendo la misura positiva segue che
$$
	\mu(B) = \underbrace{\mu(B \setminus A)}_{\geq 0} + \mu(A) \geq \mu(A).
$$

\item \label{item:misura_unione_finita} Dati due insiemi $A,B$ misurabili, vale
$$
	\mu(A \cup B) \leq \mu(A) + \mu(B).
$$

\textbf{Dimostrazione.}
La disuguaglianza segue dalle seguenti uguaglianze.

\begin{align*}
	\mu(A) & = \mu(A \setminus B) + \mu(A \cap B) \\
	\mu(B) & = \mu(B \setminus A) + \mu(A \cap B) \\
	\mu(A \cup B) & = \mu(A \setminus B ) + \mu(B \setminus A) + \mu(A \cap B).
\end{align*}

\item Data una successione di insiemi $E_1 \subset E_2 \subset \cdots \subset \cdots$, si ha
$$
	\mu \left( \bigcup_{i} E_i \right) = \sup_i \mu(E_i) = \lim_i \mu (E_i).
$$

\item Data una successione di insiemi $E_1 \supset E_2 \supset \cdots \supset \cdots$ e $\mu(E_1) < +\infty$, si ha
$$
	\mu \left( \bigcap_{i} E_i \right) = \lim_i \mu (E_i).
$$
\end{enumerate}

\textbf{Esercizio} (Numerabile subaddittività).
Dato $\ds E \in \mc{A}, E \subset \bigcup_i E_i$ dove $E_i \in \mc{A}$. Allora
$$
	\mu(E) \leq \sum_{i}^{} \mu(E_i).
$$

\textbf{Dimostrazione} (Idea).
Basta dimostrare che $\ds \mu \left( \bigcup_i E_i \right) \leq \sum_{i}^{} \mu(E_i)$. Infatti per quanto visto prima $\ds \mu(E) \leq \mu \left( \bigcup_i E_i \right)$. Prima dimostriamo per induzione $\ds \mu \left( \bigcup_{i = 1}^N E_i \right) \leq \sum_{i=1}^{N} \mu(E_i)$. 

Il passo base $n = 2$ è stato visto al punto \ref{item:misura_unione_finita}. Una volta dimostrata la proprietà sopra, si nota che $\ds \sum_{i=1}^{N} \mu(E_i) $ è limitata per ogni $N$, e dunque è limitato anche il suo limite, da cui la tesi.
\qed

\subsection{Funzioni misurabili rispetto alla misura di Lebesgue}

Si ricorda che le funzioni \textit{continue}, \textit{semplici} e \textit{semicontinue} sono classi di funzioni misurabili.
Due osservazioni sulle funzioni semicontinue.
\begin{itemize}
\item Le funzioni semicontinue sono \textit{boreliane}.

\item La proprietà di misurabilità delle funzioni semicontinue è necessaria per l'enunciato della disuguaglianza di Jensen.
\end{itemize}

\textbf{Controesempio} (disuguaglianza di Jensen).
Notiamo che l'ipotesi di semicontinuità inferiore della funzione $f$ è necessaria per la validità della disuguaglianza di Jensen.
Infatti, definiamo $f$ come segue
%
$$
f(x) = 
\begin{cases}
0 \qquad \; x \in (0,1) \\
+ \infty \quad \text{altrimenti} 
\end{cases}.
$$
%
Osserviamo che la funzione $f$ così definita è convessa ma non semicontinua inferiormente.

Ora definiamo la funzione $u : X \to \R$ con $X = (0,2)$, come la funzione costante di valore $1/2$.
Calcoliamo l'integrale di $u(x)$ su $X$.
%
$$
	\int_{X} u(x) \dd x = 1. 
$$
%
In tal caso vale $\ds f \left( \int_X u(x) \dd x \right) = + \infty$.
D'altra parte $\ds \int_X f \compose u \dd x = 0$, dunque l'ipotesi di semicontinuità inferiore è necessaria.

\textbf{Fatto.}
Date $\myphi_1, \myphi_2$ funzioni semplici su $\R$ con misura di Lebesgue.
Allora $\myphi_1 \vee \myphi_2$ e $\myphi_1 \wedge \myphi_2$ sono ancora funzioni semplici.

\textbf{Lemma.}
Data $f \colon X \to [0, +\infty]$ misurabile
$$
\int_X f \dd \mu = 0 \quad \longiff \quad f = 0 \; \text{q.o. su } X.
$$

\textbf{Dimostrazione.}
\begin{itemize}

\item[$\boxed{\Rightarrow}$] Dato che $f$ è non negativa, il dominio $X$ può essere riscritto come
$$
	X = \left\{ x \in X \mymid f(x) \geq 0 \right\} = \left\{ x \in X \mymid f(x) > 0 \right\} \cup \left\{ x \in X \mymid f(x) = 0 \right\}
$$
ricordiamo che $(0, +\infty) = \bigcup_{n \geq 1} (\frac{1}{n}, +\infty)$ da cui segue
$$
	\left\{ x \in X \mymid f(x) > 0 \right\} =  \bigcup_{n \in \N \setminus \left\{ 0 \right\}} \left\{ x \in X \mymid f(x) \geq \frac{1}{n} \right\},
$$
e passiamo alle misure
$$
	\mu \left( \left\{ x \in X \mymid f(x) > 0 \right\} \right) 
	= \lim_{n \to +\infty} \mu\left(\left\{ x \in X \mymid f(x) \geq \frac{1}{n} \right\}\right),
$$
in questo modo otteniamo la seguente caratterizzazione dell'insieme su cui $f$ è positiva
$$
	\mu \left( \left\{ x \in X \mymid f(x) > 0 \right\} \right) >0 
	\longiff
	\exists \bar{n} \mid \mu \left( \left\{ x \in X \mymid f(x) \geq 1 / \bar{n} \right\} \right) > 0.
$$

% X @aziis98: Boh magari potremmo definire tipo L_n := { blob sotto l'integrale } per rendere un po' più leggibile quel dominio di integrazione.
% OK @aziis98: Oppure anche in notazione "probabilistica" $\{ f \geq \frac{1}{n} \}$ che è più corto
Allora possiamo maggiorare come segue
$$
	0 = \int_X f \dd \mu 
	\geq \;\int_{\left\{ f \,\geq\, \frac{1}{n} \right\}} \; f \dd \mu \geq	\frac{1}{n} \mu \left(  \left\{ x \mymid f(x) \geq \frac{1}{n} \right\} \right). 
$$
Dunque abbiamo
$$
	\mu \left(  \left\{ x \mymid f(x) \geq \frac{1}{n} \right\} \right) = 0 \qquad \forall n.
$$
Si conclude osservando che
$$
	\mu \left(  \left\{ x \mymid f(x) > 0 \right\} \right) = \lim_n \mu \left( \left\{ x \mymid f(x) \geq \frac{1}{n} \right\} \right) = 0.
$$

\item[$\boxed{\Leftarrow}$]
Dal fatto che $f$ è positiva possiamo scrivere
$$
	\int_X f \dd \mu = \sup_{\substack{g \leq f \\ g \; \text{semplice}}} \int_X g \dd \mu = \sup \sum_{i}^{} \alpha_i \mu(E_i) = 0. 
$$
\qed

\end{itemize}

\textbf{Osservazione} (sup essenziale di funzioni misurabili).
Data $f$ misurabile, definiamo
$$
	\norm{ f }_{\infty, X} \coloneqq \inf \left\{ m \in [0,+\infty] \mymid \left| f(x) \right| \leq m \quad \text{quasi ovunque}  \right\}.
$$

Se $\norm{ f }_{\infty} < + \infty$, allora diciamo che esiste una costante $L > 0$ con $L = \norm{ f }_{\infty, X}$, tale che 
$$
	\left| f(x) \right| \leq L
$$
quasi ovunque. 
Infatti, per definizione di $\inf$, $L = \lim_n m_n$, dove $m_n$ verificano
$$
	\left| f(x) \right| \leq m_n \quad \forall x \in X \setminus N_m, \quad \mu(N_m) = 0.
$$
Definiamo $N = \bigcup_{m} N_m$, da cui si ottiene
$$
	\mu(N) \leq \sum_{n=1}^{\infty} \mu (N_m) = 0. 
$$
Ovvero $N$ è trascurabile.
Preso $x \in X \setminus N$, vale
$$
	\left| f(x) \right| \leq m_n \quad \forall n \in \N.
$$


\subsection{Formula di cambio di variabile applicata a funzioni radiali}

Sia $f \colon [0,+\infty) \to \R$ misurabile (di solito si richiede misurabile e positiva oppure sommabile).
Vala la seguente
$$
	\int_{0}^{+\infty} f\left( \left| x \right| \right) \dd x = c_n \cdot \int_{0}^{+\infty} f(\rho) \rho^{n-1} \dd \rho,
$$
dove $\ds c_n = n \mathscr L^n \left( \mc{B}(0,1) \right)$.

Applichiamo questa formula alla stima di integrali di funzioni positive.

\textbf{Esercizio.}
Sia
$$
	\psi (x) = \frac{1}{\norm{ x }^{\alpha}}
$$
% X @aziis98: Boh secondo me possiamo usare anche solo B(0, 1) invece di \mc B(0, 1), sempre sul tema di fare con \mc solo l'insieme delle parti e le sigma-algebre?
su $\mc{B}(0,1) \in \R^n$. Notiamo che $\psi(x) = f(\norm{ x })$ con $f = 1 / t^\alpha$.
Usiamo la formula appena introdotta per determinare gli $\alpha \in \R$ per i quali $\psi$ è sommabile su $\mc{B}(0,1)$.
%
$$
\int_{\mc{B}(0,1)}^{} \psi(x) \dd x 
= c_n \int_0^1 \frac{1}{\rho^\alpha} \rho^{n -1} \dd \rho 
= c_n \int_0^1 \rho^{n-1-\alpha} \dd \rho =
\begin{cases}
	\log (\rho) \quad n = \alpha \\
	\dfrac{\rho^{n-\alpha}}{n - \alpha} \quad \text{altrimenti} 
\end{cases} 
$$
%
Concludendo,
%
$$
	\int_{\mc{B}(0,1)}^{} \frac{1}{\norm{ x }^\alpha} \dd x < + \infty \longiff n > \alpha.
$$
%
\textbf{Esercizio.}
Con passaggi analoghi al precedente otteniamo
%
$$
	\int_{\R^n \setminus \; \mc{B}(0,1)} \frac{1}{\norm{ x }^\alpha} \dd x < + \infty \longiff n < \alpha.
$$
%
\textbf{Esercizio.}
Vediamo per quali valori di $\beta$ l'integrale
%
$$
	\int_{\mc{B}(0,1)}^{} \frac{1}{\left( \norm{ x } + \norm{ x }^2 \right)^\beta} \dd x
$$
%
converge.

Vale la seguente catena di uguaglianze.
%
$$
\int_{\R^n}^{} \frac{1}{\left( \norm{ x } + \norm{ x }^2 \right)^\beta} \dd x 
= \int_{\mc{B}(0,1)}^{} \frac{1}{\left( \norm{ x } + \norm{ x }^2 \right)^\beta} \dd x 
+ \int_{\R^n \setminus \; \mc{B}(0,1)}^{} \frac{1}{\left( \norm{ x } + \norm{ x }^2 \right)^\beta} \dd x.
$$
%
Studiamo separatamente i due pezzi dell'integrale.
%
\begin{align*}
	\int_{\mc{B}(0,1)}^{} \frac{1}{\left( \norm{ x } + \norm{ x }^2 \right)^\beta} \dd x 
	& = c_n \int_{\mc{B}(0,1)}^{} \frac{1}{(\rho + \rho^2)^\beta} \rho^{n-1} \dd \rho
	= c_n \int_{0}^{1} \frac{1}{\rho^\beta} \cdot \frac{\rho^{n-1}}{(1 + \rho)^\beta} \dd \rho \\
	& \approx \int_{0}^{1} \rho^{n-1-\beta} \dd \rho < + \infty \longiff  \beta < n.
\end{align*}
%
Inoltre,
%
$$
	\int_{\R^n \setminus \mc{B}(0,1)}^{} \frac{1}{\left( \norm{ x } + \norm{ x }^2 \right)^\beta} \dd x 
	= \int_{\R^n \setminus \; \mc{B}(0,1)}^{} \frac{1}{\rho^{2\beta}} \cdot \frac{\rho^{n-1}}{\left( \frac{1}{\rho} + 1 \right)^\beta} \dd \rho 
	\approx \int_{1}^{+\infty} \frac{\rho^{n-1}}{\rho^{2\beta}} \dd \rho < + \infty \longiff 2\beta > n.  
$$
%
In conclusione, l'integrale è finito se $n > \beta > n / 2$.

% @aziis98: Molto probabilmente metterò un disegnino con assi $n$ e $\beta$ per far vedere "meglio" l'insieme dei valori buoni

\textbf{Esercizio.}
Studiare l'insieme di finitezza al variare del parametro $\alpha$ dell'integrale
$$
	\int_{[0,1]^n}^{} \frac{1}{\norm{ x }_1^\alpha} \dd x .
$$
Osserviamo che la norma 1 e 2 sono legate dalle seguenti disuguaglianze
$$
	\frac{\norm{ x }_1}{n} \leq \norm{ x }_2 \leq \norm{ x }_1.
$$
%
Studiamo una maggiorazione per l'integrale
%
$$
	\int_{[0,1]^n}^{} \, \frac{1}{\norm{ x }_1^\alpha} \dd x 
	\leq \;\int_{[0,1]^n}^{} \, \frac{1}{\norm{ x }^\alpha} \dd x 
	\leq \;\;\int_{B(0,\sqrt{n})}^{}\;\; \frac{1}{\norm{ x }^\alpha} \dd x < + \infty \longiff \alpha < n,
$$
%
dunque
%
$$
\int_{[0,1]^n}^{} \, \frac{1}{\norm{ x }_1^\alpha} \dd x < +\infty \quad \text{se} \; \alpha < n.
$$
%
Vediamo ora una minorazione.

$$
	\int_{[0,1]^n}^{} \frac{1}{\norm{ x }_1^\alpha} \dd x 
= \frac{1}{2^n}	\int_{[-1,1]^n}^{} \frac{1}{\norm{ x }_1^\alpha} \dd x 
\geq \frac{1}{2^n} \ \int_{\mc{B}(0,1)}^{} \frac{1}{\norm{ x }_1^\alpha} \dd x 
\approx \int_{\mc{B}(0,1)}^{} \frac{1}{\norm{ x }^\alpha} \dd x < + \infty 
\longiff \alpha < n.
$$

Dunque l'integrale $\ds \int_{[0,1]^n}^{} \frac{1}{\norm{x}_1^\alpha} \dd x$ converge se solo se $\alpha < n$.

\newpage

\textbf{Esercizi per casa.}
\begin{enumerate}[label=(\arabic*)]

\item Dimostrare che date $f,g$ misurabili ed $r,p_1,p_2 > 0$ tali che  $1 / r = 1 / p_1 + 1 / p_2$.
Allora vale
$$
	\norm{ f \cdot g }_r \leq \norm{ f }_{p_1} \cdot \norm{ g }_{p_2}.
$$
\textit{Suggerimento.} Usare Hölder osservando che $\ds 1 = \frac{r}{p_1} + \frac{r}{p_2} = \frac{1}{\left( p_1/r \right)} + \frac{1}{\left( p_2/r \right)}$.

\textbf{Dimostrazione.} Vale quanto segue.
\begin{align*}
	\norm{f \cdot g}_r^r 
	&= \int_X \left| f \cdot g \right|^r \dd \mu
	= \int_X |f| \cdot |g| \dd \mu
	\overset{\text{Holder}}{\leq} \norm{f^r}_{p_1 / r} \cdot \norm{f^r}_{p_2 / r} \\
	& = \left( \int_X |f|^{r \cdot p_1 / r} \right)^{r / p_1} \cdot \left( \int_X |g|^{r \cdot p_2 / r} \right)^{r / p_2}
	= \norm{f}_{p_1}^r \cdot \norm{g}_{p_2}^r
	= \left( \norm{f}_{p_1} \cdot \norm{g}_{p_2} \right)^r \\
	& \Longrightarrow \norm{f \cdot g}_r \leq \norm{f}_{p_1} \cdot \norm{g}_{p_2}.
\end{align*}
\qed

\item Dimostrare che date $f_1,\ldots, f_n$ misurabili e $p_i > 0$ tali che $1/p_1 + \ldots + 1/p_n = 1$ si ha
$$
	\norm{ f_1 \cdots f_n }_1 \leq \norm{ f_1 }_{p_1} \cdots \norm{ f_n }_{p_n}.
$$
\textit{Suggerimento.} Fare il primo passo dell'induzione e usare la formula precedente scegliendo $r$ in modo corretto.

\textbf{Dimostrazione.} Dimostriamo per induzione la seguente proprietà più generale.

Siano $f_1,\ldots, f_n$ misurabili e $r > 1$. Allora, per i $p_i > 0$ tali che $1/p_1 + \ldots + 1/p_n = r$ si ha
$$
	\norm{ f_1 \cdots f_n }_r \leq \norm{ f_1 }_{p_1} \cdots \norm{ f_n }_{p_n}.
$$

\textit{Passo base.} Vero per il punto (1).

\textit{Passo induttivo} $[n-1 \Rightarrow n]$. Supponiamo di aver dimostrato per ogni $r > 1$ la disuguaglianza sopra. Allora
%
$$
\norm{f_1 \cdots f_n}_r 
= \norm{(f_1 \cdots f_{n-1}) \cdot f_n}_r 
\overset{(1)}{\leq} \norm{f_1 \cdots f_{n-1}}_p \cdot \norm{f_n}_{p_n}, \qquad \text{dove} \; r = 1/p + 1/p_n.
$$
%
Notando che $1/p = 1/r - 1/p_{n-1} = 1/p_1 + \cdots + 1/p_{n-1}$ e usando l'ipotesi induttiva otteniamo la tesi.
\qed

\end{enumerate}


% Completezza spazi L^p
%
% Lezione del 6 Ottobre 2021
%

\section{Costruzione spazi $L^p$}

Fissiamo $(X, \mc A, \mu)$ come sempre.

\textbf{Definizione.}
Sia $\mathscr L^p$ l'insieme delle funzioni $f \colon X \to \R$ o $\R^d$ misurabili tali che $\norm{f}_p < +\infty$.

\textbf{Osservazioni.}
\begin{itemize}
	\item $\mathscr L^p$ è un sottospazio vettoriale dello spazio vettoriale dato da $\{ f \colon X \to \R \mid f \text{ misurabile} \}$ e $\norm{\curry}_p$ è una semi-norma.

		\textbf{Dimostrazione.}
		\begin{itemize}
			\item $\mathscr L^p$ è chiuso per somma e moltiplicazione per scalari.

			% \item $f_1, f_2 \in \mathscr L^p \implies f_1 + f_2 \in \mathscr L^p$

			\item Dalla definizione segue subito $\norm{\lambda f}_p = |\lambda| \cdot \norm{f}_p$ l'omogeneità della norma.

			\item Dalla disuguaglianza di Minkowski segue che $\norm{\curry}_p$ è una semi-norma.
		\end{itemize}

	\item In particolare non è una norma se $\{ 0 \} \subsetneq \{ f \mid \norm{f}_p = 0 \}$ ovvero se $\mc A$ contiene insiemi non vuoti di misura nulla.

	\item In generale dato $V$ spazio vettoriale e $\norm{\curry}$ semi-norma su $V$ possiamo introdurre $N \coloneqq \{ v \mid \norm{v} = 0 \}$. $N$ risulta essere un sottospazio di $V$ e la norma data da $\norm{[v]} \coloneqq \norm{v}$ per $[v] \in \sfrac{V}{N}$ è ben definita ed è proprio una norma su $\sfrac{V}{N}$.

	\item Nel caso della della norma $\norm{\curry}_p$ abbiamo che $[f_1] = [f_2] \iff [f_1 - f_2] = 0 \iff f_1 - f_2 = 0$ quasi ovunque. 
\end{itemize}

\textbf{Definizione.}
Poniamo $N \coloneqq \{ f \mid \norm{f}_p = 0 \}$ e definiamo gli \textbf{spazi $L^p$} come
$$
L^p := \sfrac{\mathscr L^p}{N} = \sfrac{\mathscr L^p}{\sim} 
\qquad
\norm{[f]}_p \coloneqq \norm{f}_p
$$

\textbf{Notazione.}
Ogni tanto serve precisare meglio l'insieme di partenza e di arrivo degli spazi $L^p$ ed in tal caso useremo le seguenti notazioni
$$
L^p = L^p(X) = L^p(X, \mu) = L^p(X, \mc A, \mu) = L^p(X, \mu; \R^d).
$$

\textbf{Nota.}
Nella pratica non si parla mai di ``classi di funzioni'' e si lavora direttamente parlando di ``funzioni in $L^p$''. Le ``operazioni'' comuni non creano problemi però in certi casi bisogna stare attenti di star lavorando con oggetti ben definiti. Ad esempio:
\begin{itemize}
	\item Preso $x_0 \in X$, consideriamo l'insieme $\{ f \in L^p \mid f(x_0) = 0 \}$. Notiamo che non è un sottoinsieme ben definito (a meno che $\mu(\{ x_0 \}) > 0$ ovvero che la misura sia atomica) di $L^p$, in quanto possiamo variare $f$ su un insieme di misura nulla.

	\item Invece ad esempio il seguente insieme è ben definito
		$$
		\left\{ f \in L^1 \;\middle|\; \int_X f \dd \mu = 0 \right\}
		$$
\end{itemize}

\subsection{Prodotto scalare su $L^2$}

Date $f_1,f_2 \in L^2(X)$ si pone
%
$$
\langle f_1, f_2 \rangle \coloneqq \int_{X}^{} f_1 \cdot f_2 \dd \mu.
$$
%
\textbf{Osservazioni.}

\begin{itemize}
\item La definizione di $\langle f_1, f_2 \rangle$ è ben posta, infatti basta far vedere che $\ds \int_X \left| f_1 f_2 \right| \dd \mu < +\infty$ ma per H\"older abbiamo
%
$$
\int_{X}^{} \left| f_1 f_2 \right| \dd \mu \leq \norm{f_1}_2 \norm{f_2}_2 < +\infty
$$
%

\item $\norm{f}_2^2 = \langle f,f \rangle$ per ogni $f \in L^2(X)$.

\item Inoltre, $\ds \left| \int_X f_1 f_2 \dd \mu \, \right| \leq \int_X \left| f_1 f_2 \right| \dd \mu$ quindi
%
$$
\left| \langle f_1, f_2 \rangle \right| \leq \norm{f_1}_2 \norm{f_2}_2 \quad \; \text{(\textit{Cauchy-Schwartz})}.
$$

\item L'operatore $\langle \curry, \curry \rangle$ è un prodotto scalare definito positivo.

\end{itemize}

\textbf{Osservazioni.}
\begin{itemize}
\item Dato $C$ spazio vettoriale reale con prodotto scalare $\langle \curry, \curry \rangle$, allora $\langle \curry, \curry \rangle$ si ricava dalla norma associata $\norm{\curry}$ tramite l'identità di polarizzazione:
%
$$
\langle v_1, v_2 \rangle = \frac{1}{4} \left( \norm{v_1 + v_2}^2 - \norm{v_1 - v_2}^2 \right).
$$
%

\item  Dato $V$ come sopra, vale l'identità del parallelogramma:
%
$$
\norm{v_1 + v_2}^2 + \norm{v_1 - v_2}^2 = 2 \norm{v_1}^2 + 2 \norm{v_2}^2 \quad \forall v_1,v_2 \in V.
$$
%
 Usando questa identità di dimostra che la norma di $L^p$ deriva da un prodotto scalare solo per $p=2$.

\end{itemize}

\textbf{Proprietà.}
Sia $V$ uno spazio vettoriale con norma $\norm{\curry}$. Allora vale l'identità del parallelogramma se solo se $\norm{\curry}$ deriva da un prodotto scalare.

\textbf{Esempio.}
La norma di $L^p \left( [-1,1] \right)$, deriva da un prodotto scalare solo per $p=2$.
Prendiamo $f_1 = \One_{[-1,0]}$ e $f_2 = \One_{[0,+1]}$.
Allora
\vspace{-5mm}
\begin{align*}
	\norm{f_1 + f_2}_p^p = \int_{-1}^{1} 1 \dd x = 2 \Rightarrow \norm{f_1 + f_2}_p = 2^{1/p} \\
	\norm{f_1 - f_2}_p = \norm{f_1 + f_2}_p = 2^{1/p}, \quad \norm{f_1}_p = \norm{f_2}_p = 1
\end{align*}
Se vale l'identità del parallelogramma allora
$$
\norm{f_1 + f_2}_p^2 + \norm{f_1 - f_2}_p^2 = 2 \norm{f_1}_p^2 + 2 \norm{f_2}_p^2
$$
cioè
$$
2^{2/p} + 2^{2/p} = 2 \cdot 1 + 2 \cdot 1 \iff p = 2.
$$

\textbf{Domanda.} Per quali $X,\mc{A},\mu$ vale la stessa conclusione?


\section{Completezza degli spazi $L^p$}

Vediamo ora la proprietà più importante degli spazi $L^p$.

\textbf{Teorema.}
Lo spezio $L^p$ è completo per ogni $p \in [1,+\infty]$.

% @aziis98: Per ora mi pare non si possano usare direttamente \label e \ref in questi punti, cioè linka alla sezione.
\hypertarget{prop:completeness_lemma_1}{}
\textbf{Lemma 1.} 
Dato $(Y, d)$ spazio metrico, allora
\begin{enumerate}
	\item
		Ogni successione $(y_n)$ tale che
		$$
		\sum^{\infty}_{n=1} d(y_n, y_{n+1}) < +\infty
		$$
		è di Cauchy.

	\item \label{item:def_completeness_1}
		Se ogni $(y_n)$ tale che $\ds \sum^{\infty}_{n=1} d(y_n, y_{n+1}) < +\infty$ converge allora $Y$ è completo.
\end{enumerate}

\textbf{Osservazione.} Non tutte le successioni di Cauchy $(y_n)$ soddisfano quella condizione. Ad esempio la successione $(-1)^n / n$ definita su $\R$ è di Cauchy però
$$
\sum_{n=1}^\infty \left| \frac{(-1)^{n+1}}{n+1} - \frac{(-1)^n}{n} \right| 
= \sum_{n=1}^\infty \frac{2n + 1}{n^2 + n}
\approx \sum_{n=1}^\infty \frac{1}{n} \to \infty.
$$

\textbf{Nota.} Per mostrare la completezza degli spazi $L^p$ è sufficiente verificare la convergenza per una sottoclasse propria delle successioni di Cauchy.

\textbf{Dimostrazione.}
\begin{enumerate}
	\item Vorremmo vedere che $\forall \epsilon \, \exists N$ tale che $\forall m, n > N$ si ha $d(y_m, y_n) \leq \epsilon$. 
		Presi $n > m$ abbiamo che 
		$$
		d(y_m, y_n) \leq \sum_{k=m}^{n-1} d(y_k, y_{k+1}) \leq \sum_{k=m}^\infty d(y_k, y_{k+1}) \to 0
		$$
		in quanto \textit{coda di una serie convergente}, quindi 
		$$
		\forall \epsilon > 0 \mquad \exists m_\epsilon \text{ tale che } \sum_{k = m_\epsilon}^\infty d(y_k, y_{k+1})< \epsilon \implies \forall n > m \geq m_\epsilon \; d(y_m, y_n) \leq \epsilon
		$$ 

	\item
		Vogliamo vedere che ogni successione di Cauchy converge, ma osserviamo che basta mostrare che data $(y_n)$ di Cauchy esiste una sottosuccessione $y_{n_k}$ tale che
		$$
		\sum_{k=1}^\infty d(y_{n_k}, y_{n_{k+1}}) < +\infty
		$$
		ma a quel punto per ipotesi questa sottosuccessione converge ad $y \in Y$ e dunque anche $y_n \to y$. 
		Questa sottosuccessione esiste in quanto $\forall k \; \exists n_k$ tale che $\forall n, m \geq n_k \; d(y_m, y_n) \leq 1 /2 ^k$ e dunque $d(y_{n_k}, y_{n_{k+1}}) \leq 1 / 2^k$.
		\qed

		% @aziis98: In realtà non so se era più chiaro come stava prima boh

		% Basta far vedere che data $(y_n)$ di Cauchy esiste una sottosuccessione $y_{n_k}$ tale che
		% $$
		% \sum_{k=1}^\infty d(y_{n_k}, y_{n_{k+1}}) < +\infty
		% $$
		% ma $\forall k \; \exists n_k$ tale che $\forall n, m \geq n_k \; d(y_m, y_n) \leq 1 /2 ^k$ dunque $d(y_{n_k}, y_{n_{k+1}}) \leq 1 / 2^k$.

		% Quindi per ipotesi $y_{n_k}$ converge a qualche $y \in Y$ ed anche $y_n \to y$.
		% \qed
\end{enumerate}


% @aziis98: Il latex mi vuole male e mi metteva solo Corollario e la prima frase alla fine della pagina e l'enumerate alla nuova, però se aggiungiamo cose prima poi lo togliamo.

\hypertarget{prop:completeness_lemma_2}{}
\textbf{Lemma 2.} 
Dato $Y$ spazio normato, i seguenti fatti sono equivalenti
\begin{enumerate}
	\item $Y$ è completo.

	% \item $\sum_{n=1}^\infty y_n$ converge in $Y$ per ogni $(y_n)$ tale che $\sum_{n=1}^\infty \norm{y_n} < +\infty$ ovvero 
	\item Per ogni successione $(y_n)$ tale che $\ds \sum_{n=1}^\infty \norm{y_n} < +\infty$, la serie $\ds \sum_{n=1}^\infty y_n$ converge\footnote{Nel senso che esiste $y$ tale che $\norm{y - \sum_{n=1}^N y_n} \to 0$.}.
\end{enumerate}

\textbf{Dimostrazione.} 
È un corollario del lemma precedente. 
\qed

\hypertarget{prop:completeness_lemma_3}{}
\textbf{Lemma 3} 
(Minkowski per somme infinite). 
Date delle funzioni $(g_n)$ funzioni positive su $X$ allora
$$
\norm{\sum_{n=1}^\infty g_n}_p \leq \sum_{n=1}^\infty \norm{g_n}_p
$$

% @aziis98: Si bisognerebbe scrivere qualche parola in più magari a questa dimostrazione.
\textbf{Dimostrazione.}
Per ogni $N$ abbiamo che
$$
\norm{\sum_{n=1}^N g_n}_p^p 
\leq \left( \,\sum_{n=1}^N \norm{g_n}_p \right)^p 
\leq \left( \,\sum_{n=1}^\infty \norm{g_n}_p \right)^p 
$$
e per convergenza monotona possiamo passare il termine di sinistra al limite
$$
\lim_N \norm{\sum_{n=1}^N g_n}_p^p 
= \lim_N \int_X \left( \sum_{n=1}^N g_n \right)^p \dd \mu
= \int_X \left( \lim_N \sum_{n=1}^N g_n \right)^p \dd \mu
= \norm{\sum_{n=1}^\infty g_n}_p^p
$$
\qed

\textbf{Dimostrazione} (Completezza spazi $L^p$).
\begin{itemize}
	\item 
		Se $p = +\infty$: si tratta di vedere che data $(f_n)$ di Cauchy in $L^\infty(X)$ esiste $E$ con $\mu(E) = 0$ tale che $(f_n)$ è di Cauchy rispetto allora norma del sup in $X \setminus E$. [TODO: Finire]

	\item 
		Se $p < +\infty$: per il \hyperlink{prop:completeness_lemma_2}{Lemma 2}, basta far vedere che data $(f_n) \subset L^p(X)$ tale che $\sum_{n=1}^\infty \norm{f_n}_p < +\infty$ allora $\sum_n f_n$ converge a qualche $f \in L^p(X)$.

		La dimostrazione è suddivisa in tre passi, prima costruiamo $f$, poi mostriamo che $f_n$ converge a $f$ ed infine mostriamo $f \in L^p(X)$.

		\begin{itemize}
			\item 
				\textit{Passo 1:} 
				Per ipotesi abbiamo
				$$
				\infty 
				> \sum_{n=1}^\infty \norm{f_n}_p 
				= \sum_{n=1}^\infty \norm{|f_n|}_p 
				\geq \norm{\sum_{n=1}^\infty |f_n|}_p 
				= \left( \int \left( \sum_{n=1}^\infty |f_n(x)| \right)^p \dd \mu(x) \right)^{1/p}
				$$
				quindi $\sum_{n=1}^\infty |f_n(x)| < +\infty$ per ogni $x \in X \setminus E$ con $\mu(E) = 0$. Quindi $\sum_{n=1}^\infty f_n(x)$ converge a qualche $f(x)$ per ogni $x \in X \setminus E$ ed a questo punto ci basta estendere $f$ a zero in $E$ \footnote{Una costruzione alternativa degli spazi $L^p$ potrebbe anche partire da \textit{funzioni definite quasi ovunque}, questo ovvierebbe al problema di estendere a $0$ la funzione $f$ appena costruita. Però diventa più complicato mostrare di essere in uno spazio vettoriale poiché per esempio serve ridefinire $+$ a funzioni definite quasi ovunque.}.
			
			\item 
				\textit{Passo 2:}
				Fissiamo $N$ ed osserviamo che $\forall x \in X \setminus E$ abbiamo
				$$
				\left| f(x) - \sum_{n=1}^N f_n(x) \right| 
				= \left| \sum_{n=N+1}^\infty f_n(x) \right| 
				\leq \sum_{n=N+1}^\infty |f_n(x)|
				$$
				da cui otteniamo
				$$
				\norm{f - \sum_{n=1}^N f_n}_p 
				\leq \norm{\sum_{n=N+1}^\infty |f_n|}_p
				\leq \sum_{n=N+1}^\infty \norm{f_n}_p
				$$
				dove l'ultimo termine è la coda di una serie convergente.
			
			\item 
				\textit{Passo 3:}
				In particolare rileggendo il passo precedente per $N = 0$ otteniamo
				$$
				\norm{f}_p \leq \sum_{n=1}^\infty \norm{f_n}_p < +\infty \implies f \in L^p
				$$
				\qed

		\end{itemize}
\end{itemize}

\textbf{Esercizio.}\footnote{In questo corso non è strettamente necessario ricordarsi come si facciano tutti questi esercizi tecnici di teoria della misura ma è bene saperli applicare in automatico quando serve.}
Sia $f \colon X \to [0, +\infty]$ allora $\ds \int_X f \dd \mu < +\infty \implies f(x) < +\infty$ per quasi ogni $x$.

\textbf{Dimostrazione.}
Sia $E := \{ x \mid f(x) = +\infty \}$, allora l'idea è che
$$
\infty > \int_X f \dd \mu \geq \int_E f \dd \mu = +\infty \cdot \mu(E).
$$
Oppure, osserviamo che $\forall m \in [0, +\infty)$ abbiamo $f \cdot \One_E \geq m \cdot \One_E$ per ogni $x \in E$ quindi integrando ricaviamo
$$
\underbrace{\int_E f \dd \mu}_{I} \geq m \cdot \mu(E) 
\implies \mu(E) \leq \frac{I}{m} \xrightarrow{m \to +\infty} 0
$$
\qed



% Nozione di convergenza per funzioni, approssimax di funzioni
%
% Lezione del 07 ottobre
%

\section{Nozioni di convergenza per successioni di funzioni}
Fissiamo $X,\mc{A},\mu$ e prendiamo $f, f_n \colon X \to \R$ (o $\R^k$) misurabili.

\textbf{Definizione.}
Riportiamo le definizioni di alcune nozioni di convergenza.

\begin{itemize}

\item \textbf{Uniforme}: $\forall \epsilon \; \exists n_{\epsilon}$ tale che $\norm{f(x) - f_n(x)} < \epsilon \mquad \forall n > n_\epsilon$.

\item \textbf{Puntuale}: $f_n(x) \to f(x) \mquad \forall x \in X$.

\item \textbf{Puntuale} $\mu$\textbf{-quasi ovunque}: $f_n(x) \to f(x)$ per $\mu$-q.o. $x \in X$.

\item \textbf{In} $L^p$: $\norm{f_n - f}_p \xrightarrow{n \to \infty} 0$.

\item \textbf{In misura}: $\ds \forall \epsilon > 0 \quad \mu \left( \left\{ x \mymid \left| f_n(x) - f(x) \right| \geq \epsilon \right\} \right) \xrightarrow{n \to +\infty} 0$.

\end{itemize}

\textbf{Osservazione.}
Abbiamo le seguenti implicazioni ovvie delle diverse nozioni di convergenza:

\begin{center}
	uniforme $\Rightarrow$ puntuale $\Rightarrow$ puntuale $\mu \almosteverywhere$
\end{center}


\textbf{Proposizione.}
Valgono le seguenti.
\begin{enumerate}

\item \label{item:convergenza_i} Data $f_n \to f$ q.o. e $\mu(X) < +\infty$, allora $f_n \to f$ in misura.

\item \label{item:convergenza_ii} (\textit{Severini-Egorov}): Data $f_n \to f$ q.o. e $\mu(X) < +\infty$, allora $\forall \delta > 0$ esiste  $E \in \mc{A}$ tale che $\mu(E) < \delta$ e $f_n \to f$ uniformemente su $X \setminus E$.

\item \label{item:convergenza_iii} $f_n \to f $ in $L^p$, $p < +\infty$, allora $f_n \to f$ in misura.

\item[iii')] \label{item:convergenza_iv} $f_n \to f$ in $L^\infty$, allora $\exists E $ tale che $\mu(E) = 0$ e $f_n \to f$ uniformemente su $X \setminus E$.

\item $f_n \to f$ in misura, allora $\exists n_k$ tale che $f_{n_k} \to f$ $\mu$-q.o.

\item $f_n \to f$ in $L^p$, allora $\exists n_k$ tale che $f_{n_k} \to f$ $\mu$-q.o.

\end{enumerate}

\textbf{Osservazione.}
In i) e ii) l'ipotesi $\mu(X) < +\infty$ è necessaria.
Infatti, preso $X = \R$ e $f_n = \One_{[n,+\infty)}$ si ha che $f_n \to 0$ ovunque ma $f_n$ non converge a $0$ in misura, e $f_n$ non converge a $0$ uniformemente in $\R \setminus E$ per ogni $E$ di misura finita.

\textbf{Lemma} (disuguaglianza di Markov).
Data $g \colon X \to [0,+\infty]$ misurabile e $m > 0$ si ha
%
$$
\mu \left( \left\{ x \in X \mymid g(x) \geq m \right\} \right) \leq \frac{1}{m} \int_X g \dd \mu
$$
%

\textbf{Dimostrazione.}
Poniamo $\ds E \coloneqq \left\{ x \in X \mymid g(x) \geq m \right\}$.
Osserviamo che $g \geq m \cdot \One_E$.
Dunque vale
%
$$
\int_X g \dd \mu \geq \int_X m \cdot \One_E \dd \mu = m \cdot \mu \left( \left\{ x \in X \mymid g(x) \geq m \right\} \right).
$$
%
\qed

\textbf{Lemma} (Borel-Cantelli).
Dati $(E_n) \subset \mc{A}$ tali che $\sum \mu(E_n) \leq +\infty$, l'insieme
%
$$
E \coloneqq \left\{ x \in X \mymid x \in E_n \; \text{frequentemente} \right\}
$$
%
ha misura nulla.
Cioè per $\mu$-q.o. $x$, $x \notin E_n$ definitivamente (in $n$.)

\textbf{Dimostrazione.}
Osserviamo che
%
$$
E = \bigcap_{m=1}^\infty \Big( \underbrace{\bigcup_{n=m}^\infty E_n}_{F_m} \Big).
$$
%
Allora
%
$$
\mu(E) \quad \underbrace{=}_{\mathclap{F_m \downarrow E \text{ e } \mu(F_1) < +\infty}} \quad  \lim_{m \to \infty} \mu(F_m) \leq \lim_{m \to \infty} \underbrace{\sum_{n=m}^{\infty} \mu(E_n)}_{\mathclap{\text{coda serie convergente}}} = 0.
$$
%

\textbf{Osservazione.}
L'ipotesi $\sum \mu(E_n) < +\infty$ non può essere sostituita con $\mu(E_n) \to 0$.

Ora dimostriamo la proposizione.

\textbf{Dimostrazione della Proposizione.} Definiamo gli insiemi
\begin{align*}
A_n^\epsilon & \coloneqq \left\{ x \mymid \left| f_n(x) - f(x) \right| \geq \epsilon \right\}, \\
B_m^\epsilon & \coloneqq \left\{ x \mymid \left| f_n(x) - f(x) \right| \geq \epsilon \; \text{per qualche} \; n \geq m\right\} = \bigcup_{n = m}^\infty A_n^\epsilon, \\
B^\epsilon & \coloneqq \left\{ x \mymid \left| f_n(x) - f(x) \right| \geq \epsilon \; \text{frequentemente}  \right\} = \left\{ x \in A_n^\epsilon \; \text{frequentemente}  \right\} = \bigcap_{m = 1}^\infty B_m^\epsilon.
\end{align*}

\begin{enumerate}
	\item Per ipotesi, $f_n \to f$ quasi ovunque, cioè $\mu(B^\epsilon) = 0$ per ogni $\epsilon > 0$, ma $B_m^\epsilon \downarrow B^\epsilon$ e $\mu(X) < +\infty$.
	Allora
	$$
	\lim_{m \to +\infty} \mu(B_m^\epsilon) = \mu(B^\epsilon) = 0 \Longrightarrow \lim_{m \to \infty} \mu(A_m^\epsilon) = 0.
	$$

	\item Dalla dimostrazione precedente, abbiamo $\ds \lim_{m \to \infty} \mu(B_m^\epsilon) = 0$. 
	Allora per ogni $k$ esiste un $m_k$ tale che $\mu \left( B_{m_k}^{1/k} \right) \leq \delta / 2^k$.
	Poniamo $\ds E \coloneqq  \bigcup_{k} B_{m_k}^{1/k}$ per ogni $k$; allora $\mu(E) \leq \delta$.
	Inoltre,
	\begin{align*}
	x \in X \setminus E & \Longrightarrow x \notin B_{m_k}^{1/k} \; \forall k \iff x \notin A_n^{1/k} \mquad \forall k,n \geq m_ k \\
	& \Longrightarrow \left| f(x) - f_n(x) \right| < \frac{1}{k} \mquad \forall k,n \geq m_k \\
	& \Longrightarrow \sup_{x \in X \setminus E} \left| f(x) - f_n(x) \right| \leq \frac{1}{k} \mquad \forall k,n \geq m_k \\
	& \Longrightarrow f - f_m \; \text{uniformemente su} \; X \setminus E.
	\end{align*}

	\item Dobbiamo mostrare che per ogni $\epsilon > 0$ $\mu(A_n^\epsilon) \xrightarrow{n} 0$.
	Usando la disuguaglianza di Markov otteniamo
		$$
		\mu(A_n^\epsilon) 
		= \mu \Big( \Big\{ x \Big| \overbrace{\left| f_n(x) - f(x) \right|^p}^{g} \geq \epsilon^p \Big\} \Big)
		\leq \frac{1}{m} \int_{X}^{} g \dd \mu = \frac{1}{\epsilon^p} \norm{f_n - f}_p^p \xrightarrow{n \to +\infty} 0.
		$$

	\item[iii')] Definiamo $\ds E_n \coloneqq  \left\{ x \mymid \left| f_n(x) - f(x) \right| > \norm{f_n - f}_\infty \right\}$ per ogni $n$, allora $\mu(E_n) = 0$.
	Poniamo $E = \bigcup_{n} E_n$ e $\mu(E) = 0$, dunque
		$$
		\sup_{x \in X\setminus E} \left| f_n(x) - f(x) \right| \leq \norm{f_n - f}_\infty \xrightarrow{n \to \infty} 0.
		$$

	\item Per ipotesi, $f_n \to f$ in misura, cioè
		\begin{align*}
		& \forall \epsilon > 0 \quad \mu \left( A_n^\epsilon \right) \xrightarrow{n \to +\infty} 0 \\
		& \Longrightarrow \forall k \; \exists n_k \colon \mu \left( A_{n_k}^{1/k} \right) \leq \frac{1}{2^k} \\
		& \Longrightarrow \sum_{k}^{} \mu \left( A_{n_k}^{1/k} \right) < +\infty. 
		\end{align*}
		Allora per Borel-Cantelli, si ha per $\mu$-quasi ogni $x$, $x \notin A_{n_k}^{1/k}$ definitivamente in $k$, cioè $\norm{f_{n_k}(x) - f(x)} < 1/k$ definitivamente in $k$, cioè $\ds f_{n_k}(x) \xrightarrow{k} f(x)$.

	\item[v)] Vogliamo mostrare che $f_n \to f$ in $L^p \Rightarrow \exists n_k$ tale che $f_{n_k} \to f$ quasi ovunque. Consideriamo due casi
		\begin{itemize}
			\item Se $p < +\infty$, allora $f_n \to f$ in $L^p \implies f_n \to f$ in misura da cui $\exists n_k$ tale che $f_{n_k} \to f$ quasi ovunque.

			\item Se $p = +\infty$, allora $f_n \to f$ uniformemente su $X \setminus E$ con $\mu(E) = 0 \implies f_n \to f$ puntualmente su $X \setminus E \implies f_n \to f$ quasi ovunque (quindi semplicemente con $n_k = k$). 
		\end{itemize}

\end{enumerate}



% Controesempi convergenza di funzioni
%
% Lezione dell'11 Ottobre 2021
%

\section{Controesempi sulle convergenze}

Vediamo un controesempio che mostra che tutte le implicazioni sui vari tipi di convergenza sono ottimali ovvero
\begin{enumerate}
	\item \label{item:ce_1}
		$f_n \to f$ in misura $\centernot\implies f_n \to f$ q.o.
	\item \label{item:ce_2}
		$f_n \to f$ in $L^p$ con $p < +\infty \centernot\implies f_n \to f$ q.o.
	\item \label{item:ce_3}
		$\mu(E_n) \to 0 \centernot\implies$ per q.o $x$ si ha $x \notin E_n$ definitivamente. 
\end{enumerate}

\textbf{Dimostrazione.}
Consideriamo gli insiemi $I_1 = \left[ 1, 1 + \frac{1}{2} \right], I_2 = \left[1 + \frac{1}{2}, 1 + \frac{1}{2} + \frac{1}{3} \right], \dots$
$$
I_n \coloneqq \left[ \; \sum_{k=1}^n \frac{1}{k}, \; \sum_{k=1}^{n+1} \frac{1}{k} \; \right]
$$
e consideriamo la loro proiezione ``modulo'' $[0, 1]$ usando la funzione $p \colon \R \to [0, 1)$ \textit{parte frazionaria} data da
$$
p(x) \coloneqq x - \lfloor x \rfloor
$$
e chiamiamo $E_n \coloneqq p(I_n)$. Per ogni $n$ abbiamo che $|I_n| = |E_n| = 1 / n$ e $\bigcup_n I_n = [1, +\infty)$ (in quanto $\sum_{k=1}^\infty \frac{1}{k} = +\infty$) e quindi ogni $x \in [0, 1)$ appartiene ad $E_n$ per infiniti $n$ ed in particolare questo mostra la \ref{item:ce_3}. 

[TODO: Disegnino]

Per la \ref{item:ce_1} basta notare che $\One_{E_n} \to 0$ in misura (in quanto $|E_n| \to 0$) ma $\One_{E_n} \centernot\to 0$ q.o., anzi $\forall x \in [0, 1) \; \One_{E_n}(x) \centernot\to 0$ e la \ref{item:ce_2} segue analogamente.
\qed

\section{Approssimazioni di funzioni in $L^p$}

Vediamo ora alcune classi di funzioni dense in $L^p$ che risulteranno essere un utile strumento da usare nelle dimostrazioni.

\textbf{Nota.} Ricordiamo la nozione di insieme denso in uno spazio metrico.
Sia $(X,d)$ uno spazio metrico e $Y \subset X$. Allora Y è denso in X se solo se
per ogni $x \in X$, esiste una successione $(y_n)_{n \in \N}$ in $Y$ che tale che $x = \lim_n y_n$.

Per ora sia $(X, \mc A, \mu)$ in generale.

\textbf{Proposizione 1.} 
Le funzioni limitate in $L^p$ sono dense in $L^p$.

\textbf{Dimostrazione.}
Data $f \in L^p(X)$ cerchiamo una successione di funzioni $f_n \in L^p(X)$ limitate tali che $f_n \to f$ in $L^p$, consideriamo
$$
f(x) \coloneqq (f(x) \land n) \lor (-n)
$$
vorremmo mostrare che $f_n \to f$ in $L^p$ ovvero
$$
\norm{f_n - f}_p^p = \int_X |f_n - f|^p \dd \mu \to 0
$$
intanto osserviamo che, per la \textit{convergenza puntuale}, basta osservare che se $n \geq |f(x)|$ abbiamo che $\forall x \; f_n(x) = f(x) \implies f_n(x) \xrightarrow{n} f(x) \implies |f_n(x) - f(x)|^p \to 0$.

Per concludere basta applicare \textit{convergenza dominata} usando come dominazione direttamente $|f(x) - f_n(x)| \leq |f(x)| \implies |f(x) - f_n(x)|^p \leq |f(x)|^p$ e notiamo che $|f|^p \in L^1(X)$.
\qed

\textbf{Proposizione 2.}
Sia\footnote{Lo span è inteso come combinazioni lineari} $\tilde{\mathscr S} \coloneqq \spn(\{ \One_E \mid E \in \mc A, \mu(E) < +\infty \})$, allora $\tilde{\mathscr S}$ è denso in $L^p(X)$.

\textbf{Dimostrazione.}
Data $f \in L^p(X)$ cerchiamo una successione che approssima $f$ in $\tilde{\mathscr S}$.
\begin{itemize}
	\item 
		\textit{Caso 1}: Se $f \geq 0$ allora fissiamo $\epsilon > 0$ e per ogni $k = 1, 2, \dots$ e poniamo
		$$
		A_\epsilon^k := \{ x \mid k \epsilon \leq f(x) \leq (k+1) \epsilon \}
		$$
		risulta che $A_k^\epsilon$ è misurabile ed ha misura finita\footnote{È misurabile in quanto preimmagine di un misurabile ed ha misura finita in quanto se così non fosse esisterebbero $\epsilon > 0, k \geq 1$ tali che $\mu(A_\epsilon^k) = +\infty$ da cui $\norm{f}_p^p \geq \norm{k \epsilon \One_{A_\epsilon^k}}_p^p \geq +\infty \implies f \notin L^p(X)$ \absurd.}. Ora consideriamo la successione di funzioni parametrizzata da $\epsilon$ data da
		$$
		f_\epsilon(x) := \sum_{1 \leq k \leq 1 / \epsilon^2} k \epsilon \cdot \One_{A_\epsilon^k}(x) \in \tilde{\mathscr S}
		$$
		[TODO: Disegnino]

		Osserviamo che vale anche $\max f_\epsilon(x) = \max\{ k \epsilon \mid k \epsilon \leq f(x) \text{ e } k \leq 1 / \epsilon^2 \}$ e mostriamo la seguente\footnote{Notiamo che qui stiamo applicando il teorema di \textit{convergenza dominata} su una famiglia parametrizzata da $\epsilon$ e non su una successione ma si può verificare facilmente che il teorema (ed anche gli altri risultati di convergenza di successioni di funzioni) si può estendere semplicemente prendendo $\epsilon = 1 / n$ per $n \to \infty$.}
		$$
		\int_X |f(x) - f_\epsilon(x)|^p \dd \mu(x) \xrightarrow{\epsilon \to 0} 0
		$$
		\begin{itemize}
			\item \textit{Convergenza puntuale}: Per l'identità precedente abbiamo che $0 \leq f(x) - f_\epsilon(x) \leq \epsilon$ se $f(x) \leq 1 / \epsilon$.
			% @aziis98: Cioè per me questo passaggio è ancora un po' mistico quindi poi voglio provare a spiegarlo meglio
			% ($f(x) \leq 1 / \epsilon \implies k \epsilon \leq 1 / \epsilon \implies k \leq 1 / \epsilon^2$ [TODO])
			\item \textit{Dominazione}: Possiamo usare nuovamente $|f(x) - f_\epsilon(x)|^p \leq |f(x)|^p < +\infty$ in quanto $f \in L^p(X)$.
		\end{itemize}

		[TODO: Disegnino]

	\item 
		\textit{Caso 2}:
		Sia $f \colon X \to \R$ allora si può rifare la dimostrazione precedente oppure si può semplicemente considerare $f_\epsilon \coloneqq (f^+)_\epsilon - (f^-)_\epsilon$.

	\item 
		\textit{Caso 3}:
		Generalizziamo la proposizione al caso di $f \colon X \to \R^d$ come segue

		\textbf{Proposizione 2'} (Generalizzata). Sia $\ds \tilde{\mathscr S} \coloneqq \left\{ \sum_i \alpha_i \One_{E_i} \mid \alpha_i \in \R^d, E_i \in \mc A, \mu(E_i) < +\infty \right\}$. Allora $\tilde{\mathscr S}$ è denso in $L^p(X; \R^d)$.

		\textbf{Dimostrazione.} (Idea)
		Basta approssimare componente per componente.
\end{itemize}
\qed

Sia ora $X$ uno \textit{spazio metrico} e $\{ \text{aperti} \} \subset \mc A$.


\hypertarget{prop:apprfunz_prop3}{}
\textbf{Proposizione 3.}
Sia $\tilde{\mathscr S}_\ell \coloneqq \{ \sum_i \alpha_i \One_{E_i} \mid \alpha_i \in \R^d, E_i \in \mc A, \mu(E_i) < +\infty, E_i \text{ limitati} \}$ allora $\tilde{\mathscr S}_\ell$ è denso in $L^p(X; \R^d)$ per $p < +\infty$.

\textbf{Osservazione.} 
In generale l'enunciato non vale per $p = +\infty$. Ad esempio preso $L^\infty(\R)$ e $f = 1$ non si può approssimare con funzioni a supporto limitato (come quelle in $\tilde{\mathscr S}_\ell$. In particolare data $g$ con supporto $A$ limitato $|f - g| = 1$ su $\R \setminus A$ e siccome $|\R \setminus A| > 0$ abbiamo $\norm{f - g}_\infty \geq 1$).

\textbf{Dimostrazione.} ($\tilde{\mathscr S}_\ell$ è denso in $L^p$)
Per prima cosa vediamo un lemma che useremo assieme alla proposizione precedente.

% @aziis98: Boh questo lemma non so se metterlo prima o se lasciarlo qua

\textbf{Lemma 1.}
Dato $E \in \mc A, \mu(E) < +\infty$ esiste $E_n \in \mc A$ con $E_n$ limitati tali che $E_n \subset E$ e $\mu(E \setminus E_n) \to 0$ e quindi $\norm{\One_{E} - \One_{E_n}}_p = \mu(E \setminus E_n)^{1/p} \xrightarrow{n} 0$ (e $\One_{E_n} \in \tilde{\mathscr S}_\ell$).

\textbf{Dimostrazione.}
Dato $E$ con $\mu(E) < +\infty$ prendiamo $x_0 \in X$ e poniamo $E_n \coloneqq E \cap \mc B(x_0, n)$; $E_n \subset E$ e $E \setminus E_n \downarrow \varnothing \implies \mu(E \setminus E_n) \xrightarrow{n} 0$.

Intuitivamente $\tilde{\mathscr S}_\ell$ è denso in $\tilde{\mathscr S}$ che a sua volta è denso in $L^p$ (usando la definizione di densità topologica la tesi è quasi ovvia mentre usando la definizione per successioni bisogna passare per un procedimento diagonale).
\qed

Ora siano $X \subset \R^n, \mu = \mathscr L^n$ e $C_C(\R^n) \coloneqq \{ \text{funzioni a supporto compatto} \}$, precisiamo che il \textbf{supporto} di una funzione è definito come la chiusura dell'insieme dei punti in cui è non zero 
$$
\operatorname{supp}(f) := \overline{\{ x \mid f(x) \neq 0 \}}
$$
in quanto per le funzioni continue l'insieme $\{ x \mid f(x) \neq 0 \}$ è sempre aperto e dunque mai veramente compatto, a parte quando è vuoto.

% notiamo che $C_C(\R^n) \subset L^p$ per ogni $p$.

\textbf{Proposizione 4.}
Le funzioni in $C_C(\R^n)$ \textit{ristrette a $X$} sono dense\footnote{È denso anche l'insieme delle funzioni \textit{continue} a supporto compatto ristretto a $X$ e si indica con $\mc{C}_C^0(\R^n)$.} in $L^p(X)$ per $p < +\infty$.

Vediamo prima alcuni lemmi.

\hypertarget{prop:apprfunz_lemma2}{}
\textbf{Lemma 2.} (di Urysohn)
Dati $C_0, C_1$ chiusi disgiunti in $X$ spazio metrico esiste una funzione $f \colon X \to [0, 1]$ continua tale che $f = 0$ su $C_0$ e $f = 1$ su $C_1$.

\textbf{Dimostrazione.}
Posta $d(x, C) = \inf \{ d(x, y) \mid y \in C \}$ basta considerare
$$
f(x) =
\frac{d(x, C_0)}{d(x, C_0) + d(x, C_1)}.
$$

\hypertarget{prop:apprfunz_lemma3}{}
\textbf{Lemma 3.}
Dato $E \subset \R^n$ limitato (e quindi di misura finita) esiste $f_\epsilon \in C_C(\R^n)$ tale che $f_\epsilon \xrightarrow{\epsilon \to 0} \One_E$ in $L^p(\R^n)$ e quindi in $L^p(X)$.

\textbf{Dimostrazione.}
Per regolarità della misura di Lebesgue abbiamo che per ogni $\epsilon$ esistono $C_\epsilon \subset E \subset A_\epsilon$ tali che $|A_\epsilon \setminus C_\epsilon| \leq \epsilon$. Per il \hyperlink{prop:apprfunz_lemma2}{Lemma 2} possiamo definire la classe di funzioni $f_\epsilon \colon \R^n \to [0, 1]$ continue tali che
$$
f_\epsilon = 1 \text{ su $C_\epsilon$}
\qquad
f_\epsilon = 0 \text{ su $\R^n \setminus A_\epsilon$}.
$$
In particolare, sappiamo che su $A_\epsilon \setminus C_\epsilon$ vale $|f_\epsilon - \One_E| \leq 1$. Allora,
%
\begin{align*}
\norm{f_\epsilon - \One_E}_p^p & = \int_{A_\epsilon \setminus C_\epsilon} |f_\epsilon - \One_E|^p \dd x \leq \int_{A_\epsilon \setminus C_\epsilon } \One_{A_\epsilon \setminus C_\epsilon } \dd x = \left| A_\epsilon \setminus C_\epsilon  \right|^{1/p} \leq \epsilon^{1/p} \\
& \implies \norm{f_\epsilon - \One_E}_p^p \xrightarrow{\epsilon \to 0} 0.
\end{align*}
\qed


\textbf{Dimostrazione Proposizione 4.} 
Per la \hyperlink{prop:apprfunz_prop3}{Proposizione 3} basta approssimare la classe delle funzioni a supporto limitato (e finito). Dunque, per il \hyperlink{prop:apprfunz_lemma3}{Lemma 3} si ha la tesi.



% Esercitazione
% 
% Lezione del 13 Ottobre 2021
% 

\section{Esercitazione del 13 ottobre}

\subsection{Esercizi su spazi $L^p(X)$ al variare di $p$ e dello spazio $X$}

Sia $X \subset \R^n$, $\mu$ la misura di Lebesgue e $ 1 \leq p_1 \leq p_2$.

\textbf{Domanda.} Possiamo confrontare gli spazi $L^{p_1}(X)$ e $L^{p_2}(X)$?
In generale no. 

Vediamo informalmente perché.
Posto $X = (0,+\infty)$, gli integrali 
$$
\int_{0}^{+\infty} \frac{1}{(1+x)^{\beta p}} \dd x
\qquad 
\int_{0}^{+\infty} \frac{1}{x^{\beta p}} \cdot \One_{[0,1]} (x) \dd x 
= \int_{0}^{1} \frac{1}{x^{\beta p}} \dd x
$$
%
sono maggiorati dall'integrale di $1 / x^{\alpha}$ dove l'esponente  $\alpha$ è rispettivamente più piccolo e più grande di $\beta \cdot p$.

Utilizziamo questa intuizione per vedere formalmente che gli spazi $L^p(0,+\infty)$ non sono confrontabili.

Cerchiamo una funzione $f \in L^{p_1}(0,+\infty) \setminus \, L^{p_2}(0,+\infty)$ e una funzione $g \in L^{p_2}(0,+\infty) \setminus \, L^{p_1}(0,+\infty)$.
Sia $\beta$ un parametro e sia funzione $f$ definita come segue
$$
f(x) \coloneqq 
\begin{cases}
1 / x^\beta \mquad x \in (0,1) \\
0 \mquad x \geq 1
\end{cases} 
$$
ed ora imponiamo i vincoli sulle due norme 
$$
\begin{aligned}
	\norm{f}_{p_1}^{p_1} &= \int_{0}^{+\infty} f(x)^{p_1} \dd x
	= \int_{0}^{1} \frac{1}{x^{\beta p_1}} \dd x < +\infty \iff \beta \cdot p_1 < 1 \\
	\norm{f}_{p_2}^{p_2} &= \int_{0}^{+\infty} f(x)^{p_2} \dd x
	= \int_{0}^{1} \frac{1}{x^{\beta p_2}} \dd x = +\infty \iff \beta \cdot p_2 \geq 1
\end{aligned}
$$
dunque poiché $p_1 < p_2$ basta prendere $\beta \in [1/p_2, 1/p_1)$.

Ora cerchiamo $g \in L^{p_2} (0,+\infty) \setminus \, L^{p_1}(0,+\infty)$.
Definiamo $g(x)$ come segue
$$
g(x) \coloneqq \frac{1}{(1+x)^\alpha}
$$
da cui
$$
\int_{0}^{+\infty} g(x)^{p_2} \dd x < +\infty \longiff \alpha \cdot p_2 > 1
\quad
\text{e}
\quad
\int_{0}^{+\infty} g(x)^{p_1} \dd x = +\infty \longiff \alpha \cdot p_1 \leq 1
$$

\textit{Conclusione.} In generale non c'è confrontabilità fra gli spazi $L^p$. La confrontabilità, dipende infatti dall'insieme $X$ su cui sono definiti.

\textbf{Nota.} Un caso particolare è dato ponendo $p_1 < p_2$ e $\mu(X) < +\infty$. In tal caso $L^{p_2}(X) \subset L^{p_1}(X) $.

Data $f \in L^{p_2}(X)$, cioè con $\int_X \left| f \right|^{p_2} \dd \mu < +\infty$ vediamo che  $\int_X \left| f \right|^{p_1} \dd \mu < +\infty$.

% TODO(aziis98): Poi riaggiusto meglio questo conto

Usiamo H\"older:
\begin{align*}
\int_X \left| f \right|^{p_1} \dd \mu & \leq \Bigg( \int_X \overbrace{\left| h(x) \right|^p}^{\left| f(x) \right|^{p_1 p}}  \dd \mu  \Bigg)^{1/p} \cdot \left( \int_X 1^q \dd \mu  \right)^{1/q}
\underbrace{\leq}_{p = p_1 / p_2} \left( \int_X \left| f \right|^{p_2} \dd \mu  \right)^{p_2 /p_1} \left( \int_X 1^q \dd \mu  \right)^{1/q} \\
& \underbrace{=}_{\mathclap{q = \left( 1 - \frac{1}{p} \right)^{-1} = \frac{p}{p-1} = \frac{p_2 / p_1}{p_2 - p_1}}} \norm{f}_{L^{p_2}(X)}^{p_1} \cdot \mu(X)^{\frac{p_2 - p_1}{p_2}}.
\end{align*}

% [...]

Dunque
$$
\norm{f}_{p_1} \leq \norm{f}_{p_2} \cdot \mu(X)^{\frac{p_2 - p_1}{p_1 p_2}}
$$
L'inclusione
$$
\begin{array}{rccc}
	i \colon & L^{p_2}(X) & \to & L^{p_1}(X) \\
	& f & \mapsto & f
\end{array}
$$


è ben definita per quanto fatto sopra.

\textbf{Esercizio.} [TO DO] Vedere con quale topologia l'inclusione risulta continua.

\textbf{Esercizio.} [TO DO] Dato $p \geq 1$, stabilire se esistono $X, \mu$ tali che $f \in L^p(X)$ e $\forall q \geq 1$ con $q \neq p$ valga $f \notin L^q(X)$ (ovvero se esiste una funzione che appartiene ad un solo $L^p$). \\
\textit{Suggerimento.} Pensare a $X = (0,+\infty)$, $\mu$ misura di Lebesgue.

\textbf{Osservazione.} $L^p(X)$ è uno spazio vettoriale di dimensione infinita, ossia ogni base algebrica ha cardinalità infinita. 
Vediamo il caso $X = (0,1)$.
Per trovare una base infinita, cerchiamo per ogni $N \in \N$, un insieme di funzioni $f_1,\ldots , f_N \in L^p(0,1)$ tali che siano linearmente indipendenti.
Vale a dire, presi $\lambda_1,\ldots ,\lambda_N \in \R$ vale $\lambda_1 f_1 + \ldots +f_N = 0$ se solo se $\lambda_1 = \ldots = \lambda_N = 0$.

Ad esempio, definiamo $f_i \coloneqq \One_{i/N, (i+1)/N}$ (questa costruzione si può riprodurre per ogni $N \in \N$).

Ricordiamo che, essendo $L^p(X)$ uno spazio metrico, dato $Y \subset L^p$ vale la seguente caratterizzazione:
\begin{center}
$Y$ è compatto $\longiff$ $Y$ è compatto per successioni $\longiff $ $Y$ chiuso e totalmente limitato.
\end{center}

\textbf{Osservazione.} $Y \subset L^p(X)$ è un sottoinsieme che eredita la norma $\norm{\curry}_{L^p}$:
\begin{center}
$Y$ è completo $\longiff$ $Y$ è chiuso.
\end{center}

\textbf{Osservazione.} In $L^p$ i sottoinsiemi chiusi e limitati non sono compatti\footnote{Uno spazio metrico è compatto se solo se è completo e totalmente limitato. Ricordiamo che uno spazio metrico $X$ si dice \textbf{totalmente limitato} se $\forall \epsilon > 0$ esiste $B_\epsilon^1,\ldots,B_\epsilon^n$ tale che $X \subset \bigcup_{i=1}^n B_\epsilon^i$.}!
In particolare le palle
%
$$
Y = \left\{ f \in L^p \mymid \norm{f}_{L^p} \leq 1 \right\}
$$
%
non sono compatte.

Ad esempio, mostriamo che in $L^p(0,1)$ le palle 
%
$$
B = \left\{ f \in L^p \mymid \norm{f}_{L^p} \leq 1 \right\}
$$
%
non sono compatte.
Per farlo, esibiamo una successione $\left\{ f_n \right\}_{n \in \N} \subset B$ che non ammetta sottosuccessioni convergenti.
La costruiamo in modo che non abbia sottosuccessioni di Cauchy
$$
f_n \colon (0,1) \to \R, \quad \norm{f_n - f_m}_{L^p} \geq c_0 > 0 \quad  \forall n \neq m.
$$
Cerchiamo $A_n \subset (0,1)$ tale che $\left| A_n \cap A_m \right| = 0$ per ogni $n \neq m$.
Definiamo $f_n$ come segue
$$
f_n(x) \coloneqq 
\begin{cases}
0 & \text{se } x \in (0,1) \setminus (1 / (n+1), 1/ n) \\
c_n > 0 & \text{altrimenti}
\end{cases} 
$$
%
dove $c_n$ è tale che 
%
$$
\left( \int_{1 / n+1}^{1 / n} c_n^{\,p}  \right)^{1/p} = 1 \longiff 
c_n^{\,p} \cdot (1/n - 1/(n+1)) = 1 \longiff c_n^{\,p} = n \cdot (n+1).
$$
%
Calcoliamo ora $\norm{f_n - f_m}^p_{L^p}$ con $n \neq m$ :
%
$$
\int_0^1 \left| f_n(x) - f_m(x) \right|^p \dd x = \int\limits_{\mathclap{(1/n,1/n+1) \cup (1/ m+1,1/m)}} \left| f_n - f_m \right|^p \dd x = \int_{1 / n+1}^{1 / n} \left| f_n \right|^p \dd x + \int_{0}^{1} \left| f_m \right|^p \dd x = 1 + 1 = 2.
$$
%

Si osserva che quanto detto sopra vale anche per $p = + \infty$.

\textbf{Esercizio.} [TO DO] Sia $E = \left\{ f \in L^1(1,+\infty) \mymid \left| f(x) \right| \leq 1 / x^2 \mquad \text{e} \;x \in [1,+\infty) \right\}$.

\begin{itemize}

\item $E$ è limitato in $L^1$?

\item $E$ è chiuso in $L^1$?

\item $E$ è compatto in $L^1$?

\end{itemize}

\textbf{Soluzione.}
\begin{enumerate}

	\item Dimostriamo che $\norm{f}_{L^1} < C$ per ogni $f \in E$.
	%
	$$
	\norm{f}_{L^1} = \int_1^\infty |f(x)| \dd x \leq \int_1^\infty 1/x^2 \dd x < C.
	$$
	%

	\item $E$ è chiuso. Ci basta dimostrare che se $\left\{ f_n \right\} \in E$ è convergente a $f$ allora $f \in E$. Questo equivale a dimostrare che $|f(x)| < 1/x^2$.
	Dal fatto che $\{f_n\} \in E$ è convergente in $L^1$, abbiamo che esiste una sottosuccessione $\{f_{n_k}\}$ che converge puntualmente a $f$. Essendo che $|f_{n_k}| < 1/x^2$ per ogni $x \in [1,+\infty)$, per la continuità del modulo segue la tesi.

	\item Da fare [TO DO]

\end{enumerate}

\textbf{Esercizio.} [TO DO] 
\begin{itemize}
\item Dire se $f_n(x) = x^n$, $n = 0,\ldots , N$ è un insieme di funzioni linearmente indipendenti in $L^p([0,1])$.

\item Dire se $\left\{ f_n \right\} \subset L^p(0,1)$ è compatta in $L^p(0,1)$.
\end{itemize}
\textit{Suggerimento.} Studiare il limite puntuale.

\textbf{Soluzione.}
\begin{enumerate}

	\item Dimostriamolo per induzione.
	
	\textit{Passo base.} [TO DO]

	\textit{Passo induttivo.} ($n-1 \Rightarrow n$)
	Vediamo che se $a_1 \cdot 1 + a_2 \cdot x + \cdots + a_{n-1} \cdot x^{n-1} + a_n \cdot x^n = 0 \Longrightarrow a_1 = \ldots = a_n = 0$.
	\begin{align*}
		& a_1 \cdot 1 + a_2 \cdot x + \cdots + a_{n-1} \cdot x^{n-1} = - a_n \cdot x^n \\
		& \qquad \qquad \qquad \downarrow + a_n \cdot x^n \\
		& (a_1 + a_n) \cdot 1 + (a_2 + a_n) \cdot x + \cdots + (a_{n-1} + a_n) \cdot x^{n-1} = 0
	\end{align*}
	essendo che $1,x^1,\ldots,x^{n-1}$ sono linearmente indipendenti per ipotesi induttiva, vale $(a_i + a_n) = 0$ per ogni $i = 1,\ldots,n-1$, da cui $a_i = 0$ per ogni $i = 1,\ldots,n$.

	\item Dimostriamo che non è compatto. Se per assurdo lo fosse, dalla successione $(f_n)$ potremmo estrarre una sottosuccessione convergente $(f_{n_k})$ in $L^p([0,1])$; denotiamo il limite con $f$. Per i risultati visti sulla convergenza, da $(f_{n_k})$ potremmo estrarre una sottosuccessione convergente quasi ovunque a $f$. Ma questo è assurdo perchè $\lim_n f_n = +\infty$.
\end{enumerate}

\subsection{Spazi $\ell^p$}

Prendiamo $X = \N$ e $\mu = \#$ la misura che conta i punti.

\textbf{Osservazione.} Definiamo
%
$$
\ell^p = L^p(\N, \#) = \left\{ \left( x_n \right)_{n \in \N} \mymid \sum_{n=0}^{+\infty} \left| x_n \right|^p < +\infty  \right\}
$$
con $p \geq 1$ e $p \neq +\infty$, e
%
$$
l^{\infty} = \left\{ \text{successioni limitate} \right\} = \left\{ \left( x_n \right) \mymid \sup_{n \in \N}\left| x_n \right| < +\infty \right\}.
$$
%

\textbf{Esempio} (di insieme non compatto in $\ell^1$). Consideriamo la successione $\left( e_i \right)$ definita come
%
$$
(e_i)_n \coloneqq 
\begin{cases}
0 \quad \text{se} \; n \neq i \\
1 \quad \text{se} \; n = i
\end{cases} 
$$
%
si osserva inoltre che le successioni così definite sono linearmente indipendenti e generano se sono infiniti.

\textbf{Esempio} (di insieme compatto in $\ell^1$). Sia $F = \left\{ (x_n)_n \in \ell^1 \mymid \left| x_n \right| \leq 1 / n^2 \quad \forall n \in \N \right\}$.
Noto subito che $F$ è limitato, infatti, presa
%
$$
\underline{x} = (x_n) \in F, \quad  \norm{\underline{x}}_{\ell^1} = \sum_{n=0}^{+\infty} \left| x_n \right| \leq
\sum_{n=0}^{+\infty} 1 / n^2 < +\infty.  
$$
%
$F$ è anche chiuso.

\textbf{Osservazione.} Data una successione $(\underline{x}^k) \subset \ell^1$, se $\underline{x}^k \xrightarrow{\ell^1} \underline{x}^{\infty}$, vuol dire che
%
$$
\norm{\underline{x}^k - \underline{x}^\infty}_{\ell^1} = \sum_{n = 0}^{+\infty}  \left| x_n^k - x_n^\infty \right| \xrightarrow{k} 0. 
$$
%
In particolare, per ogni $n \in \N$ fissato, $\lim_k (x_n^k - x_n^\infty) = 0$.

$F$ è chiuso perché se $(\underline{x}^k) \subset F$ e $\underline{x}^k \xrightarrow{\ell^1} \underline{x}^\infty$, allora per ogni $n \in \N$ vale 
%
$$
\left| x_n^k \right| \leq 1 / n^2 \quad \text{e} \quad \underbrace{\lim_{n \to +\infty} \left| x_n^k  \right|}_{x_n^\infty} \leq 1/n^2.
$$
%

Dimostriamo che è compatto per successioni.
Prendiamo $( \underline{x}^k ) \subset F$, ogni componente $x_n$ è equilimitata, quindi a meno di sottosuccessioni $x_n^{k_j}$ converge a $x_n^\infty$.
A meno di diagonalizzare, possiamo supporre che le successione $k_j$ non dipenda da $n$.
Inoltre gli elementi $x_n^{k_j}$ sono dominati da $y = (1 / n^2)$. Concludiamo usando il teorema di convergenza dominata di Lebesgue.



% Complementi: teorema di Lusin, ecc..
% 
% Lezione del 14 Ottobre 2021
% 

\section{Complementi su approssimazioni di funzioni in $L^p$}

Sia $X$ misurabile in $\R^n$ con $\mu = \mathscr L^n$ su $X$. In precedenza abbiamo visto che

\textbf{Proposizione 3.} Le funzioni in $C_C(\R^n)$ \textit{ristrette a $X$} sono dense in $L^p$ se $p < +\infty$.

\begin{wrapfigure}{r}{130pt}
	\centering
	\vspace{-1.7\baselineskip}
	\inputfigure{supporto-compatto}
	\vspace{-3.5\baselineskip}
\end{wrapfigure}

\textbf{Osservazione.}
Si vede facilmente che $C_C(\R^n) \subset L^p(\R^n)$.

\textbf{Domanda.} Vale un risultato analogo per le funzioni $C_C(X)$?

Notiamo che dato $X \subset \R^n$ le funzioni continue su $X$ hanno supporto compatto solo se $X$ è aperto in quanto il supporto ha veramente distanza non nulla dal bordo e possiamo estendere la funzione a $0$ fuori da $X$. [TODO: Esempio con un chiuso in cui le cose non fungono?]

\textbf{Proposizione 4.} 
Sia $X$ aperto di $\R^n, \mu = \mathscr L^n$ allora $C_C(X)$ è denso in $L^p$ per ogni $p < +\infty$

\textbf{Dimostrazione.}
\begin{itemize}
	\item
		$\mathscr S_C := \{ \text{funzioni semplici con supporto compatto in $X$} \}$ è denso in $L^p(X)$ per ogni $p < +\infty$.

	\item
		Dato $E$ relativamente compatto\footnote{Un sottospazio relativamente compatto di uno spazio topologico è un sottoinsieme dello spazio topologico la cui chiusura è compatta.} in $X$ esiste $f_n \in C_C(X)$ tale che $f_n \to \One_E$ in $L^p$ per ogni $p < +\infty$.
\end{itemize}

La Proposizione 3 non vale per $p = +\infty$, intuitivamente in quanto data $f \in L^\infty(X)$ discontinua, se trovassimo $f_n \to f$ in $L^\infty(X)$ con $f_n$ continue avremmo $f_n \to f$ \textit{uniformemente} e dunque $f$ continua.

\textbf{Fatto.} 
In generale vale che data $f \colon X \to \R$ misurabile, $\norm{f}_\infty \leq \sup_{x \in X} |f(x)|$ (detta anche \textit{norma del sup})

\textbf{Esercizio.} 
Se $X$ è aperto in $\R^n$ e $\mu = \mathscr L^n$ e $f \colon X \to \R$ continua, allora $\norm{f}_\infty = \sup_{x \in X} |f(x)|$.

\textbf{Soluzione.}
Se per assurdo $\exists x \in X$ tale che $\norm{f}_\infty < |f(x)|$ allora la continuità di $f$ implica che esiste un intorno di $x$ in cui $\norm{f}_\infty < |f(x)|$; ma un intorno contiene una palla aperta di misura positiva. \absurd

\begin{wrapfigure}{r}{125pt}
	\centering
	\vspace{-2.5\baselineskip}
	\inputfigure{aperto-a-misura-positiva}
	\vspace{-3.5\baselineskip}
\end{wrapfigure}

In particolare possiamo anche estenderci a $X \subseteq \R^n$ tali che ogni $A$ aperto relativamente a $X$ abbia misura positiva.

Per spiegare meglio il perché la Proposizione 3 non si estende al caso $p = +\infty$ consideriamo
$$
f(x) =
\begin{cases}
	1 & x \geq 0 \\
	0 & x < 0
\end{cases}
$$
e vediamo che $\nexists f_n \colon \R \to \R$ tale che $f_n \to f$ in $L^\infty$. 

Se esistesse $(f_n)_n$, allora sarebbe di Cauchy rispetto alla norma $\norm{\curry}_\infty$ allora per continuità $(f_n)_n$ è di Cauchy anche rispetto alla norma del sup $\implies f_n \to \tilde f$ uniformemente con $\tilde f$ continua, quindi $\tilde f = f$ quasi ovunque ma questo non è possibile per la $f$ definita sopra.

(In particolare dato $E = \{ x \mid f(x) = \tilde f(x) \}$, prendiamo $x_n, y_n \in E$ tali che $x_n \uparrow 0$ e $y_n \downarrow 0$ ma i limiti di $f$ sono $0$ e $1$ \absurd)

\textbf{Teorema} (di Lusin).
Dato $X \subset \R^d, \mu = \mathscr L^d$ e data $f \colon X \to \R$ o $\R^m$ misurabile e $\epsilon > 0$, esiste $E$ aperto in $X$ con $|E| \leq \epsilon$ tale che $f$ è continua su $X \setminus E$ (la restrizione di $f$ a $X \setminus E$ è continua)

\textbf{Osservazione.} 
In generale $f$ può essere non continua in tutti i punti di $X$, infatti $E$ può essere denso e $X \setminus E$ avere parte interna vuota.

\textbf{Lemma} (di estensione di Tietze). Dato $X$ spazio metrico e $C \subset X$ chiuso, $f \colon C \to \R$ continua allora $f$ si estende a una funzione continua su $X$.

Usando questo lemma possiamo rienunciare il teorema precedente come segue

\textbf{Teorema} (di Lusin$'$).
Data $f \colon X \to \R$ misurabile e $\epsilon > 0$, $\exists E$ aperto con $|E| \leq \epsilon$ e $g \colon X \to \R$ continua tale che $f = g$ su $X \setminus E$, inoltre se $f \in L^p(X)$ e $p < +\infty$ si può anche chiuedere che $\norm{f - g}_p \leq \epsilon$.

\textbf{Dimostrazione.}
Basta trovare $E$ misurabile (per ottenere $E$ aperto si usa la regolarità della misura)
\begin{itemize}
	\item \textit{Caso 1}:
		$f \in L^1(X)$ e $|X| < +\infty$

		Abbiamo che $f \in L^1 \implies \exists f_n$ continue tali che $f_n \to f$ in $L^1 \implies f_n \to f$ in misura e per Severini-Egorov esiste $E$ tale che $|E| \leq \epsilon$ e $f_n \to f$ uniformemente su $X \setminus E \implies f$ è continua su $X \setminus E$.

	\item \textit{Caso 2}:
		$f$ qualunque misurabile e $|X| < +\infty$

		\textbf{Lemma.}
		Dati $X, \mc A, \mu$ con $\mu(X) < +\infty$ e data $f \colon X \to \R$ misurabile e $\epsilon > 0$ esiste $F$ misurabile con $\mu(F) \leq \epsilon$ tale che $f$ è limitata su $X \setminus F$.
		
		\textbf{Dimostrazione.}
		$\forall m > 0$ sia $F_m := \{ x \mid |f(x)| > m \}$ allora $F_m \downarrow \varnothing \implies \mu(F_m) \downarrow 0$ e quindi esiste $m$ tale che $\mu(F_m) \leq \epsilon$.

		Quindi data $f$ qualunque misurabile e $|X| < +\infty$ esiste $F$ misurabile tale che $|F| \leq \epsilon / 2$ e con $f$ limitata su $X \setminus F \implies f \in L^\infty(X \setminus F) \subset L^1(X \setminus F)$, dunque per il \textit{Caso 1} esiste $E$ misurabile tale che $|E| \leq \epsilon / 2$ e $f$ è continua su $X \setminus (E \cup F)$ e $\mu(E \cup F) \leq \epsilon$

	\item \textit{Caso 3}:
		$f$ qualunque misurabile

		Per ogni $n$ poniamo $X_n \coloneqq X \cap B(0, n)$ per il \textit{Caso 2} esistono $E_n$ misurabili con $|E_n| \leq \epsilon / 2^n$ tali che $f$ è continua su $X_n \setminus E_n$, infine prendo $E \coloneqq \bigcup_{n=1}^\infty E_n$ con $\mu(E) \leq \epsilon \implies f$ è continua su $X_n \setminus E$ per ogni $n \implies f$ è continua su $X \setminus E$. 

\end{itemize}
\qed



% Appendice. Inizio convoluzione
%
% Lezione dell'18 Ottobre 2021
%

\section{Alcune proprietà degli spazi normati}

\textbf{Proposizione.} Siano $V,W$ spazi normati, $T \colon V \to W$ lineare.
Sono fatti equivalenti
\begin{enumerate}
\item $T$ è continua in $0$.

\item $T$ è continua.

\item $T$ è lipschitziana, cioè esiste una costante $c < +\infty$ tale che $\norm{Tv - Tv'}_W \leq c \norm{v - v'}_V$.

\item Esiste una costante $c$ tale che $\norm{Tv}_W \leq c \norm{v}_V$ per ogni $v \in V$.

\item Esiste una costante $c$ tale che $\norm{Tv}_W \leq c$ per ogni $v \in V$, $\norm{v}_V = 1$.
\end{enumerate}

\textbf{Dimostrazione.}
v) $\Rightarrow$ iv). Vale la seguente
%
$$
\norm{Tv}_W \underbrace{=}_{v = \lambda \tilde{v}, \norm{\tilde{v}}_V = 1} \left| \lambda \right| \norm{T \tilde{v}}_W \leq c \lambda = c \norm{v}_V \leq 1.
$$
%
iv) $\Rightarrow$ iii). Vale la seguente
%
$$
\norm{Tv - Tv'}_W = \norm{T(v - v')}_W \leq c \norm{v - v'}_W.
$$
%
iii) $\Rightarrow$ ii). \\
i) $\Rightarrow$ v). $T$ continua in $0$, dunque esiste $\delta > 0$ tale che
%
$$
\norm{Tv - T0}_W \leq 1 \quad \text{se} \quad \norm{v - 0}_V \leq \delta,
$$
%
cioè
%
$$
\norm{Tv} \leq 1 \quad \text{se} \quad \norm{v} \leq \delta,
$$
%
da cui segue che $\norm{Tv} \leq 1/ \delta$ se $\norm{v} \leq 1$.
\qed

\textbf{Osservazione.} Le costanti ottimali iii), iv), v) sono uguali e valgono
%
$$
c = \sup_{\norm{v}_V \leq 1} \norm{Tv}_W.
$$
%

\textbf{Esempi.}
\begin{enumerate}
\item Sia $X, \mc{A}, \mu$ coma al solito, con $\mu(X) < +\infty$.
Allora, dati $1 \leq p_1 < p_2 \leq +\infty$, vale
\begin{equation} \tag{$\star$} \label{eq:star_1}
L^{p_2}(X) \subset L^{p_1}(X).
\end{equation}
Inoltre, l'inclusione $i \colon L^{p_2}(X) \to L^{p_1}(X)$ è continua.

\textbf{Dimostrazione.} La dimostrazione di \eqref{eq:star_1} segue dalla stima
%
$$
\norm{u}_{p_1} \underbrace{\leq}_{\mathclap{\text{Hölder generalizzato}}} \norm{\One_X}_q \norm{u}_{p_2} \quad \text{dove} \quad q = \frac{p_1 p_2}{p_2 - p_1}.
$$
%
Dove
%
$$
\norm{\One_X}_{\frac{p_1 p_2}{p_2 - p_1}} \norm{u}_{p_2} = \left( \mu(X) \right)^{\frac{1}{p_1} - \frac{1}{p_2}} \norm{u}_{p_2}.
$$
%
Quanto sopra soddisfa la condizione al punto iv).
\qed

\item L'applicazione $\ds L^1(X) \ni u \mapsto \int u \dd \mu \in \R$ è continua.

\textbf{Dimostrazione.} Infatti, vale
%
$$
\left| \int_{X} u \dd \mu \right| \leq \int_{X} \left| u \right| \dd \mu = \norm{u}_1.
$$
%
Quanto sopra soddisfa la condizione al punto iv).
\qed

\item Cosa possiamo dire invece dell'applicazione $\ds L^p(X) \ni u \mapsto \int u \dd \mu \in \R$?
Se $\mu(X) < +\infty$ la continuità segue dagli esempi i) e ii) sopra.
Se invece $\mu(X) = +\infty$? Per esempio $L^2(\R)$? [TO DO].
\end{enumerate}

\section{Convoluzione}

\textbf{Definizione.} Date $f_1,f_2 \colon  \R^d \to \R$ misurabili, il \textbf{prodotto di convoluzione} $f_1 \ast f_2$ è la funzione (da $\R^d$ a $\R$) data da
%
$$
f_1 \ast f_2(x) = \int_{\R^d} f_1(x-y) f_2(y) \dd y
$$
%
%
\textbf{Osservazioni.}
\begin{enumerate}
\item La definizione sopra è ben posta se $f_1,f_2 \geq 0$ ($f_1 \ast f_2(x)$ può essere anche $+\infty$).
In generale non è ben posta per funzioni a valori reali (non è detto che l'integrale esista). 

Ad esempio, se prendiamo $f_1 = 1$ e $f_2 = \sin x$ con $d = 1$, allora $f_1 \ast f_2(x)$ non è definito per alcun $x$.


\item Se $f_1 \ast f_2(x)$ esiste, allora $\ds f_1 \ast f_2(x) = f_2 \ast f_1(x)$, infatti
%
$$
f_1 \ast f_2 (x) 
= \int_{\R^d} f_1(x-y) f_2(y) \dd y 
% \underbrace{=}_{} 
= \left( 
{\footnotesize \begin{gathered}
	t \coloneqq x - y \\ 
	\dd t = \dd y
\end{gathered}}
\right) =
\int_{\R^d} f_1(t) f_2(x-t) \dd t 
= f_2 \ast f_1(x).
$$
%

\item È importante che $f_1,f_2$ siano definite su $\R^d$ e che la misura sia quella di Lebesgue.

In realtà, si può generalizzare quanto sopra rimpiazzando $(\R^d, \mathscr L^d)$ con $(G,\mu)$, dove $G$ è un gruppo commutativo e $\mu$ una misura su $G$ invariante per traslazione\footnote{Ovvero per ogni $g \in G$ e $E \subset G$, posto $E + g \coloneqq \{ x + g \mid x \in E \}$, allora vale $\mu(E) = \mu(E + g)$}. Per esempio, $\Z$ con la misura che conta i punti. Cioè $f_1,f_2 \colon \Z \to \R$, vale
%
$$
f_1 \ast f_2(n) \coloneqq \sum_{n \in \Z} f_1(n - m) f_2(m).
$$
%

\item Data $f$ distribuzione di massa (continua) su $\R^3$, il potenziale gravitazionale generato è
%
$$
v(x) = \int_{y \in \R^d} \frac{1}{\left| x - y \right|} \rho(y) \dd y
$$
%
cioè $v = g \ast \rho$, dove  $g (x) = 1 / \left| x \right|$ è il potenziale di una massa puntuale in $0$.

\item Se $X_1, X_2$ sono variabili aleatorie (reali) con distribuzione di probabilità continua $p_1,p_2$ e $X_1,X_2$ sono indipendenti, allora $X_1 + X_2$ ha distribuzione di probabilità $p_1 \ast p_2$. (Facile per $X_1,X_2$ in $\Z$).

\end{enumerate}



% Convoluzione
%
% Lezione dell'20 Ottobre 2021
%

% \textbf{Definizione.}
% Date $f_1, f_2 \colon \R^d \to \R$ misurabile allora il \textbf{prodotto di convoluzione} è dato da
% $$
% f_1 \ast f_2 (x) \coloneqq \int_{\R^d} f_1(x - y) f_2(y) \dd y
% $$
% e se $f_1$ e $f_2$ sono positive allora $f_1 \ast f_2(x) \in [0, +\infty]$. Ma ad esempio se prendiamo $f_1 = 1$ e $f_2 = \sin x$ con $d = 1$ allora $f_1 \ast f_2(x)$ non è definito per alcun $x$.

\textbf{Proposizione 1.}
Se $|f_1| \ast |f_2| (x) < +\infty$ allora $f_1 \ast f_2(x)$ è ben definito, in quanto $|f_1 \ast f_2(x)| \leq |f_1| \ast |f_2|(x)$.
% in quanto
% $$
% |f_1 \ast f_2(x)| \leq |f_1| \ast |f_2|(x)
% $$

\textbf{Dimostrazione.}
Basta osservare che,
%
\begin{align*}
f_1 \ast f_2(x) & = \int_{\R^d} f_1(x-y) \cdot f_2(y) \dd y 
\leq \left| \int_{\R^d} f_1(x-y) \cdot f_2(y) \dd y \right| \\
& \leq \int_{\R^d} |f_1(x-y) \cdot f_2(y)| \dd y
= |f_1| \ast |f_2|(x) < +\infty.
\end{align*}
\qed

\textbf{Corollario 2.}
Se $|f_1| \ast |f_2| \in L^p(\R^d)$ con $1 \leq p \leq +\infty$ allora $f_1 \ast f_2(x)$ è ben definito per quasi ogni $x \in \R^d$ e $\norm{f_1 \ast f_2}_p \leq \norm{|f_1| \ast |f_2|}_p$.

\textbf{Dimostrazione.}
Segue subito dalla proposizione precedente.
\qed

\textbf{Teorema 3} (disuguaglianza di Young per convoluzione).
Se $f_1 \in L^{p_1}(\R^d)$ e $f_2 \in L^{p_2}(\R^d)$ e preso $r \geq 1$ tale che
\begin{equation}\label{eqn:conv_th3_cond}
	\frac{1}{r} = \frac{1}{p_1} + \frac{1}{p_2} - 1,
	\tag{$\star$}
\end{equation}
allora $f_1 \ast f_2$ è ben definito quasi ovunque e
\begin{equation}\label{eqn:conv_th3_thesis}
	\norm{f_1 \ast f_2}_r \leq \norm{f_1}_{p_1} \cdot \norm{f_2}_{p_2}
	\tag{$\star\star$}
\end{equation}

\textbf{Osservazioni.}
\begin{itemize}
	\item 
		Nel caso di prima $1$ e $\sin x$ sono solo in $L^\infty$ infatti viene $r = -1$ e la disuguaglianza non ha senso.

	\item
		Supponiamo di avere $\norm{f_1 \ast f_2} \leq C \cdot \norm{f_1}_{p_1}^{\alpha_1} \cdot \norm{f_2}_{p_2}^{\alpha_2}$ allora vediamo che per ogni $f_1, f_2$ positiva deve valere necessariamente $\alpha_1 = \alpha_2 = 1$ e la condizione (\ref{eqn:conv_th3_cond}).

		\textbf{Dimostrazione.}
		Per ogni $\lambda > 0$ consideriamo $\lambda f_1$ e $f_2$, allora 
		$$
		\norm{(\lambda f_1) \ast f_2}_r = \norm{\lambda (f_1 \ast f_2)}_r = \lambda \norm{f_1 \ast f_2}_r
		$$
		ma abbiamo anche
		$$
		\norm{(\lambda f_1) \ast f_2}_r \leq C \cdot \norm{\lambda f_1}_{p_1}^{\alpha_1} \cdot \norm{f_2}_{p_2}^{\alpha_2} = C \cdot \lambda^{\alpha_1} \norm{f_1}_{p_1}^{\alpha_1} \cdot \norm{f_2}_{p_2}^{\alpha_2},
		$$
        da cui necessariamente $\alpha_1 = 1$ e di conseguenza $\alpha_2 = 1$.

		A questo punto richiediamo anche che $f_1$ e $f_2$ siano tali che $\norm{f_1}_{p_1}, \norm{f_2}_{p_2} < +\infty$ e $\norm{f_1 \ast f_2} > 0$ (questo possiamo farlo in quanto basta prendere $f_1 = f_2 = \One_B$ con $B$ una palla, nel caso segue proprio che $f_1 \ast f_2 (x) > 0$ se $|x| < 1$).

		Data $f \colon \R^d \to \R$ e $\lambda > 0$ poniamo $R_{\lambda} f(x) \coloneqq f(x / \lambda)$ allora abbiamo
        \begin{align*}
        \norm{(R_\lambda f_1) \ast (R_\lambda f_2)}_r
        & = \norm{\int_{\R^d} f_1 \left( \frac{x-y}{\lambda} \right) \cdot f_2\left( \frac{y}{\lambda} \right) \dd y} 
        = 
        \begin{pmatrix}
        t = \frac{y}{\lambda} \\
        \lambda \dd t = \dd y
        \end{pmatrix} \\
        & = \lambda^d \cdot \norm{\int_{\R^d} f_1 \left( \frac{x}{\lambda} - t \right) \cdot f_2(t) \dd t} \\
        & = \lambda^d \cdot \norm{R_\lambda (f_1 \ast f_2)}_r.
        \end{align*}
        Per il punto successivo abbiamo $\norm{R_\lambda(g)}_r = \lambda^{d/r} \norm{g}_r$, da cui otteniamo
		$$
		\norm{(R_\lambda f_1) \ast (R_\lambda f_2)}_r 
		= \lambda^{d \left( 1 + \frac{1}{r} \right)} \norm{f_1 \ast f_2}_r.
		$$
		Ma anche
		$$
		\norm{(R_\lambda f_1) \ast (R_\lambda f_2)}_r 
		\leq C \cdot \norm{R_\lambda f_1}_{p_1} \cdot \norm{R_\lambda f_2}_{p_2}
		= \lambda^{d \left( \frac{1}{p_1} + \frac{1}{p_2} \right)} \cdot \norm{f_1}_{p_1} \cdot \norm{f_2}_{p_2}.
		$$
		Dunque sicuramente abbiamo $\lambda^{d \left(1 + {1 / r} \right)} \leq C \cdot \lambda^{d\left( {1 / p_1} + {1 / p_2} \right)}$ per ogni $\lambda > 0$ e quindi $1 + {1 / r} = {1 / p_1} + {1 / p_2}$.
		\qed

	\item
		$\norm{R_\lambda f}_p = \lambda^{d / p} \norm{f}_p$ ed in realtà possiamo ricavare l'esponente $d / p$ per \textit{analisi dimensionale}\footnote{Ovvero studiando le potenze delle unità di misura delle varie quantità.}.
        % \footnote{In particolare ad Istituzioni di Analisi si vedono le disuguaglianze di Sobolev ed anche in quel caso tutte le condizioni sugli esponenti si riescono a ricavare per analisi dimensionale...}. 
        Consideriamo l'espressione
		$$
		\norm{f}_p^p = \int_{\R^d} f(x) \dd x.
		$$
		Se $f(x)$ è una \textit{quantità adimensionale} allora $\int_{\R^d} f \dd x$ ha dimensione di un \textit{volume} $\mathsf L^d$, da cui $\norm{f}_p$ ha dimensione di $\mathsf L^{d / p}$.

		Similmente, per ottenere $\norm{R_\lambda(f_1 \ast f_2)}_r = \lambda^{d(1 + 1 / r)} \norm{f_1 \ast f_2}_r$, basta osservare che
		$$
		f_1 \ast f_2 (x) = \int_{\R^d} f_1(x - y) f_2(y) \dd y
		$$
		ha dimensione $\mathsf L^d$, da cui
		$$
		\norm{f_1 \ast f_2}_r = \bigg( \int_{\R^d} \underbrace{|f_1 \ast f_2|^r}_{\mathsf L^{dr}} \underbrace{\dd x}_{\mathsf L^d} \bigg)^{1 / r}
		$$
		ha dimensione di $\mathsf L^{d(1 + 1/r)}$.
\end{itemize}

\textbf{Dimostrazione Teorema 3.}
Per via del Corollario 2. ci basta dimostrare (\ref{eqn:conv_th3_thesis}) se $f_1, f_2 \geq 0$.
\begin{itemize}
	\item
		\textit{Caso facile.} Se $p_1 = p_2 = 1$ e $r = 1$
		$$
		\begin{aligned}
			\norm{f_1 \ast f_2}_1
			&= \int f_1 \ast f_2 (x) \dd x 
			= \iint f_1(x - y) f_2(y) \dd y \dd x 
			= \int f_2(y) \int f_1(x - y) \dd x \dd y = \\
			&= \int \norm{f_1}_1 \cdot f_2(y) \dd y 
			= \norm{f_1}_1 \cdot \norm{f_2}_1
		\end{aligned}
		$$

	\item
		\textit{Caso leggermente meno facile.} Se $p_1 = p, p_2 = 1$ e $r = p$.
		Vogliamo vedere che
		$$
		\norm{f_1 \ast f_2}_p \leq \norm{f_1}_p \cdot \norm{f_2}_1
		$$
		allora
		$$
		\begin{aligned}
			\norm{f_1 \ast f_2}_p
			&= \int_{\R^d} (\underbrace{f_1 \ast f_2}_{h})^p \dd x
			= \int h \cdot h^{p-1} \dd x 
			= \iint f_1(x - y) f_2(y) h^{p-1}(x) \dd y \dd x = \\
			&= \iint f_1(y - x) h^{p-1}(x) \dd x f_2(y) \dd y 
			\overset{\text{H\"older}}{\leq} 
			\int \norm{f_1(y - \curry)}_p {\| h^{p-1} \|}_{p'} f_2(y) \dd y
		\end{aligned}
		$$
		con $p'$ esponente coniugato a $p$. Inoltre notiamo che $\norm{f_1(y - \curry)}_p = \norm{f_1}$ per invarianza di $\mathscr L^d$ per riflessioni e traslazioni. Infine otteniamo
		$$
		\norm{f_1}_p {\| h^{p-1} \|}_{p'} \norm{f_2}_1
		= \norm{f_1}_p \norm{h}_{p'}^{p-1} \norm{f_2}_1.
		$$
		Dunque, $\norm{f_1 \ast f_2}_p^p \leq \norm{f_1 \ast f_2}_p^{p-1} \norm{f_1}_p \norm{f_2}_1 \implies \norm{f_1 \ast f_2}_p \leq \norm{f_1}_p \norm{f_2}_1$. Quest'ultima implicazione però è valida solo nel caso in cui $0 < \norm{f_1 \ast f_2}_p < +\infty$. Resterebbero da controllare i due casi in cui la norma è $0$ oppure $+\infty$. Il primo è ovvio; il secondo invece si fa per approssimazione e passando al limite.

		Consideriamo $f_1, f_2$ e approssimiamole con $f_{1,n}, f_{2,n}$ limitate a supporto compatto, allora vale $\norm{f_{1,n} \ast f_{1,n}}_p \leq \norm{f_{1,n}}_p \cdot \norm{f_{2,n}}_1$ e passando al limite si ottiene la tesi. In particolare possiamo costruire le $f_n$ come
		$$
		f_n(x) \coloneqq (f(x) \cdot \One_{\mc B(0, n)}(x)) \land n
		$$

		\textbf{Osservazione.}
		Se $f_2 \geq 0$ e $\int f_2 \dd x = 1$ allora $\norm{f_1 \ast f_2}_p \leq \norm{f_1}_p$ è una versione semplificata della proposizione precedente, in particolare la dimostrazione si semplifica in quanto possiamo pensare a $f_2$ come distribuzione di probabilità e quindi $f_1 \ast f_2$ è una ``media pesata'' delle traslazioni di $f_1$ o più precisamente una combinazione convessa ``integrale''.

	\item
		\textit{Caso generale.} Non lo facciamo perché servono mille mila parametri e non è troppo interessante.
\end{itemize}
\qed

Nel caso $r = +\infty$ gli esponenti $p_1$ e $p_2$ sono proprio coniugati e possiamo rafforzare la tesi del teorema precedente.

\textbf{Teorema 4} (caso $r = +\infty$ del Teorema 3).
Dati $p_1$ e $p_2$ esponenti coniugati e $f_1 \in L^{p_1}(\R^d), f_2 \in L^{p_2}(\R^d)$, allora
\begin{enumerate}
	\item \label{item:20ott_th4_1} 
		$f_1 \ast f_2(x)$ è ben definito per ogni $x \in \R^d$

	\item \label{item:20ott_th4_2}
		$|f_1 \ast f_2(x)| \leq \norm{f_1}_{p_1} \norm{f_2}_{p_2}$

	\item \label{item:20ott_th4_3} 
		$f_1 \ast f_2$ è uniformemente continua

	\item \label{item:20ott_th4_4}
		Se $1 < p_1, p_2 < +\infty$ allora $f_1 \ast f_2 \to 0$ per $|x| \to +\infty$
\end{enumerate}

Premettiamo i seguenti risultati.

\textbf{Proposizione 5.}
Data $f \in L^p(\R^d)$ con $p < +\infty$ la mappa
$$
\begin{array}{cccc}
	\tau_h f : & \R^d & \to & L^p(\R^d) \\
	& h & \mapsto & f(\curry - h)
\end{array}
$$
è continua.

\textbf{Lemma 6.}
Lo spazio $C_0(\R^d) = \{ f \colon \R^d \to \R \text{ continue con } f(x) \to 0 \text{ per } |x| \to \infty \}$ è chiuso rispetto alla convergenza uniforme.


\textbf{Dimostrazione Teorema 4.}
\begin{enumerate}
\item Osserviamo che
$$
|f_1| \ast |f_2|(x) = \int_{\R^d} |f_1(x-y)| \cdot |f_2(y)| \dd y
\overset{\text{Hölder}}{\leq} \norm{f_1(x - \curry)}_{p_1} \norm{f_2}_{p_2}
= \norm{f_1}_{p_1} \norm{f_2}_{p_2}
$$
e concludiamo per la Proposizione 1.

\item Dal punto precedente abbiamo che $|f_1| \ast |f_2|(x) \leq \norm{f_1}_{p_1} \norm{f_2}_{p_2}$, da cui si conclude banalmente.

\item Uno tra $p_1$ e $p_2$ è finito; supponiamo lo sia $p_1$.
Fissiamo $x,h \in \R^d$
%
$$
f_1 \ast f_2(x+h) - f_1 \ast f_2(x) = \int_{\R^d} \left( f_1(x+h-y) - f_1(x-y) \right) f_2(y) \dd y
$$
%
quindi
\begin{align*}
\left| f_1 \ast f_2(x+h) - f_1 \ast f_2(x) \right|
& \leq \int \left| f_1(x+h - y) - f_1(x - y) \right| | f_2| \dd y  \\
& \underset{\text{Holder}}{\leq} \norm{f_1(x+h - \curry) - f_1(x - \curry)}_{p_1} \norm{f_2}_{p_2} \\
& = \norm{f_1(\curry - h) - f_1(\curry)}_{p_1} \norm{f_2}_{p_2} \\
& = \underbrace{\norm{\tau_h f_1 - f_1}_{p_1}}_{\xrightarrow[\text{Proposizione 5}]{h \to 0} 0} \norm{f_2}_{p_2}.
\end{align*}

\item Approssimiamo $f_1$  e $f_2$ con $f_{1,n}$ e $f_{2,n} \in \mc{C}_C(\R^d)$ in $L^{p_1}$ e $L^{p_2}$ rispettivamente.

Osserviamo che $f_{1,n} \ast f_{2,n} \in \mc{C}_C(\R^d) \subset \mc{C}_0(\R^d)$.
Per il Lemma 6 basta dimostrare che $f_{1,n} \ast f_{2,n} \longrightarrow f_1 \ast f_2$ uniformemente
%
\begin{align*}
\norm{f_{1,n} \ast f_{2,n} - f_1 \ast f_2}_\infty
& = \norm{\left( f_{1,n} \ast f_{2,n} - f_{1,n} \ast f_2 \right) + \left( f_{1,n} \ast f_2 - f_1 \ast f_2 \right)}_\infty \\
& \underset{\mathclap{\text{lin della conv}}}{\leq} \norm{f_{1,n} \ast \left( f_{2,n} - f_2 \right)}_\infty + \norm{\left( f_{1,n} - f_1 \right) \ast f_2}_\infty \\
& \underset{ii)}{\leq} \underbrace{\norm{f_{1,n}}_{p_1}}_{\to \norm{f_1}_{p_1}} \underbrace{\norm{f_{2,n} - f_2}_{p_2}}_{\to 0} + \underbrace{\norm{f_{1,n} - f_1}_{p_1}}_{\to 0} \norm{f_2}_{p_2}.
\end{align*}

Quindi $\norm{f_{1,n} \ast f_{2,n} - f_1 \ast f_2}_\infty \to 0$.
\qed
\end{enumerate}

% \textbf{Dimostrazione \ref{item:20ott_th4_1} e \ref{item:20ott_th4_2}.}
% Seguono subito da (\ref{eqn:conv_th3_cond}) per $f_1, f_2 \geq 0$ (con il Corollario 2.), se $f_1, f_2 \geq 0$ allora
% $$
% f_1 \ast f_2 (x) 
% = \int_{\R^d} f_1(x - y) f_2(y) \dd y 
% \leq \norm{f_1(x - \curry)}_{p_1} \norm{f_2}_{p_2} 
% = \norm{f_1}_{p_1} \norm{f_2}_{p_2}
% $$

% \textbf{Proposizione 5.}
% Data $f \in L^p(\R^d)$ con $p < +\infty$ la mappa
% $$
% \begin{array}{cccc}
% 	\tau_h f : & \R^d & \to & L^p(\R^d) \\
% 	& h & \mapsto & f(\curry - h)
% \end{array}
% $$
% è continua.

% \textbf{Lemma 6.}
% Lo spazio $C_0(\R^d) = \{ f \colon \R^d \to \R \text{ continue con } f(x) \to 0 \text{ per } |x| \to 0 \}$ è chiuso rispetto alla convergenza uniforme.

% \textbf{Dimostrazione \ref{item:20ott_th4_3}.} 
% Supponiamo $p_1 < +\infty$ allora
% $$
% \begin{aligned}
% 	|f_1 \ast f_2(x + h) - f_1 \ast f_2(x)|
% 	&\leq \int |f_1(x + h - y) - f_1(x - y)| \cdot |f_2(y)| \dd y \\
% 	&\leq \norm{f_1(x + h - \curry) - f(x - \curry)}_{p_1} \norm{f_2}_{p_2} \\
% 	&= \norm{\tau_h f_1 - f_1}_{p_2} \norm{f_2}_{p_2}
% \end{aligned}
% $$
% \qed



% Esercitazione
%
% Lezione del 21 Ottobre 2021 
%

\mysection{Esercitazione del 21 ottobre}

% Piccolo recap di quanto fatto le scorse lezioni.

Data $T \colon X \to Y$ lineare tra $X,Y$ spazi normati, allora $T$ è continua se solo se esiste  $C > 0$ tale che $\norm{T(x)}_Y \leq C \norm{x}_X$ per ogni $x \in X$.

Applichiamo questo risultato.

\begin{enumerate}
\item Sia $X = \R^d$. L'applicazione $\ds L^1(\R^d) \ni u \xmapsto{\; T \;} \int_{\R^d} u \dd x$ è lineare e continua in quanto limitata. Infatti:
%
$$
\left| T(u) \right| = \left| \int_{\R^d} u \dd x  \right| \leq \int_{\R^d} \left| u \right| \dd x = \norm{u}_{L^1(\R^d)}.
$$
%

\item Studiamo ora il caso per $p > 1$. Data $u \in L^p(\R^d)$, l'applicazione
%
$$
u \mapsto \int_{\R^d} u \dd x 
$$
%
potrebbe non essere ben definita.

Ad esempio se restringiamo il dominio a $L^p(\R^d) \cap L^1(\R^d)$ l'applicazione sopra è ben definita, ma in generale non è continua.
Più formalmente, la mappa
%
$$
T \colon \left( L^p \cap L^1(\R^d), \norm{\cdot}_{L^p} \right) \to \R
$$
%
è lineare ma non continua.

Studiamo il caso reale, ovvero $d = 1$.

Per verificare quanto sopra, utilizziamo la definizione di continuità per successioni.
Definiamo una successione di funzioni a supporto compatto $u_n$, che sappiamo essere in tutti gli spazi $L^p$, e verifichiamo che $\lim_n T(u_n) \neq T(u_\infty)$ dove $u_\infty \coloneqq \lim_n u_n$.

Definiamo la successione come segue (fare disegno):
%
$$
u_n (x) =
\frac{1}{n} \One_{[n, 2n]} =
\begin{cases}
\frac{1}{n} \quad \text{se} \quad n \leq x \leq 2n \\
0 \quad \text{altrimenti}. 
\end{cases} 
$$
%
Dunque, $\ds T(u_n) = \int_{\R} u_n \dd x = \frac{1}{n} \left| E_n \right| = 1$, dove $E_n = [n,2n]$.
Segue che $T(u_n) \equiv 1$ ma rispetto alla convergenza in $L^p$, $T(u_n) \not\to T(u_\infty) = T(0) = 0$.

% e non è vero che $T(u_\infty = u_n \to 0) = 0$.

Più in generale, quando $u \in L^p(\R^d)$ con $p > 1$, una costruzione come sopra non funziona, infatti
%
$$
\norm{u_n}_{L^p(\R)}^p 
= \int_{\R} u_n^p(x) \dd x 
= \frac{1}{n^p} \cdot \left| E_n \right| 
= \frac{n}{n^p} 
= \frac{1}{n^{p-1}} 
\xrightarrow{n \to \infty} 0.
$$
%                    
Un altro modo per dimostrare quanto sopra è verificare che, per ogni $C > 0$, esiste una funzione $u \in L^p(\R^d) \cap L^1(\R^d)$ tale per cui 
%
$$
\left| \int_{\R} u \dd x  \right| > C \left( \int_{\R} \left| u \right|^p \dd x  \right)^{1/p}.
$$
%

Notiamo che questo è proprio l'esercizio che segue.

\end{enumerate}


\textbf{Esercizio.} Fissato $C > 0$, trovare $u \in L^p \cap L^1 (\R)$ tale che

\begin{equation} \tag{$\star$} \label{eq:lez-21ott_es1}
	\left| \int_{\R} u \dd x\right| > C \norm{u}_{L^p(\R)}
\end{equation}

\textbf{Dimostrazione.} Fissato $C > 0$, cerchiamo una funzione in $L^p$ il cui integrale in modulo sia maggiore di $C$ per la sua norma $L^p$.
Per trovare $u$ consideriamo la successione di funzioni definita come segue
%
$$
f_n = 
\begin{cases}
1/x \quad & 1 \leq x \leq n \\
0 \quad  & \text{altrimenti} 
\end{cases}. 
$$
%
Le funzioni $f_n$ sono a supporto compatto e stanno in ogni $L^p$.
Notiamo che $f_n \uparrow f$ definita come
%
$$
f = 
\begin{cases}
1/x \quad & x > 1 \\
0 \quad & \text{altrimenti}
\end{cases}
$$
%
ed $f \in L^p$ per ogni $p > 1$.
In particolare, $\ds \left( \int_\R |f_n|^p \right)^{1/p} \leq \left( \int_\R |f|^p \right)^{1/p} \underbrace{\leq}_{\mathclap{f \in L^p \; \forall p > 1}} +\infty$ per ogni $n$.

Quindi il secondo membro di \eqref{eq:lez-21ott_es1} è maggiorato da una costante che non dipende da $n$. D'altra parte, per Beppo Levi $\ds \lim_n \int_{\R} f_n \dd x = \int_{\R} f \dd x$ che non sta in $L^1$. In conclusione, esiste un $n$ abbastanza grande per cui vale \eqref{eq:lez-21ott_es1}.
\qed

%%%
\textbf{Esercizio.} Sia $p \geq 1$ e $\ds E = \left\{ u \in L^p(-1,1) \colon \fint_{-1}^1 u \dd x = 0 \right\}$.

\begin{enumerate}
\item Dire se $E$ è limitato in $L^p(-1,1)$.

\item Dire se $E$ è chiuso in $L^p(-1,1)$.
\end{enumerate}

\textbf{Soluzione.}
\begin{enumerate}
\item Dimostrare che $E$ è limitato in $L^p(-1,1)$ equivale a dimostrare che esiste $M > 0$ tale che ogni  $u \in L^p(-1,1)$, $\ds \fint_{-1}^1 u \dd x = 0$ verifica $\norm{u}_{L^p} \leq M$.

Vediamo che $E$ non è limitato.
Preso $M > 0$, riesco sempre a trovare una funzione maggiore di $M$ in norma.
Ad esempio la funzione definita come
%
$$
u(x) \coloneqq
\begin{cases}
M \quad \text{se} \quad x \in (0,1) \\
-M \quad \text{se} \quad x \in (-1,0)
\end{cases} 
$$
%
ha norma $\ds \norm{u}_{L^p}^p = 2 M^p$.

\textbf{Nota.} Aveva senso cercare il controesempio nella classe delle funzioni dispari e limitate, perché hanno media zero, e perché sono in tutti gli $L^p$.

\item Vediamo che $E$ è chiuso.

\textbf{Nota.} Possiamo dimostrarlo usando i teoremi di convergenza, ma seguiremo un'altra strada.

\begin{itemize}
\item \textit{Caso} $p > 1$. Definiamo l'operatore 
\begin{align*}
T \colon L^p(-1,1) & \to \R \\
u & \mapsto \int_{-1}^{1} u \dd x 
\end{align*}
è ben definito.
Infatti, per Hölder vale
%
$$
\left| \int_{-1}^{1} 1 \cdot u \dd x  \right| \leq \left( \int_{-1}^{1} \left| u \right|^p \dd x  \right)^{1/p} \left( 1^q \right)^{1/q}
$$
%
dove $q = \frac{p}{p-1}$.
Allora
 %
$$
\left| T(u) \right| \leq \norm{u}_{L^p(-1,1)} \cdot 2^{\frac{p}{p-1}}.
$$
%
Dunque $T$ è continuo in $L^p$ per ogni $p > 1$.

\item \textit{Caso} $p = 1$. L'operatore sopra è continuo anche per $p = 1$. Grazie alla stima vista prima
%
$$
\left| T(u) \right| = \left| \int_{-1}^{1} u \dd x  \right| \leq \int_{-1}^{1} \left| u \right| \dd x = \norm{u}_{L^1}.
$$
%
Dunque $T$ è continua e $T^{-1}(0) = E$, dunque $E$ è chiuso.

\end{itemize}

\end{enumerate}

\textbf{Esercizio.} [TO DO] Sia $p \geq 1$. Definiamo 
%
$$
F = \left\{ v \in L^p(\R) \mymid \int_{0}^{1} v(x) \dd x - 2 \int_{-1}^{0} v(x) \dd x = 3 \right\}.
$$
%
Dire se $F$ è chiuso in $\left( L^p(\R), \norm{\curry}_{L^p(\R)} \right)$.


\textbf{Esercizio.} [TO DO] Sia 
%
$$
G = \left\{ v \in L^p(0,2\pi) \mymid \int_{0}^{2\pi} v(x) \sin(x) \dd x = 1  \right\}.
$$
%
Dire se $G$ è chiuso in $L^2(0,2\pi)$.

\textbf{Domanda.} Dato $L^p (X,\mu)$ e $V$ sottospazio di $L^p(X,\mu)$, posso dire che $V$ è chiuso?

In generale no! Infatti esistono sottospazi densi in $L^p(X,\mu)$.

Ad esempio in $\ell^2$ consideriamo l'insieme denso
%
$$
V = \left\{ \left\{ x_n \right\} \mymid x_n = 0 \quad \text{definitivamente}  \right\}.
$$
%
Vediamo che non è chiuso. Sia $\underline{x} \in \ell^2$, definita come $\underline{x} = \{ 1 / n \}_{n \in \N \setminus \{0 \}}$, diciamo che $\ds \underline{x} = \lim_{n \to +\infty} \underline{x}^n$ dove
%
$$
x_n^k = 
\begin{cases}
\frac{1}{n} \quad 1 \leq n \leq k \\
0 \quad n > k, n = 0
\end{cases} 
$$
%
abbiamo che
%
$$
\norm{\underline{x} - \underline{x}^k}_{\ell^2}^2 = \sum_{n=1}^{\infty} \left| x_n -x_n^k \right|^2 = \sum_{n = k+1}^{+\infty} \left| x_n \right|^2 = \sum_{n = k+1}^{\infty} \frac{1}{n^2} \xrightarrow{k \to +\infty} 0.   
$$
%

Vediamo un altro esempio. Siano $X = \R$, $\mu$ la misura di Lebesgue e $p > 1$.
In tal caso, l'insieme $L^p \cap L^1 (\R)$ è un sottospazio denso in $\left( L^1(\R), \norm{\cdot }_{L^1} \right)$ e $\left( L^p(\R), \norm{\cdot }_{L^p} \right)$.

\textbf{Nota.} L'insieme $L^2(\R) \cap L^1(\R)$ è un sottospazio proprio di $L^1(\R)$. Diciamo che non è chiuso in $\norm{\curry}_{L^1(\R)}$ perché è denso.
Infatti, 
%
$$
\mc{C}_C^0(\R) \subset L^2(\R) \cap L^1(\R).
$$
%

\mysubsection{Convoluzione}

Sia $f \in L^1(\R^d)$ e sia $g \colon \R^d \to \R$ continua a supporto compatto\footnote{In tal caso $g$ è lipschitziana.}.
%
$$
\left| g(x) - g(y) \right| \leq M \left| x - y \right|_{\R^d}.
$$
%

\textbf{Esercizio.} Dimostrare che $f \ast g$ è ben definita e lipschitziana, dove $f \in L^1(\R^d)$ e $g \in \mc{C}_C^0 (\R^d)$.

Verifichiamo che la convoluzione è ben definita.
Dal fatto che $g \in \mc{C}_C^0(\R^d)$ abbiamo in particolare che $g$ è limitata, da cui
%
$$
f \ast g = \int_{\R^d} f(x-y) \cdot g(y) \dd y \overset{|g| \leq M}{\leq} M \cdot \int_{\R^d} f(x-y) \dd y \overset{f \in L^1(\R^d)}{<} +\infty.
$$
%

Ora verifichiamo che $f \ast g$ è lipschitziana. Consideriamo $x_1,x_2 \in \R^d$
%
$$
\left| f \ast g (x_1) - f \ast g(x_2) \right| = \left| \int_{\R^d} f(x_1 - x) g(y) \dd y  - \int_{\R^d} f(x_2 - y) g(y) \dd y  \right|
$$
%
Usiamo la proprietà che, essendo $f \ast g$ ben definita, si ha $ f \ast g(x) = g \ast f(x) $. Da cui
\begin{align*}
\left| f \ast g(x_1) - f \ast g(x_2) \right|  & = \left| \int_{\R^d} g(x_1 - y) f(y) \dd y \int_{\R^d} g(x_2 - y) f(y) \dd y   \right| \\
& = \left| \int_{\R^d} \left( g(x_1 - y) - g(x_2 - y) \right) f(y) \dd y  \right| \\
& \leq \int_{\R^d} \left| g(x_1 - y) - g(x_2 - y) \right| \left| f(y) \right| \dd y \\
& \leq \int_{\R^d} M \left| (x_1 - y) - (x_2 - y) \right| \left| f(y) \right| \dd y \leq M \left| x_1 - x_2 \right| \cdot \norm{f}_{L^1(\R^d)}.
\end{align*}

\textbf{Esercizio.} [TO DO] Se $f \in L^1(\R^d)$ e $g$ a supporto compatto è $\alpha$-Hölderiana allora anche $f \ast g$ lo è.

\textbf{Esercizio.} [TO DO] Presa $f(x) = \One_{[0,1]}$ in $\R$, calcolare $f \ast f$.

\mysubsection{Separabilità degli spazi $L^p$}

\textbf{Proposizione.} Si ha che $L^p\left( \R^d, \mu \right)$ con $\mu $ la misura di Lebesgue, è separabile se solo se $p \neq +\infty$. Lo stesso risultato vale per $\ell^p$.

\textbf{Osservazione.} La proposizione è valida anche per $L^p(X,\mu)$ con $X \subset \R^d$ aperto.

Sia $1 \leq p < +\infty$, $L^p(\R^d, \mu)$ con $\mu$ la misura di Lebesgue.
Le funzioni semplici costituite da somme finite di insiemi di misura finita sono dense in $L^p(\R^d)$.

Prendiamo una base numerabile di $\R^d$ e la indichiamo con $\mc{B}$. L'insieme
%
$$
Y = \left\{ \sum_{i=1}^{n} \alpha_i \One_{B_i} \mymid B_i \in \mc{B}, \alpha_i \in \Q \right\}
$$
%
è numerabile. Vediamo che è denso in $L^p(\R^d)$.

\textit{Idea.} È sufficiente approssimare le funzioni semplici a somma finita $\sum_{i=1}^{N} \alpha_i \One_{E_i} $. In particolare, ci basta approssimare $\alpha \cdot \One_E$. Essendo $\alpha \in \R$ troviamo una successione di razionali $\alpha_j$ tali che $\alpha_j \xrightarrow{j \to \infty} \alpha$. Dunque, rimane da approssimare l'insieme $E$.

Fissiamo $E$ e supponiamo dapprima $E$ aperto. Possiamo scrivere $E$ come unione arbitraria di elementi della base $\mc{B}$
%
$$
E = \bigcup_{i = 1}^\infty B_i.
$$
%
Per approssimare $E$ consideriamo gli insiemi $E_N = \bigcup_{i = 1}^N B_i$.
Otteniamo $\ds \left| E \right| = \lim_N \left| E_N \right|$, da cui $\left| E \setminus E_N \right| \xrightarrow{N \to +\infty} 0$.
Concludiamo notando che il caso $E$ arbitrario si fa approssimandolo con una famiglia di aperti.

\vspace{3mm}

Per $\ell^p$ con $p < +\infty$ definiamo
%
$$
Y = \left\{ \left\{ x_n \right\} \mymid x_n = 0 \quad \text{definitivamente}, x_n \in \Q \right\}
$$
%
e verifico che è numerabile e separabile.

\textbf{Domanda.} Cosa succede per $p = +\infty$?

Consideriamo $L^\infty ([0,+\infty], \mu)$ con $\mu$ di Lebesgue e $E_n = [n,n+1]$.
Definiamo l'insieme
 %
$$
Z = \left\{ \forall J \subset \N \quad u = \sum_{j \in J} \One_{E_j} \right\}.
$$
%
$Z$ ha la cardinalità delle parti di $\N$ cioè è più che numerabile. Osserviamo che per ogni $u,v \in Z$, $u \neq v$ si ha che $\norm{u - v}_{L^\infty(\R)} = 1$.
Se per assurdo esistesse un insieme denso e numerabile $D$ in $\ell^\infty$, per definizione di insieme denso dovremmo trovare per ogni palla di raggio minore di 1 e centro in un qualsiasi elemento di $Z$, un elemento di $D$. Ma questo è impossibile in quanto $D$ ha cardinalità numerabile e $Z$ la cardinalità del continuo.

Vediamo in un altro modo che $l^\infty$ non è separabile. 
Se per assurdo $Y = \left\{ \underline{x}^k \right\}_{k \in \N}$ fosse denso in $L^\infty$, allora potremmo definire un elemento $z \in l^\infty$ tale che $\norm{\underline{x}^k - \underline{z}}_{l^\infty} \geq 1$ per ogni $k$.

Definiamo $z = \left\{ z_n \right\}$ come segue
%
$$
z_n = 
\begin{cases}
0 \quad \text{se} \quad | x_n^n | > 1 \\
2 \quad \text{se} \quad | x_n^n | \leq 1
\end{cases}. 
$$




% Rimanenze dalla lezione precedente. Derivata e convoluzione
%
% Lezione del 25 Ottobre 2021
%

% @aziis98: Secondo me possiamo rilocarle abbastanza le cose di questa lezione boh
% \section{Aggiunte sulle lezioni precedenti}
\section{Rimanenze dalla lezione precedente}

\textbf{Proposizione.}
Data $f \in L^p(\R^d)$ con $1 \leq p < +\infty$ allora la funzione $\tau_h f \colon \R^d \to L^p(\R^d)$ data da $h \mapsto f(\cdot - h)$ è continua.

\textbf{Dimostrazione.}
Per prima cosa notiamo che basta vedere solo la continuità in $0$ in quanto
$$
\tau_{h'} f - \tau_h f = \tau_{h} (\tau_{h' - h} f - f) 
\implies \norm{\tau_{h'} f - \tau_{h} f}_p = \norm{\tau_{h' - h} f - f}_p.
$$
Dimostriamo ora la proposizione in due passi.
\begin{itemize}
	\item
		\textit{Caso 1:} $f \in C_C(\R^d)$
		$$
		\norm{\tau_h f - f}_p^p 
		= \int_{\R^d} |f(x - h) - f(x)|^p \dd x \xrightarrow{|h| \to 0} 0
		$$
		per convergenza dominata, verifichiamo però che siano rispettate le ipotesi
		\begin{enumerate}
			\item
				La convergenza puntuale, ovvero $|f(x - h) - f(x)|^p \xrightarrow{|h| \to 0} 0$ segue direttamente dalla continuità di $f$.
			\item
				Come dominazione invece usiamo $|f(x - h) - f(x)|^p \leq (2 \norm{f}_\infty)^p \cdot \One_{\mc B(0, R + 1)}$ usando che $f \in C_C \implies \operatorname{supp}(f) \subset \overline{B(0, R)}$ e poi che 
				$$
				\operatorname{supp}(f(\curry - h) - f(\curry)) \subset \overline{\mc B(0, R + |h|)}
				$$
				infine se $|h| < 1$ come raggio ci basta prendere $R + 1$.
		\end{enumerate}
	\item 
		\textit{Caso 2:} $f$ qualunque
		Dato $\epsilon > 0$ prendiamo $g \in C_C(\R^d)$ tale che $\norm{g - f} \leq \epsilon$ allora aggiungiamo a sottraiamo $g + \tau_h g$ e raggruppiamo in modo da ottenere
		$$
		\begin{gathered}
			\tau_h f - f = \tau_h(f - g) + (\tau_h g - g) + (g - f) \\
			\implies \norm{\tau_h f - f}_p 
			\leq \underbrace{\norm{\tau_h(f - g)}_p}_{\leq \epsilon} 
			+ \norm{\tau_h g - g}_p
			+ \underbrace{\norm{g - f}_p}_{\leq \epsilon} 
			\leq 2 \epsilon + \underbrace{\norm{\tau_h g - g}_p}_{\to 0 \text{ per \textit{Caso 1}}}
		\end{gathered}
		$$
		dunque $\limsup_{|h| \to 0} \norm{\tau_h f - f}_p \leq 2\epsilon$ ma per arbitrarietà di $\epsilon$ otteniamo anche che $\norm{\tau_h f - f}_p \to 0$ per $|h| \to 0$.
\end{itemize}
\qed

\textbf{Teorema.}
Siano $f_1 \in L^{p_1}(\R^d)$ e $f_2 \in L^{p_2}(\R^d)$ con $p_1$ e $p_2$ esponenti coniugati, allora $f_1 \ast f_2$ è definita per ogni $x$ e uniformemente continua
$$
	|f_1 \ast f_2(x)| \leq \norm{f_1}_{p_1} \cdot \norm{f_2}_{p_2} \quad \forall x.
$$

\textbf{Dimostrazione.}
Prendiamo $f_{1,n}, f_{2, n} \in C_C(\R^d)$ tali che $f_{1, n} \to f_1$ in $L^{p_1}$ e $f_{2, n} \to f_2$ in $L^{p_2}$.
\begin{itemize}
	\item 
		Per prima cosa verifichiamo che $f \ast g$ è ben definita. Notiamo che $f_{1,n} \ast f_{2,n}$ ha supporto limitato, infatti se $\supp(f_{i,n}) \subset \overline{\mc B(0, r_{i,n})}$ per $i = 1, 2$ allora
		$$
		\supp(f_{1,n} \ast f_{2,n}) \subset \overline{\mc B(0, r_{1,n} + r_{2,n})}
		$$
		e basta notare che l'espressione
		$$
		f_1 \ast f_2(x) = \int_{\R^d} f_1(x - y) f_2(y) \dd y
		$$
		ha integranda nulla per ogni $y$ se $|x| \geq r_{1,n} + r_{2,n}$.
	
	\item
		Vediamo che $f_{1,n} \ast f_{2,n} \to f_1 \ast f_2$ uniformemente
		$$
		f_{1,n} \ast f_{2,n} - f_1 \ast f_2 
		= (f_{1,n} - f_1) \ast f_{2,n} - f_1 \ast (f_{2,n} - f_2)
		$$
		$$
		\begin{aligned}
			\norm{f_{1,n} \ast f_{2,n} - f_1 \ast f_2}_p
			&\leq \norm{(f_{1,n} - f_1) \ast f_{2,n}}_p + \norm{f_1 \ast (f_{2,n} - f_2)}_p \\
			&\leq 
			\underbrace{\norm{f_{1,n} - f_1}_{p_1}}_{\to 0}
			\cdot \underbrace{\norm{f_{2,n}}_{p_2}}_{\to \norm{f_2}_{p_2}}
			+ \underbrace{\norm{f_1}_{p_1}}_{\text{cost.}}
			\cdot \underbrace{\norm{f_{2,n} - f_2}_{p_2}}_{\to 0}
			\to 0
		\end{aligned}
		$$

	\item 
		$C_0(\R^d)$ è chiuso per convergenza uniforme [TODO: da fare per esercizio]
\end{itemize}

\section{Derivata e Convoluzione}

\textbf{Osservazione.}
Osserviamo che la convoluzione si comporta bene con l'operatore di traslazione definito precedentemente, infatti $\tau_h (f_1 \ast f_2) = (\tau_h f_1) \ast f_2$ in quanto
$$
f_1 \ast f_2 (x - h) 
= \int f_1(x - h - y) \cdot f_2(y) \dd y = \int \tau_h f(x - y) \cdot f_2(y) \dd y
= (\tau_h f_1) \ast f_2
$$
quindi ``formalmente'' possiamo calcolare il seguente rapporto incrementale
$$
\frac{\tau_h(f_1 \ast f_2) - f_1 \ast f_2}{h}
= \frac{\tau_h f_1 - f_1}{h} \ast f_2
\implies (f_1 \ast f_2)' = (f_1)' \ast f_2
$$

Vediamo ora di formalizzare questo risultato.

\textbf{Teorema.}
Dati $p_1$ e $p_2$ esponenti coniugati, se

\begin{itemize}
	\item $f_1 \in C^1(\R^d)$, $\nabla f_1 \in L^{p_1}(\R^d)$

	\item $f_2 \in L^{p_2}(\R^d)$ 
\end{itemize}

allora $f_1 \ast f_2 \in C^1$ con $\nabla(f_1 \ast f_2) = (\nabla f_1) \ast f_2$\footnote{Ha senso anche se $\nabla f_1$ è a valori vettoriali. In tal caso  $\frac{\pd}{\pd x}(f_1 \ast f_2) = \left( \frac{\pd f_1}{\pd x_i} \right) \ast f_2 
		\quad \text{per } i = 1, \dots, d $.}. 

\textbf{Dimostrazione.}
\begin{itemize}
	\item $d = 1$:
		Sappiamo che $f_1 \ast f_2$ è continua e $f_1' \ast f_2$ è continua. Vediamo che coincidono usando il teorema fondamentale del calcolo integrale. L'uguaglianza $(f_1 \ast f_2)' = f_1' \ast f_2$ segue da
		$$
		\int_a^b f_1' \ast f_2 \dd x = f_1 \ast f_2(b) - f_1 \ast f_2(a)
		\quad \forall a < b
		$$
		ed in effetti
		$$
		\begin{aligned}
			\int_a^b f_1' \ast f_2 (x) \dd x
			&= \int_a^b \int_{-\infty}^\infty f_1'(x - y) f_2(y) \dd y \dd x \\
			&\overset{\text{($*$)}}{=} \int_{-\infty}^\infty \int_a^b f_1'(x - y) \dd x \cdot f_2(y) \dd y \\
			&= \int_{-\infty}^\infty (f_1(b - y) - f_1(a - y)) \cdot f_2(y) \dd y \\
			&= \int_{-\infty}^\infty f_1(b - y) f_2(y) \dd y - \int_{-\infty}^\infty f_1(a - y) f_2(y) \dd y \\
			&= f_1 \ast f_2(b) - f_1 \ast f_2(a).
		\end{aligned}
		$$
		In particolare in ($*$) stiamo usando Fubini-Tonelli in quanto
		$$
		\int_a^b \int_{-\infty}^\infty |f_1'(x-y)| \cdot |f_2(y)| \dd y
		\leq \int_a^b \norm{f_1'(x - \curry)}_{p_1} \cdot \norm{f_2}_{p_2} \dd x 
		= \norm{f_1'}_{p_1} \cdot \norm{f_2}_{p_2} \cdot (b - a).
		$$

	\item
		per $d > 1$ dato $i = 1, \dots, d$ basta semplicemente considerare le proiezioni infatti
		$$
		\int_a^b \frac{\pd f_1}{\pd x_i} \ast f_2 (x_1, \dots, \overset{\text{($i$)}}{t}, \dots, x_d) \dd t
		= f_1 \ast f_2 (x_1, \dots, \overset{\text{($i$)}}{b}, \dots, x_d) - f_1 \ast f_2 (x_1, \dots, \overset{\text{($i$)}}{a}, \dots, x_d)
		$$
		\qed
\end{itemize}

\textbf{Corollario.}
Data $f_1 \in C_C^\infty(\R^d)$ (da cui segue $\nabla^k \in L^q(\R^d)$ per ogni $k = 0, 1, \dots$ e $1 \leq q < +\infty$) e $f_2 \in L^p(\R^d)$ allora $f_1 \ast f_2 \in C^\infty(\R^d)$ (anzi $\nabla^k(f_1 \ast f_2) \in C_0(\R^d)$ per ogni $k$) e vale la formula nota\footnote{Vista in termini di gradienti la formulazione è più compatta ma non poi così intuitiva: bisognerebbe definire la convoluzione tre una funzione a valori vettoriali ed uno scalare etc... Altrimenti basta scrivere le singole identità usando \textit{derivate parziali e multiindici}.}
$$
\nabla^k (f_1 \ast f_2) = (\nabla^k f_1) \ast f_2.
$$

\textbf{Dimostrazione.}
Dimostriamo il corollario per approssimazione usando il seguente teorema.

\begin{wrapfigure}{r}{200pt}
	\centering
	\vspace{-1.5\baselineskip}
	\inputfigure{sigma-delta-transform}
	\vspace{-2.5\baselineskip}
\end{wrapfigure}

\textbf{Definizione.} 
Per prima cosa data una funzione $g \colon \R^d \to \R$ e $\delta \neq 0$ poniamo
$$
\sigma_\delta g(x) \coloneqq \frac{1}{\delta^d} g\left( \frac{x}{\delta} \right)
$$
e notiamo che questa trasformazione preserva la norma $L^1$. Infatti, il valore $1/\delta^d$ è proprio il modulo del determinante dello Jacobiano del cambio di variabile.

\hypertarget{thm:lez25ott_teodelta}{%
\textbf{Teorema.}}
Data $f \in L^p(\R^d)$ e $g \in L^1(\R^d)$ con $1 \leq p < +\infty$ e posto $\ds m \coloneqq \int_{\R^d} g(x) \dd x$, allora 
$$
f \ast \sigma_\delta g \xrightarrow{\delta \to 0} m f \in L^p(\R^d).
$$

\textbf{Osservazione.}
Se $g_2 \geq 0$ con $\int g \dd x = 1$ (dunque $g$ distribuzione di probabilità) allora $f \ast g$ possiamo pensarla come media pesata di traslate di $f$, dunque facendo $f \ast \sigma_\delta g$ stiamo pesando sempre di più i valori delle traslate vicino a $0$. 

% minuto 1:28:00 definire la funzione e fare disegnino (questa funzione si usa anche
% in altri casi per vedere che una certa proprietà non vale in $L^\infty$
Inoltre per $p = +\infty$ non vale ed il controesempio è sempre il solito. [TODO: scrivere la funzione].

\textbf{Dimostrazione.}
Per ora consideriamo $g$ generica e ripercorriamo una dimostrazione simile a quella fatta per la disuguaglianza di Young
$$
\begin{aligned}
	\norm{f \ast g - m f}_p^p 
	&= \int_{\R^d} {\underbrace{|f \ast g - m f|}_h}^p \dd x \\
	&= \int |f \ast g - m f| \cdot h^{p-1} \dd x \\
	&= \int \left| \int \left( f(x - y) g(y) - f(x) \int g(y) \right) \dd y \right| \cdot h^{p-1}(x) \dd x \\
	&\leq \int \int |f(x - y) - f(x)| \cdot |g(y)| \dd y \cdot h^{p-1}(x) \dd x \\
	&\overset{\text{($*$)}}{=} \int \left(\int |f(x - y) - f(x)| h^{p-1}(x) \dd x \right) |g(y)| \dd y,
\end{aligned}
$$
dove in ($*$) abbiamo usato Fubini-Tonelli. Ora prendiamo $q$ tale che $1/p + 1/q = 1$ allora per H\"older abbiamo
$$
\begin{aligned}
	&\leq \int \norm{f(\curry - y) - f(\curry)}_p \| h^{p-1} \|_q \cdot |g(y)| \dd y \\
	&= \norm{h}_p^{p-1} \int_{\R^d} \norm{\tau_y f - f}_p \cdot |g(y)| \dd y \\
\end{aligned}
$$
dunque abbiamo ricavato che
$$
\norm{f \ast g - m f}_p^p 
\leq \norm{f \ast g - m f}_p^{p-1} \int_{\R^d} \norm{\tau_y f - f}_p \cdot |g(y)| \dd y
$$
ed ora applicando questa stima a $\sigma_\delta g$ invece che a $g$ otteniamo
$$
\norm{f \ast \sigma_\delta g - m f}_p
\leq \int_{\R^d} \norm{\tau_y f - f}_p \cdot |\sigma_\delta g(y)| \dd y
$$
infine ponendo $z = y / \delta$ e $\dd z = 1/\delta^d \dd y$ e sostituendo nell'integrale otteniamo
$$
= \int_{\R^d} \norm{\tau_{\delta z} f - f}_p \cdot |g(z)| \dd z \xrightarrow{\delta \to 0} 0
$$
usando \textit{convergenza dominata} in quest'ultimo passaggio, verifichiamone le ipotesi
\begin{enumerate}
	\item La convergenza puntuale segue in quanto $\norm{\tau_{\delta z} f - f}_p \xrightarrow{\delta \to 0} 0$ per ogni $z$.
	\item Come dominazione prendiamo $2 \norm{f}_p \cdot |g| \in L^1$.
\end{enumerate}
\qed

\textbf{Corollario.}
Sia $g \in C_C^\infty(\R^d)$ con $\ds \int g \dd x = 1$ e $f \in L^p(\R^d)$ e $1 \leq p < +\infty$ allora $\sigma_\delta g \ast f \xrightarrow{\delta \to 0} f$ in $L^p(\R^d)$ e $\sigma_\delta g \ast f \in C^\infty(\R^d)$.



% Spazi di Hilbert
%
% Lezioni del 27-28 Ottobre 2021
%

\chapter{Spazi di Hilbert}

Sia $H$ spazio vettoriale reale con prodotto scalare $\left<\cdot, \cdot \right>$ definito positivo e norma indotta $\norm{\curry}$ definita come $\norm{x} = \sqrt{\left<x,x \right>}$.

Si ricorda l'identità di polarizzazione
%
$$
	\left<x_1,x_2 \right> = \frac{1}{4} \left( \norm{x_1 + x_2}^2 - \norm{x_1 - x_2}^2 \right).
$$
%

\textbf{Nota.} Siccome $\norm{\curry}$ è continua, dalla formula di polarizzazione segue che il prodotto scalare è continuo.

\textbf{Definizione.} $H$ si dice \textbf{spazio di Hilbert} se è completo.

\textbf{Esempi.} 
\begin{itemize}

	\item Dato $(X, \mc{A}, \mu )$, gli spazi $L^2(X), L^2(X, \R^m)$ sono spazi di Hilbert.

	\item Lo spazio $\ds \ell^2 = \left\{ (x_n) \mymid \sum_{n=0}^{\infty} x_n^2 < +\infty  \right\}$ è uno spazio di Hilbert.

\end{itemize}

\textbf{Definizione.} $\mc{F} \subset H$ è un \textbf{sistema ortonormale} se
%
$$
	\norm{e} = 1 \mquad \forall e \in \mc{F}, \qquad  \left<e,e' \right> = 0 \mquad \forall e \neq e' \in \mc{F}.
$$
%


\textbf{Definizione.} $\mc{F}$ si dice \textbf{completo} se $\overline{\spn(\mc{F})} = H$\footnote{Lo span sono combinazioni lineari finite.}. In tal caso $\mc{F}$ si dice \textbf{base di Hilbert}.

\vs

\textbf{Osservazione.} In generale una base di Hilbert $\mc{F} \subset H$ non è anche una base algebrica di $H$.

L'esempio che segue spiega quanto appena detto.

\textbf{Esempio.} In $\ell^2$ una base ortonormale è $\mc{F} = \left\{ e_n \mymid n \in \N \right\}$ con $e_n = (0,\ldots ,0,\overset{(n)}{1},0,\ldots )$. \\
Infatti, il fatto che siano ortonormali è banale; verifichiamo che sia una base. 

Studiamo $\spn(\mc{F}) = \left\{ x = (x_0,x_1,\ldots ) \mymid x_n \quad \text{è definitivamente nullo}  \right\}$: dato $x \in \ell^2$ e $m = \N$, definiamo
%
$$
	P_mx \coloneqq (x_0,x_1,\ldots , x_m,0,\ldots ).
$$
%
Allora $\spn(\mc{F}) \supset P_m x \xrightarrow{m \to +\infty} x$ in $\ell^2$.
Infatti, 
%
$$
	x - P_m x = (0,\ldots ,0, x_{m+1}, x_{m+2},\ldots ).
$$
%
Dunque
%
$$
	\norm{x - P_m x} = \sum_{n = m+1}^{\infty} x_n^2 \xrightarrow{m \to +\infty} 0.
$$
%

\mybox{%
\textbf{Teorema~1.} (della base di Hilbert.) Dato $H$ spazio di Hilbert, $\mc{F}$ sistema ortonormale al più numerabile, ovvero $\mc{F} = \left\{ e_n \mymid n \in \N \right\}$.
Definiamo per ogni $x \in H$, $n \in \N$ l'elemento $x_n = \left<x, e_n \right>$.
Allora
\begin{enumerate}

	\item \label{item:27ott_thm1_1}
	Vale $\ds \sum_{n} x_n^2 \leq \norm{x}^2$ (\textbf{Disuguaglianza di Bessel}).

	\item \label{item:27ott_thm1_2}
	La somma $\ds \sum_n x_n e_n$ converge a qualche $\overline{x} \in H$ e $\overline{x}_n = x_n$  per ogni $n$.

	\item \label{item:27ott_thm1_3}
	Vale $\ds \norm{\overline{x}}^2 = \sum_n x_n^2 \leq \norm{x}^2$.

	\item \label{item:27ott_thm1_4}
	Se $x - \overline{x} \perp \mc{F}$, allora $x - \overline{x} \perp \overline{\spn(\mc{F})}$, ovvero $\overline{x}$ è la proiezione di $x$ su $\overline{\spn(\mc{F})}$.

	\item \label{item:27ott_thm1_5}
	Se $\mc{F}$ è completo, allora $x = \overline{x}$ e in particolare
	%
	$$
		x = \sum_{n=0}^{\infty} x_n e_n, \qquad \norm{x}^2 = \sum_{n=0}^{\infty} x_n^2  \qquad \text{(\textbf{Identità di Parseval})}.
	$$
	%

\end{enumerate}
}

%\textbf{Nota.} Il punto \ref{item:27ott_thm1_2} non segue dal fatto che la serie è assolutamente convergente. Infatti
%%
%$$
%\sum \norm{x_n e_n} = \sum \left| x_n \right|
%$$
%%
%può essere $+\infty$.


Alla dimostrazione del teorema premettiamo il seguente lemma.

\vs

\textbf{Lemma~2.} Siano $H$ e $\mc{F}$ come nel teorema.
Data $(a_n) \in \ell^2$, allora
\begin{enumerate}
	\item La serie $\ds \sum_n a_n e_n$ converge a qualche $\overline{x} \in H$. 

	\item $\overline{x}_n = a_n$ per ogni $n$.

	\item $\ds \norm{\overline{x}}^2 = \sum_n a_n^2$.
\end{enumerate}

\textbf{Dimostrazione.}
\begin{enumerate}
\item Dimostriamo che $\ds y_n = \sum_{n = 1}^{m} a_n e_n$ è di Cauchy in $H$.
Se $m' > m$, vale
%
$$
	y_{m'} - y_m = \sum_{n = m+1}^{m'} a_n e_n
	\Longrightarrow  \norm{y_{m'} - y_m}^2 = \norm{\sum_{n = m+1}^{m'} a_n e_n}^2 
	= \sum_{n = m+1}^{m'} a_n^2 \leq \sum_{n = m+1}^{\infty} a_n^2 < +\infty.
$$
%
Dunque, per ogni $\epsilon$ esiste $m_\epsilon$ tale che $\ds \sum_{m_\epsilon + 1}^{\infty} a_n^2 \leq \epsilon^2 $, per cui 
%
$$
	\norm{y_{m'} - y_m}^2 \leq \sum_{m+1}^{m'} a_n^2 \leq \sum_{m_\epsilon + 1}^{\infty} a_n^2 \leq \epsilon^2 \quad \forall m,m' \geq m_\epsilon.
$$
%

\item Se $m \geq n$, $\left<y_m, e_n \right> = a_n$, dunque, per continuità del prodotto scalare
%
$$
	a_n = \left<y_m, e_n \right> \xrightarrow{m \to \infty} \left<\overline{x},e_n \right> = \ol{x}_n.
$$
%

\item Si ha l'uguaglianza $\ds \norm{y_m}^2 = \sum_{n=1}^{m} a_n^2$, per cui passando al limite per $m \to +\infty$ otteniamo 
\begin{align*}
	& \norm{y_m}^2 \xrightarrow{m \to \infty} \norm{\ol{x}}^2 \\
	& \quad \rotatebox[origin=c]{90}{=} \\
	& \sum_{n=0}^{m} a_n^2 \xrightarrow{m \to \infty} \sum_{n=0}^{\infty} a_n^2 
\end{align*}
\qed

\end{enumerate}


\textbf{Dimostrazione Teorema~1.}

\begin{enumerate}
\item Studiamo la somma $\ds x = \sum_{n=0}^{m} x_n e_n + \overbrace{y}^{\text{resto}}$.

Notiamo che $x$ è somma di vettori ortogonali, infatti $y$ è ortogonale a $\ds \sum_{n=0}^{m} x_n e_n$ :
%
$$
	\left<y,e_i \right> = \left<x - \sum_{n=0}^{m} x_n e_n, e_i \right>
	= \left<x,e_i \right> - \sum_{n=0}^{m} x_n \underbrace{\left<e_n,e_i \right>}_{\delta_{i,n}} = x_i - x_i = 0.
$$
%
Essendo che $x$ è somma di vettori ortogonali abbiamo
%
$$
	\norm{x}^2 = \sum_{n=1}^{m} x_n^2 + \norm{y}^2 \geq \sum_{n=1}^{m} x_n^2.
$$
%
Passando al limite per $m \to +\infty$ otteniamo
%
$$
	\norm{x}^2 \geq \sum_{n=1}^{\infty} x_n^2. 
$$
%

\item Segue dal lemma notando che il punto precedente ci dice che la successione $(x_n)$ è a quadrato sommabile.

\item Segue banalmente dai primi due punti.

\item Notiamo che $\left<x - \ol{x}, e_n \right> = x_n - \ol{x}_n \overset{\text{ii)}}{=} 0$ per ogni $n$. Cioè
%
$$
	x - \ol{x} \perp e_n \Longrightarrow x - \ol{x} \perp \spn(\mc{F})
	\Longrightarrow x - \ol{x} \underset{\substack{\text{continuità} \\ \text{pr. scalare}}}{\perp} \ol{\spn(\mc{F})}
$$
%

\item $x - \overline{x} \perp \overline{\spn(\mc{F})} \underbrace{=}_{\mathclap{\text{$\mc{F}$ è completo}}} H \Longrightarrow x - \overline{x} = 0$, cioè $x = \overline{x}$.
\end{enumerate}
\qed

\textbf{Corollario~3.} Siano $H$ spazio di Hilbert, $\mc{F} = \left\{ e_n \mymid n \in \N \right\}$ base di Hilbert, $x,x' \in H$. Valgono le seguenti.
\begin{enumerate}
\item $x_n = x_n' \mquad \forall n \in \N \longiff x = x'$ ($\Leftarrow$ è ovvia).

\item $\ds \left<x,x' \right> = \sum_{n=0}^{\infty} x_n  x_n'$ (\textbf{Identità di Parseval}).

\item L'applicazione $H \ni x \mapsto (x_n) \in \ell^2$ è un'isometria surgettiva\footnote{In particolare è bigettiva ma l'iniettività è ovvia.}.
\end{enumerate}

\vs 

\textbf{Dimostrazione.}

\begin{enumerate}
	\item Per l'enunciato \ref{item:27ott_thm1_5} se due vettori hanno la stessa rappresentazione rispetto a una base di Hilbert coincidono.

	\item La tesi segue usando l'identità di polarizzazione congiuntamente all'enunciato \ref{item:27ott_thm1_5} del teorema:
	%
	\begin{align*}
		\left<x,x' \right> & = \frac{1}{4} \left( \norm{x + x'}^2 - \norm{x - x'}^2 \right)
		= \frac{1}{4} \Big( \sum_n \overbrace{(x_n + x_n')^2}^{x_n^2 + x_n'^2 + 2x_n x_n'} - \sum_n \underbrace{(x_n - x_n')^2}_{x_n^2 + x_n'^2 - 2 x_n x_n'}  \Big) \\
		& = \frac{1}{4} \left( \cancel{\sum x_n^2} + \cancel{\sum x_n'^2} + 2 \sum x_n x_n' - \cancel{\sum x_n^2} - \cancel{\sum x_n'^2} + 2 \sum x_n x_n' \right).
	\end{align*}

	\item Il fatto che l'applicazione sia un'isometria segue da Parseval; che sia iniettiva dal fatto che $\mc{F}$ è una base di Hilbert e che sia surgettiva dai punti i) e ii) del Lemma~2.

	\qed
\end{enumerate}

\textbf{Osservazioni.}
\begin{itemize}
\item Gli enunciati \ref{item:27ott_thm1_1} e \ref{item:27ott_thm1_5} non richiedono $H$ completo, mentre \ref{item:27ott_thm1_2} non è vero se $H$ non è completo.

\item Se $H$ è uno spazio di Hilbert e $\mc{F}$ sistema ortonormale infinito, allora $\mc{F}$ non è mai una base algebrica\footnote{Per base algebrica s'intende un insieme di vettori di uno spazio vettoriale le cui combinazioni lineari generano tutto lo spazio.}. Dunque, combinazioni lineari finite di $\mc{F}$ non sono mai uguali ad $H$, ovvero $\spn(\mc{F}) \subsetneq H $.

\textbf{Dimostrazione.} Presi $(e_n) \subset \mc{F}$, consideriamo $\ds \ol{x} = \sum_{n=0}^{\infty} 2^{-n}e_n $. Allora  $\ds \ol{x} \in H \setminus \spn(\mc{F})$.
%
% \textbf{Nota.} I coefficienti sono univocamente determinati perché ottenuti tramite prodotto scalare. Notiamo che non si può usare il teorema di algebra lineare sull'unicità della rappresentazione poiché stiamo trattando combinazioni lineari infinite.

\item Siano $H$ uno spazio di Hilbert di dimensione infinita e $\mc{F}$ una base di Hilbert. Allora $\mc{F}$ è numerabile se solo se $H$ è separabile.

\textbf{Dimostrazione.}

\begin{itemize}

\item[$\boxed{\Rightarrow}$] Vale $ H = \overline{\spn(\mc{F})} = \overline{\spn_\Q (\mc{F})} $. Concludiamo notando che $\overline{\spn_\Q (\mc{F})}$ è numerabile se  $\mc{F}$ è numerabile.

\item[$\boxed{\Leftarrow}$] Se $\mc{F}$ non fosse numerabile, siccome $\norm{e - e'} = \sqrt{2}  \quad \forall e,e' \in \mc{F}$, potremmo definire per ogni elemento di $\mc{F}$ una palla di raggio $\sqrt{2}/2$, dunque potremmo definire un insieme di palle disgiunte. Dato un sottoinsieme denso di $H$, per definizione, deve intersecare ogni palla e dunque deve essere più che numerabile, dunque $H$ non sarebbe separabile.

\end{itemize}


\textbf{Esempio.} Lo spazio $H = L^2(X)$, con $X = \R^n$, $\mu $ misura di Lebesgue ha base di Hilbert numerabile.

\newpage

\item Dato $\mc{F}$ sistema ortonormale in $H$, allora $\mc{F}$ è completo se solo se $\mc{F}$ è massimale (nella classe dei sistemi ortonormali rispetto all'inclusione).

\textbf{Dimostrazione.}
\begin{itemize}

\item[$\boxed{\Rightarrow}$] Dato che $\mc{F}$ è completo segue che $\overline{\spn(\mc{F})} = X$, quindi
%
$$
\mc{F}^{\perp} = \left( \spn(\mc{F}) \right)^\perp
\underbrace{=}_{\mathclap{\substack{\text{continuità del} \\ \text{prodotto scalare}}}} \overline{\spn(\mc{F})}^\perp = H^\perp = \{ 0 \}.
$$
%
dunque $\mc{F}$ è massimale.

\item[$\boxed{\Leftarrow}$] Se  $\mc{F}$ non è completo, esiste $x \in H \setminus \spn(\mc{F})$.
Definiamo $\overline{x}$ come nel Teorema 1. Notiamo che $x - \overline{x} \perp \spn(\mc{F})$, dunque $x - \overline{x} \perp \mc{F}$ e $x - \overline{x} \neq \{ 0 \}$, da cui $\ds \mc{F} \; \cup \; \left\{ \frac{x - \overline{x}}{\norm{x - \overline{x}}} \right\}$ è un sistema ortonormale che include strettamente $\mc{F}$. \absurd

\end{itemize}


\textbf{Osservazione.} Nell'implicazione $\boxed{\Rightarrow}$ non abbiamo usato la completezza di $H$.

\item Ogni sistema ortonormale $\mc{F}$ si completa a $\tilde{\mc{F}}$ base di Hilbert di $H$.

\textbf{Dimostrazione.} Sia $X = \left\{ \mc{F} \; \text{sistema ortonormale} H \text{ tale che } \tilde{\mc{F}} \subset \mc{F} \right\}$.
Per Zorn, $X$ contiene un elemento massimale. Denotiamolo con $\tilde{\mc{F}}$. Allora $\tilde{\mc{F}}$ è una base di Hilbert.


\end{itemize}


\mybox{%
\textbf{Teorema~4.} Dato $V$ sottospazio vettoriale chiuso di $H$. Allora
\begin{enumerate}
	\item $H = V + V^\perp$, cioè per ogni $x \in H$ esiste $\overline{x} \in V$ e $\tilde{x} \in V^\perp$ tale che $x = \overline{x} + \tilde{x}$.

	\item Gli elementi $\overline{x}$ e $\tilde{x}$ sono univocamente determinati (e indicati con $x_V$ e $x_V^\perp$).

	\item $\overline{x}$ è caratterizzato come l'elemento di $V$ più vicino a $x$.
\end{enumerate}
}

\textbf{Dimostrazione.}
\begin{enumerate}

\item Dato che $V$ è chiuso, $V$ è completo, cioè $V$ è un sottospazio di $H$, dunque $V$ ammette base ortonormale $\mc{F} = \left\{ e_n \mymid n \in \N \right\}$.
Definiamo $\overline{x} \in \overline{\spn(\mc{F})}$ come nel Teorema~1 e $\tilde{x} \coloneqq x - \overline{x} \perp \overline{\spn(\mc{F})} = V$ (per \ref{item:27ott_thm1_4}) dunque $\tilde{x} \in V^\perp$.

\item Se $x = \overline{x} + \tilde{x} = \overline{x}' + \tilde{x}'$, dove $\overline{x}, \overline{x}' \in V$ e $\tilde{x}, \tilde{x}' \in V^\perp$, allora
%
$$
	\overline{x} - \overline{x}' = \tilde{x}' - \tilde{x} \underbrace{\Longrightarrow}_{V \cap V^\perp = \{0 \} }
	\overline{x} - \overline{x}' = \tilde{x}' - \tilde{x} = 0.
$$
%

\item Per ogni $y \in V$ sia $f(y) = \norm{x - y}^2$. Mostriamo che $\overline{x}$ è l'unico minimo di $f$.
%
$$
f(y) = \norm{x - y}^2 = \lVert{\overbrace{x - \overline{x}}^{\in V^\perp} + \overbrace{\overline{x} - y}^{\in V}}\rVert^2 = \norm{x - \overline{x}}^2 + \norm{\overline{x} - y}^2
= f(\overline{x}) + \norm{\overline{x} - y}^2 \geq f(\overline{x}).
$$
%
\qed
\end{enumerate}

\vs

\textbf{Osservazione.}
Serve $V$ chiuso. Se per esempio $V$ è denso in $H$ ma $V \neq H$, allora
%
$$
\overline{V^\perp} = \overline{V}^\perp = H^\perp = \{0\} \Longrightarrow V \subseteq V + V^\perp \subseteq V + \ol{V^\perp} = V \subsetneq H.
$$
%
Un esempio di tale $V$ è $\spn(\mc{F})$ con $\mc{F}$ base di $H$ ($H$ di dimensione infinita).

\newpage

\hypertarget{thm:lez2728ott-teo3}{%
\textbf{Teorema~5} (di rappresentazione di Riesz.)} Sia $H$ spazio di Hilbert. Dato $\Lambda \colon  H \to \R$ lineare e continuo, esiste $x_0 \in H$ tale che 
\begin{equation} \tag{$\ast$}
	\Lambda(x) = \left<x,x_0 \right> \quad \text{per ogni} \; x \in H.
\end{equation}

\textbf{Lemma~6.} Dato $\Lambda \colon X \to \R$ lineare, $\left( \ker \Lambda \right)^\perp$ ha dimensione 0 o 1.

\textbf{Dimostrazione} Se per assurdo $\dim (\ker \Lambda)^\perp \geq 2$, allora $(\ker \Lambda)^\perp$ conterrebbe un sottospazio $W$ di dimensione $2$.
Dunque, $\dim \left( \ker \restr{\Lambda}{W} \right) = \{1,2\}$, essendo che $\dim \R = 1$. Ma questo non è possibile, in quanto abbiamo definito $W \subset \ker^\perp$. 
\qed

\textbf{Dimostrazione Teorema~5.}
Sia $V \coloneqq \ker \Lambda$. Dato che $\Lambda$ è continuo segue che $V$ è chiuso.
Se $V = H \Longrightarrow \Lambda \cong 0$ e prendiamo $x_0 = 0$.

Se $V \neq H$, allora $V^\perp \neq \{0\}$ e definiamo $x_1 \in V^\perp$ con $\norm{x_1} = 1$.
Poniamo $x_0 \coloneqq  c x_1$ con $c \coloneqq  \Lambda x_1$ e $\tilde{\Lambda}(x) \coloneqq \left<x,x_0 \right>$. Abbiamo che
\begin{itemize}

	\item $x \in V \Longrightarrow x \perp x_1 \Longrightarrow x \perp x_0 \Longrightarrow \tilde{\Lambda} = 0 = \Lambda(x)$. Quindi $\tilde{\Lambda} = \Lambda$ su $V$.

	\item $\tilde{\Lambda}(x_1) = \left<x_1,x_0 \right> = \left<x_1, c x_1 \right> = c \norm{x_1}^2 = c = \Lambda(x_1)$. Quindi $\tilde{\Lambda} = \Lambda$ su $\spn(x_1) = V^\perp$ che ha dimensione 0 o 1 per il Lemma~6.

	\item $\tilde{\Lambda} = \Lambda$ su $V + V^\perp = H$.

\qed
\end{itemize}

\textbf{Osservazione.}
Esistono funzioni $\Lambda \colon  H \to \R$ lineari ma non continue se $H$ ha dimensione infinita.

\textbf{Dimostrazione.} Prendo $\Lambda \colon H \to \R$ lineare definito come
%
$$
	\begin{cases}
	\Lambda(e_n) = n \quad \forall n \\
	\Lambda(e) = \text{qualsiasi } e \in \mc{G} \setminus \left\{ e_n \right\}.
	\end{cases} 
$$
%
Allora
%
$$
	+\infty = \sup_n \left| \Lambda(e_n) \right| \leq \sup_{\norm{x} \leq 1} \left| \Lambda(x) \right| 
$$
%
da cui segue che $\Lambda$ non è continuo.


\section{Spazi di Hilbert complessi}

\textbf{Definizione.}
Sia $H$ una spazio vettoriale su $\C$ con prodotto hermitiano $\langle\curry; \curry\rangle$, ovvero tale che
\begin{itemize}
	\item $\langle \curry; \curry \rangle$ è lineare nella prima variabile
	\item $\langle x; x' \rangle = \overline{\langle x', x \rangle}$ ovvero è antilineare nella seconda variabile.
	\item $\langle x; x \rangle \geq 0$ per ogni $x$ e vale $0$ se e solo se $x = 0$.
\end{itemize}

Analogamente si pone $\norm{x} \coloneqq \sqrt{\langle x; x \rangle}$. C'è un'identità di polarizzazione ma è leggermente diversa dalla versione reale.

\textbf{Definizione.} $H$ si dice di Hilbert se è \textbf{completo}.

\textbf{Esempio.}
Su $L^2(X; \C)$ si mette il prodotto scalare dato da
$$
\langle u; v \rangle \coloneqq \int_X u \cdot \overline v \dd \mu.
$$

\textbf{Teorema.} (della base di Hilbert per spazi complessi)
Dato $\mathcal F = \{ e_n \}$ sistema ortonormale in $H$ e $x \in H$ allora per ogni $n$ si pone\footnote{E non $\langle e_n; x \rangle$!}
$$
x_n = \langle x; e_n \rangle
$$
Vale anche l'identità di Parseval $\norm{x^2} = \sum |x_n|^2$ dove $|\curry|$ è il modulo di un numero complesso, in particolare nella versione con prodotto scalare diventa
$$
\langle x, x' \rangle = \sum_n x_n \overline{x_n'}.
$$


\section{Esempi di basi Hilbertiane}

\subsection{Polinomi}

La base data da $ \{ 1, x, x^2, \dots, x^n, \dots \} $ opportunamente ortonormalizzata è una base di $L^2[0, 1]$ (anche di $L^2(\R)$) per il teorema di Stone-Weirstrass.


\subsection{Base di Haar}

\begin{wrapfigure}{r}{225pt}
	\centering
	\vspace{-4\baselineskip}
	\inputfigure{base-di-haar-1} 
	\vspace{-4.5\baselineskip}
\end{wrapfigure}

Vediamo la base di Haar data da due indici $n, k$ dove $n$ indica l'ampiezza delle ``onde'' (anche dette \textit{wavelet}) e $k$ il posizionamento dell'onda. Sia $n \in \N$ e $k = 1, \dots, 2^n$ e poniamo
$$
g^{0,0} \coloneqq \One_{[0,1]}
\qquad
g^{n,k} \coloneqq 2^\frac{n-1}{2} \left( \One_{\left[ \frac{2k - 2}{2^n}, \frac{2k - 1}{2^n} \right]} - \One_{\left[ \frac{2k - 1}{2^n}, \frac{2k}{2^n} \right]} \right)
$$

Inoltre $\| g^{n, k} \|_{L^2[0, 1]} = 1$ ed anche $\| g^{0,0} \|_{L^2[0, 1]} = 1$. Vedremo che $\{ g^{n,k} \mid n \geq 1, k = 1, \dots, 2^n \} \cup \{ g^{0,0} \}$ formano un sistema ortonormale.

\begin{itemize}
	\item $\langle g^{n,k}, g^{0,0} \rangle = 0$: È ovvio in quanto le $g^{n,k}$ hanno media nulla.

	\item $\langle g^{n,k}, g^{n',k'} \rangle = 0$: Se $n = n'$ i supporti sono sempre disgiunti altrimenti $n \neq n'$, se supponiamo $n < n'$ allora i supporti o sono disgiunti e si conclude come prima o il supporto di $g^{n',k'}$ è contenuto in quello di $g^{n,k}$. In tal caso però $g^{n,k}$ è costante su $g^{n',k'}$ e dunque l'integrale è sempre nullo.
\end{itemize}

Inoltre è anche una base hilbertiana, per combinazioni algebriche si ottengono tutti gli intervalli della forma
$$
I_k \coloneqq \left[ \frac{k-1}{2^n}, \frac{k}{2^n} \right]
\qquad
\rightsquigarrow
\qquad
\One_{I_k}
$$
ad esempio normalizzando $g^{n,k} + 2^\frac{n-1}{2} g^{0, 0}$ otteniamo uno degli intervalli di sopra di lunghezza $1 / 2^{n+1}$.

Vedremo che possiamo estendere la base di Haar a tutto $\R$ però è più difficile... [TODO: Ehm aggiungere la parte dopo quando verrà fatta]


% Esercitazione
%
% Lezione del 3 Novembre 2021
%

\section{Esercitazione del 3 Novembre 2021}

\subsection{Basi Hilbertiane e proiezioni}

\textbf{Esercizio.}
Sia $H = L^2(-1, 1)$ e sia $V = \spn\{ 1, x, x^2 \}$. Verificare che $V$ è un sottospazio chiuso e calcolare la proiezione di $\sin x$ su $V$. \\
\textit{Notazione.} Indichiamo con $\norm{\curry}$ la norma $\norm{\curry}_{L^2(-1, 1)}$ e con $\langle \curry, \curry \rangle$ il prodotto scalare su $L^2$.

\textbf{Soluzione.}
Vediamo come risolvere questo esercizio in tre modi diversi.

\begin{enumerate}
% \textbf{Osservazione.}
% Su $L^2(\R)$ e su $L^2(-1, 1)$ esistono di sicuro basi Hilbertiane di cardinalità numerabile in quanto abbiamo già visto che sono spazi separabili.

\item Dato $H$ spazio di Hilbert separabile, e dato un sottospazio $V \subset H$, vediamo come trovare la proiezione di un elemento $x \in H$ su $V$. Procediamo come segue.
\begin{itemize}
	\item Controlliamo che $V$ sia chiuso.

	\item Calcoliamo una base hilbertiana di $V$ che indichiamo con $\{ e_1, \dots, e_n, \dots \}$. 
\end{itemize}
Il tal caso, la proiezione di un elemento $x \in H$  su $V$ è data da
$$
	p_V(x) = \sum_n \langle x, e_n \rangle e_n.
$$

\textbf{Esercizio.} [TO DO: per casa] Ogni sottospazio di dimensione finita  di uno spazio di Hilbert o di $L^p$ è chiuso (e in particolare ha parte interna vuota).

Abbiamo una base di $V$ data da $\{1, x, x^2\}$ (è una base in quanto sono linearmente indipendenti: si può verificare mostrando che $\forall x \in [-1, 1] \; \lambda_1 + \lambda_2 x + \lambda_3 x^2 = 0 \implies \lambda_1 = \lambda_2 = \lambda_3 = 0$ usando la teoria sulle equazioni di II grado oppure si può derivare e man mano ottenere più informazioni su $\lambda_3, \lambda_2, \lambda_1$).

Volendo usare la base scritta sopra per calcolare la proiezione di $\sin x$ su $V$, dovremmo prima applicare Gram-Schmidt alla base $\{1,x,x^2\}$ per determinare una base Hilbertiana:
$$
\begin{aligned}
	e_1 &= \frac{1}{\norm{1}} = \frac{1}{\sqrt{2}} \\
	e_2 &= \frac{x - \langle x, \frac{1}{\sqrt 2} \rangle \cdot 1}{\norm{x - \langle x, \frac{1}{\sqrt 2} \rangle \cdot 1}} \\
	e_3 &= \frac{x^2 - \langle x^2, e_1 \rangle \cdot e_1 - \langle x^2, e_2 \rangle \cdot e_2}{\norm{x^2 - \langle x^2, e_1 \rangle \cdot e_1 - \langle x^2, e_2 \rangle \cdot e_2}} \\
\end{aligned}
$$
e successivamente calcolare $p_V(\sin x)$ con la formula scritta sopra.

\item Alternativamente, possiamo direttamente cercare la proiezione di $\sin x$ su $V$. Determiniamo $a, b, c$ tali che $a + b x + c x^2$ sia $p_V(x) = \sin x$ allora posto $f(x) \coloneqq \sin x - a - b x - c x^2$ abbiamo $f(x) \in V^\perp \iff $ si verificano le seguenti condizioni
$$
\langle f(x), 1 \rangle = 0
\qquad
\langle f(x), x \rangle = 0
\qquad
\langle f(x), x^2 \rangle = 0
$$
Ad esempio da $\langle f(x), 1 \rangle = 0$ otteniamo
$$
0 = \int_{-1}^1 (\sin x - a - b x - c x^2) \cdot 1 \dd x 
= \underbrace{\int_{-1}^1 \sin x \dd x}_{=0} - 2a - b \underbrace{\int_{-1}^1 x \dd x}_{=0} - c \int_{-1}^1 x^2
\implies 0 = -2a - \frac{2}{3}c
$$
ed analogamente si procede con $x$ e $x^2$... [TODO: Da finire]

\item Un altro modo è considerare la funzione $g(a, b, c) \coloneqq \norm{\sin x - a - bx - cx^2}_{L^2(-1, 1)}$ che è continua, coerciva, etc. e imponendo $\nabla_{a,b,c,} g = 0$ si minimizza e si ottengono $\bar a, \bar b, \bar c$ che verificano $p_V(\sin x)$.

\end{enumerate}

% \textbf{Osservazione.}
% Su $L^2(\R)$ e su $L^2(-1, 1)$ esistono di sicuro basi Hilbertiane di cardinalità numerabile in quanto abbiamo già visto che sono spazi separabili.

% Dato $H$ spazio di Hilbert separabile, e dato un sottospazio $V \subset H$, vediamo come trovare la proiezione di un elemento $x \in H$ su $V$.

% Procediamo come segue.
% \begin{itemize}
% 	\item Controlliamo che $V$ sia chiuso.

% 	\item Calcoliamo una ``base hilbertiana di $V$'' che indichiamo con $\{ e_1, \dots, e_n, \dots \}$. 
% \end{itemize}
% Il tal caso, la proiezione di un elemento $x \in H$  su $V$ è data da
% $$
% 	p_V(x) = \sum_n \langle x, e_n \rangle e_n.
% $$

% \textbf{Esercizio.} [TO DO: per casa] Ogni sottospazio di dimensione finita  di uno spazio di Hilbert o di $L^p$ è chiuso (e in particolare ha parte interna vuota).

% \textbf{Soluzione.}
% Indichiamo con $\norm{\curry}$ la norma $\norm{\curry}_{L^2(-1, 1)}$ e con $\langle \curry, \curry \rangle$ il prodotto scalare su $L^2$.

% Abbiamo una base di $V$ data da $\{1, x, x^2\}$ in quanto sono linearmente indipendenti (si può verficare mostrando che $\forall x \in [-1, 1] \; \lambda_1 + \lambda_2 x + \lambda_3 x^2 = 0 \implies \lambda_1 = \lambda_2 = \lambda_3 = 0$ usando la teoria sulle equazioni di II grado oppure si può derivare e man mano ottenere più informazioni su $\lambda_3, \lambda_2, \lambda_1$).

% Volendo usare la base scritta sopra per calcolare la proiezione di $\sin x$ su $V$, dovremmo prima applicare Gram-Schmidt.
% $$
% \begin{aligned}
% 	e_1 &= \frac{1}{\norm{1}} = \frac{1}{\sqrt{2}} \\
% 	e_2 &= \frac{x - \langle x, \frac{1}{\sqrt 2} \rangle \cdot 1}{\norm{x - \langle x, \frac{1}{\sqrt 2} \rangle \cdot 1}} \\
% 	e_3 &= \frac{x^2 - \langle x^2, e_1 \rangle \cdot e_1 - \langle x^2, e_2 \rangle \cdot e_2}{\norm{x^2 - \langle x^2, e_1 \rangle \cdot e_1 - \langle x^2, e_2 \rangle \cdot e_2}} \\
% \end{aligned}
% $$

% Alternativamente, possiamo direttamente cercare la proiezione di $\sin x$ su $V$. Determiniamo $a, b, c$ tali che $a + b x + c x^2$ sia $p_V(x) = \sin x$ allora posto $f(x) \coloneqq \sin x - a - b x - c x^2$ abbiamo $f(x) \in V^\perp \iff $ si verificano le seguenti condizioni
% $$
% \langle f(x), 1 \rangle = 0
% \qquad
% \langle f(x), x \rangle = 0
% \qquad
% \langle f(x), x^2 \rangle = 0
% $$
% Ad esempio da $\langle f(x), 1 \rangle = 0$ otteniamo
% $$
% 0 = \int_{-1}^1 (\sin x - a - b x - c x^2) \cdot 1 \dd x 
% = \underbrace{\int_{-1}^1 \sin x \dd x}_{=0} - 2a - b \underbrace{\int_{-1}^1 x \dd x}_{=0} - c \int_{-1}^1 x^2
% \implies 0 = -2a - \frac{2}{3}c
% $$
% ed analogamente si procede con $x$ e $x^2$... [TODO: Magari finire questo esercizio veramente]
%
% Un altro modo è considerare la funzione $g(a, b, c) \coloneqq \norm{\sin x - a - bx - cx^2}_{L^2(-1, 1)}$ che è continua, coerciva, etc. e imponendo $\nabla_{a,b,c,} g = 0$ si minimizza e si ottengono $\bar a, \bar b, \bar c$ che verficano $p_V(\sin x)$.


\textbf{Esercizio.}
Sia $X = \{ u \in L^2(\R) \mid \int_0^2 u \dd x = 0 \}$, dire se è un sottospazio chiuso, calcolare $X^\perp$ per una generica $u \in L^2(\R)$ e determinare le proiezioni $p_X(u)$ e $p_{X^\perp}(u)$.

\textbf{Soluzione.}
La mappa $T$ lineare data da
$$
u \mapsto \int_0^2 u \dd x
$$
è ben definita, lineare e continua, allora $X$ è proprio $T^{-1}(0)$ dunque è un sottospazio chiuso. 

\textbf{Osservazione.} Notiamo che
$$
T(u) 
= \int_\R u(x) \cdot \One_{[0, 2]}(x) \dd x
= \langle u, g \rangle_{L^2(\R)},
\qquad g = \One_{[0, 2]}(x) \in L^2(\R).
$$

% E dunque $X = \{ u \in L^2 \mid \langle u, g \rangle = 0 \}$.

Calcoliamo ora $X^\perp$ e le proiezioni $p_X$, $p_{X^\perp}$. 
Abbiamo che $X = \{ u \in L^2(\R) \mid \langle u, g \rangle = 0 \}$ dove $g = \One_{[0,2]}(x)$, dunque $X^\perp = \spn(g)$.
Notiamo\footnote{Con $g / \norm{g^\perp}_{L^2}$ indichiamo una base normalizzata di $g^\perp$.} che
$$
L^2(\R) = \spn_\R\left\{\frac{g}{\norm{g}_{L^2}}\right\} \oplus \left\{ \frac{g^\perp}{\norm{g^\perp}_{L^2}} \right\}.
$$
Calcoliamo $p_X(u)$ come segue
%
$$
p_X(u) = u - \left<u, \frac{g}{\norm{g}_{L^2}} \right> \cdot \frac{g}{\norm{g}_{L^2}},
$$
%
dove
$$
\norm{g}_{L^2} = \left( \int_\R \One_{[0, 2]}(x)^2 \dd x \right)^{1/2} = \sqrt 2
$$
e dunque
$$
p_X(u) = u - \left( \int_0^2 u \dd x \right) \cdot \frac{\One_{[0,2]}}{2}
$$

Un controllo veloce per verificare di aver fatto i conti corretti è quello di vedere che $p_X(u) \in X$, dunque di verificare che $\int_0^2 p_X(u) \dd x = 0$.

Per calcolare $p_{X^\perp}(u)$ usiamo la seguente.

\textbf{Osservazione.} Vale $u = p_X(u) + p_{X^\perp}(u)$.

In conclusione,
%
$$
p_{X^\perp}(u) = u - p_X(u) = \left( \int_0^2 u \dd x \right) \cdot \frac{\One_{[0,2]}}{2}.
$$
%


\textbf{Esercizio.} [TO DO: per casa.]
Sia $V = \{ \underline x = (x_n)_{n \in \N} \in \ell^2 \mid x_1 + x_3 + x_5 = 0 \}$. 
Dire se $V$ è chiuso in $\ell^2$ e calcolare $p_V$ e $p_{V^\perp}$.



\subsection{Approssimazioni per convoluzione}

Abbiamo visto che data $g \in L^1(\R^d)$ con $\int g \dd x = 1$ allora per ogni $f \in L^p(\R^d)$ abbiamo $f_\delta \coloneqq f \ast \sigma_\delta g \xrightarrow{\delta \to 0} f$ in $L^p(\R^d)$ per $p \neq \infty$.

\textbf{Esercizio.}
Dire se esiste $v \in L^1(\R)$ tale che sia elemento neutro della convoluzione, ovvero
$$
\forall f \in L^1(\R) \qquad f \ast v = f.
$$

\textbf{Soluzione.} Una tale $v$ non esiste, per vederlo scegliamo opportunamente $\bar f$ e usiamo l'equazione. 
Scelgo $g \in C_C(\R)$ e defiamo $\sigma_\delta g = 1 / \delta g(1/\delta)$. Abbiamo che $\sigma_\delta g \ast v = \sigma_\delta g$ per ogni $\delta$.
Per il \hyperlink{thm:lez25ott_teodelta}{teorema} abbiamo che $\sigma_\delta g \ast v = \sigma_\delta g \xrightarrow{\delta \to 0} v$ in $L^1(\R)$, ma $\sigma_\delta g \xrightarrow{\delta \to 0} 0$ quasi ovunque in $L^1(\R)$. Allora $v = 0$ q.o. in $L^1(\R)$, dunque non può valere $f \ast v = f$ per ogni $f \in L^1(\R)$.

\textbf{Esercizio.} [TO DO: per casa.] 
Sia $f$ misurabile su $\R^d$ tale che $\int_E f \dd x = 0$ per ogni $E$ misurabile di $\R^d$. Dimostrare che $f = 0$ q.o. su $\R^d$.

\textit{Suggerimento.} Considerare l'integrale sull'insieme $\{ x \in \R^d \mid f(x) = 0 \} \cup \{ x \in \R^d \mid f(x) \neq 0 \}$ e verificare che, se denotiamo $A =  \{ x \in \R^d \mid f(x) \neq 0 \}$, allora $|A| = 0$.

\textbf{Esercizio.} [TO DO: per casa.]
Sia $f$ Lebesgue-misurabile su $\R^d$ tale che $\forall B$ palla su $\R^d$
$$
\int_B f \dd x = 0
$$
Dimostrare che che $f = 0$ quasi ovunque su $\R^d$.

% \textit{Suggerimenti.} Usare la convoluzione con opportuni nuclei; notare che $\ds \int_B f = 0 \longiff f \ast \One_B = 0$ per ogni palla $B$.

\textbf{Soluzione.} Considero $g = \One_{[0,1]} \in L^1(\R^d)$. Per quanto fatto in classe so che
$$
	\sigma_\delta g \ast f \xlongrightarrow{\delta \to 0} m f
$$
dunque se $\sigma_\delta g \ast f = 0$ per ogni $\delta$ ho finito.
$$
	\sigma_\delta g \ast f(y) = \int_{\R^d} \sigma_\delta \One_{[0,1]}(y-x) \cdot f(x) \dd x = \frac{1}{\delta^d} \int_{B(x,\delta)} f(x) \dd x = 0
$$



\textbf{Esercizio.}
Sia $p \geq 1$ allora $\{ u \in L^p(\R) \mid \int u \dd x = 0 \} \subseteq L^p(\R)$ è denso in $L^p(\R)$?



% Spazi di Hilbert complessi. Serie di Fourier
%
% Lezione del 4 Novembre 2021
%


\chapter{Serie di Fourier}

Lo scopo della serie di Fourier (complessa) è di rappresentare una funzione $f \colon [-\pi, \pi] \to \C$ (o più in generale una funzione $f \colon \R \to \C$ $2\pi$-periodica) come
$$
f(x) = \sum_{n=-\infty}^\infty c_n e^{i n x}
$$
In particolare, chiamiamo i coefficienti $c_n$ \textbf{coefficienti di Fourier} di $f(x)$ e tutta l'espressione a destra \textbf{serie di Fourier} di $f(x)$.

\textbf{Motivazione.}
La rappresentazione in serie di Fourier serve ad esempio a risolvere certe equazioni alle derivate parziali ed è anche utilizzata per la ``compressione dati''.

\textbf{Problemi.}
\begin{itemize}
	\item Come si trovano (se esistono) i coefficienti di Fourier?
	\item Ed in che senso la serie converge?
\end{itemize}

\textbf{Osservazione.}
La serie appena vista è indicizzata da $-\infty$ a $+\infty$, più avanti vedremo che la definizione esatta non sarà importante ma per ora usiamo la definizione
$$
\sum_{n=-\infty}^\infty a_n \coloneqq \lim_{N \to +\infty} \sum_{n = -N}^N a_n
$$
ed ogni tanto scriveremo anche $\ds \sum_{n \in \Z} a_n$ per brevità.

\mybox{%
\textbf{Teorema 1.}
L'insieme $\ds\mathcal F = \left\{ e_n(x) \coloneqq \frac{e^{inx}}{\sqrt{2\pi}} \mquad n \in \Z \right\}$ è una base ortonormale di $L^2([-\pi, \pi]; \C)$.
}

Da cui segue che
$$
\begin{aligned}
	f(x) 
	&= \sum_{n \in \Z} \langle f; e_n \rangle \cdot e_n
	= \sum_{n \in \Z} \left( \int_{-\pi}^\pi f(t) \overline{\frac{e^{int}}{\sqrt{2\pi}}} \dd t \right) \frac{e^{inx}}{\sqrt{2\pi}} \\
	&= \sum_{n \in \Z} \, \underbrace{\frac{1}{2\pi} \left( \int_{-\pi}^\pi f(t) e^{-int} \dd t \right)}_{c_n} \, e^{inx}
\end{aligned}
$$

\textbf{Definizione.}
Data $f \in L^2([-\pi, \pi]; \C)$ i coefficienti di Fourier di $f$ sono
$$
	c_n = c_n(f) \coloneqq \frac{1}{2\pi} \int_{-\pi}^\pi f(x) e^{-inx} \dd x.
$$
Notiamo in particolare che è anche ben definito anche per $f \in L^1$ (anche se per ora non ci dice molto in quanto $L^1$ non è uno spazio di Hilbert).

\mybox{%
\textbf{Corollario.}
Per ogni $f \in L^2([-\pi, \pi]; \C)$ abbiamo
\begin{enumerate}
	\item \label{item:4nov2021_cor1_1} 
		La serie $\ds \sum_{n \in \Z} c_n e^{inx}$ converge a $f$ in $L^2$.

	\item Vale l'identità di Parseval
		$$
		\norm{f}_2^2 = 2 \pi \sum_{n \in \Z} |c_n|^2
		\qquad
		\qquad
		\langle f, g \rangle = 2\pi \sum_{n \in \Z} c_n(f) \overline{c_n(g)}
		$$.
\end{enumerate}
}

\textbf{Osservazione.}
Usando la \ref{item:4nov2021_cor1_1} ed il fatto che la convergenza in $L^2$ implica la convergenza quasi ovunque a meno di sottosuccessioni otteniamo che $\forall f \; \exists N_n \uparrow \infty$ tale che
$$
	\sum_{n = -N_k}^{N_k} c_n e^{inx} \xrightarrow{\;\;k\;\;} f(x) \qquad \foralmostall x \in [-\pi, \pi].
$$

In particolare nel 1966 Carleson ha dimostrato che in realtà vale proprio
$$
	\sum_{n = -N}^N c_n e^{inx} \xrightarrow{\;\;N\;\;} f(x) \qquad \foralmostall x.
$$

Prima di dimostrare il Teorema~1 riportiamo il teorema di Stone-Weierstrass.

\mybox{%
\textbf{Teorema} (di Stone-Weierstrass.)
Sia $K$ uno spazio compatto e $T_2$ (essenzialmente è uno spazio metrico compatto) e siano
\begin{itemize}

	\item $C(K)$ l'insieme delle funzioni continue reali su $K$,

	\item $C(K; \C)$ le funzioni continue complesse su $K$

\end{itemize} 
entrambe dotate della norma del $\sup$.

Dato $\mathcal A \subset C(K)$ diciamo che è una \textbf{sottoalgebra} se è uno spazio vettoriale e chiuso rispetto al prodotto e diciamo che \textbf{separa i punti} se $\forall x_1, x_2 \in K$ con $x_1 \neq x_2$ allora $\exists f \in \mathcal A$ tale che $f(x_1) \neq f(x_2)$.
\begin{itemize}
	\item \textit{Caso reale:}
	se $\mathcal A$ è una sottoalgebra di $C(K)$ che separa i punti e contiene le costanti allora $\overline{\mathcal A} = C(K)$.
	
	\item \textit{Caso complesso:}
	se $\mathcal A$ è una sottoalgebra di $C(K; \C)$ che separa i punti, contiene le costanti ed è \textit{chiusa per coniugio} allora $\overline{\mathcal A} = C(K; \C)$.
	
\end{itemize}
}

\textbf{Osservazioni.}
\begin{itemize}
	\item Se $K = [0, 1]$, $\mathcal A = \text{``polinomi reali''} \implies \overline{\mathcal A} = C(K; \C)$.
	
	\item L'ipotesi di separare i punti è necessaria, se ad esempio $\exists x_1, x_2$ tali che $x_1 \neq x_2$ e per ogni $f$ abbiamo $f(x_1) = f(x_2)$ allora varrà analogamente anche per ogni funzione nella chiusura ma le funzioni continue separano i punti\footnote{Basta prendere una funzione lineare.}.

	\item È anche necessario che $\mathcal A \supset \text{``costanti''}$, ad esempio dato $x_0 \in K$ ed $\mathcal A \coloneqq \{ f \in C(K) \mid f(x_0) = 0 \}$ abbiamo che $\overline{\mathcal A} = \mathcal A \subsetneq C(K)$.

	\item Anche la chiusura per coniugio è necessaria, infatti ad esempio preso $K = \{ z \in \C \mid |z| \leq 1 \}$, $\mathcal A = \text{``polinomi complessi''}$, $\mathcal A$ separa i punti e contiene le costanti però $\overline{\mathcal A}$ sono solo le funzioni olomorfe su $K$.
\end{itemize}

Vorremmo applicare questo teorema alle funzioni $2\pi$-periodiche ristrette a $[-\pi, \pi]$ che però non verificano la separazione dei punti in quanto per la periodicità $f(-\pi) = f(\pi)$. Nel seguente corollario vediamo come possiamo estendere leggermente il teorema passando ai quozienti topologici.

\textbf{Corollario.}
Sia $\mathcal A$ una sottoalgebra di $C(K)$ (o analogamente per $C(K; \C)$) che contiene le costanti (e nel caso complesso anche chiusa per coniugio). Definiamo la relazione di equivalenza $x_1 \sim x_2$ se $f(x_1) = f(x_2)$ per ogni $f \in \mathcal A$. Allora,
$$
	\overline{\mathcal A} = \{ f \in C(K) \mid f(x_1) = f(x_2) \text{ se } x_1 \sim x_2 \}.
$$
\begin{minipage}{\textwidth - 2.5em}
\parskip 1ex
\setlength{\parindent}{0pt}

\begin{wrapfigure}{r}{100pt}
	\centering
	\vspace{-1.5\baselineskip}
	\begin{tikzcd}
		K \ar[r, "g"] \ar[d, "\pi"] & \C \\
		\sfrac{K}{\sim} \ar[ru, dashed, swap, "\wtilde g"]
	\end{tikzcd}
	\vspace{-1.5\baselineskip}
\end{wrapfigure}

\textbf{Dimostrazione Corollario.}
È chiaro che $\mc{A} \subset X \coloneqq \{ f \in C(K) \mid f(x_1) = f(x_2) \text{ se } x_1 \sim x_2 \}$. Data $g \in X$, definiamo $\wtilde{g} \colon \sfrac{K}{\sim} \to \C$ in modo che $g = \wtilde{g} \circ \pi$.
Osserviamo che $\sfrac{K}{\sim}$ è compatto e $T_2$ e che $\wtilde{\mc{A}} = \{\wtilde{f} \colon  f \in \mc{A}\}$ soddisfa le ipotesi del teorema di Stone-Weierstrass, quindi $\ol{\wtilde{\mc{A}}} = C\left( \sfrac{K}{\sim} ;\C \right)$, quindi per ogni $g \in X$ esiste una successione $\wtilde{g}_n \in \wtilde{\mc{A}}$ tale che $\wtilde{g}_n \to \wtilde{g}$ uniformemente e quindi $g_n \to g$ uniformemente.
% Vogliamo applicare il teorema di Stone-Weierstrass a $\sfrac{k}{\sim}$. Definiamo $X \coloneqq \{ f \in C(K) \mid f(x_1) = f(x_2) \text{ se } x_1 \sim x_2 \}$; è chiaro che $\overline{\mathcal A} \subset X$. Vediamo che $X \subset \overline{\mathcal A}$. 

% Data $g \in X$ troviamo $g_n \in \mathcal A$ tale che $g_n \to g$ uniformemente allora $\exists \wtilde g \colon \sfrac{K}{\sim} \to \C$ tale che $g = \wtilde g \compose \pi$, consideriamo $\mathcal A = \{ \wtilde f \mid f \in \mathcal A \}$ che è una sottoalgebra di $C(\overline{\sfrac{K}{\sim}}; \C)$ che separa i punti, etc. 
\qed
\end{minipage}

\vss

\textbf{Dimostrazione Teorema~1.}
Vogliamo vedere che
\begin{enumerate}
	\item $\mathcal F$ è un sistema ortonormale.

		\textit{Dimostrazione.}
		Basta calcolare $\langle e_n; e_m \rangle$ per ogni $n, m \in \Z$
		$$
		\langle e_n; e_m \rangle
		= \int_{-\pi}^\pi \frac{e^{inx}}{\sqrt{2\pi}} \cdot \overline{\frac{e^{inx}}{\sqrt{2\pi}}} \dd x
		=
		\begin{cases}
			\ds \frac{1}{2\pi} \int_{-\pi}^\pi 1 \dd x = 1 & \text{se } n = m \\[15pt]
			\ds \frac{1}{2\pi} \left[ \frac{e^{i(n - m)x}}{i(n - m)} \right]_{-\pi}^\pi = 0& \text{se } n \neq m \\
		\end{cases}
		$$

	\item $\mathcal F$ è completo.

		\textit{Dimostrazione.} Questo punto richiede il teorema di Stone-Weierstrass.

		Consideriamo\footnote{Le combinazioni lineari sono finite.}
		$$
			\mathcal A = \spn_\C(\mathcal F) = \left\{ \sum_n a_n e^{inx} \right\} = \{ p(e^{inx}) \mid p \text{ polinomio a esponenti interi} \}.
		$$
		Segue che $\mathcal A$ è una sottoalgebra che separa i punti di $K$ tranne $-\pi$ e $\pi$ ed è chiusa per coniugio.

		% Per il corollario\footnote{Notiamo che la topologia su $\mathcal A$ è quella data dalla norma del $\sup$ delle funzioni continue quindi la chiusura è rispetto a tale norma e la indichiamo con $\overline{\mathcal A}^{\,C}$.} $\overline{\mathcal A}^{\,C} = \{ f \in C([-\pi, \pi]; \C) \mid f(-\pi) = f(\pi) \}$. Dato che la convergenza uniforme implica la convergenza in $L^2$ per spazi di misura finita, abbiamo:
		% $$
		% 	\overline{\mathcal A}^{\,L^2} \supseteq \{ f \in C([-\pi, \pi]; \C) \mid f(-\pi) = f(\pi) \}.
		% $$

		% Inoltre, $\overline{\mathcal A}^{\,L^2} \supseteq \{ f \in C([-\pi, \pi]; \C) \}$ in quanto, una $f \in C([-\pi, \pi]; \C)$ può essere approssimata in $L^2$ tramite funzioni $f_n$ che coincidono in $\{-\pi,\pi\}$. Definiamo $f_n = f \cdot \varphi_n$, dove le $\varphi_n$ sono tali che $\varphi_n(-\pi) = \varphi_n(\pi) = 0$, $\varphi_n = 1$ su $[1/n - \pi, \pi - 1/n]$; notiamo che $f_n \to f$ in $L^2$.

		% [TODO: Disegnino delle $\varphi_n$]

		% Infine poiché le funzioni continue sono dense in $L^2$ segue che $\overline{\mathcal A}^{L^2} = L^2$.
		Per il Corollario, indicando con $\ol{\mc{A}}^C$ la chiusura di $\mc{A}$ rispetto alla norma del sup e con $\ol{\mc{A}}^{L_2}$ la chiusura rispetto alla norma $L^2$, si ottiene
		\begin{gather*}
			\ol{\mc{A}}^C = \left\{ g \in C([-\pi,\pi;\C]) \colon g(-\pi) = g(\pi) \right\} \\
			\text{\rotatebox{270}{$\Longrightarrow$}} \\
			\ol{\mc{A}}^{L^2} \supset \left\{ g \in C([-\pi,\pi;\C]) \colon g(-\pi) = g(\pi) \right\} \\
			\text{\rotatebox{270}{$\stackrel{\scalebox{1.2}{$\ast$}}{\Longrightarrow}$}} \\
			\ol{\mc{A}}^{L^2} \supset C([-\pi,\pi;\C]) \\
			\text{\rotatebox{270}{$\Longrightarrow$}} \\
			\ol{\mc{A}}^{L^2} \underbrace{=}_{\mathclap{\text{densità di $C$ in $L^2$}}} L^2([-\pi,\pi;\C])
		\end{gather*}
		dove $\scalebox{1.2}{$\ast$}$ deriva dal fatto che 
		$$
			\ol{\left\{ g \in C([-\pi,\pi;\C]) \colon g(-\pi) = g(\pi) \right\}}
			= C([-\pi,\pi];\C).
		$$
		Infatti, data $g \in C([-\pi,\pi;\C])$ definisco la successione $(g_n)$ come $g_n \coloneqq g \cdot \myphi_n$ dove $\varphi_n$ sono tali che $\varphi_n(-\pi) = \varphi_n(\pi) = 0$, $\varphi_n = 1$ su $[1/n - \pi, \pi - 1/n]$ e le $\myphi_n$ sono continue.
		Concludiamo osservando che $g_n \to g$ in $L^2$. [TO DO: disegno delle $\myphi_n$]

		
\qed
\end{enumerate}


% Serie di Fourier
%
% Lezione dell'8 Novembre 2021
%

% inserire recap ?

% TODO: Magari mettere tutto in display style
\textbf{Esempio} (calcolo coefficienti di Fourier).
\begin{itemize}
\item $\cos x = \frac{e^{ix} - e^{-ix}}{2} = \frac{1}{2} e^{ix} + \frac{1}{2} e^{-ix}$, allora $ c_n =
\begin{cases}
	\frac{1}{2} \quad n = \pm 1 \\
	0 \quad \text{altrimenti} 
\end{cases}. 
$

\item $(\sin x)^2 = (\frac{e^{ix} - e^{-ix}}{2i})^2 = \frac{1}{4} e^{2ix} + \frac{1}{2} - \frac{1}{4} e^{-2ix}$, allora $c_n =
\begin{cases}
	-\frac{1}{4} \quad n = \pm 2 \\
	\frac{1}{2} \quad n = 0 \\
	0 \quad \text{altrimenti} 
\end{cases} $.

\item $f(x) = x, c_0 = \frac{1}{2\pi} \int_{-\pi}^\pi f(x) \dd x = 0$. Per $n \neq 0$ :
%
$$
c_n = \frac{1}{2\pi} \int_{-\pi}^\pi xe^{-inx} \dd x = \frac{1}{2\pi} \left| \frac{x e^{-inx}}{-in} \right|_{-\pi}^\pi - \frac{1}{2\pi} \int_{-\pi}^\pi \frac{e^{-inx}}{-in} \dd x = \frac{(-1)^n i}{n}.
$$
%

Calcoliamo ora  $\ds \sum_{n \in \Z} \left| c_n \right|^2$.
Valgono le uguaglianze
%
\begin{align*}
\sum_{n \in \Z} \left| c_n \right|^2i & = 2 \sum_{n=1}^{+\infty} \frac{1}{n^2} \\
\sum_{n \in \Z} \left| c_n \right|^2i & = \frac{1}{2\pi } \norm{x}_2^2 = \frac{1}{2\pi } \int_{-\pi}^\pi x^2 \dd x = \frac{2}{2\pi} \cdot \frac{\pi^3}{3} = \frac{\pi^2}{3}.
\end{align*}
%
Dunque $\ds \sum_{n=1}^{+\infty} \frac{1}{n^2} = \frac{\pi^2}{6}$.

\end{itemize}

\section{Regolarità di $f$ e dei coefficienti}

\mybox{%
\textbf{Proposizione 1.} Data $f \in [-\pi,\pi] \to \C$ tale che
\begin{itemize}
\item[(R)] $f \in C^1$ (basta $f$ continua e  $C^1$ a tratti).

\item[(CB)] $f(-\pi) = f(\pi)$.
\end{itemize}

Allora $\ds c_n(f') \overset{(\star)}{=} in \ c_n(f)$.
}

Derivazione formale della formula
%
$$
	f(x) = \sum_{n \in \Z} c_n e^{inx} \xrightarrow{\text{derivata}} f'(x) = \sum_{n \in \Z} in \ c_n e^{inx} 
$$
%

\textbf{Dimostrazione.} Vale quanto segue
\begin{align*}
	c_n(f') & = \frac{1}{2\pi} \int_{-\pi}^\pi f(x)' e^{-inx} \dd x \\
	& \underbrace{=}_{\text{int. per parti}} \frac{1}{2\pi} \overbrace{\cancel{\left| f(x) e^{-inx} \right|_{-\pi}^\pi}}^{f(-\pi) = f(\pi), \; e^{-in\pi} = e^{-in (-\pi)}} - \frac{1}{2\pi} \int_{-\pi}^\pi f(x) (-in) e^{-inx} \dd x \\
	& = (in) \frac{1}{2\pi} \int_{-\pi}^\pi f(x) e^{-inx} \dd x = in \ c_n(f).
\end{align*} 
\qed

\textbf{Nota.} La formula della derivata nel caso di cui non valga la condizione al bordo (CB) si dimostra allo stesso modo ma il primo termine dell'integrazione per parti non si cancella. 

\textbf{Osservazione.} In verità, basta ancora meno. Possiamo riformulare la Proposizione 1 come segue.

\textbf{Proposizione~1'.} Data $f \colon [-\pi,\pi] \to \C$ tale che
\begin{itemize}
	\item[(R')] $f$ è continua.

	\item[(CB)] esiste $\ds g \in L^1([-\pi,\pi], \C)$ tale che $\ds f(x) = f(-\pi) + \int_{-x}^x g(t) \dd t $.
	% \item[(R')] Data $f$ continua ed esiste $\ds g \in L^1([-\pi,\pi], \C) $ tale che $\ds f(x) = f(-\pi) + \int_{-x}^x g(t) \dd t $ 

	% \item[(CB)] $\ds f(-\pi) = f(\pi) \longiff \int_{-x}^x g(t) \dd t = 0 $.
\end{itemize}
Allora la formula $(\star)$ diventa $\ds c_n(g) = in \ c_n(f)$.

\vs

La seguente dà un'indicazione sul comportamento asintotico dei coefficienti di Fourier.

\mybox{%
\textbf{Proposizione 2.} Data $f$ come nella Proposizione~1, valgono le seguenti
\begin{enumerate}
\item $\ds \sum_{n \in \Z} n^2 \left| c_n(f) \right|^2 = \frac{\norm{f'}_2^2}{2\pi} < +\infty$.

\item $\ds \sum_{n \in \Z} \left| n \right|^\alpha \left| c_n(f) \right| < +\infty $ per ogni $\alpha < 1/2$.

\item La serie di Fourier converge.
\end{enumerate}
}

\newpage

\textbf{Dimostrazione.}
\begin{enumerate}

	\item $\ds \sum_{n \in \Z} n^2 \left| c_n(f) \right|^2 \overbrace{=}^{(\ast)} \sum_{n \in \Z} \left| c_n(f') \right|^2
	\underbrace{=}_{\text{Parseval}} \norm{f'}_2^2 / 2\pi \overbrace{<}^{\mathclap{f' \in L^\infty([-\pi,\pi], \C) \subset L^2([-\pi,\pi], \C) }} +\infty$.

	\item $\ds \sum_{n \in \Z} \left| n \right|^\alpha \left| c_n(f) \right| 
	\underbrace{=}_{(n \neq 0)} \sum_{n \in \Z} \left| n \right| \left| c_n(f) \right| \cdot \frac{1}{\left| n \right|^{1-\alpha }}
	\underbrace{\leq}_{\text{C-S per } \ell^2} \overbrace{\left( \sum_{n \in \Z} \left| n \right|^2 \left| c_n(f) \right|^2  \right)^{1/2}}^{< +\infty \text{ per il punto ii)}} \cdot \overbrace{\left( \sum_{n \in \Z} \frac{1}{\left| n \right|^{2-2\alpha }}  \right)^{1/2}}^{\alpha < 1/2 \; \Rightarrow \; <+\infty} < +\infty. 
	$

	\item Dal punto precedente con $\alpha = 0$ otteniamo $\ds \sum \norm{c_n(f) e^{inx}}_\infty = \sum_{n \in \Z} \left| c_n(f) \right| < +\infty $.
\qed
\end{enumerate}

\hypertarget{prop:2021-08nov_prop_3}{%
\textbf{Proposizione 3.}} Data $f \in [-\pi,\pi] \to \C$ tale che
\begin{itemize}
	\item[$(R_k)$] $f \in C^k$ (oppure $f \in C^{k-1}$ e $D^{k-1}f$ è $C^1$ a tratti).

	\item[$(CB_{k-1})$] $D^h f(-\pi) = D^h f(\pi)$ per $h = 0,1,\ldots, k-1$.
\end{itemize}
Allora
\begin{enumerate}
	\item $\ds c_n (D^h f) = (in)^h c_n(f)$ per ogni $n \in \Z$ per ogni $h = 1,\ldots,k$.

	\item $\ds \sum \left| n \right|^{2k} \left| c_n(f) \right|^2 = \frac{\norm{D^k f}_2^2}{2\pi} < +\infty$.

	\item $\ds \sum_{n \in \Z} \left| n \right|^\alpha \left| c_n(f) \right| < +\infty $ per ogni $\alpha < k - 1/2$.

	\item La serie di Fourier di $f$ converge totalmente con tutte le derivate fino all'ordine $k-1$.
\end{enumerate}

\mybox{%
\textbf{Proposizione 4.} Se $f$ è continua e $\ds \sum_{n \in \Z} \left| n \right|^{k-1} \left| c_n(f) \right| < +\infty $ allora $f \in C^{k-1}$ e soddisfa $(CB_{k-1})$.
}

% \textbf{Proposizione 4.} Se $f$ è continua e $\ds \sum_{n \in \Z} \left| n \right|^k \left| c_n(f) \right| < +\infty $ allora $f \in C^k$ e soddisfa $(CB_k)$.

\textbf{Dimostrazione.} Preso $h=0,1,\ldots,k-1$ vale
\begin{align*}
	D^h (c_n(f) e^{inx}) & = c_n(f) (in)^h e^{inx} \\
	\norm{D^h (c_n(f) e^{inx})} & = \left| c_n(f) \right| \left| n \right|^h \leq \left| c_n(f) \right| \left| n \right|^{k-1}.
\end{align*}
% Dunque $\ds \sum D^h \left( c_n(f) e^{inx} \right)$ converge totalmente
%e quindi uniformemente per ogni $h \leq k-1$ ad $\wtilde{f} \colon  [-\pi,\pi] \to \C$ di classe $C^{k-1}$.
%Ma
%%
%$$
%	\frac{1}{\sqrt{2\pi} } \norm{\sum_{-N}^N c_n e^{inx} - \wtilde{f}(x)}_2
%\leq \norm{\sum_{-N}^N c_n e^{inx} - \wtilde{f}(x)}_\infty \xrightarrow{N \rightarrow +\infty} 0.
%$$
%%
%Ma $\ds \sum c_n e^{inx} \to \wtilde{f}$ uniformemente, allora $\ds \sum_{-N}^N c_n e^{inx} \to \wtilde{f}$ in $L^2$.
%Allora $f = \wtilde{f}$ nel senso $L^2$. Siccome $f,\wtilde{f}$ sono continue e coincidono quasi ovunque, vale $f = \wtilde{f}$.
%Abbiamo usato il lemma
Dunque $\ds \sum D^h \left( c_n(f) e^{inx} \right)$ converge totalmente (e quindi uniformemente) con tutte le derivate fino all'ordine $k-1$ dunque per il teorema di derivazione per serie segue la tesi.

% \textbf{Teorema} (di derivazione per serie.) Data una successione di funzioni $f_n$ di classe $C^h$, se la serie $\sum_{n} f_n $ converge uniformemente alla funzione $S$ e se la serie delle derivate $k$-esime per $k \leq h$ converge uniformemente alla funzione $W$ allora $S$ è di classe $C^h$ e $W = \dd / \dd x^k S$, cioè
% $$
% 	W = \sum_n \frac{\dd}{\dd x^k} f_n = \frac{\dd}{\dd x^k} \left( \sum_n f_n \right) = \frac{\dd}{\dd x^k} S.
% $$


%\vs

%\textbf{Lemma.} Date $f,\wtilde{f}$ continue e $f(x) = \wtilde{f}(x)$ per quasi ogni $x$, allora  $f(x) = \wtilde{f}(x)$ per ogni $x$.
\qed

\vs

\textbf{Osservazione.} $f \in C^{k-1}([-\pi,\pi]) + (CB_{k-1})$ se solo se $f $ è la restrizione a $[-\pi,\pi]$ di una funzione $2\pi$-periodica e $C^{k-1}$.


\section{Convergenza puntuale della serie di Fourier}

\mybox{%
\textbf{Teorema.} Data $f \in L^1([-\pi,\pi], \C) $ (estesa in modo $2\pi$-periodico a tutto $\R$), $\ol{x} \in \R$ e $f$ è $\alpha$-Hölderiana in $\ol{x}$ con $\alpha > 0$, cioè esistono $\delta > 0 $ e $M < +\infty$ per cui
%
$$
	\left| f(\ol{x} +t) - f(\ol{x}) \right| \leq M \left| t \right|^\alpha \qquad \text{per $t$ tale che } \left| t \right| < \delta,
$$
%
allora $\ds \sum_{-\infty}^\infty c_n(f) e^{in\ol{x}}$ converge a $f(\ol{x})$.
Cioè $\ds \sum_{-N}^N c_n(f) e^{in\ol{x}}\xrightarrow{N \to \infty} f(\ol{x})$ 
}

\textit{Lavoro preparatorio}: rappresentare somme parziali di serie di Fourier con ``convoluzione'':

Data $f \in L^1([-\pi,\pi], \C) $, $N = 1,2,\ldots$ (estesa a funzioni $2\pi$-periodiche su $\R$).
%
$$
S_N f(x) \coloneqq \sum_{-N}^N c_n e^{inx}  
$$
%
Riscriviamo
%
\begin{align*}
	S_N f(x) \coloneqq \sum_{-N}^N c_n e^{inx} 
	& = \sum_{-N}^N \frac{1}{2\pi} \left( \int_{-\pi}^\pi f(y) e^{-iny} \dd y \right) e^{inx} \\
	& = \frac{1}{2\pi} \int_{-\pi}^\pi f(y) \left( \sum_{n=-N}^{N} e^{in(x-y)}  \right) \dd y.
\end{align*}

Poniamo $\ds D_N(z) \coloneqq \sum_{n=-N}^{N} e^{inz} $ che si definisce \textbf{nucleo di Dirichlet}. Allora
%
\begin{align*}
	S_N f(x) & = \frac{1}{2\pi} \int_{-\pi}^\pi f(y) D_N(x-y) \dd y \overbrace{=}^{x-y=t, \dd y = - \dd t} \frac{1}{2\pi} \int_{x-\pi}^{x+\pi} f(x-t) D_N(t) \dd t \\
	& \overbrace{=}^{(\star)} \frac{1}{2\pi} \int_{-\pi}^\pi f(x-t) D_N(t) \dd t 
	= \frac{1}{2\pi} f \ast D_N(x)
\end{align*}
dove $\ast$ è il prodotto di convoluzione su $\R / 2\pi\Z$. Infine $(\star)$ è il seguente lemma.

\textbf{Lemma.} Se $g$ è $T$-periodica e $g \in L^1([-\pi,\pi], \C) $, allora
%
$$
	\int_0^T g(\tau) \dd \tau = \int_c^{c+T} g(\tau-s) \dd \tau \quad \forall s \; \forall c.
$$
%
Ne segue che
%
$$
	S_N f(x) \coloneqq \sum_{-N}^{N} c_n(f) e^{inx} = \frac{1}{2\pi} \int_{-\pi}^\pi f(x-t) D_N(t) \dd t 
$$
%
dove
%
$$
	D_N(t) = \sum_{-N}^{N} e^{int} = \frac{\sin( (N+ 1/2)t)}{\sin(t/2)}.
$$
%
Infatti,
%
\begin{align*}
	D_N(t) & = \sum_{-N}^N e^{int} = \sum_{-N}^N ( e^{it} )^n = e^{-iNt} \cdot \sum_{n=0}^{2N} ( e^{it} )^n \\
	& = \frac{e^{-i(N + 1/2)t}}{e^{-i t/2}} \cdot \frac{e^{(2N+1) it} - 1}{e^{it} - 1} = \frac{e^{(N+1/2)it} - e^{-(N+1/2)it}}{e^{i t/2} - e^{-i t/2}} \\
	& = \frac{\sin ( (N+1/2)t)}{\sin(t/2)}.
\end{align*}



% Conclusione serie di Fourier. Applicazioni della serie di Fourier
%
% Lezione del 10 Novembre 2021
%

% @aziis98: Ok alla fine ho commentato questa parte qua sotto che è uguale a sopra ed ho lasciato il lemma di recap tipo

% \section{Convergenza puntuale della serie di Fourier}

% \textbf{Teorema.} (di convergenza puntuale della serie di Fourier).
% Data $f \in L^1([-\pi, \pi])$ estesa a $\R$ per periodicità e dato $\bar x \in \R$ tale che $f$ è $\alpha$-H\"olderiana in $\bar x$ con $\alpha > 0$ (cioè $\exists M < +\infty, \delta > 0$ tali che $|f(\bar x + t) - f(\bar x)| \leq M |t|^\alpha$ se $|t| \leq \delta$) allora
% $$
% S_n f(\bar x) \xrightarrow{N \to \infty}  f(\bar x)
% $$
% dove $S_n f(\bar x) = \ds \sum_{n=-N}^N c_n e^{inx}$.

\textbf{Lemma.} (di rappresentazione di $S_n f$ come convoluzione)
Ricapitolando data $f$ come sopra abbiamo visto che
$$
S_n f(x) = \frac{1}{2\pi} \int_{-\pi}^\pi f(x - t) D_n(t) \dd t
\qquad
\text{con }
D_N(t) \coloneqq \sum_{n=-N}^N e^{int} = \frac{\sin\left(\left(N + \frac{1}{2}\right)t\right)}{\sin \left(\frac{t}{2}\right)}
$$
\textbf{Osservazione.}
In particolare vale $\ds \frac{1}{2\pi} \int_{-\pi}^\pi D_N(t) \dd t = 1$.

\textbf{Lemma.} (di Riemann-Lebesgue (generalizzato)).
Data $g \in L^1(\R)$ e $h \in L^\infty(\R)$ con $h$ $T$-periodica, allora
$$
\int_\R g(x) h(yx) \dd x \xrightarrow{y \to \pm \infty}
\underbrace{\left(\int_\R g(x) \dd x \right)}_a
\underbrace{\left(\avint_0^T h(x) \dd x \right)}_m
$$

Se supponiamo $\supp g \subseteq [0, 1]$ allora
%
$$
	 \int_0^1 g(x) h(yx) \dd x \xrightarrow{y \to \pm \infty} \int_0^1 g(x) \dd x \cdot \avint_0^T h(x) \dd x \approx \int_0^1 g(x) \dd x \cdot \avint_0^1 h(yx) \dd y.
$$
%
In particolare è abbastanza intuitivo il risultato per $g$ costante a tratti infatti su un intervallo otterremmo
$$
\int_0^{x_1} g(x) h(yx) \dd x = c \int_0^{x_1} h(yx) \dd x = (c x_1) m
$$
Però ci sarebbero delle correzzioni da fare per dimostrare le cose in questo modo in generale. Vediamo invece un'altra dimostrazione un po' più elegante.

[TODO: Disegnino nel caso $g$ costante a tratti]

\textbf{Dimostrazione.}
Per ogni $s, y$ poniamo $\Phi(y, s) \coloneqq \int_\R g(x) h(yx + s) \dd x$ con $s, y \in \R$ allora la tesi è che $\Phi(y, 0) \xrightarrow{y \to \pm\infty} a m$.
Vedremo che valgono le seguenti
\begin{enumerate}
	\item $\forall y \; \ds \avint_0^T \Phi(y, s) \dd s = am$.
	\item $\forall s \; \Phi(y, s) - \Phi(y, 0) \xrightarrow{y \to \pm \infty} 0$.
\end{enumerate}
da cui segue subito che
$$
\Phi(y, 0) - ma = \avint_0^T \Phi(y, 0) - \Phi(y, s) \dd s \xrightarrow{y \to \pm \infty} 0
$$
per convergenza dominata dove dalla ii). segue la convergenza puntuale e come dominazione usiamo
$$
|\Phi(y, 0) - \Phi(y, s)| \leq 2 \norm{g}_1 \norm{h}_\infty
$$
da cui segue la tesi. Mostriamo ora i due punti.
\begin{enumerate}
	\item Esplicitiamo e applichiamo Fubini-Tonelli
		$$
		\begin{aligned}
			\int_0^T \Phi(y, s) \dd s
			&= \int_0^T \int_\R g(x) h(yx + s) \dd x \dd s \\
			&= \int_\R \underbrace{\avint_0^T h(yx + s) \dd s}_m \cdot g(x) \dd x \\
			&= m \int_\R g(x) \dd x = m a
		\end{aligned}
		$$
		e possiamo usare Fubini-Tonelli in quanto 
		$$
		\ds \int_\R \avint_0^T |h(yx - s)| \dd s \cdot |g(x)| \dd x \leq \int_\R \norm{f}_\infty |g(x)| \dd x = \norm{h}_\infty \cdot \norm{g}_1
		$$

	\item {} [TODO: inventare delle parole a caso]
		$$
		\Phi(y, s) 
		= \int_\R g(x) h\left(y \left( x + \frac{s}{y} \right)\right) \dd x 
		$$
		ora applichiamo la sostituzione $\ds t = x + \frac{s}{y}$ da cui $\dd t = \dd x$
		$$
		= \int_\R g \left( t - \frac{s}{y} \right) h(yt) \dd t
		$$
		ed a questo punto otteniamo
		$$
		\Phi(y, s) - \Phi(y, 0) = \int_\R \left(g\left( t - \frac{s}{y}\right) - g(t) \right) h(yt) \dd t 
		$$
		$$
		\implies
		|\Phi(y, s) - \Phi(y, 0)| = \int_\R |\tau_{\frac{s}{y}} g - g| \cdot |h(yt)| \dd t
		\leq \norm{\tau_{\frac{s}{y}} g - g}_1 \cdot \norm{h}_\infty \xrightarrow{y \to \pm \infty} 0
		$$
\end{enumerate}
\qed

\textbf{Dimostrazione del Teorema.}
$$
\begin{aligned}
	S_N f(\bar x) - f(\bar x) 
	&= \frac{1}{2\pi} \int_{-\pi}^\pi f(\bar x - t) D_N(t) \dd t - \frac{1}{2\pi} \int_{-\pi}^\pi f(\bar x) D_N(t) \dd t \\
	&= \frac{1}{2\pi} \int_{-\pi}^\pi (f(\bar x - t) - f(\bar x)) D_N(t) \dd t \\
	&= \frac{1}{2\pi} \int_{-\pi}^\pi \frac{f(\bar x - t) - f(\bar x)}{\sin \frac{t}{2}} \sin \left(\left(N + \frac{1}{2}\right) t\right) \dd t \\
	&= \int_{-\pi}^\pi g(t) \cdot \sin \left(\left(N + \frac{1}{2}\right) t\right)
	\xrightarrow{\text{RL}} \left(\int g(x) \dd x\right) \cdot \avint_0^\pi \sin x \dd x
\end{aligned}
$$
in particolare per applicare Riemann-Lebesgue serve $g \in L^1([-\pi, \pi])$ ma infatti per $|t| \leq \delta$
$$
|g(t)| \leq \frac{|f(\bar x - t) - f(\bar x)|}{|\sin \frac{t}{2}|} \leq \frac{M |t|^\alpha}{|t| / \pi} = \frac{M \pi}{|t|^{1 - \alpha}} \in L^1([-\delta, \delta])
$$
invece per $\delta \leq |t| \leq \pi$ basta
$$
|g(t)| \leq \frac{|f(\bar x - t)| + |f(\bar x)|}{\sin \frac{\delta}{2}} \in L^1([-\pi, \pi])
$$
\qed

Data $f \in L^1([-\pi, \pi])$ estesa per periodicità e dato $\bar x$ tale che esistano i limiti a destra e sinistra di $f$ in $\bar x$ detti $L^+$ e $L^-$ ed $f$ $\alpha$-H\"olderiana a sinistra e destra si può vedere che vale
$$
S_N f(\bar x) \xrightarrow{N} \frac{L^+ + L^-}{2}
$$

\chapter{Applicazioni della serie di Fourier}

\section{Equazione del calore}

Sia $\Omega$ un aperto di $\R^d$ e $u(t, x) \colon [0, T) \times \Omega \to \R$ e chiamiamo $x$ la \textit{variabile spaziale} e $t$ la \textit{variabile temporale}. In dimensione $3$ l'insieme $\Omega$ rappresenta un solito di materiale conduttore omogeneo  e $u(t, x) = $ temperatura in $x$ all'istante $t \implies u$ risolve l'equazione del calore
$$
u_t = c \cdot \Delta u
$$
dove con $u_t$ indichiamo la derivata parziale di $u$ rispetto al tempo, $c$ è una costante fisica che porremo uguale ad $1$ e $\Delta u$ è il laplaciano rispetto alle dimensioni spaziali ovvero
$$
\Delta u = \sum_{i=1}^d \frac{\pd^2 u}{\pd x_i^2} = \operatorname{div}(\nabla_x u)
$$

Vedremo che la soluzione di $u_t = \Delta u$ esiste ed è unica specificando $u(0, \curry) = u_0$ condizione iniziale con $u_0 \colon \Omega \to \R$ data e delle condizioni al bordo come ad esempio
\begin{itemize}
	\item \textit{Condizioni di Dirichlet}: $u = v_0$ su $[0, T) \times \pd \Omega$ con $v_0$ funzione fissata. Possiamo pensare come fissare delle sorgenti di calore costanti sul bordo.
	\item \textit{Condizioni di Neumann:} $\ds \frac{\pd u}{\pd \nu}$ con $\nu$ direzione normale al bordo. Essenzialmente ci sta dicendo che non c'è scambio di calore con l'esterno.
\end{itemize}
In particolare scriveremo
$$
\left\{
\begin{aligned}
	& u_t = \Delta u \quad \text{su $\Omega$} \\
	& u(0, \curry) = u_0 \\
	& \text{\footnotesize Una delle condizioni al bordo su $\pd \Omega$...}
\end{aligned}
\right.
$$

\subsection{Derivazione dell'equazione del calore}

Partiamo da due leggi fisiche
\begin{itemize}
	\item \textit{Trasmissione del calore attraverso pareti sottili:}

		Siano $u^-$ e $u^+$ le temperature a sinistra e destra di una parete di spessore $\delta$ ed area $a$. Allora ``la quantità di calore che attraversa la parete per unità di tempo è proporzionale a $u^- - u^+$, all'area della parete e inversamente proporzionale allo spessore.
		$$
		\Delta E = c_1 \frac{\Delta u}{\delta} a \Delta t
		$$
		In particolare per $\delta \to 0$ otteniamo che su una superficie $\Sigma$ vale
		$$
		\Delta E = c_1 \frac{\pd u}{\pd \nu} |\Sigma| \Delta t
		$$
		Passando ulteriormente al caso continuo otteniamo
		$$
		\frac{\Delta E}{\Delta t} = c_1 \frac{\pd u}{\pd \nu} |\Sigma| 
		\implies \frac{\pd E}{\pd t} = \int_{\pd A} \frac{\pd u}{\pd \nu}
		$$

	\item \textit{Legge fisica 2:}

		L'aumento di temperatura in un solito è proporzionale alla quantità di calore immessa e inversamente proporzionale al volume.
		$$
		\Delta u = \frac{1}{c_2} \frac{\Delta E}{V}
		$$
		passando al continuo otteniamo $\ds \frac{\pd E}{\pd t} = \int_A c_2 \frac{\pd u}{\pd t}$.
\end{itemize}
Dunque infine otteniamo che
$$
\forall A \subseteq \Omega \, \forall t
\qquad
\int_A c_2 \frac{\pd u}{\pd t} = \int_{\pd A} \frac{\pd u}{\pd \nu} = \int_A \operatorname{div}(\nabla u) = \int_A \Delta u
$$
dove abbiamo usato il teorema della divergenza. Ed infine
$$
\implies \int_A c_2 \frac{\pd u}{\pd t} = \int_A c_1 \Delta u 
\implies c_2 \frac{\pd u}{\pd t} = c_1 \Delta u 
\implies \frac{\pd u}{\pd t} = \frac{c_1}{c_2} \cdot \Delta u
$$



% Esercitazione
%
% Lezione 11 novembre
%

\section{Esercizi}

\subsection{Serie di Fourier}

Consideriamo $L^2([-\pi,\pi], \C) $. Ricordiamo che $e^{inx} = \cos(nx) + i \sin(nx)$. 
Abbiamo
%
\begin{align*}
\sum_{n=-N}^N c_n e^{inx} & = c_0(f) + \sum_{n=1}^N \left( c_n(f) e^{inx} + c_{-n} e^{inx} \right) \\
& = c_0(f) + \sum_{n=1}^{N} \left[ \left( c_n(f) + c_{-n}(f) \right) \cos(nx) + i\left( c_n(f) - c_{-n}(f) \right) \sin(nx) \right] \\
& = c_0(f) + \sum_{n=1}^{N} a_n(f) \cos(nx) + b_n(f) \sin(nx)  
\end{align*}
con
%
$$
\begin{cases}
a_n(f) = c_n(f) + c_{-n}(f) \\
b_n(f) = i c_n(f) - i c_{-n} (f) \\
a_0(f) = c_0(f)
\end{cases} 
$$
%

Passando al limite per $N \to +\infty$
%
$$
f(x) \overbrace{=}^{L^2} \sum_{n=-N}^N c_n e^{inx} 
\overset{(\star)}{=} c_0(f) + \sum_{n=1}^{+\infty} a_n(f) \cos(nx) + b_n(f) \sin(nx) 
$$
%

\textbf{Nota.} L'uguaglianza $(\star)$ ha bisogno di qualche spiegazione: come sappiamo che la serie a destra converge? Usiamo il fatto, che mostriamo sotto, che $\{1, \cos(nx), \sin(nx), n > 1 \}$ sono un sistema ortogonale, dunque per la disuguaglianza di Bessel segue la convergenza.

\textbf{Osservazione.} Gli elementi $\left\{ 1, \sin(nx), \cos(nx) \right\}$ sono ortogonali per $n \geq 1$ in $L^2([-\pi,\pi], \C) $.
Infatti, ricordiamo che
\begin{align*}
\cos(nx) & = \frac{e^{inx} + e^{-inx}}{2}\\
\sin(nx) & = \frac{e^{inx} - e^{-inx}}{2i}. 
\end{align*}

\begin{itemize}

	\item È banale verificare che $\left<1,\cos(nx) \right> = \left<1, \sin(nx) \right> = 0$; 

	\item Verifichiamo ora che valga $\left<\cos(nx), \sin(mx) \right> = 0$ per ogni $n,m$, dunque calcoliamo:
	%
	$$
		\left<\sin(nx), \sin(mx) \right> = \frac{1}{4} \left<e^{-inx} - e^{inx}, e^{-imx} - e^{imx} \right> = 0.
	$$
	%

\end{itemize}

Ora normalizziamo: $\ds \left\{ \frac{1}{\sqrt{2\pi}}, \frac{\cos(nx)}{\sqrt{\pi}}, \frac{\sin(nx)}{\sqrt{\pi}}, n \geq 1 \right\}$.

In conclusione, abbiamo ottenuto che
\begin{itemize}

	\item $L^2( [-\pi,\pi], \C)$ ha base Hilbertiana $\left\{ \frac{1}{\sqrt{2\pi}}, \frac{\cos(nx)}{\sqrt{\pi}}, \frac{\sin(nx)}{\sqrt{\pi}} \right\}_{\C}$


	\item $L^2( [-\pi,\pi], \R)$ ha base Hilbertiana $\left\{ \frac{1}{\sqrt{2\pi}}, \frac{\cos(nx)}{\sqrt{\pi}}, \frac{\sin(nx)}{\sqrt{\pi}} \right\}_{\R}$ 

\end{itemize}

\textbf{Esercizio.}  Se $f$ è a valori reali, dimostrare che $a_n(f)$ e $b_n(f)$ sono anch'essi reali. [TO DO]

\textit{Sketch.} Si dimostra che $a_n(f) = \ol{a_n(f)}, b_n(f) = \ol{b_n(f)}$ e per farlo si usano le espressioni di $a_n, b_n$ in funzione dei coefficienti di Fourier complessi scritte sopra.


\textbf{Esercizio.} Trovare la base di Fourier complessa e reale di $L^2([a,b], \C) $.

\textbf{Soluzione.}
Data $f(x) \in L^2([a,b],\C)$, definiamo la funzione
%
$$
	F(y) \coloneqq  f\left( (y+\pi)\frac{b-a}{2\pi} + a \right) = f(x).
$$
%
Notiamo che $F \in L^2([-\pi,\pi],\C)$, dunque ha espansione in serie di Fourier:
%
\begin{align*}
	F(y) = \sum_{n \in \Z} c_n(F) e^{iny}, \qquad 
	c_n(F) & = \frac{1}{2\pi} \int_{-\pi}^\pi F(y) e^{-iny} \dd y \\
	& = \frac{1}{2\pi} \int_{-\pi}^\pi f \left( (y+ \pi) \frac{b-a}{2\pi} + a \right) e^{-iny} \dd y.
\end{align*}
Usando il cambio di variabile $\ds y = (x-a) \frac{2\pi}{b-a} - \pi$ si ottiene
\begin{align*}
	c_n(f) & = \frac{1}{b-a} \int_a^b f(x) \exp{\left[ \left((x-a)\frac{2\pi}{b-a} - \pi\right)(-in)\right]} \dd x \\
	& = \frac{(-1)^n}{b-a}\exp{\left[ \frac{2\pi a}{b-a} in \right]} \int_a^b f(x) \exp{\left[ \frac{2\pi x}{b-a} (-in) \right]} \dd x 
\end{align*}
Da cui
%
$$
	f(x) = \sum_{n \in \Z} c_n(f) \exp{\left[ \frac{2\pi}{b-a}x in \right]}, \qquad 
	c_n(f) = \frac{1}{b-a} \int_a^b f(x) \exp{\left[- \frac{2\pi}{b-a}x in \right]}.
$$
%






% \textbf{Esercizio.} Sia $f \in L^2([-\pi,\pi], \C) $ (l'estensione di) $f$ è $2\pi / N$ periodica. Dimostrare che $c_n(f) \neq 0$ se solo se $n$ multiplo di $N$.

\textbf{Esercizio.} Dato un intero $N \geq 2$, dimostrare che $f$ è $2\pi/N$-periodica se e solo se $c_n = 0$ per ogni $n$ che non è multiplo di $N$.

\textbf{Soluzione.} [TO DO: da fare]


\textbf{Esercizi classici.} Fissata una funzione $f \in L^2$, calcolare i coefficienti di Fourier complessi (e reali).

Calcoliamo i coefficienti di $f(x) = x^2$.

\begin{align*}
	c_n(f) & = \frac{1}{2\pi} \int_{-\pi}^\pi x^2 e^{-inx} \dd x = \frac{1}{2\pi} \left| \frac{e^{-inx}}{-in} x^2 \right|_{-\pi}^\pi - \int_{-\pi}^\pi \frac{e^{-inx}}{-in} 2x \dd x \\
	& = - \int_{-\pi}^\pi \frac{e^{-inx}}{-in} 2x \dd x = -\frac{i}{\pi n} \left[ \left| \frac{e^{-inx}}{-in} \right|_{-\pi}^\pi - \int_{-\pi}^\pi \frac{e^{-inx}}{-in} \dd x  \right] \\
	& = \frac{-i \pi}{\pi n} \frac{\cos(-n\pi) + \cos(-n (-\pi))}{-in} + \frac{i}{\pi n} \int_{-\pi}^\pi \frac{e^{-inx}}{-inx} \dd x \\
	& = \frac{2}{n^2} + \frac{i}{\pi n} \cdot 0 = \frac{2}{n^2} (-1)^n \\
	& \Longrightarrow c_n(f) = \frac{2}{n^2} (-1)^n.
\end{align*}

Infine
%
$$
c_0(f) = \frac{1}{2\pi} \int_{-\pi}^\pi x^2 \dd x = \frac{1}{2\pi} \frac{2}{3} \pi^3 = \frac{1}{3} \pi^2.
$$
%

Per Parseval $\ds \norm{f}_{L^2}^2 = 2\pi \sum_{n \in \Z} \left| c_n(f) \right|^2$.
Da cui
%
$$
\int_{-\pi}^\pi x^4 \dd x = \norm{x^2}_{L^2}^2 = 2\pi \cdot \left[ \sum_{n=1}^{+\infty} \frac{4}{n^4} + \frac{\pi^2}{3}  \right].
$$
%

\textbf{Nota.} Potevamo ottenere i coefficienti di $f(x) = x^2$, applicando il teorema della derivata.

\textbf{Domande.}
\begin{itemize}

	\item Abbiamo visto che $\ds c_n(x^2) = \frac{2 (-1)^n}{n^2}$ e dedotto che $\ds c_n(2x) = in \frac{2(-1)^n}{n^2} = \frac{2(-1)^n i}{n}$.


	\item Vorremmo calcolare $c_n(2)$, possiamo applicare il teorema sulla formula della derivata?

\end{itemize}

\textbf{Esercizio.}
\begin{enumerate}

	\item Calcolare i coefficienti di Fourier complessi di $x^3$ e vedere se vale $c_n(3x^2) = i n c_n(x^3)$.


	\item Calcolare i coefficienti reali di $x^2$.

\end{enumerate}

\textbf{Esercizio.} Sia $f(x)$ definita da $\ds \sum_{n \in \Z} \gamma_n e^{inx} $ con $\ds  %
\begin{cases}
	\ds \gamma_n = \frac{\cos(n)}{\left| n \right|^{3 / 2}} \\
	\gamma_0 = 1
\end{cases}
$

\textbf{Domande.}
\begin{enumerate}
\item $f$ è ben definita?

\item $f$ è continua?

\item $f$ è derivabile?
\end{enumerate}

\textbf{Dimostrazione.}
\begin{enumerate}

	\item Sì, infatti $\ds 2 \sum_{n=1}^{+\infty} \left| \gamma_n \right|^2 \leq 2 \sum_{n=1}^{+\infty} \frac{1}{n^3} < +\infty $.


	\item \textit{Suggerimento.} Usare la \hyperlink{prop:2021-08nov_prop_3}{Proposizione 3} della parte della regolarità dei coefficienti della serie di Fourier.

\end{enumerate}


% Risoluzione dell'equazione del calore


\section{Risoluzione dell'equazione del calore (su $\mathbb S^1$)}

Come conduttore consideriamo un anello di materiale omogeneo e sottile che parametrizziamo con $[-\pi, \pi]$. Dunque consideriamo $u \colon [0, T) \times [-\pi, \pi] \to \R$ con le condizioni
\begin{equation}
\tag{P} \label{eq:15nov2021_problem_1}
\left\{
\begin{aligned}
	& u_t = u_{xx} \\
	& u(\curry, \pi) = u(\curry, -\pi) \\
	& u_x(\curry, \pi) = u_x(\curry, -\pi) \\
	& u(0, \curry) = u_0
\end{aligned}
\right.
\end{equation}
in particolare la (ii) e la (iii) condizione non sono né quelle di Dirichlet né di Neumann, sono delle condizioni che effettivamente ci dicono che siamo ``su $\mathbb S^1$''\footnote{Bastano solo queste condizioni sulla funzione e sulla sua derivata perché intuitivamente le altre seguono applicando la (i).}; invece l'ultima è la condizione iniziale ed $u_0$ è data.

\subsection{Risoluzione formale}

Scriviamo $u$ in serie di Fourier rispetto a $x$ cioè
$$
u(t, x) = \sum_{n \in \Z} c_n e^{inx}
$$
con $c_n \coloneqq c_n(u(t, \curry))$ da cui derivando formalmente dentro le sommatorie otteniamo che $u_t$ e $u_{xx}$ sono
$$
\begin{gathered}
	\sum_{n \in \Z} \dot c_n(t) e^{inx} = u_t = u_{xx} = \sum_{n \in \Z} -n^2 c_n(t) e^{inx} \\
	u_t = u_{xx} \longiff  \dot c_n(t) = -n^2 c_n(t) \mquad \forall n \, \forall t
	\quad
	\text{e} 
	\quad
	u(0, \curry) = u_0 \longiff c_n(0) = c_n(u_0) \eqqcolon c_n^0
\end{gathered}
$$
Dunque risolvere \eqref{eq:15nov2021_problem_1} equivale per ogni $n$ che $c_n$ che risolva il problema di Cauchy dato da
\begin{equation}
	\tag{P$'$} \label{eq:15nov2021_problem_2}
	\left\{
	\begin{aligned}
		& \dot y = -n^2 y \\
		& y(0) = c_n^0
	\end{aligned}
	\right.
\end{equation}
con soluzione $y(t) = \alpha e^{-n^2 t}$ cioè $c_n(t) = c_n^0 e^{-n^2 t}$ e quindi abbiamo
\begin{equation}
	\tag{$*$} \label{eq:15nov2021_solution}
	u(t, x) = \sum_{n \in \Z} c_n^0 e^{-n^2 t} e^{inx}
\end{equation}

Studiando questa soluzione formale possiamo fare le seguenti osservazioni che poi diventeranno dei teoremi
\begin{itemize}
	\item \textit{La soluzione esiste per $t \in [0, +\infty)$ ed è molto regolare per $t > 0$}

		Vedremo che la soluzione formale è proprio una soluzione al problema per $t \geq 0$, in particolare il termine $e^{-n^2 t} \to 0$ in modo più che polinomiale ed infatti vedremo che la soluzione sarà proprio $C^\infty$ per $t > 0$.

	\item \textit{La soluzione è unica}

		Tutti i problemi di Cauchy per i coefficienti $c_n(t)$ hanno un'unica soluzione dunque anche la soluzione $u$ è unica.

	\item \textit{In generale non esiste soluzione nel passato.}

		Se il numero di coefficienti $c_n^0 \neq 0$ è infinito allora il termine $e^{-n^2 t} \to +\infty$ molto velocemente per $t < 0$ e la serie diverge.

\end{itemize}

\textbf{Teorema 1} (Esistenza e Regolarità).

Se $u_0 \colon [-\pi, \pi] \to \C$ (presa in $L^2$) è tale che $\ds \sum_{n \in \Z} |c_n^0| < +\infty$ (ad esempio se $u_0 \in C^1$ ed è $2\pi$-periodica) allora
$$
u(t, x) = \sum_{n \in \Z} \underbrace{c_n^0 e^{-n^2 t} e^{inx}}_{u_n(t,x)}
$$
definisce una funzione $u \colon [0, +\infty) \times \R \to \C$ tale che
\begin{enumerate}
	\item $u$ è $2\pi$-periodica in $x$ ed è reale se $u_0$ è reale.
	\item $u$ è continua.
	\item $u$ è $C^\infty$ su $(0, +\infty) \times \R$.
	% \item Risolve (\ref{eq:15nov2021_problem_1}) per $t > 0$ sulla condizione al bordo, $u(0, \curry) = u_0$ su $[-\pi, \pi]$ e $u_t = u_{xx}$ su $t > 0$.
	\item Risolve \eqref{eq:15nov2021_problem_1}. In particolare vale $u_{tt} = u_{xx}$ e valgono le condizioni di periodicità per $t > 0$; e infine vale $u(0, \curry) = u_0$ su $[-\pi,\pi]$.
\end{enumerate}

Vediamo alcuni lemmi tecnici preparatori e sia $R$ un rettangolo di $\R^d$ ovvero prodotto di intervalli con estremi aperti o chiusi.

\textbf{Lemma 4.}
Date $v_n \colon R \to \C$ di classe $C^k$ con $k = 1, 2, \dots, +\infty$ tali che
\begin{itemize}
	\item $v_n \to v$ uniformemente.

	\item $\forall \underline h = (h_1, \dots, h_d) \in \N^d$ con $|\underline h| \coloneqq h_1 + \dots + h_d \leq k$ (se $k = +\infty$ allora basta $|\underline h| < +\infty$) posto
		$$
		D^{\underline h} \, v_n \coloneqq
		\frac{\pd^k}{\pd x_1^{h_1} \cdots \pd x_d^{h_d}} v_n
		$$
		$D^{\underline h} \, v_n \to D^{\underline h} \, v$ converge uniformemente.
\end{itemize}
allora $v \in C^k$ e $D^{\underline h} v = \lim_{n} D^{\underline h} v_n$.

\textbf{Dimostrazione.}
Si parte dal caso $d = 1$ e $k = 1$ e si procede per induzione. [TODO: Esercizio]
\qed

\textbf{Corollario. 5}
Date $u_n \colon R \to \C$ di classe $C^k$ con $k = 1, \dots, +\infty$ tali che
$$
\forall \underline h \text{ con } |\underline h| \leq k \qquad \sum_n \norm{D^{\underline h} \, u_n} < +\infty
$$
allora $u \coloneqq \sum_n u_n$ è una funzione ben definita su $R$ e $C^k$ e $D^{\underline h} \, u = \sum_n D^{\underline h} \, u_n$ per ogni $\underline h$ con $|\underline h| \leq k$.

\textbf{Lemma. 6}
Data $u \colon R \to \C$ e rettangoli $R_i \subset R$ relativamente aperti in $R$ tali che $u$ è $C^k$ sugli $R_i$ per ogni $i$ allora $u$ è di classe $C^k$ su $\tilde R \coloneqq \bigcup_i R_i$.

\textbf{Dimostrazione.}
Intuitivamente essere $C^k$ è una proprietà locale ma preso $x \in R \implies \exists i \; x \in R_i$ e dunque segue per l'ipotesi sugli $R_i$.
\qed

\textbf{Lemma. 7}
Data $f \in L^2((-\pi, \pi); \C)$
$$
f \text{ è reale q.o.} \iff c_{-n}(f) = \overline{c_n(f)}
$$

\textbf{Osservazione.}
Notiamo che se $f \in L^1$ la freccia $\boxed{\Leftarrow}$ è molto più difficile.

\textbf{Dimostrazione Teorema 1.}
\begin{enumerate}
	\item $u_0$ reale $\implies c_{-n}^0 = \overline{c_n^0} \implies c_{-n}^0 e^{-(-n)^2 t} = \overline{c_n^0 e^{-n^2 t}} \iff c_{-n}(u(t, \curry)) = \ol{c_{n}(u(t, \curry))}$.

	\item Sia $R \coloneqq [0, +\infty) \times \R$, $\norm{u_n}_{L^\infty(R)} = \norm{c_n^0 e^{-n^2 t}}_{L^\infty (R)} =  |c_n^0|$ dunque
		$\ds\sum_{n \in \Z} u_n$ converge totalmente su $R$ e quindi $u$ è ben definita e continua su $R$.

	\item Presi $h, k = 0, 1, 2, \dots$ se proviamo a calcolare $D_t^h D_x^k u_n = c_n^0 (-n)^{2h} (in)^k e^{-n^2 t} e^{inx}$ vediamo non si riesce a stimare per $t \to 0$ infatti
		$$
		\norm{D_t^h D_x^k u_n}_{L^\infty(R)}
		= |c_n^0| \cdot |n|^{2h + k} \xrightarrow{n} \infty.
		$$
		Serve prendere $\delta > 0$ e sia $R_\delta \coloneqq (\delta, +\infty) \times \R$
		$$
		\norm{D_t^h D_x^k u_n}_{L^\infty(R_\delta)}
		= |c_n^0| \cdot |n|^{2h + k} e^{-n^2 \delta}
		$$
		in particolare per ogni $h, k$ abbiamo che $|n|^{2h + k} e^{-n^2 \delta} \xrightarrow{n} 0 \implies |n|^{2h + k} e^{-n^2 \delta} \leq m_{h,k} \implies \norm{D_t^h D_x^k u_n}_{L^\infty(R_\delta)} \leq m_{h,k} \cdot |c_n^0|$ e quindi $\sum_n D_t^h D_x^k u_n$ converge totalmente su $R_\delta$.

		Quindi $u$ è $C^\infty$ su $R_\delta$ per ogni $\delta > 0$ e siccome $R_\delta$ è aperto in $R$ per il Lemma. 6 $u$ è $C^\infty$ su $\bigcup_{\delta > 0} R_\delta = (0, +\infty) \times \R$.

	\item Essendo che $u$ è $2\pi$-periodica in $x$, valgono le condizioni al bordo; inoltre $u_0$ e $u(0,\cdot)$ hanno gli stessi coefficienti di Fourier, dunque $u_0 = u(0,\cdot)$ quasi ovunque, ma essendo continue vale $u_0 = u(0,\cdot)$ su $[-\pi,\pi]$; infine, $(u_n)_t = (u_n)_{xx} \implies \sum (u_n)_t = \sum (u_n)_{xx} \implies u_t = u_{xx}$ per $t > 0$.
\end{enumerate}
\qed

Ora enunciamo il teorema di unicità, vogliamo un teorema con il minor numero di ipotesi possibile e che ci dà più informazioni; quindi in questo caso cerchiamo la più grande famiglia di funzioni (quindi la meno regolare possibile) sulla quale vale l'unicità della soluzione.

\hypertarget{thm:lez15nov_teo2}{%
\textbf{Teorema. 2} (Unicità)}
Sia $u \colon [0, T) \times [-\pi, \pi] \to \C$ continua, $C^1$ nel tempo e $C^2$ nello spazio per $t > 0$. Se $u$ risolve \eqref{eq:15nov2021_problem_1} su $t > 0$ allora $u$ è unica.

\textbf{Definizione.}
Dato $R$ un rettangolo e $u \colon R \to \C$ diciamo che $u$ è $C^k$ nella variabile $x_i$ se $\left(\frac{\pd}{\pd x_i}\right)^h u$ esiste per $h = 1, \dots, k$ ed è continua su $R$.

\textbf{Lemma. 8}
Data $u \colon I \times [-\pi, \pi] \to \C$ di classe $C^k$ in $t \implies c_n(D_t^h u(t, \curry)) = D_t^h c_n(u(t, \curry))$ per $h \leq k$.

\textbf{Dimostrazione.}
$$
c_n(t) = \frac{1}{2\pi} \int_{-\pi}^\pi u(t, x) e^{-inx} \dd x
\implies \dot c_n(t) = \frac{1}{2\pi} \int_{-\pi}^\pi u_t(t, x) e^{-inx} \dd x = c_n(u_t(t, \curry))
$$
per il teorema di derivazione sotto il segno di integrale (Analisi 2?)
\qed

\textbf{Dimostrazione Teorema 2.}
Poniamo $c_n(t) \coloneqq c_n(u(t, \curry))$. Sappiamo che per $t > 0$ vale $\dot c_n(t) = c_n(u_t(t, \curry)) = c_n(u_{xx}(t, \curry)) = -n^2 c_n(t) \implies c_n$ risolve il problema di Cauchy
$$
\left\{
\begin{aligned}
	& \dot y = -n^2 y \\
	& y(0) = c_n^0
\end{aligned}
\right.
$$ 

\textbf{Nota.} Sia $y \colon [0,T) \to \R^k$ funzione continua su $[0,T)$ e derivabile su $(0,T)$ che risolve l'equazione differenziale ordinaria $\dot y = f(t,y)$ su $(0,T)$ con $f\colon  [0,T) \times \R^k \to \R^k$ continua.
Allora $y$ è $\mc{C}^1$ su $[0,T)$ e risolve $\dot y = f(t,y)$ su $[0,T)$.

Dalla nota sopra otteniamo che $c_n$ è unico.
% $\implies c_n$ è unico. (in particolare abbiamo usato che $\dot y = f(t, y)$ per $t > 0$ e $y(0) = y_0$ e $y$ è continua in $0 \implies y$ è $C^1$ anche in $0$ e risolve $\dot y = f(t, y)$ per $t \geq 0$)
\qed












% Equazioni delle onde
%
% Lezione del 17 novembre
%

\textbf{Notazione.} $\cper^k = \{ f\colon  \R \to \C \; \pi\text{-periodiche e } \mc{C}^k \} $.

\mybox{%
\textbf{Teorema 3} (di non esistenza nel passato).
Esiste $u_0 \in \cper^\infty$ tale che per ogni $\delta > 0$ non esiste $u \colon (-\delta,0] \times [-\pi,\pi] \to \C$ soluzione di \eqref{eq:15nov2021_problem_1} ($u$ continua, $\mc{C}^1$ in $t$ e $\mc{C}^2$ in $x$ per $t < 0$)
}

\textbf{Dimostrazione.} Sia $u$ su $(-\delta,0) \times [-\pi,\pi]$ un'eventuale soluzione.
Sia $c_n(t)$ al solito.
Dalla dimostrazione del \hyperlink{thm:lez15nov_teo2}{Teorema 2} abbiamo che $c_n$ risolve \eqref{eq:15nov2021_problem_2}.

Quindi $c_n(t) = c_n^0 e^{-n^2 t}$. Scelgliamo $c_n^0$ (cioè $u_0$) in modo che

\begin{itemize}

	\item $c_n^0 = O(|n|^{-a})$ per $n \to \pm \infty$ per ogni $a > 0$. ($\Rightarrow \sum |n|^k |c_n^0| < +\infty \mquad \forall k \Rightarrow u_0 \in \cper^\infty$).


	\item $c_n^0 e^{-n^2 t} \nrightarrow 0$ per ogni $t < 0$.

\end{itemize}

Con un tale $c_n^0$ la soluzione non esiste al tempo $t$. Infatti, se per assurdo esistesse, i coefficienti di Fourier $c_n(t)$ sarebbero quadrato sommabili\footnote{Per la disuguaglianza di Bessel.}, ovvero dovrebbero tendere a zero \absurd.

Prendiamo $c_n^0 = e^{-|n|}$.
\qed

\textbf{Esercizio.} Dato $u_0$ sia $T_\ast $ il massimo $T$ per cui \eqref{eq:15nov2021_problem_1} ammette soluzione su $(-T,0] \times [-\pi,\pi]$.
Caratterizzare $T_\ast$ in termini del comportamento asintotico di $c_n^0$ per $n \to \pm \infty$.

\textit{Suggerimento.} Guardare $\ds \log(|c_n^0|) / n^2$.


\section{Equazione delle onde}

Sia $\Omega \subset \R^d$ aperto, $I$ intervallo temporale, $u \colon I \times \ol{\Omega} \to \R$, l' equazione delle onde è
%
$$
	u_{tt} = v^2 \nabla u = \nabla_x u = \sum_{i=1}^{d} \frac{\partial^2 u}{\partial x_i^2}
$$
%
dove $v$ si chiama \textbf{velocità di propagazione}.

La soluzione è univocamente determinata specificando

\begin{itemize}

	\item Le condizioni al bordo (come per il calore), ad esempio quelle di Dirichlet: $u = v_0$ su $I \times \partial \Omega$ oppure di Neumann: $\partial u / \partial \nu = 0$ su $I \times \partial \Omega$.


	\item Condizioni iniziali: $u(0,\cdot) = u_0, u_t(0,\cdot) = u_1$.

\end{itemize}

\textbf{Esempio 1.} Per $d = 1$, $\Omega = [0,1]$ rappresenta una sbarra sottile di materiale elastico. La sbarra è soggetta a vibrazioni longitudinali (onde sonore).
La funzione $u(t,x)$ rappresenta lo spostamento dalla posizione di riposo $x$ al tempo $t$.
In tal caso, l'equazione delle onde è
%
$$
	u_{tt} = v^2 u_{xx}.
$$
%


\textbf{Esempio 2.} Per $d=2$, $\Omega$ rappresenta una sbarra sottile di materiale elastico che vibra trasversalmente. La funzione $u(t,x)$ rappresenta lo spostamento verticale del punto di coordinata $x \in \Omega$ a riposo. Allora $u$ soddisfa\footnote{Per oscillazioni piccole.}
%
$$
	u_{tt} = v^2 \nabla v.
$$
%


\section{Risoluzione dell'equazione delle onde}

Consideriamo il caso uno dimensionale. In tal caso l'equazione delle onde è la seguente.
%
\begin{equation}
\tag{P} \label{eq:17nov2021_problem_1}
\begin{cases}
	u_{tt} = v^2 u_{xx} \\
	u(\cdot, \pi) = u(\cdot, -\pi) \\
	u_x(\cdot, \pi) = u_x(\cdot, -\pi) \\
	u(0,\cdot) = u_0 \\
	u_t(0,\cdot) = u_1
\end{cases} 
\end{equation}

\subsection{Risoluzione formale}

Scriviamo $\ds u(t,x) = \sum_{n \in \Z} c_n(t) e^{inx}$. Deriviamo in $t$ e due volte in $x$.
\begin{align*}
	u_{tt} & = \sum_{n \in \Z} \ddot c_n e^{inx} \\
	u_{xx} & = \sum_{n \in \Z} - v^2 n^2 c_n e^{inx}
\end{align*}
Abbiamo che
\begin{gather*}
	u_{tt} = v^2 u_{xx} \longiff \ddot c_n = -v^2 n^2 c_n \\
	u(0, \cdot) = u_0 \longiff c_n(0) = c_n^0 \coloneqq c_n(u_0)
	\qquad
	u_t(0, \cdot) = u_1 \longiff \dot c_n(0) = c_n^1 \coloneqq c_n(u_1)
\end{gather*}
Quindi $u$ risolve \eqref{eq:17nov2021_problem_1} se solo se per ogni $n$, $c_n$ risolve
%
\begin{equation}
\tag{P$'$} \label{eq:17nov2021_problem_2}
\begin{cases}
	\ddot y = -n^2 v^2 y \\
	y(0) = c_n^0 \\
	\dot y(0) = c_n^1
\end{cases} 
\end{equation}
%
Dunque,
\begin{itemize}

	\item Per $n = 0$, $\ddot y = 0$ se solo se $y$ è un polinomio di primo grado, ovvero $c_0(t) = c_0^0 + c_0^1 t$.


	\item Per $n \neq 0$, $y = \alpha_{n}^+ e^{invt} + \alpha_n^- e^{-invt} $ con 
	%
	$$
		\alpha_n^{\pm} = \frac{1}{2} \left( c_n^0 \pm \frac{c_n^1}{inv} \right)
	$$                                 

\end{itemize}
%
Quindi, la soluzione è
\begin{equation}
 \tag{$\ast$} \label{eq:17nov2021_solution_1}
	u(t,x) = c_0^0 + c_0^1 t + \sum_{n\neq 0} \left[ \alpha_n^+ e^{in(x + vt)} + \alpha_n^- e^{in(x - vt)} \right]
\end{equation}

Inoltre,
\begin{equation}
\tag{$\ast \ast$} \label{eq:17nov2021_solution_2}
	u(t,x) = c_0^0 + c_0^1 t + \myphi^+(x + vt) + \myphi^-(x - vt)
\end{equation}
con $\myphi^{\pm}$ funzioni con coefficienti di Fourier $\alpha_n^{\pm}$ che si dicono \textbf{onde viaggianti}.

\textbf{Nota.} La \eqref{eq:17nov2021_solution_2} è specifica delle equazioni delle onde.


% Esercitazione
%
% Lezione del 18 Novembre 2021
%

\section{Esercitazione del 18 Novembre 2021}

\subsection{Esercizi preliminari}

Data $f \colon [-\pi, \pi] \to \R$ o $\C$ uno degli esercizi più comuni è doverne calcolare lo sviluppo di Fourier complesso o reale.

\textbf{Osservazione.}
Ricordiamo che $c_n(f)$ può essere calcolato anche solo se $f \in L^1$ inoltre
$$
\begin{gathered}
	\operatorname{SF}_\C(f) = \sum_{n \in \Z} c_n(f) e^{inx}
	\qquad
	c_n(f) = \frac{1}{2\pi} \int_{-\pi}^\pi f(x) e^{-inx} \dd x \\
	\text{e con base hilbertiana } \left\{ \frac{e^{inx}}{\sqrt{2\pi}} \,\middle|\, n \in \Z \right\}
\end{gathered}
$$
invece nel caso reale abbiamo visto
$$
\begin{gathered}
	\operatorname{SF}_\R(f) = \sum_{n=1}^\infty \left[ a_n \cos(nx) + b_n \sin(nx) \right] + a_0 \\
	\qquad
	a_n(f) = c_n(f) + c_{-n}(f)
	\qquad
	b_n(f) = i(c_n(f) - c_{-n}(f))
	\qquad
	a_0(f) = c_0(f) \\
	a_n = \frac{1}{\pi} \int_{-\pi}^\pi f(x) \cos(nx) \dd x
	\qquad
	b_n = \frac{1}{\pi} \int_{-\pi}^\pi f(x) \sin(nx) \dd x \\
	\text{e con base hilbertiana } \left\{ \frac{1}{\sqrt{2\pi}}, \frac{\cos(nx)}{\sqrt{\pi}}, \frac{\sin(nx)}{\sqrt{\pi}} \,\middle|\, n \geq 1 \right\}
\end{gathered}
$$

\textbf{Esercizio.}
Sia $f(x) = \cos^2(x) \sin(3x)$, calcolare i coefficienti di Fourier\footnote{Con funzioni ottenute come combinazioni di prodotti di potenze di funzioni trigonometriche (anche con argomento moltiplicato per un naturale) conviene calcolare lo sviluppo complesso e poi passare a quello reale.}.

\textbf{Svolgimento.}
Usiamo lo sviluppo complesso
$$
\cos(x) = \frac{e^{ix} + e^{-ix}}{2}
\qquad
\sin(x) = \frac{e^{ix} - e^{-ix}}{2i}
$$
Dunque possiamo riscrivere $f(x)$ come
$$
\begin{aligned}
	&= \frac{i}{8} \left( e^{i2x} + e^{-i2x} + 2 \right) \left( e^{-i3x} - e^{i3x} \right) = \\
	&= \frac{i}{8} \left( e^{-ix} + e^{-i5x} + 2e^{-i3x} - e^{i5x} - e^{ix} - 2e^{i3x} \right) = \\
	&= \frac{i}{8} e^{-ix} + \frac{i}{8} e^{-i5x} + \frac{i}{4} e^{-i3x} - \frac{i}{8} e^{i5x} - \frac{i}{8} e^{ix} - \frac{i}{4} e^{i3x}. 
\end{aligned}
$$
Dunque possiamo già scrivere i coefficienti di Fourier complessi di $f(x)$
$$
\begin{gathered}
	c_n(f) \neq 0 \iff n = \pm 1, \pm 3, \pm 5 \\
	c_{\pm 1}(f) = \mp\frac{i}{8}  
	\quad
	c_{\pm 3}(f) = \mp\frac{i}{4}
	\quad
	c_{\pm 5}(f) = \mp\frac{i}{8}.
\end{gathered}
$$
Continuiamo ora il conto precedente e ricostruiamo la serie di Fourier reale ricomponendo i termini
$$
\begin{aligned}
	&= -\frac{i^2}{4}\left( \frac{e^{ix} - e^{-ix}}{2i} \right) -\frac{i^2}{2}\left( \frac{e^{i3x} - e^{-i3x}}{2i} \right) -\frac{i^2}{4}\left( \frac{e^{i5x} - e^{-i5x}}{2i} \right) = \\
	&= \frac{1}{4} \sin(x) + \frac{1}{2} \sin(3x) + \frac{1}{4} \sin(5x),
\end{aligned}
$$
in particolare possiamo notare che $f(\pi) = f(-\pi) = 0$ dunque potevamo già dedurre che la serie di Fourier reale sarebbe stata composta solo da seni.

\textbf{Esercizio.}
Caratterizzare i coefficienti $c_n(f)$ di una $f \colon [-\pi, \pi] \to \C$ in $L^2$ tale che $\operatorname{Im}(f) \subseteq \R$.

\textit{Suggerimento.} Si usa che per $z \in \C$ vale $z \in \R \iff z = \overline z$.

\textbf{Esercizio.}
Determinare la soluzione di (P) e stabilire unicità e regolarità della soluzione $u \colon [0, T) \times [-\pi, \pi] \to \R$ o $\C$.
\begin{equation}
	\tag{P}
	\left\{
	\begin{aligned}
		& u_t = 4 u_{xx} \\
		& u(\curry, -\pi) = u(\curry, \pi) \\
		& u_x(\curry, -\pi) = u_x(\curry, \pi) \\
		& u(0, x) = \cos^2(x) \sin(3x)
	\end{aligned}
	\right.
\end{equation}

\textbf{Svolgimento.}
Per prima cosa troviamo formalmente una soluzione in serie di Fourier $u(t, x) = \sum_{n \in \Z} c_n(t) e^{inx}$ dove $c_n(t)$ è il coefficiente di $u(t, \curry)$. 

Per il teorema di derivazione sotto il segno di integrale abbiamo
$$
u_t(t, x) = \sum_{n \in \Z} \dot c_n(t) e^{inx}
\qquad
\text{con }
c_n(t) = \frac{1}{2\pi} \int_{-\pi}^\pi u(t, x) e^{-inx} \dd x
$$
Le condizioni al bordo assicurano che $c_n(u_{xx}(t, \curry)) = - 4 n^2 c_n(u(t, \curry))$ da cui otteniamo il seguente problema di Cauchy sui coefficienti
$$
\left\{
\begin{aligned}
	& \dot c_n(t) = -4 n^2 c_n(t) \\
	& c_n(0) = c_n(u(0, \curry)) = c_n(\cos^2(x) \sin(3x))
\end{aligned}
\right.
\qquad
\forall n \in \Z
$$
Inoltre dato che $c_n^0 = 0$ se $n \neq \pm 1, \pm 3, \pm 5 \implies c_n(t) = 0$ per questi $n$, dunque complessivamente i sistemi sono
$$
\newcommand{\uffaA}[3]{%
	\left\{
	\begin{aligned}
		& \dot c_{#1}(t) = - #2 c_{#1}(t) \\
		& c_{#1}(0) = -\frac{i}{#3}
	\end{aligned}
	\right.
}
\uffaA{1}{4}{8}
\quad
\uffaA{3}{36}{4}
\quad
\uffaA{5}{100}{8}
$$
con la condizione $c_{-n}(t) = \overline{c_n(t)}$, così otteniamo
$$
\newcommand{\uffaB}[3]{%
	c_{#1}(t) = #3 e^{-#2t}
}
\begin{aligned}
	& \uffaB{1}{4}{-\frac{i}{8}}
	& \quad \uffaB{3}{36}{-\frac{i}{4}}
	& \quad\;\; \uffaB{5}{100}{-\frac{i}{8}} \\
	& \uffaB{-1}{4}{\frac{i}{8}}
	& \quad \uffaB{-3}{36}{\frac{i}{4}}
	& \quad\;\; \uffaB{-5}{100}{\frac{i}{8}} \\
\end{aligned}
$$
ed infine fattorizzando
$$
\begin{aligned}
	u(t, x) &= 
	\frac{e^{-4t}}{4} \left( -\frac{i}{2}e^{ix} + \frac{i}{2}e^{-ix} \right)
	+ \frac{e^{-36t}}{2} \left( -\frac{i}{2}e^{i3x} + \frac{i}{2}e^{-i3x} \right)
	+ \frac{e^{-100t}}{4} \left( -\frac{i}{2}e^{i5x} + \frac{i}{2}e^{-i5x} \right)
	= \\
	&= \frac{1}{4} e^{-4t} \sin(x)
	+ \frac{1}{2} e^{-36t} \sin(3x)
	+ \frac{1}{2} e^{-100t} \sin(5x)
\end{aligned}
$$

\textbf{Esercizio.}
Consideriamo il problema (P) dato da
\begin{equation}
	\tag{P}
	\left\{
	\begin{aligned}
		& u_t = u_{xx} + u \\
		& u(\curry, -\pi) = u(\curry, \pi) \\
		& u_x(\curry, -\pi) = u_x(\curry, \pi) \\
		& u(0, \curry) = u_0
	\end{aligned}
	\right.
\end{equation}
dove $u_0(x)$ è $\cos^2(x) \sin(3x)$ oppure $\ds \sum_{n \in \Z} \frac{1}{2^{|n|}} e^{inx}$.

\textbf{Svolgimento.}
Per ora lavoriamo con $u_0(x) = \cos^2(x) \sin(3x)$, notiamo subito che i coefficienti soddisfano l'equazione
$$
\left\{
\begin{aligned}
	& \dot c_n(t) = -n^2 c_n(t) + c_n(t) = (1 - n^2) c_n(t) \\
	& c_n(0) = c_n(\cos^2(x) \sin(3x))
\end{aligned}
\right.
$$
da cui $\dot c_n(t) = (1 - n^2)c_n$ con soluzione $c_n(t) = \gamma e^{(1 - n^2) t}$, quindi ad esempio abbiamo
$$
c_{\pm 1}(t) = \mp \frac{i}{8}
\qquad
c_{\pm 3}(t) = \mp \frac{i}{4} e^{-8t}
\qquad
c_{\pm 5}(t) = \mp \frac{i}{8} e^{-24t}
$$
Dunque la soluzione finale è
$$
\begin{aligned}
	u(t, x) 
	&= \frac{i}{8} e^{-ix} -\frac{i}{8} e^{ix} + \frac{i}{4}e^{3t} e^{-i3x} + \frac{i}{4}e^{3t} e^{i3x} + \frac{i}{8} e^{-24t} e^{-i5x} + \frac{i}{8} e^{-24 t} e^{i5x} = \\
	&= 
	-\frac{i}{4} \left( \frac{e^{ix} - e^{-ix}}{2} \right)
	-\frac{i}{2} e^{-3t} \left( \frac{e^{i3x} - e^{-i3x}}{2} \right)
	-\frac{i}{4} e^{-24t} \left( \frac{e^{i5x} - e^{-i5x}}{2} \right) = \\
	&= 
	\frac{1}{4} \sin(x) 
	- \frac{1}{2} e^{-3t} \sin(3x)
	- \frac{1}{4} e^{-24t} \sin(5x)
\end{aligned}
$$

Invece considerando la condizione iniziale $u_0(x) = \sum_{n \in \Z} e^{inx} / 2^{|n|}$ abbiamo che $c_n(u_0) = 1 / 2^{|n|}$, notiamo che i coefficienti sono sommabili
$$
\sum_{n \in \Z} \frac{1}{2^{|n|}} 
= 2 \sum_{n=1}^\infty \frac{1}{2^n} + 1 < +\infty
\qquad
u(t, x) \coloneqq \sum_{n \in \Z} \frac{1}{2^{|n|}} e^{(1 - n)^2 t} e^{inx}
$$
in particolare formalmente possiamo scriverla meglio come
$$
= \sum_{n \in \Z} \frac{1}{2^{|n|}} e^{(1 - n)^2 t} e^{inx}
= e^t \left(1 + \sum_{n > 0} \frac{1}{2^{|n|}} e^{(1 - n)^2} \cos(nx) \cdots\cdots \right.
$$
[TODO: Finire meglio questo conto]

\textbf{Esercizio.} (della volta scorsa)
Consideriamo la funzione
$$
f(x) = \sum_{n \neq 0} \frac{\cos(n)}{|n|^{3/2}} e^{inx}
$$
\begin{itemize}
	\item Dire se $f$ è ben definita e continua.
	\item Dire se $f$ è derivabile.
\end{itemize}

\textbf{Svolgimento.}
$$
\sum_{n \neq 0} |c_n| 
= 2 \sum_{n=1}^\infty \frac{|\cos(n)|}{|n|^{3/2}}
\leq 2 \sum \frac{1}{n^{3/2}} < +\infty
$$
dunque la serie di Fourier converge uniformemente a $f \implies$ è continua e periodica.

Se $\sum |n| \cdot |c_n| < +\infty$ si potrebbe dire che $f$ è derivabile però
$$
\sum_{n \neq 0} |n| \cdot |c_n| = 2 \sum_{n=1}^\infty \frac{|\cos(n)|}{\sqrt{n}} = +\infty
\text{ non converge assolutamente}
$$
Ma la candidata derivata ha coefficienti $i n c_n$ e non starebbe in $L^2$ ovvero
$$
\sum n^2 |c_n|^2 = +\infty \implies \sum i n c_n e^{inx} \notin L^2
$$




















% Equazioni delle onde: teoremi di esistenze, ecc

\section{Risoluzione dell'equazione delle onde}

Consideriamo il problema
\begin{equation}
	\tag{P}
	\left\{
	\begin{aligned}
			& u_{tt} = v^2 u_{xx} \\
			& u(\curry, \pi) = u(\curry, -\pi) \\
			& u_x(\curry, \pi) = u_x(\curry, -\pi) \\
			& u(0,\curry) = u_0 \\
			& u_t(0,\curry) = u_1
	\end{aligned}
	\right.
\end{equation}
ed abbiamo visto che ha soluzione
\begin{equation}
	\tag{$*$}
	u(t,x) = c_0^0 + c_0^1 t + \sum_{n \neq  0} (\alpha_n^+ e^{in(x + vt)} + \alpha_n^- e^{in(x - vt)})
\end{equation}
$$
	\alpha^{\pm}_n = \frac{1}{2}\left( c_n^0 \pm \frac{c^1_n}{i n v} \right).
$$

Inoltre, possiamo scrivere l'equazione $(*)$ come
\begin{equation}
	\tag{$**$}
	u(t,x) = c_0^0 + c_0^1 t + \myphi^+(x + vt) + \myphi^-(x - vt)
\end{equation}
dove $\varphi^+, \varphi^-$ sono funzioni $2\pi$-periodiche.

Vedremo i seguenti risultati

\begin{itemize}
	\item Esistenza usando la forma ($**$), specifico per equazione delle onde.
	\item Esistenza usando la forma ($*$), che però richiede maggiore regolarità su $u_0$ e $u_1$.
	\item Unicità.
\end{itemize}

\textbf{Teorema 1.}
Dati $u_0 \in C^1_{\text{per}}$ allora esistono $c_0^0, c_0^1$ e $\varphi^+, \varphi^- \in C^2_{\text{per}}$ tali che la $u$ in ($**$) è di classe $C^2$ su $\R \times \R$, $2\pi$-periodica in $x$ e risolve (P).

\textbf{Lemma 4.}
Date $h, g \in C^1(\R)$ con $g$ primitiva di $h$ e $T > 0$ allora $g$ è $T$-periodica $\iff h$ è $T$-periodica e $\int_0^T h(x) \dd x = 0$.

\textbf{Dimostrazione.}
Notiamo che $h$ è $T$-periodica se e solo se $\forall x \; \int_{x}^{T+x} h(x) \dd x = \text{cost.}$
$$
\int_{x}^{T+x} h(x) \dd x = g \Big|_x^{T+x} = g(T + x) - g(x) = 0 \iff \text{$g$ è $T$-periodica}
$$

\textbf{Dimostrazione Teorema 1.}

\textit{Parte 1.}
Se $c_0^0, c_0^1 \in \R$ e $\varphi^+, \varphi^- \in C^2_\text{per}$ allora la $u$ data da ($**$) è $C^2$ su $\R \times \R$ e $2\pi$-periodica in $x$ e risolve $u_{tt} = v^2 u_{xx}$.
$$
\begin{aligned}
	u_{tt} &= [\ddot \varphi^+(x + vt) + \ddot \varphi^-(x - vt)] v^2 \\
	u_{xx} &= \ddot \varphi^+(x + vt) + \ddot \varphi^-(x - vt)
\end{aligned}
\implies u_{tt} = v^2 u_{xx}
$$

\textit{Parte 2.}
$\exists c_0^0, c_0^1 \in \R$ e $\varphi^+, \varphi^- \in C^2_\text{per}$ tali che la $u$ data da ($**$) soddisfa la condizione iniziale in (P), per $t = 0$, poste $\varphi^\pm = \varphi^\pm(x \pm v0)$
$$
\begin{cases}
	c_0^0 + \varphi^+ + \varphi^- = u_0 \\
	c_0^1 + v(\dot \varphi^+ - \dot \varphi^-) = u_1
\end{cases}
\implies
\begin{cases}
	\varphi^+ + \varphi^- = u_0 - c_0^0 \\
	(\varphi^+ - \varphi^-)' = (u_1 - c_0^1) / v \\
\end{cases}
$$
ed ora fissiamo $c_0^0 = \avint_{-\pi}^\pi u_0 \dd x$ e $c_0^1 = \avint_{-\pi}^\pi u_1 \dd x$. In questo modo possiamo applicare il lemma precedente ed ottenere
$$
\begin{cases}
	\varphi^+ + \varphi^- = g_0 \\
	(\varphi^+ - \varphi^-)' = g_1'
\end{cases}
\implies
\varphi^+ = \frac{1}{2}(g_0 + g_1)
\qquad
\varphi^- = \frac{1}{2}(g_0 - g_1)
$$
\qed

% ??
% \textbf{Osservazione.}
% Il termine ``$c_0^1 t$'' rappresenta il caso in cui è l'anello di base a ruotare.

\textbf{Teorema 2.}
Siano $u_0, u_1 \in C^0_\text{per}$ tali che $\sum n^2 |c_n^0| < +\infty$ e $\sum |n| \cdot |c_n^1| < +\infty$. Allora ($*$) definisce una funzione $u \colon \R \times \R \to \C$ di classe $C^2$, $2\pi$-periodica in $x$ che risolve (P).

\textbf{Dimostrazione.}
$$
u(t,x) = c_0^0 + c_0^1 t + \sum_{n \neq  0} 
(
\underbrace{\alpha^+ e^{in(x + vt)}}_{v^+_n} 
+ 
\underbrace{\alpha^- e^{in(x - vt)}}_{v^-_n}
)
$$

\textit{Passo 1.}
Dimostriamo che $u \in C^0(\R \times \R)$ e $2\pi$-periodica in $x$. 

La funzione $u$ soddisfa le condizioni di periodicità. Per mostrare la continuità è sufficiente mostrare che la serie converga totalmente su $\R \times \R$.
$$
\norm{v^\pm_n}_{L^\infty(\R \times \R)} = |\alpha^\pm| = O\left( |c_n^0| + \frac{|c_n^1|}{n} \right)
$$
che sono sommabili in $n$.

\textit{Passo 2.}
Mostriamo che $u \in C^2(\R \times \R)$. 

Abbiamo 
%
\begin{align*}
	& D_t^h D_x^k v_n^\pm = \alpha^\pm_n e^{in(x \pm vt)} (in)^k (ivn)^h \\
	\Longrightarrow & \norm{D_t^h D_x^k v^\pm_n}_{L^\infty(\R \times \R)} = |\alpha^\pm_n| \cdot |v|^h \cdot |n|^{k+h}
	= O(|c_n^0| \cdot |n|^{k+h} + |c_n^1| \cdot |n|^{k + h - 1}) 
\end{align*}
%
che è sommabile se $k + h \leq 2$ in $n$. La serie in ($*$) converge totalmente su $\R \times \R$ con tutte le derivate di ordine $\leq 2 \implies u$ è $C^2$.

\textit{Passo 3.}
Dimostriamo che $u$ risolve l'equazione $u_{tt} = v^2 u_{xx}$.

% Infatti $u$ la risolve $c_0^0 + c_0^1 t$ e $v^\pm_n(x, t)$ e derivate e serie commutano. 
$u$ risolve l'equazione perché derivata e serie commutano e per come abbiamo impostato (P$'$) $c_n(u(0, \curry)) = c_n(u_0) \implies u(0, \curry) = u_0$. $c_n(u_t(0, \curry)) = c_n(u_1) \implies u_t(0, \curry) = u_1$.

\textbf{Teorema 3.} (Unicità)
Se $u \colon I \times [-\pi, \pi] \to \C$ è $C^2$ in $x$ e $t$ e risolve (P) allora è unica.

\textbf{Dimostrazione.}
Si ripercorre la stessa dell'equazione del calore. Dimostriamo che i coefficienti $c_n(t) = c_n(u(t, \curry))$ definiti per $t \in I$ risolvono (P$'$)...

\section{Altre applicazioni della serie di Fourier}

\subsection{Disuguaglianza isoperimetrica}

Sia $D$ un aperto limitato con frontiera $C^1$ parametrizzata da un unico cammino $\gamma$ (quindi niente buchi o più di una componente connessa). Allora $L^2 \geq 4 \pi A$ dove $L$ è la lunghezza di $\pd D$ e $A$ è l'area di $D$. Inoltre vale l'uguale se e solo se $D$ è un disco.

\textbf{Dimostrazione.}

Possiamo scegliere $\gamma \colon [-\pi, \pi] \to \R^2 \simeq \C$ e $\gamma$ parametrizzazione di $\pd D$ in senso antiorario ed a velocità costante (da cui $|\dot\gamma(t)| = L / 2\pi$)

\textit{Passo 1.}
$$
L^2 = 2\pi \int_{-\pi}^\pi |\dot \gamma|^2 \dd t = 2\pi \norm{\dot \gamma}_2 = 4 \pi^2 \sum |c_n(\dot \gamma)|^2 = 4 \pi^2 \sum n^2 |c_n|^2
$$

\textit{Passo 2.}
$$
A \overset{(*)}{=} \frac{1}{2}\langle -i\dot\gamma, \gamma \rangle = \frac{1}{2} 2\pi \sum (-i (inc_n))c_n = \pi \sum n |c_n|^2
$$
Vediamo che vale questa formula per l'area usata in ($*$), poniamo $\gamma = \gamma_x + i \gamma_y$ allora
$$
\begin{aligned}
	\langle \dot\gamma, \gamma \rangle 
	&= \int_{-\pi}^\pi \dot\gamma \; \overline\gamma \dd t \\
	&= \int_{-\pi}^\pi (\gamma_x - i \gamma_y) (\dot\gamma_x + i \dot\gamma_y) \dd t = \\
	&= \int_\gamma (x - iy) (\dd x + i \dd y) = \\
	&= \int_D 2i \dd x \mathrm d y = 2i A
\end{aligned}
$$
\textit{Passo 3.}
Infine $L^2 = 4 \pi \sum n^2 |c_n|^2$ e $4\pi A = 4\pi \sum n |c_n|^2$, dunque segue subito che $L^2 \geq 4 \pi A$ e vale l'uguale se e solo se $n^2 = n$ o se $c_n = 0$ per ogni $n \implies \gamma(t) = c_0 + c_1 e^{it}$ che è una circonferenza di centro $c_0$ e raggio $|c_1|$.






% Conclusione capitolo SdF e applicazioni
\section{Appendice}

Studiamo alcune variazioni dell'equazione del calore.

\textbf{Nota.} Un problema del tipo $u_t = a(t) \cdot u_{xx}$ si può risolvere ripercorrendo i passaggi della risoluzione dell'equazione del calore.
Viceversa, il problema $u_t = a(x) \cdot u_{xx}$ non si può risolvere allo stesso modo, in quanto, non è vero che il prodotto di serie di Fourier ha come coefficienti il prodotto dei coefficienti.

Studiamo ora variazioni alle condizioni di bordo.

\textbf{Osservazione.} Quando proviamo a risolvere $u_t = u_{xx}$, passiamo alla serie di Fourier e deriviamo; per fare questo passaggio servono le condizioni al bordo\footnote{Anche se avevamo derivato le formule formalmente anche a posteriori l'ipotesi delle condizioni al bordo era necessaria.}; dunque, togliendo le condizioni di periodicità il sistema non funziona più molto bene.

Introduciamo delle varianti della serie di Fourier.

\begin{itemize}

	\item \textbf{Serie di Fourier su} \boldmath$[-\pi,\pi]^d$. Data \unboldmath$u \in L^2([-\pi,\pi]^2, \C)$, definiamo
	%
	$$
		u(x) = \sum_{\underline{n} \in \Z^d} c_{\ul{n}} e^{i \ul{n} x} \qquad 
		c_{\ul{n}} = c_{\ul{n}} (u) \coloneqq \frac{1}{(2\pi)^d} \int_{[-\pi,\pi]^d} u(x) e^{- i \ul{n} x} \dd x 
	$$
	%
	con base di Hilbert
	%
	$$
		\mc{F} = \left\{ \frac{e^{i \ul{n} x}}{(2\pi)^{d/2}} \colon n \in \Z^d \right\}
	$$
	%

	C'è da dimostrare che $\mc{F}$ è una base di Hilbert.

	\textbf{Dimostrazione} (idea). 
	\begin{itemize}

		\item Ortonormalità. È un conto [TO DO].


		\item Completezza. Si può dimostrare come per $d = 1$, oppure si usa il seguente lemma.

		\textbf{Lemma.} Sia $\mc{F}_1 \coloneqq \{ e_n^1 \}$ base di Hilbert di $L^2(X_1, \C)$ e $\mc{F}_2 \coloneqq \{ e_n^2 \}$ base di Hilbert di $L^2(X_2, \C)$. Allora, una base di Hilbert di $L^2(X_1 \times X_2, \C)$ è
		%
		$$
			\mc{F} = \left\{ e_{n_1,n_2}(x_1,x_2) \mymid e_{n_1}^1 (x_1) e_{n_2}^2(x_2) \right\}
		$$
		%

	\end{itemize}

	\textit{Formula chiave.}
	Se $u \in C_{\text{per}}^1 (\R^d) = \left\{ \text{funzioni $2\pi$-periodiche in tutte le variabili} \right\}$. Abbiamo che
	%
	$$
		c_{\ul{n}} (\nabla u) = i \ul{n} c_n(u), \qquad 
		c_{\ul{n}} (\Delta u) = - \left| \ul{n} \right|^2 c_{\ul{n}} (u) \mquad \text{se }  u \in \mc{C}_{\text{per}}^2
	$$
	%


	\item \textbf{Serie in seni.} Data $u \in L^2([0,\pi])$, allora 
	%
	$$
		u(x) = \sum_{n=1}^\infty b_n \sin(nx) \qquad 
		b_n = b_n(u) \coloneqq  \frac{2}{\pi} \int_0^\pi u(x) \sin(nx) \dd x
	$$
	%
	con base di Hilbert 
	%
	$$
		\mc{F} = \left\{ \sqrt{\frac{2}{\pi}} \sin(nx) \mymid n \geq 1 \right\}
	$$
	%

	\textbf{Dimostrazione.} Mostriamo l'ortonormalità e la completezza.

	\textit{Ortonormalità.} Sono conti. [TO DO]

	\textit{Completezza.} Data $u \in L^2([0,\pi])$. Sia $\tilde{u}$ l'estensione dispari a $[-\pi,\pi]$. Allora 
	%
	$$
		\tilde{u} = a_0 + \sum_{n=1}^{\infty} \tilde{a}_n \cos(nx) + \tilde{b}_n \sin(nx)
		\underset{\tilde{u} \text{ dispari}} = \sum_{n=1}^{\infty} \tilde{b}_n \sin(nx).
	$$
	%

	\textbf{Osservazione.} I coefficienti $\tilde{b}_n = b_n$. Si può vedere in diversi modi, un modo possibile è questo.
	%
	$$
		\tilde{b}_n = \frac{1}{\pi} \int_{-\pi}^{\pi} \tilde{u}(x) \sin(nx) \dd x 
		= \frac{2}{\pi} \int_{0}^{\pi} \tilde{u} \sin(nx) \dd x 
		= \frac{2}{\pi} \int_{0}^{\pi} u(x) \sin(nx) \dd x 
		= b_n.
	$$
	%

	\textit{Formula chiave.} Data $u \in \mc{C}^2 ([0,\pi])$ con condizioni al bordo $u(\curry, 0) = u(\curry, \pi)$. 
	Allora
	%
	$$
		b_n(\ddot u) = -n^2 b_n(u)
	$$
	%
	dove 
	\begin{align*}
		b_n(\ddot u) & \coloneqq \frac{2}{\pi} \int_0^\pi \ddot u(x) \sin(nx) \dd x \\
		& = \frac{2}{\pi} \left| \dot u(x) \sin(nx) \right|_0^\pi - \frac{2}{\pi} \int_{0}^{\pi} \dot u(x) \cos(nx) \dd x \\
		& = -n \frac{2}{\pi} \left| u(x) \cos(nx) \right|_0^\pi - n^2  \underbrace{\left( \frac{2}{\pi} \int_{0}^{\pi} u(x) \sin(nx) \dd x \right)}_{b_{n(x)}}
	\end{align*}
	\qed

\end{itemize}


\textbf{Applicazione} (della serie in seni). Risoluzione di EDP su $[0,\pi]$ con condizioni di Dirichlet (omogenee) al bordo.

\textbf{Esempio.} Risolvere
%
\begin{equation}
\label{eq:24nov-problema-1} \tag{P}
	\begin{cases}
		u_t = u_{xx} \quad \text{su} \mquad [0,\pi] \\
		u(\curry, 0) = u(\curry,\pi) = 0 \\
		u(0,\curry) = u_0
	\end{cases} 
\end{equation}

\textbf{Soluzione.} Poniamo $b_n^0 \coloneqq b_n (u_0)$.
Scriviamo $\ds u(t,x) = \sum_{n=1}^{\infty} b_n(t) \sin(nx) $ serie di seni in $x$.

Formalmente,
%
$$
	u_t = \sum_{n=1}^\infty \dot b_n(t) \sin(nx) \qquad 
	u_{xx} = \sum_{n=1}^{\infty} -n^2 b_n(t) \sin(nx) 
$$
%
Dunque, 
%
$$
	u_t = u_{xx} \longiff \dot b_n(t) = -n^2 b_n(t) \mquad \forall t \forall n
$$
%
Cioè $b_n(t)$ risolve il problema di Cauchy.
\begin{equation}
	\label{eq:24nov-problema-2} \tag{P'}
	\begin{cases}
		\dot y = -n^2 y \\
		y(0) = \dot b_n
	\end{cases}
\end{equation}

Ovvero $b_n(t) = b_n^0 e^{-n^2 t}$, da cui  
\begin{equation}
\label{eq:24nov-soluzione-1} \tag{$\ast$}
	u(t,x) = \sum_{n=1}^\infty \underbrace{b_n^0 e^{-n^2t} \sin(nx)}_{u_n}.
\end{equation}


\textbf{Teorema 1} (di esistenza nel futuro).
Se $u_0 \colon [0,\pi] \to \R$ è continua è $\ds \sum_n |b_n^0| < +\infty$ (basta $u_0 \in \mc{C}^1$ e $u(0) = u(\pi) = 0$).
Allora la $u$ in \eqref{eq:24nov-soluzione-1} è ben definita e continua su $[0,+\infty) \times \R$ e risolve \eqref{eq:24nov-problema-1}.

\textbf{Dimostrazione.} Dimostriamo il teorema per passi.

\textit{Passo 1.}
Mostriamo che $u$ è ben definita e continua su $(0,+\infty) \times \R$: studiamo la norma del sup. Sia $R = [0, +\infty) \times \R$.
%
$$
	\norm{u_n}_{L^\infty(R)} \leq |b_n^0| \Longrightarrow u_n \mquad \text{converge totalmente su } \R.
$$
%

\textit{Passo 2.} Mostriamo che $\mc{C}^\infty$ su $(0,+\infty) \times \R$.
Sia $R_\delta = (\delta, +\infty) \times \R$.
Stimiamo le derivate.
\begin{align*}
	D_t^k D_x^h u_n = b_n^0 (-n^2)^k e^{-n^2t} \cdot n^h \cdot \underbrace{\ldots}_{\star} \\
	\Longrightarrow \norm{D_t^k D_x^h u_n}_{L^\infty(R_\delta)} 
	= |b_n^0| \underbrace{e^{-n^2\delta} \cdot |n|^{2k+h}}_{\substack{\text{limitato in $n$} \\ \text{ perché è infinitesimo in $n$ }}}
\end{align*}
Allora le norme delle derivate sono sommabili per ogni $n$, dunque $u \in \mc{C}^\infty(R_\delta)$ per ogni $\delta$, da cui $u \in \mc{C}^\infty( (0,+\infty),\R)$.

\textit{Passo 3.} Mostriamo che la $u(t,x)$ definita in \eqref{eq:24nov-soluzione-1} risolve \eqref{eq:24nov-problema-1}.
\begin{itemize}

	\item $u$ risolve $u_t = u_{xx}$ per $t > 0$.
	Infatti, l'equazione è lineare per quanto mostrato al punto sopra e dunque posso scambiare serie e derivata.


	\item $u$ soddisfa la condizione iniziale $u(0,\curry) = u_0$, perché hanno gli stessi coefficienti di Fourier.


	\item Sono soddisfatte anche le condizioni al bordo, infatti
	%
	$$
	u(\curry,0) = u(\curry,\pi) = 0
	$$
	%
\end{itemize}
\qed

\textbf{Domanda.} Quale ipotesi su $u_0$ garantisce $\sum_n |b_n^0| < +\infty$? Basta $u_0 \in \mc{C}^1$ e $u(0) = u(\pi) = 0$.

\textbf{Teorema 2} (non esistenza nel passato).
Esiste $u_0 \colon [0,\pi] \to \R$ $\mc{C}^\infty$ (+ condizioni al bordo) tale che per ogni $\delta > 0$ \eqref{eq:24nov-problema-1} non ha alcuna soluzione $u \colon (-\delta,0] \times [0,\pi] \to \R$ continua e $\mc{C}^1$ in $t$ e $\mc{C}^2 $ in $x$.

\textbf{Teorema 3} (di unicità). [TO DO: aggiungere (è sempre lo stesso).]


% Esercitazione
%
% Lezione del 25 Novembre 2021
%

\section{Esercitazione del 25 Novembre 2021}

\textbf{Esercizio.}
Consideriamo l'equazione alle derivate parziali
\begin{equation}
	\tag{P}
	\begin{cases}
		u_{ttt}(t, x) = u_{xx}(t, x) & x \in [-\pi, \pi] \\
		u(\curry, \pi) = u(\curry, -\pi) \\
		u_x(\curry, \pi) = u_x(\curry, -\pi) \\
		u(0, \curry) = u_0 \\
		u_{x}(0, \curry) = u_1 \\
		u_{xx}(0, \curry) = u_2 \\
	\end{cases}
\end{equation}
ponendo $c_n^i \coloneqq c_n(u_i) \text{ per } n \in \Z$ per $i = 1, 2, 3$. Segue subito che il problema di Cauchy sui coefficienti è
\begin{equation}
	\tag{P$'$}
	\begin{cases}
		\dddot c_n(t) = -n^2c_n(t) \\
		c_n(0) = c_n^0 \\
		\dot c_n(0) = c_n^1 \\
		\ddot c_n(0) = c_n^2 \\
	\end{cases}
	\qquad
	\forall n \in \Z
\end{equation}
che ha polinomio caratteristico $p(\lambda) = \lambda^3 + n^2 \implies \lambda^3 = -n^2$ e dunque le soluzioni sono $\lambda_i = n^{2/3} \zeta_6^{2i - 1}$ con $\zeta_6$ una radice sesta dell'unità. Per comodità per $i = 1, 2, 3$ poniamo $z_i \coloneqq n^{2/3} \omega_i$ con $\omega_i$ soluzioni di $\omega^3 = -1$ che possiamo anche riscrivere come
$$
\omega_1 = \frac{1}{2} + \frac{\sqrt 3}{2}i
\qquad
\omega_2 = \frac{1}{2} - \frac{\sqrt 3}{2}i
\qquad
\omega_3 = -1
$$
Dunque per $n \in \Z$ e $n \neq 0$ la soluzione sarà
$$
\left\{
\begin{aligned}
	& c_n(t) = A_n e^{-z_1^n t} + B_n e^{-z_2^n t} + C_n e^{-z_3^n t} \\[1ex]
	& c_n(0) = c_n^0 = A_n + B_n + C_n \\[1ex]
	& c_n(1) = c_n^1 = A_n z_1^n + B_n z_2^n + C_n z_3^n \\[1ex]
	& c_n(2) = c_n^2 = A_n (z_1^n)^2 + B_n (z_2^n)^2 + C_n (z_3^n)^2
\end{aligned}
\right.
$$
e quindi otteniamo il sistema
$$
\implies
\left\{
\begin{aligned}
	& c_n^0 = A_n + B_n + C_n \\[1ex]
	& n^{-2/3} c_n^1 = A_n \omega_1 + B_n \omega_2 + C_n \omega_3 \\[1ex]
	& n^{-4/3} c_n^2 = A_n \omega_1^2 + B_n \omega_2^2 + C_n \omega_3^2
\end{aligned}
\right.
\qquad
\textcolor{lightgray}{\rightsquigarrow
\begin{pmatrix}
	1 & 1 & 1 \\
	\omega_1 & \omega_2 & \omega_3 \\
	\omega_1^2 & \omega_2^2 & \omega_3^2
\end{pmatrix}
\begin{pmatrix}
	A_n \\
	B_n \\
	C_n
\end{pmatrix}
=
\begin{pmatrix}
	c_n^0 \\
	n^{-2/3} c_n^1 \\
	n^{-4/3} c_n^2
\end{pmatrix}}
$$
e facendo conti si ottengono $A_n, B_n$ e $C_n$ e si scopre che 
[TODO: Controllare i conti con Mathematica]
$$
\begin{aligned}
	A_n e^{n^{2/3}(t/2 + i \sqrt 3 t / 2)} &\sim e^{n^{2/3}t / 2}
	&\xrightarrow{\;\,t \to \infty\,\;} \;\, \infty \\
	B_n e^{n^{2/3}(t/2 - i \sqrt 3 t / 2)} &\sim e^{n^{2/3}t / 2}
	&\xrightarrow{\;\,t \to \infty\,\;} \;\, \infty \\
	C_n e^{-n^{2/3}} &\sim e^{-n^{2/3} t}
	&\xrightarrow{t \to -\infty} \;\, \infty \\
\end{aligned}
$$
dunque in realtà anche se il problema in partenza sembrava ben definito in realtà non ha soluzione per alcun $t \in \R$.

\textit{Conti esatti con Mathematica:}
$$
\begin{aligned}
	& A_n \to \frac{c_n^0}{3}-\frac{c_n^2}{3 n^{4/3}}-\frac{(-1)^{2/3} c_n^2}{3 n^{4/3}}-\frac{(-1)^{2/3} c_n^1}{3
	n^{2/3}}, \\
	& B_n \to \frac{c_n^0}{3}-\frac{c_n^2}{6 n^{4/3}}+\frac{i c_n^2}{2 \sqrt{3} n^{4/3}}+\frac{c_n^1}{6 n^{2/3}}+\frac{i
	c_n^1}{2 \sqrt{3} n^{2/3}}, \\
	& C_n \to \frac{c_n^0}{3}+\frac{c_n^2}{3 n^{4/3}}-\frac{c_n^1}{3 n^{2/3}}
\end{aligned}
$$

\textbf{Esercizio.}
(Equazione del calore senza una condizione al bordo)
\begin{equation}
	\tag{P}
	\begin{cases}
		u_t = u_{xx} & x \in [-\pi, \pi] \\
		u(\curry, -\pi) = u(\curry, \pi) \\
		u(0, \curry) = u_0 = \cos(x / 2)
	\end{cases}
\end{equation}
\begin{enumerate}
	\item \textit{Esiste una soluzione?}

		Sì in quanto esiste anche con una condizione in più

	\item \textit{È unica?}

		Senza periodicità per $u_x$ non è vero in generale che $c_n(u_{xx}(t, \curry)) = -n^2 c_n(u(t, \curry))$.

\end{enumerate}
Cerchiamo una soluzione della forma $u(t, x) = \cos(x/2) \psi(t)$. Abbiamo che $u_t(t, x) = \dot \psi(t) \cos(x / 2)$ e $u_{xx}(t, x) = - \cos(x / 2) \psi(t) / 4$. Dunque $\dot \psi(t) = - \psi(t) / 4$ e $\psi(0) = 1 \implies \psi(t) = e^{-t / 4}$.

\textbf{Esercizio.}
\begin{equation}
	\tag{P}
	\begin{cases}
		u_t = u_{xx} & x \in [0, \pi] \\
		u(t, 0) = 0 & t \in \R \\
		u(t, \pi) = t & t \in \R \\
		u(0, \curry) = u_0
	\end{cases}
\end{equation}
L'equazione è lineare, cerchiamo $u(t, x) = v(t, x) + \psi(t, x)$ in modo che $v(t, x) = 0$ se $x = 0, \pi$ e $\psi(t, 0) = 0$ e $\psi(t, \pi) = t$ e $\psi(t, x) = t x / \pi$.

\textbf{Esercizio.}
\begin{equation}
	\tag{P}
	\begin{cases}
		u_t = u_{xxxx} & x \in [-\pi, \pi] \\
		u(\curry, \pi) = u(\curry, -\pi) \\
		u_x(\curry, \pi) = u_x(\curry, -\pi) \\
		u_{xx}(\curry, \pi) = u_{xx}(\curry, -\pi) \\
		u_{xxx}(\curry, \pi) = u_{xxx}(\curry, -\pi) \\
		u(0, \curry) = u_0
	\end{cases}
\end{equation}

\textbf{Esercizio.}
\begin{equation}
	\tag{P}
	\begin{cases}
		u_t = u_{xxxx} & x \in [0, \pi] \\
		u(\curry, 0) = u(\curry, \pi) = 0 \\
		u_{xx}(\curry, 0) = u_{xx}(\curry, \pi) = 0 \\
		u(0, \curry) = u_0
	\end{cases}
\end{equation}

\textbf{Esercizio.}
Sia $V$ il seguente insieme
$$
V \coloneqq \left\{ f \in L^1([1, +\infty]) \;\middle|\; |f(x)| \leq \frac{1}{x^2} \text{ per q.o. } x \right\}
$$
è compatto in $L^1$? e se al posto di $L^1$ avessimo $L^2$?

[TODO: Espandere]

Intuitivamente $V \supseteq \{ f \mid |f(x)| \leq 1 / 2 \text{ q.o. in } [1, 2] \}$ che non è compatto in quanto contiene famiglie di funzioni che ``oscillano molto'' costruite sull'idea della base di Haar.

\textbf{Esercizio.}
Trovare una funzione in $L^p([0, +\infty))$ tale però che $f \notin L^q$ per $q \neq p$.

Cercare $f$ della forma
$$
f(x) = \frac{1}{x^\alpha (a + (\ln x)^\beta)}
$$














% Conclusione capitolo SdF e applicazioni 
\subsection{Considerazioni finali su SdF e serie in seni}

Notiamo che l'efficacia èer la soluzione di certe EDP dipende dal fatto che
%
$$
	c_n(u) = in c_n(u) \qquad b_n(\ddot u) = -n^2 b_n(u)
$$
%
che segue (almeno formalmente) da $(e^{inx})' = in e^{inx}$ e $(\sin(nx))'' = -n^2 \sin(nx)$.

Cioè che $\left\{ e^{inx} / \sqrt{2\pi} \right\}$ è una base ortonormale di $L^2([-\pi,\pi],\C)$ di autovettori di $D$ e $\left\{ \sqrt{2/\pi} \sin(x)  \right\}$ è una base ortonormal di autovettori di $D^2$.

Analogamente per risolvere $u_t = \Delta u$ su $\Omega$, basterebbe avere $\{ e_n \}$ base ortonormale di $L^2(\Omega)$ fatta di autovettori del laplaciano.

Per avere una base ortonormale di autovettori di un operatore $T$ serve che $T$ sia autoaggiunto (almeno in dimensione finita).

\textbf{Definizione.} Dato $H$ spazio di Hilbert complesso o reale, $D$ sottospazio denso di $H$, $T \colon D \to H$ lineare (non necessariamente continuo), dico che $T$ è \textbf{autoaggiunto} se $\left<Tx,y \right> = \left<x,Ty \right>$ per ogni $x,y \in D$.

\textbf{Proposizione.} Dato $T$ come sopra
\begin{enumerate}

	\item Se $\lambda$ è autovalore di $T$ (ovvero tale che $\exists x \neq 0$ tale che $Tx = \lambda x$) allora $\lambda$ è reale.


	\item Dati $\lambda_1 \neq \lambda_2$ autovalori allora $V_{\lambda_1} \perp V_{\lambda_2}$ dove $V_\lambda \coloneqq \{ x \mid Tx = \lambda x \}$.

\end{enumerate}

\textbf{Nota.} In dimensione infinita manca un teorema spettrale, ovvero tale che $\ds \ol{\bigoplus_\lambda V_\lambda} = H$.


\textbf{Esempio 1.} Sia $H = L^2([-\pi,\pi],\C)$, $D = \left\{ u \in \mc{C}^2 (-\pi,\pi) \mymid u(-\pi) = u(\pi) \right\}$ e $T \colon D \to H$ tale che $u \mapsto iu$.
Mostrare che
\begin{enumerate}

	\item $T$ è autoaggiunto

	\item Gli autovalori di $T$ sono $\lambda_n = n$ con $n \in \Z$ $V_{\lambda_n} = V_n = \spn \left\{ e^{inx} \right\}$.

	\item $T$ non è continuo

\end{enumerate}


In questo caso esiste una base ortonormale di $L^2$ di autovettori di $T$. [TO DO: aggiustare].

\textbf{Dimostrazione.} 
\begin{enumerate}

	\item Dati $u,c \in D (= \cper^1)$, allora
	\begin{align*}
		& \left<Tu,v \right> = \int_{-\pi}^\pi i \dot u \ol{v} \dd x = \left| i u \ol{v} \right|_{-\pi}^\pi - \int_{-\pi}^\pi i u \ol{v} \dd x \\
		& = \int_{-\pi}^\pi u \ol{iv} \dd x 
		= \left<u, Tv \right>
	\end{align*}


	\item Questo è un esercizio di equazioni differenziali ordinarie. Risolviamo il problema
	%
	$$
	\begin{cases}
		-iu = \lambda u \quad \text{su } [-\pi,\pi] \\
		u(\pi) = u(-\pi)
	\end{cases} 
	$$
	%
	da cui $\dot u - i \lambda u = 0$, che ha polinomio associaot $t - i\lambda = 0$ con radice  $i\lambda$. In conclusione la soluzione del problema sopra è $\alpha e^{i\lambda x}$.

	Dalla condizione al bordo abbiamo che $\alpha e^{i\lambda \pi} = e^{-i\lambda \pi}$ dunque $e^{i\lambda \pi} = e^{- i\lambda \pi} \longiff e^{2i\lambda \pi} = 1 \longiff \lambda \in \Z$.


	\item Siccome gli autovalori sono illimitati, $T$ non è continuo.
\qed

\end{enumerate}


\textbf{Esempio 3.} Sia $H = L^2([-\pi,\pi], \C)$ $D = \{ u \in \mc{C}^1(-\pi,\pi) \}$ e $T \colon D \to H$ tale che $u \mapsto i\dot u$.

\textbf{Dimostrazione.} Dati $u,v \in D$ abbiamo
\begin{align*}
	\left<Tu,v \right> & = \int_{-\pi}^{\pi} i \dot u \ol{v} \dd x  = \left| i u \ol{v} \right|_{-\pi}^\pi - \int_{-\pi}^\pi i u \ol{\dot v} \dd x \\
	& = i (u(\pi) \ol{v}(\pi) - u(- \pi) \ol{v}(-\pi)) + \left<u, Tv \right> \neq \left<u, Tv \right>.
\end{align*}
In quanto, in generale, il termine $u(\pi) \ol{v}(\pi) - u(- \pi) \ol{v}(-\pi)$ è diverso da zero.


\textbf{Esercizio.} Cercare $T \colon L^2([0,1]) \to L^2([0,1])$ continuo autoaggiunto senza autovalori.

\textit{Suggerimento.} Cercare $T$ del tipo $T \colon u \mapsto gu$ con $g \in L^\infty$.


\chapter{Trasformata di Fourier}

% Trasformata di Fourier: introduzione
Data $f \colon \R \to \C$ poniamo
%
$$
	f(x) \overset{(\ast)}{=} \frac{1}{2\pi} \int_{-\infty}^\infty \hat{f}(y) e^{iyx} \dd y 
	\qquad \hat{f}(y) \coloneqq \int_{-\infty}^\infty f(x) e^{-iyx} \dd x.
$$
%
Dove $\hat{u}$ si chiama \textit{trasformata di Fourier}\footnote{Sostituisce la serie di Fourier quando si passa da funzioni su $\R$ $2\pi$-periodiche a funzioni su $\R$.} di $u$ e la formula $(\ast)$ si dice \textit{formula di inversione}.

\textit{Derivazione formale} (della formula di inversione).
Prendiamo $f \in \mc{C}_C^1(\R,\C)$ e $\delta > 0$ tale che $\supp(f) \subset [-\pi / \delta, \pi / \delta]$. 

Scriviamo $f$ in serie di Fourier su $[-\pi / \delta, \pi / \delta]$ (serve un cambio di variabile per ricondursi alla serie di Fourier su $[-\pi,\pi]$).
%
\begin{gather*}
	f(x) = \sum_{n \in \Z} c_n^\delta (f) e^{in \delta x} \\
	c_n^\delta (f) \coloneqq \frac{\delta}{2\pi} \int_{-\pi / \delta}^{\pi / \delta} f(x) e^{-in \delta x} \dd x
	= \frac{\delta}{2\pi} \int_{-\infty}^\infty f(x) e^{-in \delta x} \dd x
	= \frac{\delta}{2\pi} \hat{f}(n\delta).
\end{gather*}
Dunque,
%
$$
	f(x) = \frac{1}{2\pi} \sum_{n \in \Z} \frac{\delta}{2\pi} \underbrace{\hat{f}(n\delta) e^{i(n\delta)x}}_{\hat{f}(y) e^{iyx} \; \text{calcolata in } y = n\delta}
$$
%
dove $\ds \sum_{n \in \Z} \frac{\delta}{2\pi}\hat{f}(n\delta) e^{i(n\delta)x} $ è la somma di Rienmann di $\ds \int_{-\infty}^{\infty} \hat{f} e^{iyx} \dd y$.
Dunque
%
$$
	f(x) = \frac{1}{2\pi} \sum_{n \in \Z} \frac{\delta}{2\pi} \hat{f}(n\delta) e^{i(n\delta)x}
	\xrightarrow{\delta \to 0}
	\frac{1}{2\pi} \int_{-\infty}^{\infty} \hat{f}(y) e^{iyx} \dd y.
$$
%
Quest'ultimo passaggio non è giustificato rigorosamente ma si può rendere rigoroso per $f \in \mc{C}_C^1(\R)$.

\vss

\textbf{Definizione.} Data $f \in L^1(\R; \C)$ la \textbf{trasformata di Fourier} $\hat{f}$ è definita da
%
$$
	\hat{f}(y) = \int_{-\infty}^\infty f(x) e^{-ixy} \dd x \quad \forall y \in \R.
$$
%

\newpage

\textbf{Teorema.} Data $f \in L^1(\R;\C)$, allora
\begin{enumerate}

	\item $\hat{f}$ è ben definita in ogni punto di $\R$.

	\item Vale $\norm{\hat{f}}_\infty \leq \norm{f}_1$.

	\item $\hat{f}$ è continua

	\item $\hat{f}$ è infinitesima.

\end{enumerate}

\textbf{Dimostrazione.} 
\begin{enumerate}

	\item $\hat{f}(y)$ è ben definita per ogni $y \in \R$. Infatti, $f(x) e^{-iyx} \in L^1$ dato che
	%
	$$
		\int_{-\infty}^\infty \left| f(x) e^{-iyx} \right| \dd x 
		= \int_{-\infty}^\infty \left| f(x) \right| \dd x
		= \norm{f}_1.
	$$
	%


	\item $\norm{\hat{f}}_\infty \leq \norm{f}_1$. Infatti,
	%
	$$
		|\hat{f}|_\infty \leq \int \left| f(x) e^{-iyx} \right| \dd x = \norm{f}_1
	$$
	%


	\item $\hat{f}$ è continua. Se $y_n \to y$, allora
	%
	$$
		\hat{f}(y_n) = \int_{-\infty}^\infty f(x) e^{-ixy_n} \dd x \xrightarrow{n \to \infty}
		\int_{-\infty}^\infty f(x) e^{-iny} \dd x = \hat{f}(y)
	$$
	%
	per convergenza dominata. Infatti, la convergenza puntuale segue dalla continuità dell'esponenziale; mentre la dominazione è data da $|f(x) e^{-iyx}| = |f(x)|$.


	\item $\hat{f}(y) \xrightarrow{y \to \pm \infty} 0$ per il lemma di Rienmann-Lebesgue.

\end{enumerate}
\qed



% Esercitazione
\section{Proprietà della trasformata di Fourier}

Data $f \in L^1(\R; \C)$ abbiamo posto
$$
	\forall y \in \R
	\qquad 
	\mathcal F(f)(y) = \hat f(y) \coloneqq \int_{-\infty}^{+\infty} f(x) e^{-iyx} \dd x
$$
ed abbiamo visto che

\textbf{Teorema 1.}
$\hat f \in C_0(\R; \C)$ e $\| \hat f \|_\infty \leq \norm{f}_1$.

\mybox{%
\textbf{Proposizione 2.}
Data $f \in L^1(\R; \C)$ allora
\begin{enumerate}
	\item $\forall h \in \R$ vale $\hat{\tau_h f} = e^{-ihy} \hat f$
	\item $\forall h \in \R$ vale $\hat{e^{ihx} f} =  \tau_h \hat f$
	\item $\forall \delta \neq 0$ vale $\hat{\sigma_\delta f} = \hat f(\delta y)$
\end{enumerate}
}

\textit{Derivazione.}
Partendo dalla formula di inversione
$$
\begin{aligned}
	f(x) & = \frac{1}{2\pi} \int \hat f(y) e^{iyx} \dd y \\
	f(x - h) & = \frac{1}{2\pi} \int \underbrace{\hat f(y) e^{-ihy}}_{=\hat{f(x-h)}} e^{iyx} \dd y \\
\end{aligned}
$$

\textbf{Dimostrazione.}
Facciamo il calcolo diretto
$$
\begin{aligned}
	\hat{\tau_h f}
	& = \int_{-\infty}^{+\infty} f(x - h) e^{-ixy} \dd x =\\
	& = {\footnotesize \left(\begin{gathered} t = x - h \\ \dd t = \dd x \end{gathered}\right)} 
	= \int_{-\infty}^{+\infty} f(t) e^{-i(t + h)y} \dd t = \\
	& = e^{-ihy} \int_{-\infty}^{+\infty} f(t) e^{-ity} \dd t = e^{-ihy} \hat f(t).
\end{aligned}
$$
Analogamente seguono anche le altre
\qed

\mybox{%
\textbf{Proposizione 3.}
Sia $f \in C^1(\R; \C)$ con $f, f' \in L^1$ allora $\hat{f'} = iy\hat f$ (da confrontare con $c_n(f') = in c_n(f)$ nel caso della serie di Fourier).
}

\textit{Derivazione.} Si deriva la formula di inversione
$$
	f'(x) = \frac{1}{2\pi} \int_{-\infty}^{+\infty} \hat f(y) i y e^{ixy} \dd y
$$

\textbf{Dimostrazione.}
Vediamo prima una dimostrazione che non funziona e cerchiamo di aggiustarla. Abbiamo che
$$
	\hat{f'}(y) 
	= \int_{-\infty}^{+\infty} f'(x) e^{-iyx} \dd x
	= \underbrace{\left[ f(x) e^{-iyx} \right]_{-\infty}^{+\infty}}_{(\star)} + i y \int_{-\infty}^{+\infty} f(x) e^{-iyx} \dd x = iy \hat f(y)
$$
serve che $f(x) \to 0$ per $|x| \to +\infty$ per far sì che $(\star) = 0$.

\textbf{Nota.} $f \in C \cap L^1 $ non implica che $f(x) \xrightarrow{x \to \pm \infty}$, tuttavia $f \in C^1 \cap L^1$ e $f' \in L^1$ sì, però la dimostrazione è più complicata.

Argomentiamo come segue: $f \in L^1 \implies \liminf_{|x| \to \infty} |f(x)| = 0$, in quanto se $\liminf_{|x| \to \infty} |f(x)| = \delta > 0$ allora la funzione sarebbe $> \delta$ per $|x| \to +\infty$ ed avrebbe integrale $+\infty$, dunque esistono due successioni $a_n \to -\infty, b_n \to +\infty$ tali che $f(a_n) \to 0$ e $f(b_n) \to 0$ quindi come prima abbiamo
$$
\begin{aligned}
	\hat{f'}(y) 
	& \overset{(\ast)}{=} \lim_n \int_\R \One_{[a_n, b_n]} f'(x) e^{iyx} \dd x = \\
	& = \lim_n \int_{a_n}^{b_n} f'(x) e^{iyx} \dd x = \\
	& = \lim_n \bigg( \underbrace{\left[ f(x) e^{-iyx} \right]_{a_n}^{b_n}}_{\to 0} + i y \int_{a_n}^{b_n} f(x) e^{-iyx} \dd x \bigg) = \\
	& = \lim_n iy \int_{a_n}^{b_n} f(x) e^{-iyx} \dd x = \\
	& = iy \hat f(y) \\
\end{aligned}
$$
dove $(\ast)$ segue dal teorema di convergenza dominata.
\qed

\mybox{%
\textbf{Proposizione 4.} Sia $f \in L^1$ con $xf \in L^1$, allora $\hat{f} \in C^1(\R; \C)$ e $(\hat f)' = - \hat{i x f}$.
}

\textbf{Dimostrazione.} Per il teorema di derivazione sotto il segno di integrale:
$$
	\hat f(y) = \int_{-\infty}^{+\infty} f(x) e^{-ixy} \dd x
	\implies
	(\hat{f})'(y) = \int_{-\infty}^{+\infty} f(x) (-ix) e^{-ixy} \dd x = \hat{-ix f}
$$

\textbf{Proposizione 5.} (Derivazione sotto segno di integrale)
Sia $I$ un intervallo di $\R$, $E$ misurabile in $\R^d$ e $g \colon I \times E \to \C$ tale che
\begin{enumerate}
	\item $g(\curry, x) \in C^1(I)$ per q.o. $x \in E$.
	\item $\exists h_0, h_1 \in L^1(E)$ tali che
		$$
		|g(t, x)| \leq h_0(x) 
		\quad
		\text{e}
		\quad
		\left|\frac{\pd g}{\pd t}(t, x)\right| \leq h_1(x)
		$$
\end{enumerate}
allora $G(t) \coloneqq \int_E g(t, x) \dd x$ è ben definita per ogni $t \in I$ e $G \in C^1(I)$ e
$$
G'(t) = \int_E \frac{\pd}{\pd t}g(t, x) \dd x
$$
\textbf{Traccia dimostrazione.}
\begin{itemize}
	\item \textit{Passo 1:} $G(t)$ e $\tilde G(t)$ sono ben definite $\forall t \in I$ (grazie alla dominazione) e continue in $t$ (usando convergenza dominata e le dominazioni)
	\item \textit{Passo 2:} Dobbiamo far vedere che $G$ è $C^1$ con derivata $\tilde G$, si usa la seguente forma del teorema fondamentale del calcolo integrale
		$$
		\forall t_0, t_1 \in I \text{ con } t_0 < t_1
		\qquad
		G(t_1) - G(t_0) \overset{(*)}{=} \int_{t_0}^{t_1} \tilde G(t) \dd t
		$$
		ed usando Fubini-Tonelli in ($*$).
\end{itemize}

\mybox{%
\textbf{Proposizione 6.} (Prodotto di convoluzione e trasformata di Fourier).
Siano $f_1, f_2 \in L^1(\R; \C)$, allora $f_1 \ast f_2 \in L^1$ (già visto) e vale
$$
\mathcal F(f_1 \ast f_2) = (\mathcal F f_1) \cdot (\mathcal F f_2)
$$
}

\textbf{Dimostrazione.}
$$
\begin{aligned}
	\hat{f_1 \ast f_2}(y)
	&= \int f_1 \ast f_2 (x) e^{-ixy} \dd x = \\
	&= \iint f_1(x - t) f_2(t) \dd t \, e^{-ixy} \dd x = \\
	&= \int \left(\int f_1(x - t) e^{-i(x - t)y} \dd x \right) f_2(t) e^{-ity} \dd t = \\
	&= \int \hat f_1(y) f_2(t) e^{-ity} \dd t = \hat{f_1}(y) \cdot \hat{f_2}(y)
\end{aligned}
$$

\textbf{Definizione.}
Data $g \in L^1(\R; \C)$ definiamo l'\textbf{antitrasformata di Fourier} di $g$ la funzione
$$
	\check g(x) \coloneqq \int_{-\infty}^{+\infty} g(y) e^{ixy} \dd y.
$$
Cioè $\check g(x) = \hat g(-x)$ e scriviamo anche $\check g = \mathcal F^* g$. Effettivamente $\mathcal F^*$ è l'aggiunto di $\mathcal F$, almeno formalmente\footnotemark infatti abbiamo
$$
	\langle \mathcal F f, g \rangle
	= \iint f \overline{e^{ixy} g(y)} \dd x \dd y
	= \int f(x) \overline{\check g(x)} \dd x
	= \langle f, \mathcal F^* g \rangle.
$$
\footnotetext{In $L^1$ non è definito il prodotto scalare.}

\mybox{%
\textbf{Teorema 7.}
Data $f \in L^1(\R; \C)$ tale che $\hat f \in L^1(\R; \C)$ allora
$$
	\foralmostall x \in \R
	\qquad
	\mathcal F^* \mathcal F f = 2 \pi f
	\qquad
	\text{cioè }
	\int \hat f(x) e^{ixy} \dd y = 2\pi f(x)
$$
}

\textbf{Nota.} Una funzione continua e infinitesima non è, in generale, una funzione $L^1$; in particolare, l'ipotesi $\hat{f} \in L^1$ è necessaria e non deriva dalle proprietà già note.

\textbf{Dimostrazione.}
\textit{Dimostrazione scorretta} (passando dalla Delta di Dirac dei fisici):
$$
\begin{aligned}
	\mathcal F^* \mathcal F f 
	& = \int_{-\infty}^{+\infty} \hat f(y) e^{iyx} \dd y = \\
	& = \iint f(t) e^{-iyt} \dd t e^{ixy} \dd y = \\
	\substack{\text{non si può} \\ \text{usare Fubini\footnotemark}}\to & = \int f(t) \underbrace{\int e^{i(x-t)y} \dd y}_{\text{``$\delta(x-t)$''}} \dd t = f(x)
\end{aligned}
$$
% \addtocounter{footnote}{1}
\footnotetext{Infatti $f \in L^1$ non basta per garantire che $\iint |f(t)| < \infty$.}

\textit{Dimostrazione vera:} scegliamo una funzione ausiliaria $\varphi \colon \R \to \R$ tale che
\begin{enumerate}
	\item $\varphi(0) = 1$ continua in $0$ e $\varphi$ limitata
	\item $\varphi \in L^1$
	\item $\check\varphi \in L^1$
\end{enumerate}
e poniamo $g_\delta(x) \coloneqq \ds \int_{-\infty}^{+\infty} \hat f(y) \varphi(\delta y) e^{ixy} \dd y$.
\begin{itemize}
	\item \textit{Passo 1.}
		$g_\delta(x) \to \mathcal F^* \mathcal F f(x)$ per ogni $x \in \R$ per convergenza dominata. Infatti:
		$$
			\int \hat f(y) e^{iyx} \varphi(\delta y) \dd y \xrightarrow{\delta \to 0} 
			\int \hat f(y) e^{iyx} \dd y
		$$
		e come dominazione usiamo $|\hat f(y) e^{iyx} \varphi(\delta y)| \leq |\hat f(y)| \cdot \norm{\varphi}_\infty$

	\item \textit{Passo 2.} Per Fubini Tonelli vale 
		\begin{multline*}
			g_\delta(x) = \int \hat{f} e^{ixy} \myphi(\delta y) \dd y = \int \left( \int f(t) e^{-ity} \dd t \right) e^{ixy} \varphi(\delta y) \dd y \\
			\overset{\text{Fubini}}{=} \int f(t) \left( \int \myphi(\delta y) e^{i(x-t)y} \dd y \right) \dd t,
		\end{multline*}
		dato che
		$$
			\iint \left| f(t) e^{-ity} e^{ixy} \myphi(\delta y) \right| \dd t \dd y
			= \iint \left| f(t) \right| \left| \myphi(\delta y) \right| \dd t \dd y
			= \norm{f}_1 \cdot \norm{\myphi(\delta \curry)}_1 < +\infty.
		$$
		Dunque otteniamo
		$$
		\begin{aligned}
			g_\delta(x) & = \int f(t) \overbrace{\left( \int \myphi(\delta y) e^{i(x-t)y} \dd y \right)}^{\mathclap{\substack{\text{antitrasf. di $\myphi(\delta y)$} \\ \text{calcolata in $x-t$ ovvero $\sigma_\delta \check{\myphi}(x-t)$}}}} \dd t \\
			& = \int f(t) \sigma_\delta \check{\myphi}(x-t) \dd t = f \ast \sigma_\delta \check{\myphi}(x).
		\end{aligned}
		$$

	\item \textit{Passo 3.} Siccome $f \in L^1$ e $\check \myphi \in L^1$, per il teorema di approssimazione per convoluzione segue
	$$
		g_\delta (x) = f \ast \sigma_\delta \check \myphi(x) \xrightarrow{\delta \to 0} mf(x) \quad \text{in } L^1, \qquad \text{dove} \qquad  m = \int \check \myphi(y) \dd y.
	$$
	
	\item \textit{Passo 4.} Usando il primo ed il terzo passo otteniamo $\mathcal F^* \mathcal F f = m f$ per quasi ogni $x$, in quanto la convergenza puntuale e quella in $L^1$ devono essere compatibili. 
	% in particolare, la convergenza in $L^1$ a meno di sottosuccessioni equivale alla convergenza puntuale e dunque coincidono.

	\item \textit{Passo 5.} Resta da dimostrare che $m = 2\pi$. Prendendo $\varphi(y) = e^{-|y|}$ segue che
		$$
			\check \varphi(x) = \frac{2}{1 + x^2}
		$$
		e dunque $m = 2\pi$. In realtà vale per ogni $\varphi$ che verifica le condizioni dell'ipotesi.
\end{itemize}
\qed

\mybox{%
\textbf{Corollario 8.}
Date $f_1, f_2 \in L^1$ tali che $\hat f_1 = \hat f_2 \implies f_1 = f_2$ quasi ovunque cioè $\mathcal F$ è iniettiva, cioè $f$ è univocamente determinata da $\hat f$.
}

\textbf{Dimostrazione.} Per ipotesi, $\hat{f_1} - \hat{f_2} = \hat{f_1 - f_2} = 0$.
Applicando il Teorema 7 a $\hat{f_1 - f_2}$ (possiamo farlo perché $0 \in L^1$) otteniamo
%
$$
	0 = \int \hat{f_1 - f_2}(x) e^{ixy} \dd y = 2\pi (f_1(x) - f_2(x)) 
	\Rightarrow f_1(x) = f_2(x) \quad \foralmostall x \in \R.
$$
%
\qed


\textbf{Esercizio.}
Date $f_1, f_2 \in L^1([-\pi, \pi]; \C)$ e tali che per ogni $n \in \Z$ vale $c_n(f_1) = c_n(f_2)$ allora $f_1 = f_2$ quasi ovunque (e $c_n(f) = 0$ per ogni $n \implies f = 0$ q.o.).

\textbf{Osservazione.} Data $(f_n)$ successione di funzioni in $L^1$ con $f_n \to f$ in $L^1$, allora $\hat{f_n} \to \hat{f}$ uniformemente\footnote{Basta applicare la definizione di convergenza uniforme più il teorema di convergenza dominata.}. 











% Esercitazione
\section{Esercitazione del 2 dicembre}

Ricordiamo la definizione della trasformata di Fourier
%
$$
	\mc{F}(f) = \hat{f}(y) = \int_{\R} f(x) e^{-ixy} \dd x, \quad f \in L^1(\R;\C), \mquad f \in L^1(\R)
$$
%
dove $\ds \mc{F} \colon L^1(\R;\C) \to L^\infty(\R) \cap C_0(\R)$, in quanto $\norm{\hat{f}}_\infty \leq \norm{f}_1$.


\textbf{Proprietà.} Ricordiamo le proprietà viste a lezione.

\begin{enumerate}
	\item $\ds \hat{\tau_h f}(y) = e^{-iyh} \hat{f}(y)$ per ogni $h \in \R$, dove $\sigma_h f(x) = f(x - h)$


	\item $\ds \hat{e^{ihx} f}(y) = \tau_h \hat{f}(y) $


	\item Legame tra trasformata e derivata.

	\begin{itemize}

		\item $f \in C^1(\R)$, $f,f' \in L^1(\R,\C)$, allora $\ds \hat{f'}(y) = iy \hat{f}(y)$.


		\item $f \in L^1(\R;\C)$ e $xf(x) \in L^1(\R;\C)$, $(1 + |x|) f(x) \in L^1(\R;\C)$, allora $\ds \hat{f} \in C^1(\R)$ e $(\hat{f})' = -i \hat{xf(x)}$.

	\end{itemize}

	\textbf{Nota.} Le ipotesi $f \in L^1(\R;\C)$ e $xf(x) \in L^1(\R;\C)$ sono equivalenti a $(1 + |x|) f(x) \in L^1(\R;\C)$.


	\item  Vale $\hat{f \ast g} = \hat{f} \ast \hat{g}$.
	
\end{enumerate}

Riportiamo un esercizio già posto con una soluzione alternativa. 

\textbf{Esercizio.} Dire se esiste $v \in L^1$ non banale tale che 
\begin{itemize}

	\item $v \ast v = v$

	\item $g \ast v = g$ per ogni $g \in L^1$.

\end{itemize}

\textbf{Soluzione.} La risposta è no per entrambi i punti. Infatti,

\begin{itemize}

	\item Se per assurdo valesse tale identità, passando alle trasformate si avrebbe 
	%
	$$
		\hat{v \ast v } = (\hat{v})^2 \Longrightarrow \hat{v}(\hat{v} - 1) = 0.
	$$
	%
	Ovvero, $\hat{v} = \{0,1\}$. Osserviamo subito che non è possibile che $\hat{v}$ assuma entrambi i valori in quanto funzione continua; d'altra parte non è possibile che $\hat{v} = 1$, in quanto è anche infinitesima, dunque $\hat{v} = 0 \Longrightarrow v = 0$.


	\item Analogamente al punto precedente si avrebbe $\hat{v} = 1$ ma ciò non è possibile.

\end{itemize}

\vss

\textbf{Esercizio 1.} Calcolare la trasformata di Fourier della funzione $\ds f(x) = e^{-|x|}$.

\textbf{Soluzione.}
Abbiamo
\begin{align*}
	\hat{f}(y) & = \int_{-\infty}^\infty e^{-|x|} e^{-ixy} \dd x 
	= \int_{-\infty}^\infty e^{-|x|} \cos(xy) \dd x 
	- i \int_{-\infty}^\infty \overbrace{e^{-|x|} \sin(xy)}^{\substack{\text{integrale definito} \\ \text{di fuzione dispari}} \; = \; 0 } \dd x
	= 2 \int_{0}^{\infty} e^{-|x|} \cos(xy) \dd x \\
	& = 2 \int_0^\infty e^{-x} \re e^{ixy} \dd x
	= 2 \int_0^\infty \re \left( e^{-x} \cdot e^{ixy} \right) \dd x
	= 2 \re \left[ \int_0^\infty e^{-x} e^{ixy} \right] \dd x
	= 2 \re \left| \frac{e^{x(iy - 1)}}{(iy - 1)} \right|_0^\infty \\
	& = 2 \re \left[ -\frac{1}{iy - 1} + \underbrace{\lim_{x \to \pm \infty} \frac{e^{x(iy -1)}}{iy - 1}}_{= 0}\right]
	= 2 \re \left[ -\frac{1}{iy - 1} \right]
	= 2\re \left[ -\frac{1}{iy-1} \cdot \frac{iy + 1}{iy + 1} \right] \\
	& = 2\re \left[ \frac{iy + 1}{1 + y^2} \right].
\end{align*}

In conclusione, $\ds \hat{f}(y) = \frac{2}{1 + y^2}$.

\vss

\textbf{Esercizio 2.} Calcolare la trasformata di Fourier della funzione $\ds f(x) = \frac{1}{1 + x^2}$.

\textbf{Soluzione.}
Calcoliamo $\hat{f}(y) = \int_\R \frac{e^{-ixy}}{1 + x^2} \dd x$.
Dal fatto che $f \in L^1$ e usando il teorema di convergenza dominata, possiamo scrivere $\hat{f}(y)$ come
%
$$
	\hat{f}(y) = \lim_{R \to +\infty} \int_{-R}^R \frac{e^{-ixy}}{1 + x^2} \dd x
$$
%

\textit{Idea.} Calcolare questo integrale con il metodo dei residui, ponendo $\ds \frac{e^{-ixy}}{1 + x^2} = \left. \frac{e^{-izy}}{1 + z^2} \right|_{z \text{ reale}}$

Per il teorema dei residui:
%
$$
	\int_{B_r} g(z) \dd z = 2\pi i \sum_{\mathclap{\substack{z_i \text{ singolarità} \\ \text{di } g \text{ in } B_r}}} \; \operatorname{res} (g,z_i) 
$$
%
Dove
%
$$
	\int_{B_r} g(z) \dd z = \int\limits_{\mathclap{\gamma_r \coloneqq \text{bordo  sotto}}} g(z) \dd x \quad 
	+ \quad \underbrace{\int\limits_{\mathclap{\gamma \coloneqq \text{bordo sopra}}} g(x)}_{\xrightarrow{r \to \infty} 0}
	= \int_{-r}^{r} \frac{e^{-ixy}}{1 + x^2} \dd x.
$$
%

\begin{itemize}

	\item Verifichiamo che $\ds \int_{\gamma} g(x) \xrightarrow{r \to \infty} 0$ dove $[0,\pi] \ni \theta \xmapsto{\gamma} r e^{i\theta}$.

	Poniamo $z = x + it$, dunque $yz = xy + yit$, da cui $\ds g(z) = g(x + it) = \left( e^{-ixy} e^{ty} \right) / (1 + (x + it)^2)$.

	Dunque,
	%
	$$
		\int_0^\pi g(e^{i\theta}r) r \dd \theta
		\Longrightarrow \int_0^\pi \frac{e^{-ri\cos \theta} e^{r \sin\theta y}}{(1 + r^2 e^{i2\theta})} r \dd \theta
		\xrightarrow{r \to \pm \infty} 0 \quad \text{se } y < 0. 
	$$
	%
	Per il caso $y > 0$ si ripercorre lo stesso procedimento ma si utilizza la curva $[\pi,2\pi] \ni \theta \xmapsto{\gamma} r e^{i\theta}$.


	\item Calcoliamo i residui: l'unico residuo di $g$ è nel punto $i$ (nel caso $y > 0$ la singolarità è in $-i$).
	%
	$$
		\lim_{r \to \pm \infty} 2\pi i \operatorname{res}(g,i) = \pi e^{y}
	$$
	%
	Considerando anche il caso $y > 0$ la trasformata di Fourier diviene $\pi e^{-|y|}$.

\end{itemize}

[TO DO]. Riportare il teorema dei residui con i metodi di base per calcolare i residui?



% Trasformata di Fourier su L^2

\section{Trasformata di Fourier su $L^2$}

Abbiamo visto che la \textit{serie di Fourier} si definisce naturalmente su $L^2$ (uno spazio di Hilbert) mentre la \textit{trasformata di Fourier} ha bisogno di $L^1$ che non è uno spazio di Hilbert. Vedremo ora come estendere la trasformata di Fourier ad $L^2$ e come poter fare i conti.

\textbf{Proposizione 1.}
Data $f \in L^1(\R; \C) \cap L^2(\R; \C)$ vale $\| \hat f\, \|_2 = \sqrt{2\pi} \norm{f}_2$.

\textbf{Teorema 2.}
$\mathcal F$ si estende per continuità da $L^1 \cap L^2$ a tutto $L^2$ e $\mathcal F / \sqrt{2\pi}$ risulta essere un'isometria (come operatore a valori in $L^2$).

\textbf{Corollario 3.} (Identità di Plancherel).
$\forall f_1, f_2 \in L^2(\R; \C)$ vale $\langle \hat{f_1}, \hat{f_2} \rangle = 2\pi \langle f_1, f_2 \rangle$.

\textbf{Osservazione.}
Come si può calcolare $\hat f$ per $f \in L^2 \setminus L^1$? Se per quasi ogni $y \in \R$ esiste il limite
$$
\lim_n \underbrace{\int_{-n}^{n} f(x) e^{-ixy} \dd x}_{\hat{f_n}(y)}
$$
allora coincide con $\hat f(y)$.

Infatti, per ogni $n$ posto $f_n \coloneqq f \cdot \One_{[-n, n]}$ abbiamo che $\lim_n \int_{-n}^n f(x) e^{-ixy} \dd x = \hat{f_n}(x)$.
A questo punto, osserviamo che $f_n \to f$ in $L^2$ (da controllare per esercizio) e quindi $\hat f_n \to \hat f$ in $L^2$ (segue dalla continuità della trasformata). Siccome per ipotesi $\hat f_n$ converge puntualmente quasi ovunque allora $\hat f_n \to \hat f$ puntualmente quasi ovunque.

Intuitivamente, il Teorema 2 e l'identità di polarizzazione danno il Corollario 3. mentre il Teorema 2 segue dalla Proposizione 1. più un fatto noto usando che $L^1 \cap L^2$ è denso in $L^2$.

\textbf{Fatto Noto.} Dati $X$ e $Y$ spazi metrici, $Y$ completo e $D$ denso in $X$, $g \colon D \to Y$ uniformemente continua allora $g$ ammette un'unica estensione $G \colon X \to Y$ continua. (Inoltre se $X$ e $Y$ sono spazi normati e $g$ è lineare allora anche $G$ è lineare)

\textbf{Dimostrazione Proposizione 1.}

\textit{Dimostrazione che non funziona:}
Proviamo a svolgere il calcolo diretto
$$
\begin{aligned}
	\|\hat f\,\|_2^2 
	&= \int_{-\infty}^{+\infty} \hat f(y) \hat f(y) \dd y \\
	&= \iiint f(x) e^{-ixy} \overline{f(t) e^{-ity}} \dd t \dd x \dd y = \\
	&= \iint f(x) \overline{f(t)} \bigg( \underbrace{\int_{-\infty}^{+\infty} e^{-iy(t-x)} \dd y}_{\delta(x-t)} \bigg) \dd t \dd x = \\
	&= \int \left( \int f(x) \delta(x - t) \dd x \right) \overline{f(t)} \dd t = \\
	&= 2\pi \int f(t) \overline{f(t)} \dd t = 2\pi \norm{f}_2^2
\end{aligned}
$$
vediamo però che compare l'integrale $\int_{-\infty}^{+\infty} e^{-iy(t-x)} \dd y$ e serve assumere che corrisponda a $\delta(x - t)$ dove $\delta$ è la ``funzione Delta di Dirac'', vediamo ora la dimostrazione formale usando una funzione ausiliaria.

\textit{Dimostrazione formale:}
Prendiamo $\varphi \colon \R \to [0,+\infty]$ tale che
\begin{enumerate}
	\item $\varphi$ continua in $0$, crescente per $y < 0$ e decrescente per $y > 0$ e $\varphi(0) = 1$.
	\item $\varphi \in L^1$ e $\check\varphi \in L^1$.
\end{enumerate}

Poniamo per ogni $\delta$
$$
I_\delta = \int_{-\infty}^{+\infty} |\hat f(y)|^2 \varphi(\delta y) \dd y
\xrightarrow{\text{?}}
\int_{-\infty}^{+\infty} |f(y)|^2
$$

\begin{itemize}
	\item \textit{Passo 1:}
		$I_\delta \xrightarrow{\delta \to 0} \| \hat f \|_2^2$ per convergenza monotona usando l'ipotesi di crescenza/descrescenza prima/dopo lo $0$.

	\item \textit{Passo 2:}
		$$
		\begin{aligned}
			I_\delta 
			&= \int \hat f(y) \overline{\hat f(y)} \varphi(\delta y) \dd y = \\
			&= \int \left( \int f(x) e^{-ixy} \dd x \right) \left( \int \overline{f(t)} e^{ity} \dd t \right) \varphi(\delta y) \dd y = \\
			&\overset{\mathclap{\text{FT}}}{=} 
			\iint f(x) \overline{f(t)} 
			\bigg( \underbrace{\int \varphi(\delta y) e^{i(t-x)y} \dd y}_{\sigma_\delta \check \varphi(t - x)} \bigg)
			\dd x \dd y = \\
			&= \int \left( f(x) \sigma_\delta \check\varphi(t - x) \dd x \right) \overline{f(t)} \dd t =\\
			&= \int f \ast \sigma_\delta \check \varphi(t) \cdot \overline{f(t)} \dd t = \\
			&= \langle f \ast \sigma_\delta \check \varphi; f \rangle \\
		\end{aligned}
		$$
		e possiamo applicare il teorema di Fubini-Tonelli in quanto le ipotesi sono verificate infatti
		$$
		\begin{aligned}
			&\iiint |f(x) \overline{f(t)}| e^{i(t - x)y} \varphi(\delta y) \dd x \dd t \dd y = \\
			= &\iiint |f(x)| \cdot |f(t)| \cdot |\varphi(\delta y)| \dd x \dd t \dd y = \\
			= &\norm{f}_1^2 \norm{\varphi(\delta y)}_1 < +\infty
		\end{aligned}
		$$
		e $\norm{\varphi(\delta y)}_1 < +\infty$ poiché $\varphi \in L^1$.

	\item \textit{Passo 3:}
		$I_\delta \xrightarrow{\delta \to 0} 2\pi \norm{f}_2^2 $. Infatti $I_\delta = \langle f \ast \sigma_\delta \check \varphi; f \rangle$ e
		$$
		\sigma_\delta \check \varphi \xrightarrow{\text{in $L^2$}} m f
		\qquad
		\text{con }
		m = \int \check \varphi(x) \dd x
		$$

	\item \textit{Passo 4:}
		Infine $m = 2\pi$ ad esempio prendendo $\varphi(y) = e^{-|y|}$
		$$
		\check\varphi(x) = \frac{2}{1 + x^2} \in L^1
		$$
		ed in questo caso $m$ si calcola.
\end{itemize}
\qed

\subsection{Proprietà della trasformata di Fourier in $L^2$}

\textbf{Proposizione 4.}
\begin{itemize}
	\item $\hat{\tau_h f} = e^{-ihy} \hat f$
	\item $\hat{e^{ihx} f} =  \tau_h \hat f$
	\item $\hat{\sigma_h f} = \hat f(\delta y)$
\end{itemize}

\textbf{Dimostrazione.}
Le identità valgono in $L^1 \cap L^2$ che è denso in $L^2$ e dunque si estendono per continuità ad $L^2$.

\textbf{Proposizione 5.}
Se $f \in C^1(\R; \C)$ e $f \in L^1 \cup L^2$ e $f' \in L^1 \cup L^2 \implies \hat{f'} = iy\hat{f}$.

\textbf{Dimostrazione.}
La stessa fatta per $f, f' \in L^1$. Si parte da $a_n, b_n$ tali che $a_n \to -\infty$ e $b_n \to +\infty$ con $f(a_n) \to 0$ e $f(b_n) \to 0$ e si integra per parti
$$
\begin{aligned}
	\mathcal F(f' \cdot \One_{[a_n, b_n]})
	&= \int_{a_n}^{b_n} f'(x) e^{-ixy} \dd x \\
	&= \underbrace{\left[f(x) e^{-ixy}\right]_{a_n}^{b_n}}_{\to 0} + iy \int_{a_n}^{b_n} f(x) e^{-iyx} \dd x = iy \mathcal F(f \cdot \One_{[a_n, b_n]})
\end{aligned}
$$

\textbf{Proposizione 6.}
Se $f \in C^1, f \in L^1, f' \in L^2 \implies \hat f \in L^1$ e soddisfa le ipotesi del teorema di inversione.

\textbf{Dimostrazione.}
Sappiamo che $iy \hat f = \hat{f'} \in L^2 \implies y \hat f \in L^2$.
$$
\begin{aligned}
	\int_{\R} |\hat f(y)| \dd y 
	&= \int_{|y| \leq 1} |\hat f(y)| \dd y + \int_{|y| \geq 1} |\hat f(y)| \dd y \\
	&\leq 2 \| \hat f\, \|_\infty + \int_{|y| \geq 1} |\hat f(y) y| \frac{1}{|y|} \dd y \\
	&\leq 2 \norm{f}_1 + \| \hat f y \|_2 \left( \int_{|y| \geq 1} \frac{1}{|y|^2} \dd y \right) \\
	&\leq 2 \norm{f}_1 + 2 \sqrt{\pi} \norm{f}_2
\end{aligned}
$$

\textbf{Corollario.}
$f \in C_C^1 \implies f, \hat f \in L^1$

\textbf{Proposizione 7.}
Se $f_1, f_2 \in L^2(\R; \C)$ (e dunque $f_1, f_2 \in L^1(\R; \C)$ per H\"older) allora
$$
\hat{f_1 f_2} = 2\pi \hat{f_1} \ast \hat{f_2}
$$

\textbf{Dimostrazione.}
$f_1, f_2 \in L^2 \implies f_1, f_2 \in L^1$ segue da H\"older. Dimostriamo la proposizione per $f_1, f_2 \in C_C^1 \implies f_1, f_2, f_1 f_2 \in C_C^1 \implies$ tutte in $L^1$ e con trasformate in $L^1$.
$$
\mathcal F^*(2\pi \hat{f_1} \ast \hat{f_2})
= 2\pi \mathcal F^*(\hat{f_1}) \mathcal F^*(\hat{f_2})
= 2\pi \frac{1}{2\pi} f_1 \frac{1}{2\pi} f_2
= \frac{1}{2\pi} f_1 \cdot f_2 = \mathcal F^*(\hat{f_1 f_2})
$$
ed usando che $\mathcal F^*$ è iniettiva otteniamo che $2\pi \hat{f_1} \ast \hat{f_2} = \hat{f_1 f_2}$. 

Per $f_1, f_2 \in L^2$ si procede per continuità e si approssimano $f_1$ ed $f_2$ con $f_{1,n}$ e $f_{2,n}$ in $C_C^1$.
\qed


% Esercitazione su operatori autoaggiunti e Trasformate di Fourier

\section{Esercitazione del 13 Dicembre 2021}

\subsection{Operatori autoaggiunti}

[TODO: Pezzo iniziale mancante]

\textbf{Esercizi.}
\begin{enumerate}
	\item[1)] Esempio classico di $H = L^2([-\pi, \pi]; \C)$ e $D = \{ u \in C^1([-\pi, \pi]; \C) \mid u(-\pi) = u(\pi) \}$ e $T u = i u$ allora $T$ è un operatore autoaggiunto ed ha autovalori $\lambda = n \in \Z$.
	
	\item[2)] $H = L^2([0, \pi]; \R)$ e $D = \{ u \in C^2([0, \pi]) \mid u(0) = 0 \text{ e } u(\pi) = 0 \}$ sono dette condizioni di Dirichlet con $T u = - \ddot u$.
\end{enumerate}

Ora usiamo sempre $T u = -\ddot u$ ma su domini differrenti.

\begin{enumerate}
	\item[3)] $D_3 = \{ u \in C^2([0, \pi]) \mid \dot u(0) = 0 \text{ e } \dot u(\pi) = 0 \}$ sono dette condizioni di Neumann.
	
	\item[4)] $D_4 = \{ u \in C^2([0, \pi]) \mid u(0) = 0 \text{ e } \dot u(\pi) = 0 \}$ sono dette condizioni di Robin.
\end{enumerate}
Dire per 2), 3) e 4) rispondere alle seguenti
\begin{itemize}
	\item L'operatore $T$ è autoaggiunto e controllare se il relativo $D$ è denso in $L^2$
	\item Controllare se esistono autovalori ed eventualmente dire chi sono gli autovettori.
	\item Stabilire se esiste una base Hilbertiana di autovettori.
\end{itemize}

\textbf{Risoluzione.}
\begin{enumerate}
	\item[2)] $D_2$ è denso. Vediamo l'operatore è autoaggiunto
		$$
		\begin{aligned}
			\langle Tu, v \rangle 
			&= \int_0^\pi (-\ddot u(x)) v(x) \dd x 
			= \underbrace{-\dot u(x) v(x) \bigg|_0^\pi}_{v(0) = v(\pi) = 0} - \int_0^\pi (-\dot u(x)) \dot v(x) \dd x 
			= \int_0^\pi \dot u(x) \dot v(x) \dd x \\
			\langle u, Tv \rangle 
			&= \int_0^\pi u(x) (-\ddot v(x)) \dd x 
			= \underbrace{u(x) (-\dot v(x)) \bigg|_0^\pi}_{u(0) = u(\pi) = 0} - \int_0^\pi \dot u(x)) (-\dot v(x)) \dd x 
			= \int_0^\pi \dot u(x) \dot v(x) \dd x \\
		\end{aligned}
		$$
		dunque $\langle Tu, v \rangle = \langle \dot u, \dot v \rangle = \langle u, Tv \rangle$.

		Inoltre $T$ è anche definito positivo infatti $\langle T u, u \rangle = \langle \dot u, \dot u \rangle = \| \dot u \|_{L^2} \geq 0$.

		Cerchiamo gli autovalori quindi poniamo $-\ddot u = Tu = \lambda u$ con $\lambda \geq 0$ e $u \in D_2$. Segue $p(t) = t^2 + \lambda \implies t = \pm i \sqrt{\lambda}$ se $\lambda \neq 0$. 

		Se $\lambda = 0$ invece otteniamo $\ddot u = 0 \implies u(x) = a x + b$ ma per le condizioni al bordo segue $a, b = 0$ e dunque $u = 0 \implies \lambda = 0$ non è autovalore.

		Invece se $\lambda > 0$ abbiamo $u(x) = A \cos(\sqrt{\lambda} x) + B \sin(\sqrt{\lambda} x)$ e segue $A = 0$ e $\lambda = n^2$ per $n \in \N \setminus \{ 0 \}$.

	\item[3)] $D_2$ è denso e similmente si vede che anche in questo caso $T$ è autoaggiunto. Anche in questo caso $T$ è definito positivo perché vale sempre $\langle Tu, v \rangle = \langle \dot u, \dot v \rangle$.

		Per cercare gli autovalori risolviamo il seguente sistema
		$$
		\begin{cases}
			-\ddot u = \lambda u \\
			\dot u(0) = 0 \\
			\dot u(\pi) = 0
		\end{cases}
		$$
		Se $\lambda = 0$ allora $u(x) = \text{cost.}$ è un autovettore per l'autovalore $0$.

		Se invece $\lambda \neq 0$ allora $u(x) = A \cos(\sqrt{\lambda} x) + B \sin(\sqrt{\lambda} x) \implies \cos(nx)$ è un autovettore e $\lambda = n^2$ per $n = 1, 2, \dots$.

	\item[4)]
		In questo caso vediamo che vale sempre $\langle Tu, v \rangle = \langle \dot u, \dot v \rangle$ ma per motivi diversi infatti
		$$
		\begin{aligned}
			\langle Tu, v \rangle 
			&= \int_0^\pi (-\ddot u(x)) v(x) \dd x 
			= \underbrace{-\dot u(x) v(x) \bigg|_0^\pi}_{\dot u(\pi) = 0, \; v(0) = 0} - \int_0^\pi (-\dot u(x)) \dot v(x) \dd x 
			= \int_0^\pi \dot u(x) \dot v(x) \dd x \\
			\langle u, Tv \rangle 
			&= \int_0^\pi u(x) (-\ddot v(x)) \dd x 
			= \underbrace{u(x) (-\dot v(x)) \bigg|_0^\pi}_{\dot v(\pi) = 0, \; u(0) = 0} - \int_0^\pi \dot u(x)) (-\dot v(x)) \dd x 
			= \int_0^\pi \dot u(x) \dot v(x) \dd x \\
		\end{aligned}
		$$
		Considerando il sistema $-\ddot u = \lambda u$ con le condizioni al bordo di Robin caso $\lambda = 0$ non è un autovalore mentre se $\lambda \neq 0$ abbiamo che $\dot u(x) = \sqrt{\lambda} B \cos(\sqrt{\lambda} x) = 0$ per $x = \pi$ dunque $\sqrt{\lambda} = n + 1/2$ per $n = 0, 1, 2, \dots$ e gli autovettori sono
		$$
		u_n(x) = \sin\left(\left( n + \frac{1}{2} \right)x\right)
		$$
\end{enumerate}

\textbf{Osservazione.}
$T \colon D \to L^2$ operatore lineare e continuo $\iff \exists M > 0$ tale che $\norm{Tu}_2 \leq M\norm{u}$ per ogni $u \in D$.

Vediamo ad esempio che $D_1$ non è continuo infatti gli autovalori sono $\lambda_n = n^2 \implies n^2 \norm{u_n}_2 \leq M \norm{u_n}_2 \implies M \geq n^2$ per ogni $n$. Dunque $M$ è illimitato e l'operatore non può essere continuo.

\textbf{Esempio.} Se ad esempio abbiamo $T u = -\ddot u$ con $\tilde D = \{ u \in C^2 \mid u(0) = u(\pi) = 1 \}$ allora $T$ non è autoaggiunto e basta trovare $u, v$ tali che
$$
\langle T u, v \rangle \neq \langle u, T v \rangle
$$

\textbf{Esercizio.}
\begin{enumerate}
	\item Sia $T_1 \colon \ell^2 \to \ell^2$ dato da
		$$
		T_1((x_n)_{n > 0}) = (0, x_1, x_2, \dots)
		$$
	\item Sia $T_2 \colon \ell^2 \to \ell^2$ dato da
		$$
		T_2((x_n)_{n > 0}) = (x_2, x_3, \dots)
		$$
\end{enumerate}
Dire se sono autoaggiunti ed eventualmente chi sono gli autovalori.

\textit{Esericizi più da compito sono invece cose del tipo...}

\textbf{Esercizio.}
Sia $H = L^2([0, \pi] \times [0, \pi]; \R)$ e $T u = -\Delta u = -u_{xx} -u_{yy}$
\begin{itemize}
	\item $D_1 = \{ u \in C^2([0, \pi] \times [0, \pi]; \R) \mid u|_{\pd Q} = 0 \}$
	\item $D_2 = \{ u \in C^2([0, \pi] \times [0, \pi]; \R) \mid \nabla u|_{\pd Q} = 0 \}$
	\item $D_3 = \{ u \in C^2([0, \pi] \times [0, \pi]; \R) \mid \text{$u = 0$ su due lati paralleli e $\nabla u = 0$ sugli altri due} \}$
\end{itemize}
e dire se l'operatore è autoaggiunto ed eventualmente trovare gli autovalori.

\subsection{Calcolo Trasformate di Fourier}

Abbiamo visto che le trasformate di $f(x) = e^{-|x|}$ e $g(x) = 1 / (1 + x^2)$ sono rispettivamente
$$
\hat f(y) = \frac{2}{1 + y^2}
\qquad
\hat g(y) = \pi e^{-|y|}
$$
Vorremo provare a trovare ora le trasformate funzioni come $x^2 e^{-|x|}$ o $x / (1 + x^2)$ usando le proprietà delle trasformate con le derivate. Ricordiamo che
$$
\begin{gathered}
	f, f' \in L^1 \implies \hat{f'\,}(y) = i y \hat f(y) \\
	f, x f \in L^1 \implies \hat{f}'(y) = \hat{-i x f\,}(y)
\end{gathered}
$$
dunque intuitivamente per $x^2 e^{-|x|}$ possiamo fare
$$
x^2 e^{-|x|} = i (-i) x (x e^{-|x|}) \implies \mathcal F(i (-i) x (x e^{-|x|})) = i (\mathcal F(x e^{-|x|}))'(y)
$$
ora dobbiamo calcolare $\mathcal F(x e^{-|x|})(y)$
$$
\mathcal F(x e^{-|x|})(y) = i \mathcal F(-ix e^{-|x|}) = i \mathcal F(e^{-|x|})'(y) = i \left(\frac{2}{1 + y^2}\right)' = \frac{-4iy}{(1+y^2)^2}
$$
dunque in conclusione abbiamo
$$
\mathcal F(x^2 e^{-|x|})(y) = i \left(\frac{-4iy}{(1 + y^2)^2}\right)' = 4 \left(\frac{y}{(1 + y^2)^2}\right)'
$$
Invece per quanto riguarda
$$
g(x) = \frac{x}{1+x^2} \notin L^1
$$
però è in $L^2$ ma per poterne calcolare la trasformata di Fourier dovremmo passare per delle troncate di $g(x)$. Possiamo però vedere chi dovrebbe essere il candidato formale usando le tecniche di prima
$$
\frac{x}{1+x^2} = \frac{i (-ix)}{1+x^2} \rightsquigarrow 
i \mathcal F \left( (-ix) \frac{1}{1+x^2} \right)(y)
= i \mathcal F \left( \frac{1}{1+x^2} \right)'(y) = i \pi (e^{-|y|})'(y)
$$
però notiamo che la derivata di $e^{-|y|}$ non è ben definita in $0$.

\textbf{Esercizio.}
Calcolare la trasformata di Fourier di
$$
f(x) = \One_{[-r, r]}(x)
$$
Iniziamo a svolgere il conto
$$
\hat f(y) = \int_{-\infty}^\infty \One_{[-r, r]}(x) e^{-ixy} \dd x = \int_{-r}^r e^{-ixy} \dd x
=
\begin{cases}
	2r & xy=0 \\
	\ds \int_{-r}^r e^{-ixy} \dd x & xy \neq 0 \\
\end{cases}
$$
nel caso $xy \neq 0$ contiuiamo a svolgere il conto
$$
\int_{-ry}^{ry} \frac{e^{-it}}{y} \dd t 
= \frac{1}{y} \int_{-ry}^{ry} [\cos(t) - i\underbrace{\sin(t)}_{\text{dispari}}] \dd t
= \frac{2}{y} \sin(ry)
$$
dunque in conclusione abbiamo
$$
\mathcal F(\One_{[-r, r]}(x)) =
\begin{cases}
	2r & y=0 \\
	\ds \frac{2}{y} \sin(ry) & y \neq 0 \\
\end{cases}
$$

\textbf{Esercizio.}
Un esercizio simile è calcolare $\mathcal F(\One_{[0, r]}(x))$, ovvero il caso non centrato e poi provare a calcolare (come integrale improprio di Analisi 1) l'integrale
$$
\int_{0}^{\infty} \frac{\sin(t)}{t} \dd t = \lim_{r \to \infty}\int_{0}^{r} \frac{\sin(t)}{t} \dd t
$$

\subsubsection{Trasformata della Gaussiana}

Calcoliamo ora la trasformata della funzione gaussiana $e^{-x^2 / 2}$.

\begin{itemize}
	\item \textbf{Metodo I:} Troviamo un'equazione differenziale (lineare) risolta dalla gaussiana, sia $f(x) = e^{-x^2 / 2}$ allora vale
		$$
		f'(x) = -x e^{-x^2/2} = -x f(x)
		$$
		e per il decadimento della gaussiana abbiamo che $f, f' \in L^1$ dunque
		$$
		i y \hat f(y) = \hat{f'(x)}(y) = -i \mathcal F(-i x f(x))(y) = i (\hat{f})' (y)
		$$
		dunque $\hat f = h(y)$ con $h$ tale che $h'(y) = -y h(y) \implies h(y) = k e^{-y^2 / 2}$, rimane da trovare $k$. Calcoliamo direttamente $h(0)$
		$$
		\begin{gathered}
			h(0) 
			= \hat{e^{-x^2/2}(0)} 
			= \int_{-\infty}^\infty e^{-x^2 / 2} e^{-ix \cdot 0} \dd x
			= \int_{-\infty}^\infty e^{-x^2 / 2} \dd x = \sqrt{2\pi} \\
			\implies \hat{e^{-x^2/2}} = \sqrt{2\pi} e^{-y^2 / 2}
		\end{gathered}
		$$

	\item \textbf{Metodo II:} Studiamo la funzione di variabile complessa $g(z) = e^{-z^2/2}$ e integriamola lungo un percorso che passi per $[-r, r] \times \{ 0 \} \subset \R^2 \approx \C$.
		$$
		\begin{aligned}
			\mathcal F(\hat{-x^2/2})(y) 
			&= \int_\R e^{-x^2 / 2} \cdot e^{-ixy} \dd x
			= \int_\R e^{-\frac{1}{2}(x^2 + 2ixy)} \dd x \\
			&= \int_\R e^{-\frac{1}{2}(x^2 + 2ixy + y^2 - y^2)} \dd x 
			= \int_\R e^{-\frac{1}{2}(x^2 + 2ixy + y^2)} \cdot e^{-y^2 / 2} \dd x \\
			&= e^{-y^2 / 2} \int_\R e^{-\frac{1}{2}(x^2 + 2ixy + y^2)} \dd x 
			= e^{-y^2 / 2} \lim_{r \to \infty} \int_{-r}^r \underbrace{e^{-\frac{1}{2}(x + iy)^2}}_{g(x + iy)} \dd x
		\end{aligned}
		$$
		Consideriamo ora il rettangolo $D_r \coloneqq \{ z \mid \operatorname{Im} z \in [0, iy] \text{ e } \operatorname{Re} z \in [-r, r] \}$ dunque poiché $g(z)$ non ha poli su $D_r$ abbiamo
		$$
		\begin{gathered}
			\int_{\pd D_r} g = \sum \text{Res. su $D_r$} = 0 \\
			0 = \int_{-r}^r g(x + it) \dd x - \int_{-r}^r g(x + iy) \dd x 
			+ \int_{0}^{iy} g(r + iy) \dd t - \int_{0}^{iy} \underbrace{g(-r + iy)}_{\mathclap{\sim \, e^{-(r+it)^2} \xrightarrow{r \to \infty} 0}} \dd t
		\end{gathered}
		$$
		infatti più precisamente i termini verticali vanno a zero
		$$
		\int_0^y e^{-(r+it)^2 / 2} \dd t 
		= e^{-r^2 / 2} \int_0^y e^{-itr - t^2/2} \dd t = y e^{-r^2 / 2} \xrightarrow{r \to +\infty} 0
		$$
		In conclusione abbiamo
		$$
		\begin{aligned}
			\hat{e^{-x^2 / 2}} (y) 
			&= \lim_{r \to +\infty} \int_{-r}^r e^{-y^2 / 2} e^{-(x + iy)^2 / 2} \dd x 
			= e^{-y^2 / 2} \left[ \lim_{r \to +\infty} \int_{-r}^r e^{-x^2 / 2} \dd x + o(1) \right] = \\
			&= e^{-y^2 / 2} \int_{-\infty}^\infty e^{-x^2 / 2} \dd x = \sqrt{2\pi} e^{-y^2 / 2} 
		\end{aligned}
		$$
		\qed

\end{itemize}




\chapter{Integrazione di superfici}

% TODO

\newpage

\section{Indice Analitico}

Lorem ipsum dolor sit amet, consectetur adipisicing elit, sed do eiusmod
tempor incididunt ut labore et dolore magna aliqua. Ut enim ad minim veniam,
quis nostrud exercitation ullamco laboris nisi ut aliquip ex ea commodo
consequat. Duis aute irure dolor in reprehenderit in voluptate velit esse
cillum dolore eu fugiat nulla pariatur. Excepteur sint occaecat cupidatat non
proident, sunt in culpa qui officia deserunt mollit anim id est laborum.

\begin{multicols*}{2}

\makebox[3cm][l]{\absurd} Assurdo

\makebox[3cm][l]{\absurd} Assurdo

\makebox[3cm][l]{\absurd} Assurdo

\makebox[3cm][l]{\absurd} Assurdo

\vfill\null\columnbreak

\makebox[3cm][l]{\absurd} Assurdo

\makebox[3cm][l]{\absurd} Assurdo

\makebox[3cm][l]{\absurd} Assurdo

\makebox[3cm][l]{\absurd} Assurdo

\end{multicols*}

\section*{Esempi di figure}

\subsection*{Semplici}

Lorem ipsum dolor sit amet, consectetur adipisicing elit, sed do eiusmod
tempor incididunt ut labore et dolore magna aliqua. Ut enim ad minim veniam,
quis nostrud exercitation ullamco laboris nisi ut aliquip ex ea commodo consequat.

% Moralmente "example" corrisponde ad un file "src/figures/example.svg", in realtà la magia è che "example" diventa "example.pdf_tex" attraverso "\inputfigure" e poi quello il latex lo va a cercare dentro ".cache/figures/.pdf_tex/" come impostato in "prelude.tex".
\begin{figure}[h]
	\centering
	\inputfigure{example} 
\end{figure}

Duis aute irure dolor in reprehenderit in voluptate velit esse
cillum dolore eu fugiat nulla pariatur. Excepteur sint occaecat cupidatat non
proident, sunt in culpa qui officia deserunt mollit anim id est laborum.

\subsection*{Wrappate}

Lorem ipsum dolor sit amet, consectetur adipisicing elit, sed do eiusmod
tempor incididunt ut labore et dolore magna aliqua. Ut enim ad minim veniam,
quis nostrud exercitation ullamco laboris nisi ut aliquip ex ea commodo consequat.

% La pagina del file "example.svg" è larga 300pt e qui sotto va lo stesso valore, per motivi estetici i due vspace negativi riducono un po' il margine di default che sta sopra e sotto l'ambiente "wrapfigure".
\begin{wrapfigure}{r}{300pt}
	\centering
	\vspace{-1.5\baselineskip}
	\inputfigure{example}
	\vspace{-2.5\baselineskip}
\end{wrapfigure}

Duis aute irure dolor in reprehenderit in voluptate velit esse
cillum dolore eu fugiat nulla pariatur. Excepteur sint occaecat cupidatat non
proident, sunt in culpa qui officia deserunt mollit anim id est laborum.

Lorem ipsum dolor sit amet, consectetur adipisicing elit, sed do eiusmod
tempor incididunt ut labore et dolore magna aliqua. Ut enim ad minim veniam,
quis nostrud exercitation ullamco laboris nisi ut aliquip ex ea commodo
consequat. Duis aute irure dolor in reprehenderit in voluptate velit esse
cillum dolore eu fugiat nulla pariatur. Excepteur sint occaecat cupidatat non
proident, sunt in culpa qui officia deserunt mollit anim id est laborum.

\subsection*{Con caption o descrizione}

Lorem ipsum dolor sit amet, consectetur adipisicing elit, sed do eiusmod
tempor incididunt ut labore et dolore magna aliqua. Ut enim ad minim veniam,
quis nostrud exercitation ullamco laboris nisi ut aliquip ex ea commodo consequat.

\begin{figure}[h]
	\centering
	\inputfigure{example}
	\caption{
		Lorem ipsum dolor sit amet, consectetur adipisicing elit, sed do eiusmod
		tempor incididunt ut labore et dolore magna aliqua.
	}
\end{figure}

Lorem ipsum dolor sit amet, consectetur adipisicing elit, sed do eiusmod
tempor incididunt ut labore et dolore magna aliqua. Ut enim ad minim veniam,
quis nostrud exercitation ullamco laboris nisi ut aliquip ex ea commodo consequat.

\begin{wrapfigure}{r}{300pt}
	\centering
	\vspace{-1.5\baselineskip}
	\inputfigure{example}
	{\footnotesize
		Lorem ipsum dolor sit amet, consectetur adipisicing elit, sed do eiusmod
		tempor incididunt ut labore et dolore magna aliqua.
	}
	\vspace{-1\baselineskip}
\end{wrapfigure}

Duis aute irure dolor in reprehenderit in voluptate velit esse
cillum dolore eu fugiat nulla pariatur. Excepteur sint occaecat cupidatat non
proident, sunt in culpa qui officia deserunt mollit anim id est laborum.

Lorem ipsum dolor sit amet, consectetur adipisicing elit, sed do eiusmod
tempor incididunt ut labore et dolore magna aliqua. Ut enim ad minim veniam,
quis nostrud exercitation ullamco laboris nisi ut aliquip ex ea commodo
consequat. Duis aute irure dolor in reprehenderit in voluptate velit esse
cillum dolore eu fugiat nulla pariatur. Excepteur sint occaecat cupidatat non
proident, sunt in culpa qui officia deserunt mollit anim id est laborum.


\end{document}
