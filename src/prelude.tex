
%%%
%%% Use a custom geometry for the page
%%%
\usepackage{geometry}
\geometry{
  a4paper,
  left=20mm,
  right=20mm,
  top=20mm,
  bottom=25mm
}

\usepackage{lipsum}

%%%
%%% File encoding
%%%
\usepackage[utf8]{inputenc}
\usepackage[italian]{babel}
\usepackage[T1]{fontenc}

%%%
%%% Font
%%%
\usepackage{lmodern}
\usepackage{textcomp}
\usepackage[font=itshape]{quoting}

%%%
%%% References
%%%
\usepackage{hyperref}
\hypersetup{ 
  linkbordercolor={.77 .4 .20},
  citebordercolor={.31 .63 .31},
  pdftitle={Appunti}
}

%%%
%%% Pictures
%%%

% General
\usepackage{caption}
\usepackage{graphicx}

% TikZ
\usepackage{tikz}
\usetikzlibrary{cd}

% Inkscape figures
\usepackage{import}
\usepackage{xifthen}
\usepackage{pdfpages}
\usepackage{transparent}
\newcommand{\incfig}[1]{%
    \def\svgwidth{\columnwidth}
    \import{./figures}{#1.pdf_tex}
}

%%%
%%% Basic math and layout packages
%%%
\usepackage{amssymb}
\usepackage{amsmath}
\usepackage{amsthm}
\usepackage{mathrsfs}
\usepackage{mathtools}
\usepackage{xfrac}
\usepackage{array}
\usepackage[all]{xy}
\usepackage{stmaryrd}
\usepackage{multicol}

%%%
%%% Enumerations
%%%
\usepackage{enumitem}

%%%
%%% Sections and page borders styles
%%%

\usepackage{titlesec}
% Adds horizontal rules under "\section" entries
\titleformat{\section}{\Large\scshape\raggedright}{}{0em}{}[\titlerule]
\usepackage{fancyhdr}
% \pagestyle{fancy}
% \pagestyle{fancyplain} % Makes all pages in the document conform to the custom headers and footers
% \fancyhead[L]{\leftmark}% Empty left header
% \fancyhead[C]{} % Page numbering for center header  
% \fancyhead[R]{\rightmark}% Empty right header
% \fancyfoot[L]{}% Empty left footer
% \fancyfoot[C]{\thepage}% Empty center footer
% \fancyfoot[R]{}% Empty left footer
% \fancyhf{}
% \chead{\footnotesize\rightmark}
% \rfoot{\thepage}

%%%
%%% Styles by @aziis98
%%%

% Better "\setminus"
\newcommand{\mysetminusD}{%
  \hbox{\tikz{\draw[line width=0.6pt,line cap=round] (3pt,0) -- (0,6pt);}}
}
\newcommand{\mysetminusS}{%
  \hbox{\tikz{\draw[line width=0.45pt,line cap=round] (2pt,0) -- (0,4pt);}}
}
\newcommand{\mysetminusSS}{%
  \hbox{\tikz{\draw[line width=0.4pt,line cap=round] (1.5pt,0) -- (0,3pt);}}
}
\newcommand{\mysetminusT}{\mysetminusD}
\newcommand{\mysetminus}{\mathbin{\mathchoice{\mysetminusD}{\mysetminusT}{\mysetminusS}{\mysetminusSS}}}
\renewcommand{\setminus}{\mysetminus}

% Generali
\renewcommand{\implies}{\Rightarrow}
\renewcommand{\iff}{\Leftrightarrow}
\renewcommand{\epsilon}{\varepsilon}
\newcommand{\compose}{\circ}
\newcommand{\mquad}{\;\;}
\newcommand{\curry}{\,\cdot\,}
\newcommand{\C}{\mathbb C}
\newcommand{\R}{\mathbb R}
\newcommand{\Z}{\mathbb Z}
\newcommand{\N}{\mathbb N}
\newcommand{\ds}{\displaystyle}
\newcommand{\dd}{\,\mathrm{d}}
\newcommand{\pd}{\partial}

% Fulmine dell'assurdo
\usepackage{stmaryrd}
\let\stmaryrdLightning\lightning
\renewcommand{\lightning}{\stmaryrdLightning}
\newcommand{\absurd}{$\lightning$}

% Funzione indicatrice
\newcommand{\bbOne}{\text{\usefont{U}{bbold}{m}{n}1}}
\MakeRobust{\bbOne}
\newcommand{\One}{\bbOne}

%%%
%%% Styles by @<arianna>
%%%

%%
%% Nuovi comandi
%%

% Insiemi numerici
% \newcommand{\N}{\mathbb{N}}
% \newcommand{\Z}{\mathbb{Z}}
% \newcommand{\Q}{\mathbb{Q}}
% \newcommand{\R}{\mathbb{R}}
% \newcommand{\C}{\mathbb{C}}
\newcommand{\restr}[2]{\left.#1\right|_{#2}}

% Dichiarazione lettera Chi maiuscola -> \Chi
\DeclareRobustCommand{\rchi}{{\mathpalette\irchi\relax}}
\newcommand{\irchi}[2]{\raisebox{\depth}{\mbox{\Large$#1\chi$}}} % inner command, used by \rchi%
\newcommand{\Chi}{\rchi}

% Dichiarazioni nuovi ambienti
\theoremstyle{definition}
\newtheorem{definizione}{Definizione}[chapter]
\newtheorem{teorema}{Teorema}[chapter]
\newtheorem{proposizione}[definizione]{Proposizione}
\newtheorem{lemma}[teorema]{Lemma}
\newtheorem{corollario}[teorema]{Corollario}
\newenvironment{dimostrazione}{\textit{Dimostrazione:}}{\hfill$\square$\break}
\newtheorem{osservazione}{Osservazione}
\newtheorem{proprieta}[teorema]{Proprietà}
\newtheorem{esempio}[teorema]{Esempio}
\newtheorem{nota}[osservazione]{Nota}
