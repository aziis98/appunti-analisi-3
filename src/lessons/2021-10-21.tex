%
% Lezione del 21 Ottobre 2021 
%

\section{Esercitazione del 21 ottobre}

% Piccolo recap di quanto fatto le scorse lezioni.

Data $T \colon X \to Y$ lineare tra $X,Y$ spazi normati, allora $T$ è continua se solo se esiste  $C > 0$ tale che $\norm{T(x)}_Y \leq C \norm{x}_X$ per ogni $x \in X$.

Applichiamo questo risultato.

\begin{enumerate}
\item Sia $X = \R^d$. L'applicazione $\ds L^1(\R^d) \ni u \xmapsto{\; T \;} \int_{\R^d} u \dd x$ è lineare e continua in quanto limitata. Infatti:
%
$$
\left| T(u) \right| = \left| \int_{\R^d} u \dd x  \right| \leq \int_{\R^d} \left| u \right| \dd x = \norm{u}_{L^1(\R^d)}.
$$
%

\item Studiamo ora il caso per $p > 1$. Data $u \in L^p(\R^d)$, l'applicazione
%
$$
u \mapsto \int_{\R^d} u \dd x 
$$
%
potrebbe non essere ben definita.

Ad esempio se restringiamo il dominio a $L^p(\R^d) \cap L^1(\R^d)$ l'applicazione sopra è ben definita, ma in generale non è continua.
Più formalmente, la mappa
%
$$
T \colon \left( L^p \cap L^1(\R^d), \norm{\cdot}_{L^p} \right) \to \R
$$
%
è lineare ma non continua.

Studiamo il caso reale, ovvero $d = 1$.

Per verificare quanto sopra, utilizziamo la definizione di continuità per successioni.
Definiamo una successione di funzioni a supporto compatto $u_n$, che sappiamo essere in tutti gli spazi $L^p$, e verifichiamo che $\lim_n T(u_n) \neq T(u_\infty)$ dove $u_\infty \coloneqq \lim_n u_n$.

Definiamo la successione come segue (fare disegno):
%
$$
u_n (x) =
\begin{cases}
\frac{1}{n} \quad \text{se} \quad n \leq x \leq 2n \\
0 \quad \text{altrimenti}. 
\end{cases} 
$$
%
Dunque, $\ds T(u_n) = \int_{\R} u_n \dd x = \frac{1}{n} \left| E_n \right| = 1$, dove $E_n = [n,2n]$.
Segue che $T(u_n) \equiv 1$ e non è vero che $T(u_\infty = u_n \to 0) = 0$.

Più in generale, quando $u \in L^p(\R^d)$ con $p > 1$, una costruzione come sopra non funziona, infatti
%
$$
\norm{u_n}_{L^p(\R)}^p = \int_{\R} u_n^p(x) \dd x = \frac{1}{n^p} \cdot \left| E_n \right| = \frac{n}{n^p} = \frac{1}{^{p-1}} \xrightarrow{n \to \infty} 0.
$$
%                    
Un altro modo per dimostrare quanto sopra è verificare che, per ogni $C > 0$, esiste una funzione $u \in L^p(\R^d) \cap L^1(\R^d)$ tale per cui 
%
$$
\left| \int_{\R} u \dd x  \right| \nleq C \left( \int_{\R} \left| u \right|^p \dd x  \right)^{1/p}.
$$
%       
[TO DO: aggiungere soluzione] Notiamo che questo è proprio l'esercizio che segue.
\end{enumerate}


\textbf{Esercizio.} [TO DO] Fissato $C > 0$, trovare $u \in L^p \cap L^1 (\R)$ tale che
%
$$
\left| \int_{\R} u \dd x\right| \geq C \norm{u}_{L^p(\R)}
$$
%

%%%
\textbf{Esercizio.} Sia $p \geq 1$ e $\ds E = \left\{ u \in L^p(-1,1) \colon \fint_{-1}^1 u \dd x = 0 \right\}$.

\begin{enumerate}
\item Dire se $E$ è limitato in $L^p(-1,1)$.

\item Dire se $E$ è chiuso in $L^p(-1,1)$.
\end{enumerate}

\textbf{Soluzione.}
\begin{enumerate}
\item Dimostrare che $E$ è limitato in $L^p(-1,1)$ equivale a dimostrare che esiste $M > 0$ tale che ogni  $u \in L^p(-1,1)$, $\ds \fint_{-1}^1 u \dd x = 0$ verifica $\norm{u}_{L^p} \leq M$.

Vediamo che $E$ non è limitato.
Preso $M > 0$, riesco sempre a trovare una funzione maggiore di $M$ in norma.
Ad esempio la funzione definita come
%
$$
u(x) \coloneqq
\begin{cases}
M \quad \text{se} \quad x \in (0,1) \\
-M \quad \text{se} \quad x \in (-1,0)
\end{cases} 
$$
%
ha norma $\ds \norm{u}_{L^p}^p = 2 M^p$.

\textbf{Nota.} Aveva senso cercare il controesempio nella classe delle funzioni dispari e limitate, perché hanno media zero, e perché sono in tutti gli $L^p$.

\item Vediamo che $E$ è chiuso.

\textbf{Nota.} Possiamo dimostrarlo usando i teoremi di convergenza, ma seguiremo un'altra strada.

\begin{itemize}
\item \textit{Caso} $p > 1$. Definiamo l'operatore 
\begin{align*}
T \colon L^p(-1,1) & \to \R \\
u & \mapsto \int_{-1}^{1} u \dd x 
\end{align*}
è ben definito.
Infatti, per Hölder vale
%
$$
\left| \int_{-1}^{1} 1 \cdot u \dd x  \right| \leq \left( \int_{-1}^{1} \left| u \right|^p \dd x  \right)^{1/p} \left( 1^q \right)^{1/q}
$$
%
dove $q = \frac{p}{p-1}$.
Allora
 %
$$
\left| T(u) \right| \leq \norm{u}_{L^p(-1,1)} \cdot 2^{\frac{p}{p-1}}.
$$
%
Dunque $T$ è continuo in $L^p$ per ogni $p > 1$.

\item \textit{Caso} $p = 1$. L'operatore sopra è continuo anche per $p = 1$. Grazie alla stima vista prima
%
$$
\left| T(u) \right| = \left| \int_{-1}^{1} u \dd x  \right| \leq \int_{-1}^{1} \left| u \right| \dd x = \norm{u}_{L^1}.
$$
%
Dunque $T$ è continua e $T^{-1}(0) = E$, dunque $E$ è chiuso.

\end{itemize}

\end{enumerate}

\textbf{Esercizio.} [TO DO] Sia $p \geq 1$. Definiamo 
%
$$
F = \left\{ v \in L^p(\R) \mymid \int_{0}^{1} v(x) \dd x - 2 \int_{-1}^{0} v(x) \dd x = 3 \right\}.
$$
%
Dire se $F$ è chiuso in $\left( L^p(\R), \norm{\cdot }_{L^p(\R)} \right)$.


\textbf{Esercizio.} [TO DO] Sia 
%
$$
G = \left\{ v \in L^p(0,2\pi) \mymid \int_{0}^{2\pi} v(x) \sin(x) \dd x = 1  \right\}.
$$
%
Dire se $G$ è chiuso in $L^2(0,2\pi)$.

\textbf{Domanda.} Dato $L^p (X,\mu)$ e $V$ sottospazio di $L^p(X,\mu)$, posso dire che $V$ è chiuso?

In generale no! Infatti esistono sottospazi densi in $L^p(X,\mu)$.

Ad esempio in $\ell^2$ consideriamo l'insieme denso
%
$$
V = \left\{ \left\{ x_n \right\} \mymid x_n = 0 \quad \text{definitivamente}  \right\}.
$$
%
Vediamo che non è chiuso. Sia $\underline{x} \in \ell^2$, definita come $\underline{x} = \{ 1 / n \}_{n \in \N \setminus \{0 \}}$, diciamo che $\ds \underline{x} = \lim_{n \to +\infty} \underline{x}^n$ dove
%
$$
x_n^k = 
\begin{cases}
\frac{1}{n} \quad 1 \leq n \leq k \\
0 \quad n > k, n = 0
\end{cases} 
$$
%
abbiamo che
%
$$
\norm{\underline{x} - \underline{x}^k}_{\ell^2}^2 = \sum_{n=1}^{\infty} \left| x_n -x_n^k \right|^2 = \sum_{n = k+1}^{+\infty} \left| x_n \right|^2 = \sum_{n = k+1}^{\infty} \frac{1}{n^2} \xrightarrow{k \to +\infty} 0.   
$$
%

Vediamo un altro esempio. Siano $X = \R$, $\mu$ la misura di Lebesgue e $p > 1$.
In tal caso, l'insieme $L^p \cap L^1 (\R)$ è un sottospazio denso in $\left( L^1(\R), \norm{\cdot }_{L^1} \right)$ e $\left( L^p(\R), \norm{\cdot }_{L^p} \right)$.

\textbf{Nota.} L'insieme $L^2(\R) \cap L^1(\R)$ è un sottospazio proprio di $L^1(\R)$. Dico che non è chiuso in $\norm{\cdot }_{L^1(\R)}$ perché è denso.
Infatti, 
%
$$
\mc{C}_C^0(\R) \subset L^2(\R) \cap L^1(\R).
$$
%

\subsection{Convoluzione}

Sia $f \in L^1(\R^d)$ e sia $g \colon \R^d \to \R$ continua a supporto compatto\footnote{In tal caso $g$ è lipschitziana.}.
%
$$
\left| g(x) - g(y) \right| \leq M \left| x - y \right|_{\R^d}.
$$
%

\textbf{Esercizio.} Dimostrare che $f \ast g$ è ben definita e lipschitziana, dove $f \in L^1(\R^d)$ e $g \in \mc{C}_C^0 (\R^d)$.

Prendo $x_1,x_2 \in \R^d$
%
$$
\left| f \ast g (x_1) - f \ast g(x_2) \right| = \left| \int_{\R^d} f(x_1 - x) g(y) \dd y  - \int_{\R^d} f(x_2 - y) g(y) \dd y  \right|
$$
%
Usiamo la proprietà che, essendo $f \ast g$ ben definita, si ha $ f \ast g(x) = g \ast f(x) $. Da cui
\begin{align*}
\left| f \ast g(x_1) - f \ast g(x_2) \right|  & = \left| \int_{\R^d} g(x_1 - y) f(y) \dd y \int_{\R^d} g(x_2 - y) f(y) \dd y   \right| \\
& = \left| \int_{\R^d} \left( g(x_1 - y) - g(x_2 - y) \right) f(y) \dd y  \right| \\
& \leq \int_{\R^d} \left| g(x_1 - y) - g(x_2 - y) \right| \left| f(y) \right| \dd y \\
& \leq \int_{\R^d} M \left| (x_1 - y) - (x_2 - y) \right| \left| f(y) \right| \dd y \leq M \left| x_1 - x_2 \right| \cdot \norm{f}_{L^1(\R^d)}.
\end{align*}

\textbf{Esercizio.} [TO DO] Se $f \in L^1(\R^d)$ e $g$ a supporto compatto è $\alpha$-Hölderiana allora anche $f \ast g$ lo è.

\textbf{Esercizio.} [TO DO] Presa $f(x) = \One_{[0,1]}$ in $\R$, calcolare $f \ast f$.

\subsection{Separabilità degli spazi $L^p$}

\textbf{Proposizione.} Si ha che $L^p\left( \R^d, \mu \right)$ con $\mu $ la misura di Lebesgue, è separabile se solo se $p \neq +\infty$. Lo stesso risultato vale per $\ell^p$.

Sia $1 \leq p < +\infty$, $L^p(\R^d, \mu)$ con $\mu$ la misura di Lebesgue.
Le funzioni semplici costituite da somme finite di insiemi di misura finita sono dense in $L^p(\R^d)$.

Prendiamo una base numerabile di $\R^d$ e la indichiamo con $\mc{B}$. L'insieme
%
$$
Y = \left\{ \sum_{i=1}^{n} \alpha_i \One_{B_i} \mymid B_i \in \mc{B}, \alpha_i \in \Q \right\}
$$
%
è numerabile. Vediamo che è denso in $L^p(\R^d)$.

\textit{Idea.} È sufficiente approssimare le funzioni semplici a somma finita $\sum_{i=1}^{N} \alpha_i \One_{E_i} $. In particolare, ci basta approssimare $\alpha \cdot \One_E$. Essendo $\alpha \in \R$ troviamo una successione di razionali $\alpha_j$ tali che $\alpha_j \xrightarrow{j \to \infty} \alpha$. Dunque, rimane da approssimare l'insieme $E$.

Fissiamo $E$ e supponiamo dapprima $E$ aperto. Possiamo scrivere $E$ come unione arbitraria di elementi della base $\mc{B}$
%
$$
E = \bigcup_{i = 1}^\infty B_i.
$$
%
Per approssimare $E$ consideriamo gli insiemi $E_N = \bigcup_{i = 1}^N B_i$.
Otteniamo $\ds \left| E \right| = \lim_N \left| E_N \right|$, da cui $\left| E \setminus E_N \right| \xrightarrow{N \to +\infty} 0$.
Concludiamo notando che il caso $E$ arbitrario si fa approssimandolo con una famiglia di aperti.

\vspace{3mm}

Per $\ell^p$ con $p < +\infty$ definiamo
%
$$
Y = \left\{ \left\{ x_n \right\} \mymid x_n = 0 \quad \text{definitivamente}, x_n \in \Q \right\}
$$
%
e verifico che è numerabile e separabile.

\textbf{Domanda.} Cosa succede per $p = +\infty$?

Consideriamo $L^\infty ([0,+\infty], \mu)$ con $\mu$ di Lebesgue e $E_n = [n,n+1]$.
Definiamo l'insieme
 %
$$
Z = \left\{ \forall J \subset \N \quad u = \sum_{j \in J} \One_{E_j} \right\}.
$$
%
$Z$ ha la cardinalità delle parti di $\N$ cioè è più che numerabile. Osserviamo che per ogni $u,v \in Z$, $u \neq v$ si ha che $\norm{u - v}_{L^\infty(\R)} = 1$.
Se per assurdo esistesse un insieme denso e numerabile $D$ in $\ell^\infty$, per definizione di insieme denso dovremmo trovare per ogni palla di raggio minore di 1 e centro in un qualsiasi elemento di $Z$, un elemento di $D$. Ma questo è impossibile in quanto $D$ ha cardinalità numerabile e $Z$ la cardinalità del continuo.

Vediamo in un altro modo che $l^\infty$ non è separabile. 
Se per assurdo $Y = \left\{ \underline{x}^k \right\}_{k \in \N}$ fosse denso in $L^\infty$, allora potremmo definire un elemento $z \in l^\infty$ tale che $\norm{\underline{x}^k - \underline{z}}_{l^\infty} \geq 1$ per ogni $k$.

Definiamo $z = \left\{ z_n \right\}$ come segue
%
$$
z_n = 
\begin{cases}
0 \quad \text{se} \quad | x_n^n | > 1 \\
2 \quad \text{se} \quad | x_n^n | \leq 1
\end{cases}. 
$$
