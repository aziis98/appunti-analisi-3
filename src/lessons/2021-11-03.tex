%
% Lezione del 3 Novembre 2021
%

\section{Esercitazione del 3 Novembre 2021}

\subsection{Basi Hilbertiane e proiezioni}

\textbf{Esercizio.}
Sia $H = L^2(-1, 1)$ e sia $V = \spn\{ 1, x, x^2 \}$. Verificare che $V$ è un sottospazio chiuso e calcolare la proiezione di $\sin x$ su $V$. \\
\textit{Notazione.} Indichiamo con $\norm{\curry}$ la norma $\norm{\curry}_{L^2(-1, 1)}$ e con $\langle \curry, \curry \rangle$ il prodotto scalare su $L^2$.

\textbf{Soluzione.}
Vediamo come risolvere questo esercizio in tre modi diversi.

\begin{enumerate}
% \textbf{Osservazione.}
% Su $L^2(\R)$ e su $L^2(-1, 1)$ esistono di sicuro basi Hilbertiane di cardinalità numerabile in quanto abbiamo già visto che sono spazi separabili.

\item Dato $H$ spazio di Hilbert separabile, e dato un sottospazio $V \subset H$, vediamo come trovare la proiezione di un elemento $x \in H$ su $V$. Procediamo come segue.
\begin{itemize}
	\item Controlliamo che $V$ sia chiuso.

	\item Calcoliamo una base hilbertiana di $V$ che indichiamo con $\{ e_1, \dots, e_n, \dots \}$. 
\end{itemize}
Il tal caso, la proiezione di un elemento $x \in H$  su $V$ è data da
$$
	p_V(x) = \sum_n \langle x, e_n \rangle e_n.
$$

\textbf{Esercizio.} [TO DO: per casa] Ogni sottospazio di dimensione finita  di uno spazio di Hilbert o di $L^p$ è chiuso (e in particolare ha parte interna vuota).

Abbiamo una base di $V$ data da $\{1, x, x^2\}$ (è una base in quanto sono linearmente indipendenti: si può verificare mostrando che $\forall x \in [-1, 1] \; \lambda_1 + \lambda_2 x + \lambda_3 x^2 = 0 \implies \lambda_1 = \lambda_2 = \lambda_3 = 0$ usando la teoria sulle equazioni di II grado oppure si può derivare e man mano ottenere più informazioni su $\lambda_3, \lambda_2, \lambda_1$).

Volendo usare la base scritta sopra per calcolare la proiezione di $\sin x$ su $V$, dovremmo prima applicare Gram-Schmidt alla base $\{1,x,x^2\}$ per determinare una base Hilbertiana:
$$
\begin{aligned}
	e_1 &= \frac{1}{\norm{1}} = \frac{1}{\sqrt{2}} \\
	e_2 &= \frac{x - \langle x, \frac{1}{\sqrt 2} \rangle \cdot 1}{\norm{x - \langle x, \frac{1}{\sqrt 2} \rangle \cdot 1}} \\
	e_3 &= \frac{x^2 - \langle x^2, e_1 \rangle \cdot e_1 - \langle x^2, e_2 \rangle \cdot e_2}{\norm{x^2 - \langle x^2, e_1 \rangle \cdot e_1 - \langle x^2, e_2 \rangle \cdot e_2}} \\
\end{aligned}
$$
e successivamente calcolare $p_V(\sin x)$ con la formula scritta sopra.

\item Alternativamente, possiamo direttamente cercare la proiezione di $\sin x$ su $V$. Determiniamo $a, b, c$ tali che $a + b x + c x^2$ sia $p_V(x) = \sin x$ allora posto $f(x) \coloneqq \sin x - a - b x - c x^2$ abbiamo $f(x) \in V^\perp \iff $ si verificano le seguenti condizioni
$$
\langle f(x), 1 \rangle = 0
\qquad
\langle f(x), x \rangle = 0
\qquad
\langle f(x), x^2 \rangle = 0
$$
Ad esempio da $\langle f(x), 1 \rangle = 0$ otteniamo
$$
0 = \int_{-1}^1 (\sin x - a - b x - c x^2) \cdot 1 \dd x 
= \underbrace{\int_{-1}^1 \sin x \dd x}_{=0} - 2a - b \underbrace{\int_{-1}^1 x \dd x}_{=0} - c \int_{-1}^1 x^2
\implies 0 = -2a - \frac{2}{3}c
$$
ed analogamente si procede con $x$ e $x^2$... [TODO: Da finire]

\item Un altro modo è considerare la funzione $g(a, b, c) \coloneqq \norm{\sin x - a - bx - cx^2}_{L^2(-1, 1)}$ che è continua, coerciva, etc. e imponendo $\nabla_{a,b,c,} g = 0$ si minimizza e si ottengono $\bar a, \bar b, \bar c$ che verificano $p_V(\sin x)$.

\end{enumerate}

% \textbf{Osservazione.}
% Su $L^2(\R)$ e su $L^2(-1, 1)$ esistono di sicuro basi Hilbertiane di cardinalità numerabile in quanto abbiamo già visto che sono spazi separabili.

% Dato $H$ spazio di Hilbert separabile, e dato un sottospazio $V \subset H$, vediamo come trovare la proiezione di un elemento $x \in H$ su $V$.

% Procediamo come segue.
% \begin{itemize}
% 	\item Controlliamo che $V$ sia chiuso.

% 	\item Calcoliamo una ``base hilbertiana di $V$'' che indichiamo con $\{ e_1, \dots, e_n, \dots \}$. 
% \end{itemize}
% Il tal caso, la proiezione di un elemento $x \in H$  su $V$ è data da
% $$
% 	p_V(x) = \sum_n \langle x, e_n \rangle e_n.
% $$

% \textbf{Esercizio.} [TO DO: per casa] Ogni sottospazio di dimensione finita  di uno spazio di Hilbert o di $L^p$ è chiuso (e in particolare ha parte interna vuota).

% \textbf{Soluzione.}
% Indichiamo con $\norm{\curry}$ la norma $\norm{\curry}_{L^2(-1, 1)}$ e con $\langle \curry, \curry \rangle$ il prodotto scalare su $L^2$.

% Abbiamo una base di $V$ data da $\{1, x, x^2\}$ in quanto sono linearmente indipendenti (si può verficare mostrando che $\forall x \in [-1, 1] \; \lambda_1 + \lambda_2 x + \lambda_3 x^2 = 0 \implies \lambda_1 = \lambda_2 = \lambda_3 = 0$ usando la teoria sulle equazioni di II grado oppure si può derivare e man mano ottenere più informazioni su $\lambda_3, \lambda_2, \lambda_1$).

% Volendo usare la base scritta sopra per calcolare la proiezione di $\sin x$ su $V$, dovremmo prima applicare Gram-Schmidt.
% $$
% \begin{aligned}
% 	e_1 &= \frac{1}{\norm{1}} = \frac{1}{\sqrt{2}} \\
% 	e_2 &= \frac{x - \langle x, \frac{1}{\sqrt 2} \rangle \cdot 1}{\norm{x - \langle x, \frac{1}{\sqrt 2} \rangle \cdot 1}} \\
% 	e_3 &= \frac{x^2 - \langle x^2, e_1 \rangle \cdot e_1 - \langle x^2, e_2 \rangle \cdot e_2}{\norm{x^2 - \langle x^2, e_1 \rangle \cdot e_1 - \langle x^2, e_2 \rangle \cdot e_2}} \\
% \end{aligned}
% $$

% Alternativamente, possiamo direttamente cercare la proiezione di $\sin x$ su $V$. Determiniamo $a, b, c$ tali che $a + b x + c x^2$ sia $p_V(x) = \sin x$ allora posto $f(x) \coloneqq \sin x - a - b x - c x^2$ abbiamo $f(x) \in V^\perp \iff $ si verificano le seguenti condizioni
% $$
% \langle f(x), 1 \rangle = 0
% \qquad
% \langle f(x), x \rangle = 0
% \qquad
% \langle f(x), x^2 \rangle = 0
% $$
% Ad esempio da $\langle f(x), 1 \rangle = 0$ otteniamo
% $$
% 0 = \int_{-1}^1 (\sin x - a - b x - c x^2) \cdot 1 \dd x 
% = \underbrace{\int_{-1}^1 \sin x \dd x}_{=0} - 2a - b \underbrace{\int_{-1}^1 x \dd x}_{=0} - c \int_{-1}^1 x^2
% \implies 0 = -2a - \frac{2}{3}c
% $$
% ed analogamente si procede con $x$ e $x^2$... [TODO: Magari finire questo esercizio veramente]
%
% Un altro modo è considerare la funzione $g(a, b, c) \coloneqq \norm{\sin x - a - bx - cx^2}_{L^2(-1, 1)}$ che è continua, coerciva, etc. e imponendo $\nabla_{a,b,c,} g = 0$ si minimizza e si ottengono $\bar a, \bar b, \bar c$ che verficano $p_V(\sin x)$.


\textbf{Esercizio.}
Sia $X = \{ u \in L^2(\R) \mid \int_0^2 u \dd x = 0 \}$, dire se è un sottospazio chiuso, calcolare $X^\perp$ per una generica $u \in L^2(\R)$ e determinare le proiezioni $p_X(u)$ e $p_{X^\perp}(u)$.

\textbf{Soluzione.}
La mappa $T$ lineare data da
$$
u \mapsto \int_0^2 u \dd x
$$
è ben definita, lineare e continua, allora $X$ è proprio $T^{-1}(0)$ dunque è un sottospazio chiuso. 

\textbf{Osservazione.} Notiamo che
$$
T(u) 
= \int_\R u(x) \cdot \One_{[0, 2]}(x) \dd x
= \langle u, g \rangle_{L^2(\R)},
\qquad g = \One_{[0, 2]}(x) \in L^2(\R).
$$

% E dunque $X = \{ u \in L^2 \mid \langle u, g \rangle = 0 \}$.

Calcoliamo ora $X^\perp$ e le proiezioni $p_X$, $p_{X^\perp}$. 
Abbiamo che $X = \{ u \in L^2(\R) \mid \langle u, g \rangle = 0 \}$ dove $g = \One_{[0,2]}(x)$, dunque $X^\perp = \spn(g)$.
Notiamo\footnote{Con $g / \norm{g^\perp}_{L^2}$ indichiamo una base normalizzata di $g^\perp$.} che
$$
L^2(\R) = \spn_\R\left\{\frac{g}{\norm{g}_{L^2}}\right\} \oplus \left\{ \frac{g^\perp}{\norm{g^\perp}_{L^2}} \right\}.
$$
Calcoliamo $p_X(u)$ come segue
%
$$
p_X(u) = u - \left<u, \frac{g}{\norm{g}_{L^2}} \right> \cdot \frac{g}{\norm{g}_{L^2}},
$$
%
dove
$$
\norm{g}_{L^2} = \left( \int_\R \One_{[0, 2]}(x)^2 \dd x \right)^{1/2} = \sqrt 2
$$
e dunque
$$
p_X(u) = u - \left( \int_0^2 u \dd x \right) \cdot \frac{\One_{[0,2]}}{2}
$$

Un controllo veloce per verificare di aver fatto i conti corretti è quello di vedere che $p_X(u) \in X$, dunque di verificare che $\int_0^2 p_X(u) \dd x = 0$.

Per calcolare $p_{X^\perp}(u)$ usiamo la seguente.

\textbf{Osservazione.} Vale $u = p_X(u) + p_{X^\perp}(u)$.

In conclusione,
%
$$
p_{X^\perp}(u) = u - p_X(u) = \left( \int_0^2 u \dd x \right) \cdot \frac{\One_{[0,2]}}{2}.
$$
%


\textbf{Esercizio.} [TO DO: per casa.]
Sia $V = \{ \underline x = (x_n)_{n \in \N} \in \ell^2 \mid x_1 + x_3 + x_5 = 0 \}$. 
Dire se $V$ è chiuso in $\ell^2$ e calcolare $p_V$ e $p_{V^\perp}$.



\subsection{Approssimazioni per convoluzione}

Abbiamo visto che data $g \in L^1(\R^d)$ con $\int g \dd x = 1$ allora per ogni $f \in L^p(\R^d)$ abbiamo $f_\delta \coloneqq f \ast \sigma_\delta g \xrightarrow{\delta \to 0} f$ in $L^p(\R^d)$ per $p \neq \infty$.

\textbf{Esercizio.}
Dire se esiste $v \in L^1(\R)$ tale che sia elemento neutro della convoluzione, ovvero
$$
\forall f \in L^1(\R) \qquad f \ast v = f.
$$
Una tale $v$ non esiste, per vederlo scegliamo opportunamente $\bar f$ e usiamo l'equazione. 
Scelgo $g \in C_C(\R)$ e defiamo $\sigma_\delta g = 1 / \delta g(1/\delta)$. Abbiamo che $\sigma_\delta g \ast v = \sigma_\delta g$ per ogni $\delta$.
Per il \hyperlink{thm:lez25ott_teodelta}{teorema} abbiamo che $\sigma_\delta g \ast v = \sigma_\delta g \xrightarrow{\delta \to 0} v$ in $L^1(\R)$, ma $\sigma_\delta g \xrightarrow{\delta \to 0} 0$ quasi ovunque in $L^1(\R)$. Allora $v = 0$ q.o. in $L^1(\R)$, dunque non può valere $f \ast v = v$ per ogni $f \in L^1(\R)$.

\textbf{Esercizio.} [TO DO: per casa.] 
Sia $f$ misurabile su $\R^d$ tale che $\int_E f \dd x = 0$ per ogni $E$ misurabile di $\R^d$. Dimostrare che $f = 0$ q.o. su $\R^d$.

\textit{Suggerimento.} Considerare l'integrale sull'insieme $\{ x \in \R^d \mid f(x) = 0 \} \cup \{ x \in \R^d \mid f(x) \neq 0 \}$ e verificare che, se denotiamo $A =  \{ x \in \R^d \mid f(x) \neq 0 \}$, allora $|A| = 0$.

\textbf{Esercizio.} [TO DO: per casa.]
Sia $f$ Lebesgue-misurabile su $\R^d$ tale che $\forall B$ palla su $\R^d$
$$
\int_B f \dd x = 0
$$
Dimostrare che che $f = 0$ quasi ovunque su $\R^d$.

\textit{Suggerimenti.} Usare la convoluzione con opportuni nuclei; notare che $\ds \int_B f = 0 \longiff f \ast \One_B = 0$ per ogni palla $B$.



\section{Esempi di basi Hilbertiane}


\subsection{Polinomi}

La base data da $ \{ 1, x, x^2, \dots, x^n, \dots \} $ opportunamente ortonormalizzata è una base\footnote{Teorema di Stone-Weierstrass: i polinomi \\ sono densi nello spazio delle funzioni continue.} di $L^2[0, 1]$ (anche di $L^2(\R)$).


\subsection{Base di Haar}

\begin{wrapfigure}{r}{225pt}
	\centering
	\vspace{-4\baselineskip}
	\inputfigure{base-di-haar-1} 
	\vspace{-4.5\baselineskip}
\end{wrapfigure}

Vediamo la base di Haar data da due indici $n, k$ dove $n$ indica l'ampiezza delle ``onde'' (anche dette \textit{wavelet}) e $k$ il posizionamento dell'onda. Sia $n \in \N$ e $k = 1, \dots, 2^n$ e poniamo
$$
g^{0,0} \coloneqq \One_{[0,1]}
\qquad
g^{n,k} \coloneqq 2^\frac{n-1}{2} \left( \One_{\left[ \frac{2k - 2}{2^n}, \frac{2k - 1}{2^n} \right]} - \One_{\left[ \frac{2k - 1}{2^n}, \frac{2k}{2^n} \right]} \right)
$$

Inoltre $\| g^{n, k} \|_{L^2[0, 1]} = 1$ ed anche $\| g^{0,0} \|_{L^2[0, 1]} = 1$. Vedremo che $\{ g^{n,k} \mid n \geq 1, k = 1, \dots, 2^n \} \cup \{ g^{0,0} \}$ formano un sistema ortonormale.

\begin{itemize}
	\item $\langle g^{n,k}, g^{0,0} \rangle = 0$: È ovvio in quanto le $g^{n,k}$ hanno media nulla.

	\item $\langle g^{n,k}, g^{n',k'} \rangle = 0$: Se $n = n'$ i supporti sono sempre disgiunti altrimenti $n \neq n'$, se supponiamo $n < n'$ allora i supporti o sono disgiunti e si conclude come prima o il supporto di $g^{n',k'}$ è contenuto in quello di $g^{n,k}$. In tal caso però $g^{n,k}$ è costante su $g^{n',k'}$ e dunque l'integrale è sempre nullo.
\end{itemize}

Inoltre è anche una base hilbertiana, per combinazioni algebriche si ottengono tutti gli intervalli della forma
$$
I_k \coloneqq \left[ \frac{k-1}{2^n}, \frac{k}{2^n} \right]
\qquad
\rightsquigarrow
\qquad
\One_{I_k}
$$
ad esempio normalizzando $g^{n,k} + 2^\frac{n-1}{2} g^{0, 0}$ otteniamo uno degli intervalli di sopra di lunghezza $1 / 2^{n+1}$.

Vedremo che possiamo estendere la base di Haar a tutto $\R$ però è più difficile... [TODO: Ehm aggiungere la parte dopo quando verrà fatta]

\textbf{Esercizio.}
Sia $p \geq 1$ allora $\{ u \in L^p(\R) \mid \int u \dd x = 0 \} \subseteq L^p(\R)$ è denso in $L^p(\R)$?

