%
% Lezioni del 27-28 Ottobre 2021
%

\chapter{Spazi di Hilbert}

Sia $H$ spazio vettoriale reale con prodotto scalare $\left<\cdot, \cdot \right>$ definito positivo e norma indotta $\norm{\cdot}$ definita come $\norm{x} = \sqrt{\left<x,x \right>}$.

Si ricorda l'identità di polarizzazione
%
$$
\left<x_1,x_2 \right> = \frac{1}{4} \left( \norm{x_1 + x_2}^2 - \norm{x_1 - x_2}^2 \right).
$$
%

\textbf{Nota.} Siccome $\norm{\cdot }$ è continua, dalla formula di polarizzazione segue che il prodotto scalare è continuo.

\textbf{Definizione.} $H$ si dice \textbf{spazio di Hilbert} se è completo.

\textbf{Esempi.} 
\begin{itemize}
\item Dato $(X, \mc{A}, \mu )$, gli spazi $L^2(X), L^2(X, \R^m)$ sono spazi di Hilbert.

\item Lo spazio $\ds \ell^2 = \left\{ (x_n) \mymid \sum_{n=0}^{\infty} x_n^2 < +\infty  \right\}$ è uno spazio di Hilbert.

\end{itemize}

\textbf{Definizione.} $\mc{F} \subset H$ è un \textbf{sistema ortonormale} se
%
$$
\norm{e} = 1 \mquad \forall e \in \mc{F}, \qquad  \left<e,e' \right> = 0 \mquad \forall e \neq e' \in \mc{F}.
$$
%


\textbf{Definizione.} $\mc{F}$ si dice \textbf{completo} se $\overline{\spn(\mc{F})} = H$\footnote{lo span sono combinazioni lineari finite}. In tal caso $\mc{F}$ si dice \textbf{base di Hilbert}.

\textbf{Osservazione.} Se $H$ ha dimensione infinita non esistono basi ortonormali di Hilbert. \textit{Attenzione.} Le basi algebriche esistono, sono quelle ortonormali a non esistere.

\textbf{Esempi.}
In $\ell^2$ una base ortonormale è $\mc{F} = \left\{ e_n \mymid n \in \N \right\}$ con $e_n = (0,\ldots ,0,\underbrace{1}_{i},0,\ldots )$. \\
Infatti, il fatto che siano ortonormali è banale; verifichiamo che sia una base. 
Studiamo $\spn(\mc{F}) = \left\{ x = (x_0,x_1,\ldots ) \mymid x_n \quad \text{è definitivamente nullo}  \right\}$: dato $x \in \ell^2$ e $m = 0,1,2,\ldots $, definiamo
%
$$
P_mx \coloneqq (x_0,x_1,\ldots , x_m,0,\ldots ).
$$
%
Allora $\spn(\mc{F}) \supset P_m x \xrightarrow{m \to +\infty} x$ in $\ell^2$.
Infatti, 
%
$$
x - P_m x = (0,\ldots ,0, x_{m+1}, x_{m+2},\ldots ).
$$
%
Dunque
%
$$
\norm{x - P_m x} = \sum_{n = m+1}^{\infty} x_n^2 \xrightarrow{m \to +\infty} 0.
$$
%

\textbf{Teorema 1.} (della base di Hilbert.) Dato $H$ spazio di Hilbert, $\mc{F}$ sistema al più numerabile\footnote{il caso interessante è quello numerabile}, ovvero $\mc{F} = \left\{ e_n \mymid n \in \N \right\}$.
Definiamo per ogni $x \in H$, $n \in \N$ l'elemento $x_n = \left<x, e_n \right>$.
Allora
\begin{enumerate}
\item \label{item:27ott_thm1_1}
Vale $\ds \sum_{n} x_n^2 \leq \norm{x}^2$ (\textbf{Disuguaglianza di Bessel}).

\item \label{item:27ott_thm1_2}
La somma $\ds \sum_n x_n e_n$ converge a qualche $\overline{x} \in H$ e $\overline{x}_n = x_n$  per ogni $n$.

\item \label{item:27ott_thm1_3}
Vale $\ds \norm{\overline{x}}^2 = \sum_n x_n^2 \leq \norm{x}^2$.

\item \label{item:27ott_thm1_4}
Se $x - \overline{x} \perp \mc{F}$, allora $x - \overline{x} \perp \overline{\spn(\mc{F})}$, ovvero $\overline{x}$ è la proiezione di $x$ su $\overline{\spn(\mc{F})}$.

\item \label{item:27ott_thm1_5}
Se $\mc{F}$ è completo, allora $x = \overline{x}$ e in particolare
%
$$
x = \sum_{n=0}^{\infty} x_n e_n, \qquad \norm{x}^2 = \sum_{n=0}^{\infty} x_n^2  \qquad \text{(\textbf{Identità di Parceval})}.
$$
%
\end{enumerate}

\textbf{Nota.} Il punto \ref{item:27ott_thm1_2} non segue dal fatto che la serie non è assolutamente convergente. Infatti
%
$$
\sum \norm{x_n e_n} = \sum \left| x_n \right|
$$
%
può essere $+\infty$.


Alla dimostrazione del teorema premettiamo il seguente lemma.

\vs

\textbf{Lemma.} Siano $H$ e $\mc{F}$ come nel teorema.
Data $(a_n) \in \ell^2$, allora
\begin{enumerate}
\item La somma $\ds \sum_n a_n e_n$ converge a qualche $\overline{x} \in H$. 

\item $\overline{x}_n = a_n$.

\item $\ds \norm{\overline{x}}^2 = \sum_n a_n^2$.
\end{enumerate}

\textbf{Dimostrazione lemma.} 
\begin{enumerate}
\item Dimostriamo che $\ds y_n = \sum_{n = 1}^{m} x_n e_n$ è di Cauchy in $H$.
Se $m' > m$, vale
%
$$
y_{m'} - y_m = \sum_{n = m+1}^{m'} x_n e_n
\Longrightarrow  \norm{y_{m'} - y_m}^2 = \norm{\sum_{n = m+1}^{m'} x_n e_n}^2 
= \sum_{n = m+1}^{m'} x_n^2 \leq \sum_{n = m+1}^{\infty} x_n^2 < +\infty.
$$
%
Dunque, per ogni $\epsilon$ esiste $m_\epsilon$ tale che $\ds \sum_{m + 1}^{\infty} x_n^2 \leq \epsilon^2 $, per cui 
%
$$
\norm{y_{m'} - y_m}^2 \leq \sum_{m+1}^{\infty} x_n^2 \leq \sum_{m_\epsilon + 1}^{\infty} x_n^2 \leq \epsilon^2 \quad \forall m,m' \geq m_\epsilon.
$$
%

\item Per ipotesi, $\left<y_m, e_n \right> = x_n$ se $m \geq n$, dunque
%
$$
\left<y_n, e_n \right> \xrightarrow{n \to \infty} \left<\overline{x},e_n \right> = x_n.
$$
%

\item Si ha l'uguaglianza $\ds \norm{y_n}^2 = \sum_{n=1}^{m} x_n^2$, per cui passando al limite per $n \to +\infty$ otteniamo 
%
$$
\norm{y_n}^2 \to \norm{\overline{x}}^2, \qquad \sum_{n=0}^{m} x_n^2 \to \sum_{n=0}^{\infty} x_n^2.
$$
%


\end{enumerate}
\qed


\textbf{Dimostrazione teorema.}

\begin{enumerate}
\item Studiamo la somma
%
$$
x = \sum_{n=0}^{\infty} x_n e_n + \overbrace{y}^{\text{resto}}
$$
%
notiamo che $x$ è somma di vettori ortogonali, infatti $y$ è ortogonale a $\ds \sum_{n=0}^{\infty} x_n e_n$ :
%
$$
\left<y,e_i \right> = \left<x - \sum_{n=0}^{\infty} x_n e_n, e_i \right>
= \left<x,e_i \right> - \sum_{n=0}^{\infty} x_n \underbrace{\left<e_n,e_i \right>}_{\delta_{i,n}} = x_i - x_i = 0.
$$
%
Essendo che $x$ è somma di vettori ortogonali abbiamo
%
$$
\norm{x}^2 = \sum_{n=1}^{\infty} x_n^2 + \norm{y}^2 \geq \sum_{n=1}^{m} x_n^2.
$$
%
Passando al limite per $m \to +\infty$ otteniamo
%
$$
\norm{x}^2 \geq \sum_{n=1}^{\infty} x_n^2. 
$$
%

\item Segue dal punto precedente e dal lemma.

\item Segue dal punto precedente e dal lemma.

\item Notiamo che 
%
$$
\left<x - \overline{x}, e_n \right> = \left<x,e_n \right> - \left<\overline{x} , -e_n \right> = x_n - \overline{x}_n 
\overbrace{=}^{\ref{item:27ott_thm1_2}} x - \overline{x} \perp e_n \mquad \forall n \Longrightarrow x - \overline{x} \perp \spn(\mc{F}).
$$
%
Segue che $x - \overline{x} \perp \overline{\spn(\mc{F})}$. (dato $y \in$ chiusura span.., prendo $y_m \to y \in span$, $\left< x- \overline{x}, y_n \right> = 0 \Rightarrow  \left<x - \overline{x}, y \right> = 0$).

\item $x - \overline{x} \perp \overline{\spn(\mc{F})} \underbrace{=}_{\mathclap{\text{$\mc{F}$ è completo}}} H \Longrightarrow x - \overline{x} = 0$, cioè $x = \overline{x}$.
\end{enumerate}
\qed

\textbf{Corollario.} Siano $H$ spazio di Hilbert, $\mc{F} = \left\{ e_n \mymid n \in \Z \right\}$ base di Hilbert, $x,x' \in H$. Valgono le seguenti.
\begin{enumerate}
\item Se $x_n = x_n'$ per ogni $n \in \N$, allora $x = x'$ ($\Leftarrow$ è ovvia.)

\item $\ds \left<x,x' \right> = \sum_{n=0}^{\infty} x_n  x_n'$ (\textbf{Identità di Parceval}).

\item L'applicazione $H \ni X \mapsto (x_n) \in \ell^2$ è un'isometria surgettiva\footnote{in particolare è bigettiva ma l'iniettività è ovvia}.
\end{enumerate}

\vs 

\textbf{Dimostrazione.}

\begin{enumerate}
\item Per l'enunciato \ref{item:27ott_thm1_5} se due vettori hanno la stessa rappresentazione rispetto a una base di Hilbert coincidono.

\item La tesi segue usando l'identità di polarizzazione conigiuntamente all'enunciato \ref{item:27ott_thm1_5} del teorema:
%
\begin{align*}
\left<x,x' \right> & = \frac{1}{4} \left( \norm{x + x'}^2 + \norm{x - x'}^2 \right)
= \frac{1}{4} \Big( \sum_n \overbrace{(x_n + x_n')^2}^{x_n^2 + x_n'^2 + 2x_n x_n'}  \sum_n \underbrace{(x_n - x_n')^2}_{x_n^2 + x_n'^2 - 2 x_n x_n'}  \Big) \\
& = \frac{1}{4} \left( \cancel{\sum x_n^2} + \cancel{\sum x_n'^2} + 2 \sum x_n x_n' - \cancel{\sum x_n^2} - \cancel{\sum x_n'^2} + 2 \sum x_n x_n' \right).
\end{align*}

\item TO DO.

\end{enumerate}
\qed

\textbf{Osservazioni.}
\begin{itemize}
\item Gli enunciati \ref{item:27ott_thm1_1} e \ref{item:27ott_thm1_5} non richiedono $H$ completo.

\item Se $H$ è uno spazio di Hilbert e $\mc{F}$ sistema ortonormale infinito, allora $\mc{F}$ non è mai una base algebrica\footnote{per base algebrica s'intende un insieme di vettori di uno spazio vettoriale le cui combinazioni lineari generano tutto lo spazio}. Dunque combinazioni lineari finite di $H$ non sono mai uguali ad $H$, ovvero $\spn(\mc{F}) \subsetneq H $) di $H$.

\textbf{Dimostrazione.} 
Notiamo che possiamo scrivere $\overline{x}$ come
%
$$
(e_n) \subset \mc{F}, \quad  \overline{x} = \sum_{n = 0}^\infty \frac{1}{2^n} e_n, \quad  \overline{x} \notin \spn(\mc{F}).
$$
%
\textbf{Nota.} I coefficienti sono univocamente determinati perchè ottenuti tramite prodotto scalare. Notiamo che non si può usare il teorema di albebra lineare sull'unicità della rappesentazione poichè stiamo trattando combinazioni lineari infite.
\qed

\item Siano $H$ uno spazio di Hilbert di dimensione infinita e $\mc{F}$ una base di Hilbert. Allora, $\mc{F}$ è numerabile se solo se $H$ è separabile.

\textbf{Dimostrazione.}

\begin{itemize}

\item[$\boxed{\Rightarrow}$] Vale $ H = \overline{\spn(\mc{F})} = \overline{\spn_\Q (\mc{F})} $ e notando che $\overline{\spn_\Q (\mc{F})}$ è numerabile se  $\mc{F}$ è numerabile.

\item[$\boxed{\Leftarrow}$] Se $\mc{F}$ non è numerabile, siccome $\norm{e - e'} = \sqrt{2}  \quad \forall e,e' \in \mc{F}$ allora $H$ non è separabile.

\end{itemize}
\qed

\textbf{Esempio.} Lo spazio $H = L^2(X)$, con $X = \R^n$, $\mu $ misura di Lebesgue ha base di Hilbert numerabile.

\item Dato $\mc{F}$ sistema ortonormale in $H$, allora $\mc{F}$ è completo se solo se $\mc{F}$ è massimale (nella classe dei sistemi ortonormali rispetto all'inclusione).

\textbf{Dimostrazione.}
\begin{itemize}

\item[$\boxed{\Rightarrow}$] Dato che $\mc{F}$ è completo segue che $\overline{\spn(\mc{F})} = X$, quindi
%
$$
\mc{F}^{\perp} = \left( \spn(\mc{F}) \right)^\perp
\underbrace{=}_{\mathclap{\substack{\text{continuità del} \\ \text{prodotto scalare}}}} \overline{\spn(\mc{F})}^\perp = H^\perp = \{ 0 \}.
$$
%
dunque $\mc{F}$ è massimale.

\item[$\boxed{\Leftarrow}$] Se  $\mc{F}$ non è completo, esiste $c \in H \setminus \spn(\mc{F})$.
Definiamo $\overline{x}$ come nel Teorema 1. Notiamo che $x - \overline{x} \perp \spn(\mc{F})$, dunque $x - \overline{x} \perp \mc{F}$ e $x - \overline{x} \neq \{ 0 \}$, da cui $\ds \mc{F} \; \cup \; \left\{ \frac{x - \overline{x}}{\norm{x - \overline{x}}} \right\}$ è un sistema ortonormale che include strettamente $\mc{F}$. \absurd

\end{itemize}


\textbf{Osservazione.} Nell'implicazione $\boxed{\Rightarrow}$ non abbiamo usato la completezza di $H$.

\end{itemize}

\vs

\textbf{Corollario.} Ogni sistema ortonormale $\mc{F}$ si completa a $\tilde{\mc{F}}$ base di Hilbert di $H$.

\textbf{Dimostrazione.} Sia $X = \left\{ \mc{F} \; \text{sistema ortonormale} H \text{ tale che } \tilde{\mc{F}} \subset \mc{F} \right\}$
Per Zorn, $X$ contiene un elemento massimale. Denotiamolo con $\tilde{\mc{F}}$. Allora $\tilde{\mc{F}}$ è una base di Hilbert.
\qed

\vs

\textbf{Nota.} Aggiungere le note a caso.

\vs

\textbf{Teorema 2.} Dato $V$ sottospazio vettoriale chiuso di $H$. Allora
\begin{enumerate}
\item $H = V + V^\perp$, cioè per ogni $x \in H$ esiste $\overline{x} \in V$ e $\tilde{x} \in V$ tale che $x = \overline{x} + \tilde{x}$.

\item Gli elementi $\overline{x}$ e $\tilde{x}$ sono univocamente determinati (e indicati con $x_V$ e $x_V^\perp$).

\item $\overline{x}$ è caratterizzato come l'elemento di $V$ più vicino a $X$.
\end{enumerate}

\textbf{Dimostrazione.}
\begin{enumerate}
\item Dato che $V$ è chiuso, $V$ è completo, cioè $V$ è un sottospazio di $H$, dunque $V$ ammette base ortonormale $\mc{F} = \left\{ e_n \mymid n \in \N \right\}$.
Definiamo $\overline{x} \in \overline{\spn(\mc{F})}$ come nel Teorema 1 e $\tilde{x} \coloneqq x - \overline{x} \in \overline{\spn(\mc{F})} = V^\perp$ (per \ref{item:27ott_thm1_4}).

\item Se $x = \overline{x} + \tilde{x} = \overline{x}' + \tilde{x}'$, dove $\overline{x}, \overline{x}' \in V$ e $\tilde{x}, \tilde{x}' \in V^\perp$, allora
%
$$
\overline{x} - \overline{x}' = \tilde{x}' - \tilde{x} \underbrace{\Longrightarrow}_{V \cap V^\perp = \{0 \} }
\overline{x} - \overline{x}' = \tilde{x}' - \tilde{x} = 0.
$$
%

\item Per ogni $y \in V$ sia $f(y) = \norm{x - y}^2$. Mostriamo che $\overline{x}$ è l'unico minimo di $f$.
%
$$
f(y) = \norm{x - y}^2 = \lVert{\overbrace{x - \overline{x}}^{\in V^\perp} + \overbrace{\overline{x} - y}^{\in V}}\rVert^2 = \norm{x - \overline{x}}^2 + \norm{\overline{x} - y}^2
= f(\overline{x}) + \norm{\overline{x} - y}^2 \geq f(\overline{x}).
$$
%
\qed
\end{enumerate}

\vs

\textbf{Osservazione.}
Serve $V$ chiuso. Dato $\mc{F}$ base di Hilbert di $H$ ($H$ dimensione infinita), $V \coloneqq \spn(\mc{F})$, so che $V = \spn(\mc{F}) \subset \overline{V} = H$. Allora
%
$$
\overline{V^\perp} = \overline{V}^\perp = H^\perp = \{0\} \Longrightarrow V + V^\perp = V \subsetneq H.
$$
%

\textbf{Teorema 3.} Dato $\Lambda \colon  H \to \R$ lineare e continuo, esiste $x_0 \in H$ tale che 

\begin{equation} \tag{$\ast$}
	\Lambda(x) = \left<x,x_0 \right> \quad \text{per ogni} \; x \in H.
\end{equation}

\textbf{Dimostrazione.}
Supponiamo $\Lambda \not\equiv 0$. Dato che $\Lambda$ è continuo, $\ker \Lambda$ è chiuso in $H$. Definiamo $V \coloneqq \ker \Lambda$.
Per il primo enunciato del teorema precedente, $H = V + V^\perp$ e per quanto supposto $V^\perp \neq \{0\}$.

Notiamo che $\dim V^\perp = 1$. Infatti, se per assurdo $\dim(V^\perp) > 1$, allora esisterebbe $W \subset V^\perp$ con $\dim W = 2$, da cui seguirebbe che $\Lambda \colon  W \to \R$ ha $\ker$ banale. \absurd

Allora $V^\perp = \spn \left\{ x_1 \right\}$ con $\norm{x_1} = 1$. Definiamo $c \coloneqq \Lambda(x_1), x_0 = c x_1$.

Dimostriamo ora l'uguaglianza $\ast$ per passi.

\begin{enumerate}
\item Vale per $x \in V$ tale che $x \in \ker \Lambda$. Infatti $\Lambda(x)=x_0$ e $\left<x,x_0 \right> = 0$ perchè $x_0 \in V^\perp$.

\item Vale per $x = x_1$ (e quindi per $x \in V^\perp$). Infatti, 
%
$$
\Lambda(x_1) = c \quad \left<x_1,x_0 \right> = \left<x_1, c x_1 \right> = c \norm{x_1}^2 = c.
$$
%

\item Vale su $V + V^\perp = H$.
\end{enumerate}

\textbf{Osservazioni.}
\begin{itemize}
\item Esistono funzioni $\Lambda \colon  H \to \R$ lineari ma non continue se $H$ ha dimensine infinita.

\textbf{Dimostrazione.} Prendo $\Lambda \colon H \to \R$ lineare definito come
%
$$
\begin{cases}
\Lambda(e_n) = n \quad \forall n \\
\Lambda(e) = \text{qualsiasi } e \in \mc{G} \setminus \left\{ e_n \right\}.
\end{cases} 
$$
%
Allora
%
$$
+\infty = \sup_n \left| \Lambda(e_n) \right| \leq \sup_{\norm{x} \leq 1} \left| \Lambda(x) \right| 
$$
%
da cui segue che $\Lambda$ non è continuo.

\end{itemize}

