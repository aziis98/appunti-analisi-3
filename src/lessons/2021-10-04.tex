%
% Lezione del 4 ottobre
%

\section{Esercitazione del 4 ottobre}

\subsection{Esercizi di teoria della misura}

Di seguito riportiamo alcune proprietà di base di teoria della misura.

\textbf{Proprietà.}

\begin{enumerate}
\item Se $A \subset B$, allora $\mu(A) \leq \mu(B)$.

\textbf{Dimostrazione.}
Scomponiamo $B = (B \setminus A) \cup (A \cap B)$. Per ipotesi $A \cap B = A$ ed essendo la misura positiva segue che
$$
	\mu(B) = \underbrace{\mu(B \setminus A)}_{\geq 0} + \mu(A) \geq \mu(A).
$$

\item \label{item:misura_unione_finita} Dati due insiemi $A,B$ misurabili, vale
$$
	\mu(A \cup B) \leq \mu(A) + \mu(B).
$$

\textbf{Dimostrazione.}
La disuguaglianza segue dalle seguenti uguaglianze.

\begin{align*}
	\mu(A) & = \mu(A \setminus B) + \mu(A \cap B) \\
	\mu(B) & = \mu(B \setminus A) + \mu(A \cap B) \\
	\mu(A \cup B) & = \mu(A \setminus B ) + \mu(B \setminus A) + \mu(A \cap B).
\end{align*}

\item Data una successione di insiemi $E_1 \subset E_2 \subset \cdots \subset \cdots$, si ha
$$
	\mu \left( \bigcup_{i} E_i \right) = \sup_i \mu(E_i) = \lim_i \mu (E_i).
$$

\item Data una successione di insiemi $E_1 \supset E_2 \supset \cdots \supset \cdots$ e $\mu(E_1) < +\infty$, si ha
$$
	\mu \left( \bigcap_{i} E_i \right) = \lim_i \mu (E_i).
$$
\end{enumerate}

\textbf{Esercizio} (Numerabile subaddittività).
Dato $\ds E \in \mc{A}, E \subset \bigcup_i E_i$ dove $E_i \in \mc{A}$. Allora
$$
	\mu(E) \leq \sum_{i}^{} \mu(E_i).
$$

\textbf{Dimostrazione} (Idea).
Basta dimostrare che $\ds \mu \left( \bigcup_i E_i \right) \leq \sum_{i}^{} \mu(E_i)$. Infatti per quanto visto prima $\ds \mu(E) \leq \mu \left( \bigcup_i E_i \right)$. Prima dimostriamo per induzione $\ds \mu \left( \bigcup_{i = 1}^N E_i \right) \leq \sum_{i=1}^{N} \mu(E_i)$. 

Il passo base $n = 2$ è stato visto al punto \ref{item:misura_unione_finita}. Una volta dimostrata la proprietà sopra, si nota che $\ds \sum_{i=1}^{N} \mu(E_i) $ è limitata per ogni $N$, e dunque è limitato anche il suo limite, da cui la tesi.
\qed

\subsection{Funzioni misurabili rispetto alla misura di Lebesgue}

Si ricorda che le funzioni \textit{continue}, \textit{semplici} e \textit{semicontinue} sono classi di funzioni misurabili.
Due osservazioni sulle funzioni semicontinue.
\begin{itemize}
\item Le funzioni semicontinue sono \textit{boreliane}.

\item La proprietà di misurabilità delle funzioni semicontinue è necessaria per l'enunciato della disuguaglianza di Jensen.
\end{itemize}

\textbf{Controesempio} (disuguaglianza di Jensen).
Notiamo che l'ipotesi di semicontinuità inferiore della funzione $f$ è necessaria per la validità della disuguaglianza di Jensen.
Infatti, definiamo $f$ come segue
%
$$
f(x) = 
\begin{cases}
0 \qquad \; x \in (0,1) \\
+ \infty \quad \text{altrimenti} 
\end{cases}.
$$
%
Osserviamo che la funzione $f$ così definita è convessa ma non semicontinua inferiormente.

Ora definiamo la funzione $u : X \to \R$ con $X = (0,2)$, come la funzione costante di valore $1/2$.
Calcoliamo l'integrale di $u(x)$ su $X$.
%
$$
	\int_{X} u(x) \dd x = 1. 
$$
%
In tal caso vale $\ds f \left( \int_X u(x) \dd x \right) = + \infty$.
D'altra parte $\ds \int_X f \compose u \dd x = 0$, dunque l'ipotesi di semicontinuità inferiore è necessaria.

\textbf{Fatto.}
Date $\myphi_1, \myphi_2$ funzioni semplici su $\R$ con misura di Lebesgue.
Allora $\myphi_1 \vee \myphi_2$ e $\myphi_1 \wedge \myphi_2$ sono ancora funzioni semplici.

\textbf{Lemma.}
Data $f \colon X \to [0, +\infty]$ misurabile
$$
\int_X f \dd \mu = 0 \quad \longiff \quad f = 0 \; \text{q.o. su } X
$$

\textbf{Dimostrazione.}
\begin{itemize}

\item[$\boxed{\Rightarrow}$] Dato che $f$ è non negativa, il dominio $X$ può essere riscritto come
$$
	X = \left\{ x \in X \mymid f(x) \geq 0 \right\} = \left\{ x \in X \mymid f(x) > 0 \right\} \cup \left\{ x \in X \mymid f(x) = 0 \right\}
$$
ricordiamo che $(0, +\infty) = \bigcup_{n \geq 1} (\frac{1}{n}, +\infty)$, dunque possiamo riscrivere una parte di $X$ come segue e poi passare alle misure
$$
\begin{gathered}
	\left\{ x \in X \mymid f(x) > 0 \right\} =  \bigcup_{n \in \N \setminus \left\{ 0 \right\}} \left\{ x \in X \mymid f(x) \geq \frac{1}{n} \right\} \\
	\implies \mu \left( \left\{ x \in X \mymid f(x) > 0 \right\} \right) 
	= \lim_{n \to +\infty} \mu\left(\left\{ x \in X \mymid f(x) \geq \frac{1}{n} \right\}\right)
\end{gathered}
$$
in questo modo otteniamo la seguente caratterizzazione dell'insieme su cui $f$ è positiva
$$
\mu \left( \left\{ x \in X \mymid f(x) > 0 \right\} \right) >0 
\longiff
\exists \bar{n} \text{ tale che } \mu \left( \left\{ x \in X \mymid f(x) \geq 1 / \bar{n} \right\} \right) > 0
$$
A questo punto possiamo maggiorare come segue
$$
0 = \int_X f \dd \mu 
\geq \;\int_{\left\{ f \,\geq\, \frac{1}{n} \right\}} \; f \dd \mu \geq	\frac{1}{n} \mu \left(  \left\{ x \mymid f(x) \geq \frac{1}{n} \right\} \right). 
$$
da cui ricaviamo che $\forall n$ vale
$$
\mu \left(  \left\{ x \mymid f(x) \geq \frac{1}{n} \right\} \right) = 0
$$
e si conclude osservando che
$$
\mu \left(  \left\{ x \mymid f(x) > 0 \right\} \right) = \lim_n \mu \left( \left\{ x \mymid f(x) \geq \frac{1}{n} \right\} \right) = 0
$$

\item[$\boxed{\Leftarrow}$]
Dal fatto che $f$ è positiva possiamo scrivere
$$
	\int_X f \dd \mu = \sup_{\substack{g \leq f \\ g \; \text{semplice}}} \int_X g \dd \mu = \sup \sum_{i}^{} \alpha_i \mu(E_i) = 0. 
$$
\qed

\end{itemize}

\textbf{Osservazione} (sup essenziale di funzioni misurabili).
Data $f$ misurabile, definiamo
$$
	\norm{ f }_{\infty, X} \coloneqq \inf \left\{ m \in [0,+\infty] \mymid \left| f(x) \right| \leq m \quad \text{quasi ovunque}  \right\}.
$$

Se $\norm{ f }_{\infty} < + \infty$, allora diciamo che esiste una costante $L > 0$ con $L = \norm{ f }_{\infty, X}$, tale che 
$$
	\left| f(x) \right| \leq L
$$
quasi ovunque. 
Infatti, per definizione di $\inf$, $L = \lim_n m_n$, dove $m_n$ verificano
$$
	\left| f(x) \right| \leq m_n \quad \forall x \in X \setminus N_m, \quad \mu(N_m) = 0.
$$
Definiamo $N = \bigcup_{m} N_m$, da cui si ottiene
$$
	\mu(N) \leq \sum_{n=1}^{\infty} \mu (N_m) = 0. 
$$
Ovvero $N$ è trascurabile.
Preso $x \in X \setminus N$, vale
$$
	\left| f(x) \right| \leq m_n \quad \forall n \in \N.
$$


\subsection{Formula di cambio di variabile applicata a funzioni radiali}

Sia $f \colon [0,+\infty) \to \R$ misurabile (di solito si richiede misurabile e positiva oppure sommabile). Allora per il teorema di cambio di variabili vale la seguente
$$
	\int_{\R^n} f\left( \left| x \right| \right) \dd x = c_n \cdot \int_{0}^{+\infty} f(\rho) \rho^{n-1} \dd \rho,
$$
dove $\ds c_n = n \mathscr L^n \left( \mc{B}(0,1) \right)$.

Applichiamo questa formula alla stima di integrali di funzioni positive.

\textbf{Esercizio.}
Sia
$$
	\psi (x) = \frac{1}{\norm{ x }^{\alpha}}
$$
su $\mc{B}(0,1) \in \R^n$. Notiamo che $\psi(x) = f(\norm{ x })$ con $f = 1 / t^\alpha$.
Usiamo la formula appena introdotta per determinare gli $\alpha \in \R$ per i quali $\psi$ è sommabile su $\mc{B}(0,1)$.
$$
\int_{\mc{B}(0,1)}^{} \psi(x) \dd x 
= c_n \int_0^1 \frac{1}{\rho^\alpha} \rho^{n -1} \dd \rho 
= c_n \int_0^1 \rho^{n-1-\alpha} \dd \rho =
\begin{cases}
	\log (\rho) \quad n = \alpha \\
	\dfrac{\rho^{n-\alpha}}{n - \alpha} \quad \text{altrimenti} 
\end{cases} 
$$
Concludendo,
$$
	\int_{\mc{B}(0,1)}^{} \frac{1}{\norm{ x }^\alpha} \dd x < + \infty \longiff n > \alpha.
$$

\textbf{Esercizio.}
Con passaggi analoghi al precedente otteniamo
$$
	\int_{\R^n \setminus \; \mc{B}(0,1)} \frac{1}{\norm{ x }^\alpha} \dd x < + \infty \longiff n < \alpha.
$$
\textbf{Esercizio.}
Vediamo per quali valori di $\beta$ il seguente integrale converge
$$
	\int_{\mc{B}(0,1)}^{} \frac{1}{\left( \norm{ x } + \norm{ x }^2 \right)^\beta} \dd x
$$

Vale la seguente catena di uguaglianze.
%
$$
\int_{\R^n}^{} \frac{1}{\left( \norm{ x } + \norm{ x }^2 \right)^\beta} \dd x 
= \int_{\mc{B}(0,1)}^{} \frac{1}{\left( \norm{ x } + \norm{ x }^2 \right)^\beta} \dd x 
+ \int_{\R^n \setminus \; \mc{B}(0,1)}^{} \frac{1}{\left( \norm{ x } + \norm{ x }^2 \right)^\beta} \dd x.
$$
%
Studiamo separatamente i due pezzi dell'integrale.
%
\begin{align*}
	\int_{\mc{B}(0,1)}^{} \frac{1}{\left( \norm{ x } + \norm{ x }^2 \right)^\beta} \dd x 
	& = c_n \int_{\mc{B}(0,1)}^{} \frac{1}{(\rho + \rho^2)^\beta} \rho^{n-1} \dd \rho
	= c_n \int_{0}^{1} \frac{1}{\rho^\beta} \cdot \frac{\rho^{n-1}}{(1 + \rho)^\beta} \dd \rho \\
	& \approx \int_{0}^{1} \rho^{n-1-\beta} \dd \rho < + \infty \longiff  \beta < n.
\end{align*}
%
Inoltre,
%
$$
	\int_{\R^n \setminus \mc{B}(0,1)}^{} \frac{1}{\left( \norm{ x } + \norm{ x }^2 \right)^\beta} \dd x 
	= \int_{\R^n \setminus \; \mc{B}(0,1)}^{} \frac{1}{\rho^{2\beta}} \cdot \frac{\rho^{n-1}}{\left( \frac{1}{\rho} + 1 \right)^\beta} \dd \rho 
	\approx \int_{1}^{+\infty} \frac{\rho^{n-1}}{\rho^{2\beta}} \dd \rho < + \infty \longiff 2\beta > n.  
$$
%
In conclusione, l'integrale è finito se $n > \beta > n / 2$.

% @aziis98: Molto probabilmente metterò un disegnino con assi $n$ e $\beta$ per far vedere "meglio" l'insieme dei valori buoni

\textbf{Esercizio.}
Studiare l'insieme di finitezza al variare del parametro $\alpha$ dell'integrale
$$
	\int_{[0,1]^n}^{} \frac{1}{\norm{ x }_1^\alpha} \dd x .
$$
Osserviamo che la norma 1 e 2 sono legate dalle seguenti disuguaglianze
$$
	\frac{\norm{ x }_1}{n} \leq \norm{ x }_2 \leq \norm{ x }_1.
$$
%
Studiamo una maggiorazione per l'integrale
%
$$
	\int_{[0,1]^n}^{} \, \frac{1}{\norm{ x }_1^\alpha} \dd x 
	\leq \;\int_{[0,1]^n}^{} \, \frac{1}{\norm{ x }^\alpha} \dd x 
	\leq \;\;\int_{B(0,\sqrt{n})}^{}\;\; \frac{1}{\norm{ x }^\alpha} \dd x < + \infty \longiff \alpha < n,
$$
%
dunque
%
$$
\int_{[0,1]^n}^{} \, \frac{1}{\norm{ x }_1^\alpha} \dd x < +\infty \quad \text{se} \; \alpha < n.
$$
%
Vediamo ora una minorazione.

$$
	\int_{[0,1]^n}^{} \frac{1}{\norm{ x }_1^\alpha} \dd x 
= \frac{1}{2^n}	\int_{[-1,1]^n}^{} \frac{1}{\norm{ x }_1^\alpha} \dd x 
\geq \frac{1}{2^n} \ \int_{\mc{B}(0,1)}^{} \frac{1}{\norm{ x }_1^\alpha} \dd x 
\approx \int_{\mc{B}(0,1)}^{} \frac{1}{\norm{ x }^\alpha} \dd x < + \infty 
\longiff \alpha < n.
$$

Dunque l'integrale $\ds \int_{[0,1]^n}^{} \frac{1}{\norm{x}_1^\alpha} \dd x$ converge se solo se $\alpha < n$.

\newpage

\textbf{Esercizi per casa.}
\begin{enumerate}[label=(\arabic*)]

\item Dimostrare che date $f,g$ misurabili ed $r,p_1,p_2 > 0$ tali che  $1 / r = 1 / p_1 + 1 / p_2$.
Allora vale
$$
	\norm{ f \cdot g }_r \leq \norm{ f }_{p_1} \cdot \norm{ g }_{p_2}.
$$
\textit{Suggerimento.} Usare Hölder osservando che $\ds 1 = \frac{r}{p_1} + \frac{r}{p_2} = \frac{1}{\left( p_1/r \right)} + \frac{1}{\left( p_2/r \right)}$.

\textbf{Dimostrazione.} Vale quanto segue.
\begin{align*}
	\norm{f \cdot g}_r^r 
	&= \int_X \left| f \cdot g \right|^r \dd \mu
	= \int_X |f| \cdot |g| \dd \mu
	\overset{\text{Holder}}{\leq} \norm{f^r}_{p_1 / r} \cdot \norm{f^r}_{p_2 / r} \\
	& = \left( \int_X |f|^{r \cdot p_1 / r} \right)^{r / p_1} \cdot \left( \int_X |g|^{r \cdot p_2 / r} \right)^{r / p_2}
	= \norm{f}_{p_1}^r \cdot \norm{g}_{p_2}^r
	= \left( \norm{f}_{p_1} \cdot \norm{g}_{p_2} \right)^r \\
	& \Longrightarrow \norm{f \cdot g}_r \leq \norm{f}_{p_1} \cdot \norm{g}_{p_2}.
\end{align*}
\qed

\item Dimostrare che date $f_1,\ldots, f_n$ misurabili e $p_i > 0$ tali che $1/p_1 + \ldots + 1/p_n = 1$ si ha
$$
	\norm{ f_1 \cdots f_n }_1 \leq \norm{ f_1 }_{p_1} \cdots \norm{ f_n }_{p_n}.
$$
\textit{Suggerimento.} Fare il primo passo dell'induzione e usare la formula precedente scegliendo $r$ in modo corretto.

\textbf{Dimostrazione.} Dimostriamo per induzione la seguente proprietà più generale.

Siano $f_1,\ldots, f_n$ misurabili e $r > 1$. Allora, per i $p_i > 0$ tali che $1/p_1 + \ldots + 1/p_n = r$ si ha
$$
	\norm{ f_1 \cdots f_n }_r \leq \norm{ f_1 }_{p_1} \cdots \norm{ f_n }_{p_n}.
$$

\textit{Passo base.} Vero per il punto (1).

\textit{Passo induttivo} $[n-1 \Rightarrow n]$. Supponiamo di aver dimostrato per ogni $r > 1$ la disuguaglianza sopra. Allora
%
$$
\norm{f_1 \cdots f_n}_r 
= \norm{(f_1 \cdots f_{n-1}) \cdot f_n}_r 
\overset{(1)}{\leq} \norm{f_1 \cdots f_{n-1}}_p \cdot \norm{f_n}_{p_n}, \qquad \text{dove} \; r = 1/p + 1/p_n.
$$
%
Notando che $1/p = 1/r - 1/p_{n-1} = 1/p_1 + \cdots + 1/p_{n-1}$ e usando l'ipotesi induttiva otteniamo la tesi.
\qed

\end{enumerate}
