%
% Lezione dell'20 Ottobre 2021
%

% \textbf{Definizione.}
% Date $f_1, f_2 \colon \R^d \to \R$ misurabile allora il \textbf{prodotto di convoluzione} è dato da
% $$
% f_1 \ast f_2 (x) \coloneqq \int_{\R^d} f_1(x - y) f_2(y) \dd y
% $$
% e se $f_1$ e $f_2$ sono positive allora $f_1 \ast f_2(x) \in [0, +\infty]$. Ma ad esempio se prendiamo $f_1 = 1$ e $f_2 = \sin x$ con $d = 1$ allora $f_1 \ast f_2(x)$ non è definito per alcun $x$.

\textbf{Proposizione 1.}
Se $|f_1| \ast |f_2| (x) < +\infty$ allora $f_1 \ast f_2(x)$ è ben definito, in quanto $|f_1 \ast f_2(x)| \leq |f_1| \ast |f_2|(x)$.
% in quanto
% $$
% |f_1 \ast f_2(x)| \leq |f_1| \ast |f_2|(x)
% $$

\textbf{Dimostrazione.}
Basta osservare che,
%
\begin{align*}
f_1 \ast f_2(x) & = \int_{\R^d} f_1(x-y) \cdot f_2(y) \dd y 
\leq \left| \int_{\R^d} f_1(x-y) \cdot f_2(y) \dd y \right| \\
& \leq \int_{\R^d} |f_1(x-y) \cdot f_2(y)| \dd y
= |f_1| \ast |f_2|(x) < +\infty.
\end{align*}
\qed

\textbf{Corollario 2.}
Se $|f_1| \ast |f_2| \in L^p(\R^d)$ con $1 \leq p \leq +\infty$ allora $f_1 \ast f_2(x)$ è ben definito per quasi ogni $x \in \R^d$ e $\norm{f_1 \ast f_2}_p \leq \norm{|f_1| \ast |f_2|}_p$.

\textbf{Dimostrazione.}
Segue subito dalla proposizione precedente.
\qed

\textbf{Teorema 3} (disuguaglianza di Young per convoluzione.)
Se $f_1 \in L^{p_1}(\R^d)$ e $f_2 \in L^{p_2}(\R^d)$ e preso $r \geq 1$ tale che
\begin{equation}\label{eqn:conv_th3_cond}
	\frac{1}{r} = \frac{1}{p_1} + \frac{1}{p_2} - 1,
	\tag{$\star$}
\end{equation}
allora $f_1 \ast f_2$ è ben definito quasi ovunque e
\begin{equation}\label{eqn:conv_th3_thesis}
	\norm{f_1 \ast f_2}_r \leq \norm{f_1}_{p_1} \cdot \norm{f_2}_{p_2}
	\tag{$\star\star$}
\end{equation}

\textbf{Osservazioni.}
\begin{itemize}
	\item 
		Nel caso di prima $1$ e $\sin x$ sono solo in $L^\infty$ infatti viene $r = -1$ e la disuguaglianza non ha senso.

	\item
		Supponiamo di avere $\norm{f_1 \ast f_2} \leq C \cdot \norm{f_1}_{p_1}^{\alpha_1} \cdot \norm{f_2}_{p_2}^{\alpha_2}$ allora vediamo che per ogni $f_1, f_2$ positiva deve valere necessariamente $\alpha_1 = \alpha_2 = 1$ e la condizione (\ref{eqn:conv_th3_cond}).

		\textbf{Dimostrazione.}
		Per ogni $\lambda > 0$ consideriamo $\lambda f_1$ e $f_2$, allora 
		$$
		\norm{(\lambda f_1) \ast f_2}_r = \norm{\lambda (f_1 \ast f_2)}_r = \lambda \norm{f_1 \ast f_2}_r
		$$
		ma abbiamo anche
		$$
		\norm{(\lambda f_1) \ast f_2}_r \leq C \cdot \norm{\lambda f_1}_{p_1}^{\alpha_1} \cdot \norm{f_2}_{p_2}^{\alpha_2} = C \cdot \lambda^{\alpha_1} \norm{f_1}_{p_1}^{\alpha_1} \cdot \norm{f_2}_{p_2}^{\alpha_2},
		$$
        da cui necessariamente $\alpha_1 = 1$ e di conseguenza $\alpha_2 = 1$.

		A questo punto richiediamo anche che $f_1$ e $f_2$ siano tali che $\norm{f_1}_{p_1}, \norm{f_2}_{p_2} < +\infty$ e $\norm{f_1 \ast f_2} > 0$ (questo possiamo farlo in quanto basta prendere $f_1 = f_2 = \One_B$ con $B$ una palla, nel caso segue proprio che $f_1 \ast f_2 (x) > 0$ se $|x| < 1$).

		Data $f \colon \R^d \to \R$ e $\lambda > 0$ poniamo $R_{\lambda} f(x) \coloneqq f(x / \lambda)$ allora abbiamo
        \begin{align*}
        \norm{(R_\lambda f_1) \ast (R_\lambda f_2)}_r
        & = \norm{\int_{\R^d} f_1 \left( \frac{x-y}{\lambda} \right) \cdot f_2\left( \frac{y}{\lambda} \right) \dd y} 
        = 
        \begin{pmatrix}
        t = \frac{y}{\lambda} \\
        \lambda \dd t = \dd y
        \end{pmatrix} \\
        & = \lambda^d \cdot \norm{\int_{\R^d} f_1 \left( \frac{x}{\lambda} - t \right) \cdot f_2(t) \dd t} \\
        & = \lambda^d \cdot \norm{R_\lambda (f_1 \ast f_2)}_r.
        \end{align*}
        Per il punto successivo abbiamo $\norm{R_\lambda(g)}_r = \lambda^{d/r} \norm{g}_r$, da cui otteniamo
		$$
		\norm{(R_\lambda f_1) \ast (R_\lambda f_2)}_r 
		= \lambda^{d \left( 1 + \frac{1}{r} \right)} \norm{f_1 \ast f_2}_r.
		$$
		Ma anche
		$$
		\norm{(R_\lambda f_1) \ast (R_\lambda f_2)}_r 
		\leq C \cdot \norm{R_\lambda f_1}_{p_1} \cdot \norm{R_\lambda f_2}_{p_2}
		= \lambda^{d \left( \frac{1}{p_1} + \frac{1}{p_2} \right)} \cdot \norm{f_1}_{p_1} \cdot \norm{f_2}_{p_2}.
		$$
		Dunque sicuramente abbiamo $\lambda^{d \left(1 + {1 / r} \right)} \leq C \cdot \lambda^{d\left( {1 / p_1} + {1 / p_2} \right)}$ per ogni $\lambda > 0$ e quindi $1 + {1 / r} = {1 / p_1} + {1 / p_2}$.
		\qed

	\item
		$\norm{R_\lambda f}_p = \lambda^{d / p} \norm{f}_p$ ed in realtà possiamo ricavare l'esponente $d / p$ per \textit{analisi dimensionale}\footnote{Ovvero studiando le potenze delle unità di misura delle varie quantità.}.
        % \footnote{In particolare ad Istituzioni di Analisi si vedono le disuguaglianze di Sobolev ed anche in quel caso tutte le condizioni sugli esponenti si riescono a ricavare per analisi dimensionale...}. 
        Consideriamo l'espressione
		$$
		\norm{f}_p^p = \int_{\R^d} f(x) \dd x.
		$$
		Se $f(x)$ è una \textit{quantità adimensionale} allora $\int_{\R^d} f \dd x$ ha dimensione di un \textit{volume} $\mathsf L^d$, da cui $\norm{f}_p$ ha dimensione di $\mathsf L^{d / p}$.

		Similmente, per ottenere $\norm{R_\lambda(f_1 \ast f_2)}_r = \lambda^{d(1 + 1 / r)} \norm{f_1 \ast f_2}_r$, basta osservare che
		$$
		f_1 \ast f_2 (x) = \int_{\R^d} f_1(x - y) f_2(y) \dd y
		$$
		ha dimensione $\mathsf L^d$, da cui
		$$
		\norm{f_1 \ast f_2}_r = \bigg( \int_{\R^d} \underbrace{|f_1 \ast f_2|^r}_{\mathsf L^{dr}} \underbrace{\dd x}_{\mathsf L^d} \bigg)^{1 / r}
		$$
		ha dimensione di $\mathsf L^{d(1 + 1/r)}$.
\end{itemize}

\textbf{Dimostrazione Teorema 3.}
Per via del Corollario 2. ci basta dimostrare (\ref{eqn:conv_th3_thesis}) se $f_1, f_2 \geq 0$.
\begin{itemize}
	\item
		\textit{Caso facile.} Se $p_1 = p_2 = 1$ e $r = 1$
		$$
		\begin{aligned}
			\norm{f_1 \ast f_2}_1
			&= \int f_1 \ast f_2 (x) \dd x 
			= \iint f_1(x - y) f_2(y) \dd y \dd x 
			= \int f_2(y) \int f_1(x - y) \dd x \dd y = \\
			&= \int \norm{f_1}_1 \cdot f_2(y) \dd y 
			= \norm{f_1}_1 \cdot \norm{f_2}_1
		\end{aligned}
		$$

	\item
		\textit{Caso leggermente meno facile.} Se $p_1 = p, p_2 = 1$ e $r = p$.
		Vogliamo vedere che
		$$
		\norm{f_1 \ast f_2}_p \leq \norm{f_1}_p \cdot \norm{f_2}_1
		$$
		allora
		$$
		\begin{aligned}
			\norm{f_1 \ast f_2}_p
			&= \int_{\R^d} (\underbrace{f_1 \ast f_2}_{h})^p \dd x
			= \int h \cdot h^{p-1} \dd x 
			= \iint f_1(x - y) f_2(y) h^{p-1}(x) \dd y \dd x = \\
			&= \iint f_1(y - x) h^{p-1}(x) \dd x f_2(y) \dd y 
			\overset{\text{H\"older}}{\leq} 
			\int \norm{f_1(y - \curry)}_p {\| h^{p-1} \|}_{p'} f_2(y) \dd y
		\end{aligned}
		$$
		con $p'$ esponente coniugato a $p$. Inoltre notiamo che $\norm{f_1(y - \curry)}_p = \norm{f_1}$ per invarianza di $\mathscr L^d$ per riflessioni e traslazioni. Infine otteniamo
		$$
		\norm{f_1}_p {\| h^{p-1} \|}_{p'} \norm{f_2}_1
		= \norm{f_1}_p \norm{h}_{p}^{p-1} \norm{f_2}_1.
		$$
		Dunque, $\norm{f_1 \ast f_2}_p^p \leq \norm{f_1 \ast f_2}_p^{p-1} \norm{f_1}_p \norm{f_2}_1 \implies \norm{f_1 \ast f_2}_p \leq \norm{f_1}_p \norm{f_2}_1$. Quest'ultima implicazione però è valida solo nel caso in cui $0 < \norm{f_1 \ast f_2}_p < +\infty$. Resterebbero da controllare i due casi in cui la norma è $0$ oppure $+\infty$. Il primo è ovvio; il secondo invece si fa per approssimazione e passando al limite.

		Consideriamo $f_1, f_2$ e approssimiamole con $f_{1,n}, f_{2,n}$ limitate a supporto compatto, allora vale $\norm{f_{1,n} \ast f_{1,n}}_p \leq \norm{f_{1,n}}_p \cdot \norm{f_{2,n}}_1$ e passando al limite si ottiene la tesi. In particolare possiamo costruire le $f_n$ come
		$$
		f_n(x) \coloneqq (f(x) \cdot \One_{\mc B(0, n)}(x)) \land n
		$$

		\textbf{Osservazione.}
		Se $f_2 \geq 0$ e $\int f_2 \dd x = 1$ allora $\norm{f_1 \ast f_2}_p \leq \norm{f_1}_p$ è una versione semplificata della proposizione precedente, in particolare la dimostrazione si semplifica in quanto possiamo pensare a $f_2$ come distribuzione di probabilità e quindi $f_1 \ast f_2$ è una ``media pesata'' delle traslazioni di $f_1$ o più precisamente una combinazione convessa ``integrale''.

	\item
		\textit{Caso generale.} Non lo facciamo perché servono mille mila parametri e non è troppo interessante.
\end{itemize}
\qed

Nel caso $r = +\infty$ gli esponenti $p_1$ e $p_2$ sono proprio coniugati e possiamo rafforzare la tesi del teorema precedente.

\textbf{Teorema 4} (caso $r = +\infty$ del Teorema 3).
Dati $p_1$ e $p_2$ esponenti coniugati e $f_1 \in L^{p_1}(\R^d), f_2 \in L^{p_2}(\R^d)$, allora
\begin{enumerate}
	\item \label{item:20ott_th4_1} 
		$f_1 \ast f_2(x)$ è ben definito per ogni $x \in \R^d$

	\item \label{item:20ott_th4_2}
		$|f_1 \ast f_2(x)| \leq \norm{f_1}_{p_1} \norm{f_2}_{p_2}$

	\item \label{item:20ott_th4_3} 
		$f_1 \ast f_2$ è uniformemente continua

	\item \label{item:20ott_th4_4}
		Se $1 < p_1, p_2 < +\infty$ allora $f_1 \ast f_2 \to 0$ per $|x| \to +\infty$
\end{enumerate}

Premettiamo i seguenti risultati.

\textbf{Proposizione 5.}
Data $f \in L^p(\R^d)$ con $p < +\infty$ la mappa
$$
\begin{array}{cccc}
	\tau_h f : & \R^d & \to & L^p(\R^d) \\
	& h & \mapsto & f(\curry - h)
\end{array}
$$
è continua.

\textbf{Lemma 6.}
Lo spazio $C_0(\R^d) = \{ f \colon \R^d \to \R \text{ continue con } f(x) \to 0 \text{ per } |x| \to \infty \}$ è chiuso rispetto alla convergenza uniforme.


\textbf{Dimostrazione Teorema 4.}
\begin{enumerate}
\item Osserviamo che
$$
|f_1| \ast |f_2|(x) = \int_{\R^d} |f_1(x-y)| \cdot |f_2(y)| \dd y
\overset{\text{Hölder}}{\leq} \norm{f_1(x - \curry)}_{p_1} \norm{f_2}_{p_2}
= \norm{f_1}_{p_1} \norm{f_2}_{p_2}
$$
e concludiamo per la Proposizione 1.

\item Dal punto precedente abbiamo che $|f_1| \ast |f_2|(x) \leq \norm{f_1}_{p_1} \norm{f_2}_{p_2}$, da cui si conclude banalmente.

\item Uno tra $p_1$ e $p_2$ è finito; supponiamo lo sia $p_1$.
Fissiamo $x,h \in \R^d$
%
$$
	f_1 \ast f_2(x+h) - f_1 \ast f_2(x) = \int_{\R^d} \left( f_1(x+h-y) - f_1(x-y) \right) f_2(y) \dd y,
$$
%
quindi
\begin{align*}
	\left| f_1 \ast f_2(x+h) - f_1 \ast f_2(x) \right|
	& \leq \int \left| f_1(x+h - y) - f_1(x - y) \right| | f_2| \dd y  \\
	& \underset{\text{Holder}}{\leq} \norm{f_1(x+h - \curry) - f_1(x - \curry)}_{p_1} \norm{f_2}_{p_2} \\
	& = \norm{f_1(\curry - h) - f_1(\curry)}_{p_1} \norm{f_2}_{p_2} \\
	& = \underbrace{\norm{\tau_h f_1 - f_1}_{p_1}}_{\xrightarrow[\text{Proposizione 5}]{h \to 0} 0} \norm{f_2}_{p_2}
\end{align*}
da cui segue la tesi\footnote{\textit{Nota.} In generale, quanto appena mostrato ci direbbe che la funzione è continua, ma essendo che stiamo maggiorando con una quantità indipendente da $x$ segue l'uniforme continuità.}.

\item Approssimiamo $f_1$  e $f_2$ con $f_{1,n}$ e $f_{2,n} \in \mc{C}_C(\R^d)$ in $L^{p_1}$ e $L^{p_2}$ rispettivamente.

Osserviamo che $f_{1,n} \ast f_{2,n} \in \mc{C}_C(\R^d) \subset \mc{C}_0(\R^d)$.
Per il Lemma 6 basta dimostrare che $f_{1,n} \ast f_{2,n} \longrightarrow f_1 \ast f_2$ uniformemente
%
\begin{align*}
	\norm{f_{1,n} \ast f_{2,n} - f_1 \ast f_2}_\infty
	& = \norm{\left( f_{1,n} \ast f_{2,n} - f_{1,n} \ast f_2 \right) + \left( f_{1,n} \ast f_2 - f_1 \ast f_2 \right)}_\infty \\
	& \underset{\mathclap{\text{lin della conv}}}{\leq} \norm{f_{1,n} \ast \left( f_{2,n} - f_2 \right)}_\infty + \norm{\left( f_{1,n} - f_1 \right) \ast f_2}_\infty \\
	& \underset{ii)}{\leq} \underbrace{\norm{f_{1,n}}_{p_1}}_{\to \norm{f_1}_{p_1}} \underbrace{\norm{f_{2,n} - f_2}_{p_2}}_{\to 0} + \underbrace{\norm{f_{1,n} - f_1}_{p_1}}_{\to 0} \norm{f_2}_{p_2}.
\end{align*}

Quindi $\norm{f_{1,n} \ast f_{2,n} - f_1 \ast f_2}_\infty \to 0$.
\qed
\end{enumerate}

% \textbf{Dimostrazione \ref{item:20ott_th4_1} e \ref{item:20ott_th4_2}.}
% Seguono subito da (\ref{eqn:conv_th3_cond}) per $f_1, f_2 \geq 0$ (con il Corollario 2.), se $f_1, f_2 \geq 0$ allora
% $$
% f_1 \ast f_2 (x) 
% = \int_{\R^d} f_1(x - y) f_2(y) \dd y 
% \leq \norm{f_1(x - \curry)}_{p_1} \norm{f_2}_{p_2} 
% = \norm{f_1}_{p_1} \norm{f_2}_{p_2}
% $$

% \textbf{Proposizione 5.}
% Data $f \in L^p(\R^d)$ con $p < +\infty$ la mappa
% $$
% \begin{array}{cccc}
% 	\tau_h f : & \R^d & \to & L^p(\R^d) \\
% 	& h & \mapsto & f(\curry - h)
% \end{array}
% $$
% è continua.

% \textbf{Lemma 6.}
% Lo spazio $C_0(\R^d) = \{ f \colon \R^d \to \R \text{ continue con } f(x) \to 0 \text{ per } |x| \to 0 \}$ è chiuso rispetto alla convergenza uniforme.

% \textbf{Dimostrazione \ref{item:20ott_th4_3}.} 
% Supponiamo $p_1 < +\infty$ allora
% $$
% \begin{aligned}
% 	|f_1 \ast f_2(x + h) - f_1 \ast f_2(x)|
% 	&\leq \int |f_1(x + h - y) - f_1(x - y)| \cdot |f_2(y)| \dd y \\
% 	&\leq \norm{f_1(x + h - \curry) - f(x - \curry)}_{p_1} \norm{f_2}_{p_2} \\
% 	&= \norm{\tau_h f_1 - f_1}_{p_2} \norm{f_2}_{p_2}
% \end{aligned}
% $$
% \qed

