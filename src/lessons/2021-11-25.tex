%
% Lezione del 25 Novembre 2021
%

\section{Esercitazione del 25 Novembre 2021}

\textbf{Esercizio.}
Consideriamo l'equazione alle derivate parziali
\begin{equation}
	\tag{P}
	\begin{cases}
		u_{ttt}(t, x) = u_{xx}(t, x) & x \in [-\pi, \pi] \\
		u(\curry, \pi) = u(\curry, -\pi) \\
		u_x(\curry, \pi) = u_x(\curry, -\pi) \\
		u(0, \curry) = u_0 \\
		u_{x}(0, \curry) = u_1 \\
		u_{xx}(0, \curry) = u_2 \\
	\end{cases}
\end{equation}
ponendo $c_n^i \coloneqq c_n(u_i) \text{ per } n \in \Z$ per $i = 1, 2, 3$. Segue subito che il problema di Cauchy sui coefficienti è
\begin{equation}
	\tag{P$'$}
	\begin{cases}
		\dddot c_n(t) = -n^2c_n(t) \\
		c_n(0) = c_n^0 \\
		\dot c_n(0) = c_n^1 \\
		\ddot c_n(0) = c_n^2 \\
	\end{cases}
	\qquad
	\forall n \in \Z
\end{equation}
che ha polinomio caratteristico $p(\lambda) = \lambda^3 + n^2 \implies \lambda^3 = -n^2$ e dunque le soluzioni sono $\lambda_i = n^{2/3} \zeta_6^{2i - 1}$ con $\zeta_6$ una radice sesta dell'unità. Per comodità per $i = 1, 2, 3$ poniamo $z_i \coloneqq n^{2/3} \omega_i$ con $\omega_i$ soluzioni di $\omega^3 = -1$ che possiamo anche riscrivere come
$$
\omega_1 = \frac{1}{2} + \frac{\sqrt 3}{2}i
\qquad
\omega_2 = \frac{1}{2} - \frac{\sqrt 3}{2}i
\qquad
\omega_3 = -1
$$
Dunque per $n \in \Z$ e $n \neq 0$ la soluzione sarà
$$
\left\{
\begin{aligned}
	& c_n(t) = A_n e^{-z_1^n t} + B_n e^{-z_2^n t} + C_n e^{-z_3^n t} \\[1ex]
	& c_n(0) = c_n^0 = A_n + B_n + C_n \\[1ex]
	& c_n(1) = c_n^1 = A_n z_1^n + B_n z_2^n + C_n z_3^n \\[1ex]
	& c_n(2) = c_n^2 = A_n (z_1^n)^2 + B_n (z_2^n)^2 + C_n (z_3^n)^2
\end{aligned}
\right.
$$
e quindi otteniamo il sistema
$$
\implies
\left\{
\begin{aligned}
	& c_n^0 = A_n + B_n + C_n \\[1ex]
	& n^{-2/3} c_n^1 = A_n \omega_1 + B_n \omega_2 + C_n \omega_3 \\[1ex]
	& n^{-4/3} c_n^2 = A_n \omega_1^2 + B_n \omega_2^2 + C_n \omega_3^2
\end{aligned}
\right.
\qquad
\textcolor{lightgray}{\rightsquigarrow
\begin{pmatrix}
	1 & 1 & 1 \\
	\omega_1 & \omega_2 & \omega_3 \\
	\omega_1^2 & \omega_2^2 & \omega_3^2
\end{pmatrix}
\begin{pmatrix}
	A_n \\
	B_n \\
	C_n
\end{pmatrix}
=
\begin{pmatrix}
	c_n^0 \\
	n^{-2/3} c_n^1 \\
	n^{-4/3} c_n^2
\end{pmatrix}}
$$
e facendo conti si ottengono $A_n, B_n$ e $C_n$ e si scopre che 
[TODO: Controllare i conti con Mathematica]
$$
\begin{aligned}
	A_n e^{n^{2/3}(t/2 + i \sqrt 3 t / 2)} &\sim e^{n^{2/3}t / 2}
	&\xrightarrow{\;\,t \to \infty\,\;} \;\, \infty \\
	B_n e^{n^{2/3}(t/2 - i \sqrt 3 t / 2)} &\sim e^{n^{2/3}t / 2}
	&\xrightarrow{\;\,t \to \infty\,\;} \;\, \infty \\
	C_n e^{-n^{2/3}} &\sim e^{-n^{2/3} t}
	&\xrightarrow{t \to -\infty} \;\, \infty \\
\end{aligned}
$$
dunque in realtà anche se il problema in partenza sembrava ben definito in realtà non ha soluzione per alcun $t \in \R$.

\textit{Conti esatti con Mathematica:}
$$
\begin{aligned}
	& A_n \to \frac{c_n^0}{3}-\frac{c_n^2}{3 n^{4/3}}-\frac{(-1)^{2/3} c_n^2}{3 n^{4/3}}-\frac{(-1)^{2/3} c_n^1}{3
	n^{2/3}}, \\
	& B_n \to \frac{c_n^0}{3}-\frac{c_n^2}{6 n^{4/3}}+\frac{i c_n^2}{2 \sqrt{3} n^{4/3}}+\frac{c_n^1}{6 n^{2/3}}+\frac{i
	c_n^1}{2 \sqrt{3} n^{2/3}}, \\
	& C_n \to \frac{c_n^0}{3}+\frac{c_n^2}{3 n^{4/3}}-\frac{c_n^1}{3 n^{2/3}}
\end{aligned}
$$

\textbf{Esercizio.}
(Equazione del calore senza una condizione al bordo)
\begin{equation}
	\tag{P}
	\begin{cases}
		u_t = u_{xx} & x \in [-\pi, \pi] \\
		u(\curry, -\pi) = u(\curry, \pi) \\
		u(0, \curry) = u_0 = \cos(x / 2)
	\end{cases}
\end{equation}
\begin{enumerate}
	\item \textit{Esiste una soluzione?}

		Sì in quanto esiste anche con una condizione in più

	\item \textit{È unica?}

		Senza periodicità per $u_x$ non è vero in generale che $c_n(u_{xx}(t, \curry)) = -n^2 c_n(u(t, \curry))$.

\end{enumerate}
Cerchiamo una soluzione della forma $u(t, x) = \cos(x/2) \psi(t)$. Abbiamo che $u_t(t, x) = \dot \psi(t) \cos(x / 2)$ e $u_{xx}(t, x) = - \cos(x / 2) \psi(t) / 4$. Dunque $\dot \psi(t) = - \psi(t) / 4$ e $\psi(0) = 1 \implies \psi(t) = e^{-t / 4}$.

\textbf{Esercizio.}
\begin{equation}
	\tag{P}
	\begin{cases}
		u_t = u_{xx} & x \in [0, \pi] \\
		u(t, 0) = 0 & t \in \R \\
		u(t, \pi) = t & t \in \R \\
		u(0, \curry) = u_0
	\end{cases}
\end{equation}
L'equazione è lineare, cerchiamo $u(t, x) = v(t, x) + \psi(t, x)$ in modo che $v(t, x) = 0$ se $x = 0, \pi$ e $\psi(t, 0) = 0$ e $\psi(t, \pi) = t$ e $\psi(t, x) = t x / \pi$.

\textbf{Esercizio.}
\begin{equation}
	\tag{P}
	\begin{cases}
		u_t = u_{xxxx} & x \in [-\pi, \pi] \\
		u(\curry, \pi) = u(\curry, -\pi) \\
		u_x(\curry, \pi) = u_x(\curry, -\pi) \\
		u_{xx}(\curry, \pi) = u_{xx}(\curry, -\pi) \\
		u_{xxx}(\curry, \pi) = u_{xxx}(\curry, -\pi) \\
		u(0, \curry) = u_0
	\end{cases}
\end{equation}

\textbf{Esercizio.}
\begin{equation}
	\tag{P}
	\begin{cases}
		u_t = u_{xxxx} & x \in [0, \pi] \\
		u(\curry, 0) = u(\curry, \pi) = 0 \\
		u_{xx}(\curry, 0) = u_{xx}(\curry, \pi) = 0 \\
		u(0, \curry) = u_0
	\end{cases}
\end{equation}

\textbf{Esercizio.}
Sia $V$ il seguente insieme
$$
V \coloneqq \left\{ f \in L^1([1, +\infty]) \;\middle|\; |f(x)| \leq \frac{1}{x^2} \text{ per q.o. } x \right\}
$$
è compatto in $L^1$? e se al posto di $L^1$ avessimo $L^2$?

[TODO: Espandere]

Intuitivamente $V \supseteq \{ f \mid |f(x)| \leq 1 / 2 \text{ q.o. in } [1, 2] \}$ che non è compatto in quanto contiene famiglie di funzioni che ``oscillano molto'' costruite sull'idea della base di Haar.

\textbf{Esercizio.}
Trovare una funzione in $L^p([0, +\infty))$ tale però che $f \notin L^q$ per $q \neq p$.

Cercare $f$ della forma
$$
f(x) = \frac{1}{x^\alpha (a + (\ln x)^\beta)}
$$












