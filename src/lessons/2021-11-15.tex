\chapter{Applicazioni della serie di Fourier}

\section{Equazione del calore}

Sia $\Omega$ un aperto di $\R^d$ e $u(t, x) \colon [0, T) \times \Omega \to \R$ e chiamiamo $x$ la \textit{variabile spaziale} e $t$ la \textit{variabile temporale}. In dimensione $3$ l'insieme $\Omega$ rappresenta un solido di materiale conduttore omogeneo  e $u(t, x)$ rappresenta la temperatura in $x$ all'istante $t$. Dunque $u$ risolve l'equazione del calore
$$
	u_t = c \cdot \Delta u
$$
dove con $u_t$ indichiamo la derivata parziale di $u$ rispetto al tempo, $c$ è una costante fisica che porremo uguale ad $1$ e $\Delta u$ è il laplaciano rispetto alle dimensioni spaziali ovvero
$$
	\Delta u = \sum_{i=1}^d \frac{\pd^2 u}{\pd x_i^2} = \operatorname{div}(\nabla_x u).
$$

Vedremo che la soluzione di $u_t = \Delta u$ esiste ed è unica specificando $u(0, \curry) = u_0$ condizione iniziale con $u_0 \colon \Omega \to \R$ data e delle condizioni al bordo come ad esempio
\begin{itemize}
	\item \textit{Condizioni di Dirichlet}: $u = v_0$ su $[0, T) \times \pd \Omega$ con $v_0$ funzione fissata. Possiamo pensare come fissare delle sorgenti di calore costanti sul bordo.
	\item \textit{Condizioni di Neumann:} $\ds \frac{\pd u}{\pd \nu}$ con $\nu$ direzione normale al bordo. Essenzialmente ci sta dicendo che non c'è scambio di calore con l'esterno.
\end{itemize}
In particolare scriveremo
$$
\left\{
\begin{aligned}
	& u_t = \Delta u \quad \text{su $\Omega$} \\
	& u(0, \curry) = u_0 \\
	& \text{\footnotesize Una delle condizioni al bordo su $\pd \Omega$...}
\end{aligned}
\right.
$$

\section{Risoluzione dell'equazione del calore (su $\mathbb S^1$)}

Come conduttore consideriamo un anello di materiale omogeneo e sottile che parametrizziamo con $[-\pi, \pi]$. Dunque consideriamo $u \colon [0, T) \times [-\pi, \pi] \to \R$ con le condizioni
\begin{equation}
\tag{P} \label{eq:15nov2021_problem_1}
\left\{
\begin{aligned}
	& u_t = u_{xx} \\
	& u(\curry, \pi) = u(\curry, -\pi) \\
	& u_x(\curry, \pi) = u_x(\curry, -\pi) \\
	& u(0, \curry) = u_0
\end{aligned}
\right.
\end{equation}
in particolare la (ii) e la (iii) condizione non sono né quelle di Dirichlet né di Neumann, sono delle condizioni che effettivamente ci dicono che siamo ``su $\mathbb S^1$''\footnote{Bastano solo queste condizioni sulla funzione e sulla sua derivata perché intuitivamente le altre seguono applicando la (i).}; invece l'ultima è la condizione iniziale ed $u_0$ è data.

\subsection{Risoluzione formale}

Scriviamo $u$ in serie di Fourier rispetto a $x$ cioè
$$
u(t, x) = \sum_{n \in \Z} c_n e^{inx}
$$
con $c_n \coloneqq c_n(u(t, \curry))$ da cui derivando formalmente dentro le sommatorie otteniamo che $u_t$ e $u_{xx}$ sono
$$
\begin{gathered}
	\sum_{n \in \Z} \dot c_n(t) e^{inx} = u_t = u_{xx} = \sum_{n \in \Z} -n^2 c_n(t) e^{inx} \\
	u_t = u_{xx} \longiff  \dot c_n(t) = -n^2 c_n(t) \mquad \forall n \, \forall t
	\quad
	\text{e} 
	\quad
	u(0, \curry) = u_0 \longiff c_n(0) = c_n(u_0) \eqqcolon c_n^0
\end{gathered}
$$
Dunque risolvere \eqref{eq:15nov2021_problem_1} equivale per ogni $n$ che $c_n$ che risolva il problema di Cauchy dato da
\begin{equation}
	\tag{P$'$} \label{eq:15nov2021_problem_2}
	\left\{
	\begin{aligned}
		& \dot y = -n^2 y \\
		& y(0) = c_n^0
	\end{aligned}
	\right.
\end{equation}
con soluzione $y(t) = \alpha e^{-n^2 t}$ cioè $c_n(t) = c_n^0 e^{-n^2 t}$ e quindi abbiamo
\begin{equation}
	\tag{$*$} \label{eq:15nov2021_solution}
	u(t, x) = \sum_{n \in \Z} c_n^0 e^{-n^2 t} e^{inx}
\end{equation}

Studiando questa soluzione formale possiamo fare le seguenti osservazioni che poi diventeranno dei teoremi
\begin{itemize}
	\item \textit{La soluzione esiste per $t \in [0, +\infty)$ ed è molto regolare per $t > 0$}

		Vedremo che la soluzione formale è proprio una soluzione al problema per $t \geq 0$, in particolare il termine $e^{-n^2 t} \to 0$ in modo più che polinomiale ed infatti vedremo che la soluzione sarà proprio $C^\infty$ per $t > 0$.

	\item \textit{La soluzione è unica}

		Tutti i problemi di Cauchy per i coefficienti $c_n(t)$ hanno un'unica soluzione dunque anche la soluzione $u$ è unica.

	\item \textit{In generale non esiste soluzione nel passato.}

		Se il numero di coefficienti $c_n^0 \neq 0$ è infinito allora il termine $e^{-n^2 t} \to +\infty$ molto velocemente per $t < 0$ e la serie diverge.

\end{itemize}

\textbf{Teorema 1} (Esistenza e Regolarità).

Se $u_0 \colon [-\pi, \pi] \to \C$ (presa in $L^2$) continua e tale che $\ds \sum_{n \in \Z} |c_n^0| < +\infty$, allora
$$
	u(t, x) = \sum_{n \in \Z} \underbrace{c_n^0 e^{-n^2 t} e^{inx}}_{u_n(t,x)}
$$
definisce una funzione $u \colon [0, +\infty) \times \R \to \C$ tale che
\begin{enumerate}
	\item $u$ è $2\pi$-periodica in $x$ ed è reale se $u_0$ è reale.
	\item $u$ è continua.
	\item $u$ è $C^\infty$ su $(0, +\infty) \times \R$.
	% \item Risolve (\ref{eq:15nov2021_problem_1}) per $t > 0$ sulla condizione al bordo, $u(0, \curry) = u_0$ su $[-\pi, \pi]$ e $u_t = u_{xx}$ su $t > 0$.
	\item Risolve \eqref{eq:15nov2021_problem_1}. In particolare vale $u_{tt} = u_{xx}$ e valgono le condizioni di periodicità per $t > 0$; e infine vale $u(0, \curry) = u_0$ su $[-\pi,\pi]$.
\end{enumerate}

Vediamo alcuni lemmi tecnici preparatori e sia $R$ un rettangolo di $\R^d$ ovvero prodotto di intervalli con estremi aperti o chiusi.

\textbf{Lemma 4.}
Date $v_n \colon R \to \C$ di classe $C^k$ con $k = 1, 2, \dots, +\infty$ tali che
\begin{itemize}
	\item $v_n \to v$ uniformemente.

	\item $\forall \underline h = (h_1, \dots, h_d) \in \N^d$ con $|\underline h| \coloneqq h_1 + \dots + h_d \leq k$ (se $k = +\infty$ allora basta $|\underline h| < +\infty$) posto
		$$
		D^{\underline h} \, v_n \coloneqq
		\frac{\pd^k}{\pd x_1^{h_1} \cdots \pd x_d^{h_d}} v_n
		$$
		$D^{\underline h} \, v_n \to D^{\underline h} \, v$ converge uniformemente.
\end{itemize}
allora $v \in C^k$ e $D^{\underline h} v = \lim_{n} D^{\underline h} v_n$.

\textbf{Dimostrazione.}
Si parte dal caso $d = 1$ e $k = 1$ e si procede per induzione. [TODO: Esercizio]
\qed

\textbf{Corollario. 5}
Date $u_n \colon R \to \C$ di classe $C^k$ con $k = 1, \dots, +\infty$ tali che
$$
\forall \underline h \text{ con } |\underline h| \leq k \qquad \sum_n \norm{D^{\underline h} \, u_n} < +\infty
$$
allora $u \coloneqq \sum_n u_n$ è una funzione ben definita su $R$ e $C^k$ e $D^{\underline h} \, u = \sum_n D^{\underline h} \, u_n$ per ogni $\underline h$ con $|\underline h| \leq k$.

\textbf{Lemma. 6}
Data $u \colon R \to \C$ e rettangoli $R_i \subset R$ relativamente aperti in $R$ tali che $u$ è $C^k$ sugli $R_i$ per ogni $i$ allora $u$ è di classe $C^k$ su $\tilde R \coloneqq \bigcup_i R_i$.

\textbf{Dimostrazione.}
Intuitivamente essere $C^k$ è una proprietà locale ma preso $x \in R \implies \exists i \; x \in R_i$ e dunque segue per l'ipotesi sugli $R_i$.
\qed

\textbf{Lemma. 7}
Data $f \in L^2((-\pi, \pi); \C)$
$$
f \text{ è reale q.o.} \iff c_{-n}(f) = \overline{c_n(f)}
$$

\textbf{Osservazione.}
Notiamo che se $f \in L^1$ la freccia $\boxed{\Leftarrow}$ è molto più difficile.

\textbf{Dimostrazione Teorema 1.}
\begin{enumerate}
	\item $u_0$ reale $\implies c_{-n}^0 = \overline{c_n^0} \implies c_{-n}^0 e^{-(-n)^2 t} = \overline{c_n^0 e^{-n^2 t}} \iff c_{-n}(u(t, \curry)) = \ol{c_{n}(u(t, \curry))}$.

	\item Sia $R \coloneqq [0, +\infty) \times \R$, $\norm{u_n}_{L^\infty(R)} = \norm{c_n^0 e^{-n^2 t}}_{L^\infty (R)} =  |c_n^0|$ dunque
		$\ds\sum_{n \in \Z} u_n$ converge totalmente su $R$ e quindi $u$ è ben definita e continua su $R$.

	\item Presi $h, k = 0, 1, 2, \dots$ se proviamo a calcolare $D_t^h D_x^k u_n = c_n^0 (-n)^{2h} (in)^k e^{-n^2 t} e^{inx}$ vediamo non si riesce a stimare per $t \to 0$ infatti
		$$
		\norm{D_t^h D_x^k u_n}_{L^\infty(R)}
		= |c_n^0| \cdot |n|^{2h + k} \xrightarrow{n} \infty.
		$$
		Serve prendere $\delta > 0$ e sia $R_\delta \coloneqq (\delta, +\infty) \times \R$
		$$
		\norm{D_t^h D_x^k u_n}_{L^\infty(R_\delta)}
		= |c_n^0| \cdot |n|^{2h + k} e^{-n^2 \delta}
		$$
		in particolare per ogni $h, k$ abbiamo che $|n|^{2h + k} e^{-n^2 \delta} \xrightarrow{n} 0 \implies |n|^{2h + k} e^{-n^2 \delta} \leq m_{h,k} \implies \norm{D_t^h D_x^k u_n}_{L^\infty(R_\delta)} \leq m_{h,k} \cdot |c_n^0|$ e quindi $\sum_n D_t^h D_x^k u_n$ converge totalmente su $R_\delta$.

		Quindi $u$ è $C^\infty$ su $R_\delta$ per ogni $\delta > 0$ e siccome $R_\delta$ è aperto in $R$ per il Lemma. 6 $u$ è $C^\infty$ su $\bigcup_{\delta > 0} R_\delta = (0, +\infty) \times \R$.

	\item Essendo che $u$ è $2\pi$-periodica in $x$, valgono le condizioni al bordo; inoltre $u_0$ e $u(0,\cdot)$ hanno gli stessi coefficienti di Fourier, dunque $u_0 = u(0,\cdot)$ quasi ovunque, ma essendo continue vale $u_0 = u(0,\cdot)$ su $[-\pi,\pi]$\footnote{I coefficienti di Fourier sono prodotti scalari in $L^2$ dunque sono definiti a meno di insiemi di misura nulla.}.

	Per mostrare che $u_t = u_{xx}$ osserviamo che $(u_n)_t = (u_n)_{xx}$ e che l'equazione è lineare omogenea:
	$$
		u_t = \Big(\sum_n u_n\Big)_t \overset{\text{Cor 5}}{=} \sum_n (u_n)_t = \sum_n (u_n)_{xx} \overset{\text{Cor 5}}{=} \Big( \sum_n u_n \Big)_{xx} = u_{xx}.
	$$
\end{enumerate}
\qed

Ora enunciamo il teorema di unicità, vogliamo un teorema con il minor numero di ipotesi possibile e che ci dà più informazioni; quindi in questo caso cerchiamo la più grande famiglia di funzioni (quindi la meno regolare possibile) sulla quale vale l'unicità della soluzione.

\hypertarget{thm:lez15nov_teo2}{%
\textbf{Teorema. 2} (Unicità)}
Sia $u \colon [0, T) \times [-\pi, \pi] \to \C$ continua, $C^1$ nel tempo e $C^2$ nello spazio per $t > 0$. Se $u$ risolve \eqref{eq:15nov2021_problem_1} su $t > 0$ allora $u$ è unica.

\textbf{Definizione.}
Dato $R$ un rettangolo e $u \colon R \to \C$ diciamo che $u$ è $C^k$ nella variabile $x_i$ se $\left(\frac{\pd}{\pd x_i}\right)^h u$ esiste per $h = 1, \dots, k$ ed è continua su $R$.

\textbf{Lemma 8.}
% Sia Data $u \colon I \times [-\pi, \pi] \to \C$ di classe $C^k$ in $t \implies c_n(D_t^h u(t, \curry)) = D_t^h c_n(u(t, \curry))$ per $h \leq k$.
Sia $u \colon I \times [-\pi,\pi] \to \C$ e sia $c_n(t) \coloneqq c_n(u(t,\cdot))$ per ogni $n$.

Se $u$ è continua $c_n$ è continua su $I$.

Se $u$ è $C^k$ allora $c_n$ è $C^k$ su $I$ e $c_n(D_t^h u(t, \curry)) = D_t^h c_n(u(t, \curry))$ per $h \leq k$.


\textbf{Dimostrazione.}
$$
c_n(t) = \frac{1}{2\pi} \int_{-\pi}^\pi u(t, x) e^{-inx} \dd x
\implies \dot c_n(t) = \frac{1}{2\pi} \int_{-\pi}^\pi u_t(t, x) e^{-inx} \dd x = c_n(u_t(t, \curry))
$$
per il teorema di derivazione sotto il segno di integrale. 
\qed

\textbf{Lemma 9.} Sia $y \colon [0,T) \to \R^k$ funzione continua su $[0,T)$ e derivabile su $(0,T)$ che risolve l'equazione differenziale ordinaria $\dot y = f(t,y)$ su $(0,T)$ con $f\colon  [0,T) \times \R^k \to \R^k$ continua.
Allora $y$ è $\mc{C}^1$ su $[0,T)$ e risolve $\dot y = f(t,y)$ su $[0,T)$.

\textbf{Dimostrazione Teorema 2.}
Poniamo $c_n(t) \coloneqq c_n(u(t, \curry))$. Sappiamo che per $t > 0$ vale $\dot c_n(t) \stackrel{(\ast)}{=} c_n(u_t(t, \curry)) = c_n(u_{xx}(t, \curry)) \stackrel{(\ast \ast)}{=} -n^2 c_n(t)$, dove $(\ast)$ segue dal Lemma 8 e $(\ast \ast)$ segue dalla regolarità dei coefficienti. Dunque, per il Lemma 9, i coefficienti $c_n$ risolvono il problema di Cauchy
$$
\begin{cases}
	\dot y = -n^2 y \\
	y(0) = c_n^0
\end{cases} 
$$ 
che ha un'unica soluzione.
% $\implies c_n$ è unico. (in particolare abbiamo usato che $\dot y = f(t, y)$ per $t > 0$ e $y(0) = y_0$ e $y$ è continua in $0 \implies y$ è $C^1$ anche in $0$ e risolve $\dot y = f(t, y)$ per $t \geq 0$)
\qed










