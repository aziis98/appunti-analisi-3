% 
% Lezione del 13 Ottobre 2021
% 

\section{Esercitazione del 13 ottobre}

\subsection{Esercizi su spazi $L^p(X)$ al variare di $p$ e dello spazio $X$}

Sia $X \subset \R^n$, $\mu$ la misura di Lebesgue e $ 1 \leq p_1 \leq p_2$.

\textbf{Domanda.} Possiamo confrontare gli spazi $L^{p_1}(X)$ e $L^{p_2}(X)$?
In generale no. 

Vediamo informalmente perché.
Posto $X = (0,+\infty)$, gli integrali 
%
$$
\int_{0}^{+\infty} \frac{1}{(1+x)^{\beta p}} \dd x, \qquad \int_{0}^{+\infty} \frac{1}{x^{\beta p}} \cdot \One_{[0,1]}(x) \dd x = \int_{0}^{1} \frac{1}{x^{\beta p}} \dd x
$$
%
 sono maggiorati dall'integrale di $1 / x^{\alpha}$ dove l'esponente  $\alpha$ è rispettivamente più piccolo e più grande di $\beta \cdot p$.

Utilizziamo questa intuizione per vedere formalmente che gli spazi $L^p(0,+\infty)$ non sono confrontabili.

Cerchiamo una funzione $f \in L^{p_1}(0,+\infty) \setminus \, L^{p_2}(0,+\infty)$ e una funzione $g \in L^{p_2}(0,+\infty) \setminus \, L^{p_1}(0,+\infty)$.
La funzione $f$ definita come segue
%
$$
f(x) \coloneqq 
\begin{cases}
1 / x^\beta \mquad x \in (0,1) \\
0 \mquad x \geq 1
\end{cases} 
$$
%
ha integrale 
%
$$
\int_{0}^{+\infty} f(x)^{p_1} \dd x = \int_{0}^{1} \frac{1}{x^{\beta p_1}} \dd x < +\infty \longiff \beta \cdot p_1 < 1
$$
e
$$
\int_{0}^{+\infty} f(x)^{p_2} \dd x = \int_{0}^{1} \frac{1}{x^{\beta p_2}} \dd x = +\infty \longiff \beta \cdot p_2 \geq 1.
$$
%
Dunque, basta prendere $\beta \in [1/p_2, 1/p_1)$.

Ora cerchiamo $g \in L^2 (0,+\infty) \setminus \, L^{p_1}(0,+\infty)$.
Definiamo $g(x)$ come segue
$$
g(x) \coloneqq \frac{1}{(1+x)^\alpha}
$$
da cui
$$
\int_{0}^{+\infty} g(x)^{p_2} \dd x < +\infty \longiff \alpha \cdot p_2 > 1
\quad
\text{e}
\quad
\int_{0}^{+\infty} g(x)^{p_1} \dd x = +\infty \longiff \alpha \cdot p_1 \leq 1
$$

\textit{Conclusione.} In generale non c'è confrontabilità fra gli spazi $L^p$. La confrontabilità, dipende infatti dall'insieme $X$ su cui sono definiti.

\textbf{Nota.} Un caso particolare è dato ponendo $p_1 < p_2$ e $\mu(X) < +\infty$. In tal caso $L^{p_2}(X) \subset L^{p_1}(X) $.

Data $f \in L^{p_2}(X)$, cioè con $\int_X \left| f \right|^{p_2} \dd \mu < +\infty$ vediamo che  $\int_X \left| f \right|^{p_1} \dd \mu < +\infty$.

Usiamo Hölder:
\begin{align*}
\int_X \left| f \right|^{p_1} \dd \mu & \leq \Bigg( \int_X \overbrace{\left| h(x) \right|^p}^{\left| f(x) \right|^{p_1 p}}  \dd \mu  \Bigg)^{1/p} \cdot \left( \int_X 1^q \dd \mu  \right)^{1/q}
\underbrace{\leq}_{p = p_1 / p_2} \left( \int_X \left| f \right|^{p_2} \dd \mu  \right)^{p_2 /p_1} \left( \int_X 1^q \dd \mu  \right)^{1/q} \\
& \underbrace{=}_{\mathclap{q = \left( 1 - \frac{1}{p} \right)^{-1} = \frac{p}{p-1} = \frac{p_2 / p_1}{p_2 - p_1}}} \norm{f}_{L^{p_2}(X)}^{p_1} \cdot \mu(X)^{\frac{p_2 - p_1}{p_2}}.
\end{align*}
%
Dunque
%
$$
\norm{f}_{L^{p_1}(X)} \leq \norm{f}_{L^{p_2}(X)}\cdot \mu(X)^{\frac{p_2 - p_1}{p_1 p_2}}.
$$
%
L'inclusione
%
\begin{align*}
i \colon L^{p_2} & \to L^{p_1}(X) \\
f & \mapsto f
\end{align*}

è ben definita per quanto fatto sopra.

\textbf{Esercizio.} [TO DO] Vedere con quale topologia l'inclusione risulta continua.

\textbf{Esercizio.} [TO DO] Dato $p \geq 1$, stabilire se esistono $X, \mu,f \in L^p(X)$ e $f \notin L^q(X)$ per ogni $q \neq p$, $q \geq 1$. \\
\textit{Suggerimento.} Pensare a $X = (0,+\infty)$, $\mu$ misura di Lebesgue.

\textbf{Osservazione.} $L^p(X)$ è uno spazio vettoriale di dimensione infinita, ossia ogni base algebrica ha cardinalità infinita. 
Vediamo il caso $X = (0,1)$.
Per trovare una base infinita, cerchiamo per ogni $N \in \N$, un insieme di funzioni $f_1,\ldots , f_N \in L^p(0,1)$ tali che siano linearmente indipendenti.
Vale a dire, presi $\lambda_1,\ldots ,\lambda_N \in \R$ vale $\lambda_1 f_1 + \ldots +f_N = 0$ se solo se $\lambda_1 = \ldots = \lambda_N = 0$.

Ad esempio, definiamo $f_i \coloneqq \One_{i/N, (i+1)/N}$ (questa costruzione si può riprodurre per ogni $N \in \N$).

Ricordiamo che, essendo $L^p(X)$ uno spazio metrico, dato $Y \subset L^p$ vale la seguente caratterizzazione:
\begin{center}
$Y$ è compatto $\longiff$ $Y$ è compatto per successioni $\longiff $ $Y$ chiuso e totalmente limitato.
\end{center}

\textbf{Osservazione.} $Y \subset L^p(X)$ è un sottoinsieme che eredita la norma $\norm{\curry}_{L^p}$ :
\begin{center}
$Y$ è completo $\longiff$ $Y$ è chiuso.
\end{center}

\textbf{Osservazione.} In $L^p$ i sottoinsiemi chiusi e limitati non sono compatti\footnote{Uno spazio metrico è compatto se solo se è completo e totalmente limitato. Inoltre, uno spazio metrico $X$ si dice totalmente limitato se $\forall \epsilon > 0$ esiste $B_\epsilon^1,\ldots,B_\epsilon^n$ tale che $X \subset \bigcup_{i=1}^n B_\epsilon^i$.}!
In particolare le palle
%
$$
Y = \left\{ f \in L^p \mymid \norm{f}_{L^p} \leq 1 \right\}
$$
%
non sono compatte.

Ad esempio, mostriamo che in $L^p(0,1)$ le palle 
%
$$
B = \left\{ f \in L^p \mymid \norm{f}_{L^p} \leq 1 \right\}
$$
%
non sono compatte.
Per farlo, esibiamo una successione $\left\{ f_n \right\}_{n \in \N} \subset B$ che non ammette sottosuccessioni convergenti.
La costruiamo in modo che non abbia sottosuccessioni di Cauchy
%
$$
f_n \colon (0,1) \to \R, \quad \norm{f_n - f_m}_{L^p} \geq c_0 > 0 \quad  \forall n \neq m.
$$
%
Cerco $A_n \subset (0,1)$ tale che $\left| A_n \cap A_m \right| = 0$ per ogni $n \neq m$.
Definiamo $f_n$ come segue
%
$$
f_n(x) \coloneqq 
\begin{cases}
0 \quad  \text{se} \; x \in (0,1) \setminus (1 / (n+1), 1/ n) \\
c_n > 0 \quad \text{altrimenti}
\end{cases} 
$$
%
dove $c_n$ è tale che 
%
$$
\left( \int_{1 / n+1}^{1 / n} c_n^p  \right)^{1/p} = 1 \longiff 
c_n^p \cdot (1/n - 1/(n+1)) = 1 \longiff c_n^p = n \cdot (n+1).
$$
%
Calcoliamo ora $\norm{f_n - f_m}^p_{L^p}$ con $n \neq m$ :
%
$$
\int_0^1 \left| f_n(x) - f_m(x) \right|^p \dd x = \int\limits_{\mathclap{(1/n,1/n+1) \cup (1/ m+1,1/m)}} \left| f_n - f_m \right|^p \dd x = \int_{1 / n+1}^{1 / n} \left| f_n \right|^p \dd x + \int_{0}^{1} \left| f_m \right|^p \dd x = 1 + 1 = 2.
$$
%

Si osserva che quanto detto sopra vale anche per $p = + \infty$.

\textbf{Esercizio.} [TO DO] Sia $E = \left\{ f \in L^1(1,+\infty) \mymid \left| f(x) \right| \leq 1 / x^2 \mquad \text{e} \;x \in [1,+\infty) \right\}$.

\begin{itemize}

\item $E$ è limitato in $L^1$?

\item $E$ è chiuso in $L^1$?

\item $E$ è compatto in $L^1$?

\end{itemize}

\textbf{Soluzione.}
\begin{enumerate}

	\item Dimostriamo che $\norm{f}_{L^1} < C$ per ogni $f \in E$.
	%
	$$
	\norm{f}_{L^1} = \int_1^\infty |f(x)| \dd x \leq \int_1^\infty 1/x^2 \dd x < C.
	$$
	%

	\item $E$ è chiuso. Ci basta dimostrare che se $\left\{ f_n \right\} \in E$ è convergente a $f$ allora $f \in E$. Questo equivale a dimostrare che $|f(x)| < 1/x^2$.
	Dal fatto che $\{f_n\} \in E$ è convergente in $L^1$, abbiamo che esiste una sottosuccessione $\{f_{n_k}\}$ che converge puntualmente a $f$. Essendo che $|f_{n_k}| < 1/x^2$ per ogni $x \in [1,+\infty)$, per la continuità del modulo segue la tesi.

	\item Da fare [TO DO]

\end{enumerate}

\textbf{Esercizio.} [TO DO] 
\begin{itemize}
\item Dire se $f_n(x) = x^n$, $n = 0,\ldots , N$ è un insieme di funzioni linearmente indipendenti in $L^p([0,1])$.

\item Dire se $\left\{ f_n \right\} \subset L^p(0,1)$ è compatta in $L^p(0,1)$.
\end{itemize}
\textit{Suggerimento.} Studiare il limite puntuale.

\textbf{Soluzione.}
\begin{enumerate}

	\item Dimostriamolo per induzione.
	\textit{Passo base.} [TO DO]

	\textit{Passo induttivo.} ($n-1 \geq n$)
	Vediamo che se $a_1 \cdot 1 + a_2 \cdot x + \cdots + a_{n-1} \cdot x^{n-1} + a_n \cdot x^n = 0 \Longrightarrow a_1 = \ldots = a_n = 0$.
	\begin{align*}
		& a_1 \cdot 1 + a_2 \cdot x + \cdots + a_{n-1} \cdot x^{n-1} = - a_n \cdot x^n \\
		& \qquad \qquad \qquad \downarrow + a_n \cdot x^n \\
		& (a_1 + a_n) \cdot 1 + (a_2 + a_n) \cdot x + \cdots + (a_{n-1} + a_n) \cdot x^{n-1} = 0
	\end{align*}
	essendo che $1,x^1,\ldots,x^{n-1}$ sono linearmente indipendenti per ipotesi induttiva, vale $(a_i + a_n) = 0$ per ogni $i = 1,\ldots,n-1$, da cui $a_i = 0$ per ogni $i = 1,\ldots,n$.

	\item Dimostriamo che non è compatto. Se per assurdo lo fosse, dalla successione $(f_n)$ potremmo estrarre una sottosuccessione convergente $(f_{n_k})$ in $L^p([0,1])$; denotiamo il limite con $f$. Per i risultati visti sulla convergenza, da $(f_{n_k})$ potremmo estrarre una sottosuccessione convergente quasi ovunque a $f$. Ma questo è assurdo perchè $\lim_n f_n = +\infty$.
\end{enumerate}

\subsection{Spazi $\ell^p$}

Prendiamo $X = \N$ e $\mu = \#$ la misura che conta i punti.

\textbf{Osservazione.} Definiamo
%
$$
\ell^p = L^p(\N, \#) = \left\{ \left( x_n \right)_{n \in \N} \mymid \sum_{n=0}^{+\infty} \left| x_n \right|^p < +\infty  \right\}
$$
con $p \geq 1$ e $p \neq +\infty$, e
%
$$
l^{\infty} = \left\{ \text{successioni limitate} \right\} = \left\{ \left( x_n \right) \mymid \sup_{n \in \N}\left| x_n \right| < +\infty \right\}.
$$
%

\textbf{Esempio} (di insieme non compatto in $\ell^1$). Consideriamo la successione $\left( e_i \right)$ definita come
%
$$
(e_i)_n \coloneqq 
\begin{cases}
0 \quad \text{se} \; n \neq i \\
1 \quad \text{se} \; n = i
\end{cases} 
$$
%
si osserva inoltre che le successioni così definite sono linearmente indipendenti e generano se sono infiniti.

\textbf{Esempio} (di insieme compatto in $\ell^1$). Sia $F = \left\{ (x_n)_n \in \ell^1 \mymid \left| x_n \right| \leq 1 / n^2 \quad \forall n \in \N \right\}$.
Noto subito che $F$ è limitato, infatti, presa
%
$$
\underline{x} = (x_n) \in F, \quad  \norm{\underline{x}}_{\ell^1} = \sum_{n=0}^{+\infty} \left| x_n \right| \leq
\sum_{n=0}^{+\infty} 1 / n^2 < +\infty.  
$$
%
$F$ è anche chiuso.

\textbf{Osservazione.} Data una successione $(\underline{x}^k) \subset \ell^1$, se $\underline{x}^k \xrightarrow{\ell^1} \underline{x}^{\infty}$, vuol dire che
%
$$
\norm{\underline{x}^k - \underline{x}^\infty}_{\ell^1} = \sum_{n = 0}^{+\infty}  \left| x_n^k - x_n^\infty \right| \xrightarrow{k} 0. 
$$
%
In particolare, per ogni $n \in \N$ fissato, $\lim_k (x_n^k - x_n^\infty) = 0$.

$F$ è chiuso perché se $(\underline{x}^k) \subset F$ e $\underline{x}^k \xrightarrow{\ell^1} \underline{x}^\infty$, allora per ogni $n \in \N$ vale 
%
$$
\left| x_n^k \right| \leq 1 / n^2 \quad \text{e} \quad \underbrace{\lim_{n \to +\infty} \left| x_n^k  \right|}_{x_n^\infty} \leq 1/n^2.
$$
%

Dimostriamo che è compatto per successioni.
Prendiamo $( \underline{x}^k ) \subset F$, ogni componente $x_n$ è equilimitata, quindi a meno di sottosuccessioni $x_n^{k_j}$ converge a $x_n^\infty$.
A meno di diagonalizzare, possiamo supporre che le successione $k_j$ non dipenda da $n$.
Inoltre gli elementi $x_n^{k_j}$ sono dominati da $y = (1 / n^2)$. Concludiamo usando il teorema di convergenza dominata di Lebesgue.

