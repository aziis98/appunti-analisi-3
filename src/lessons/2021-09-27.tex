%
% Lezione del 27 settembre
%

\chapter{Teoria della misura}

\section{Misure astratte}

\textbf{Definizione.} 
Uno spazio misurabile è una terna $(X, \mc A, \mu)$ tale che
\begin{itemize}
	
	\item $X$ è un insieme qualunque.

	\item $\mc{A}$ è una $\sigma$-algebra di sottoinsiemi di $X$ (chiamata $\sigma$-algebra dei misurabili) ovvero una famiglia di sottoinsiemi di $X$ che rispetta le seguenti proprietà:
		\begin{itemize}
			\item $\emptyset, X \in \mc{A}$.
			\item $\mc{A}$ è chiusa per complementare, unione e intersezione numerabile.
		\end{itemize}
	
	\item $\mu$ è una misura su $X$, ossia una funzione $\mu \colon \mc A \to [0, +\infty]$ $\sigma$-addittiva, cioè tale che data una famiglia numerabile $\left\{ E_k \right\} \subset \mc A$ disgiunta e posto $E \coloneqq \bigcup E_n $, allora
		$$
		\mu(E) = \sum_{n} \mu (E_n).
		$$

\end{itemize}


\textbf{Notazione.}
Data una successione crescente di insiemi $E_1 \subset E_2 \subset \cdots E_n \subset \cdots$ con $\bigcup E_n = E$, scriviamo $E_n \uparrow E$.

\textbf{Proprietà.}
\begin{itemize}
	\item $\mu(\emptyset) = 0$
	\item \textit{Monotonia}: Dati $E,E' \in \mc{A}$ e $E \subset E'$, allora $\mu(E) \leq \mu(E')$.
	\item Data $E_n \uparrow E$, vale $\ds \mu(E) = \lim_{n \to \infty} \mu(E_n) = \sup_{n} \mu(E_n)$.
	\item Se $E_n \uparrow E$ e $\mu (E_{\bar{n}}) < +\infty$ per qualche $\bar{n}$, allora $\ds \mu(E) = \lim_{n \to \infty} \mu(E_n) = \inf_{n} \mu(E_n)$.
	\item \textit{Subadditività}: Se $\ds E \subset \bigcup E_n $, allora $\ds \mu(E) \leq \sum_{n} \mu(E_n)$.
\end{itemize}

\textbf{Osservazione.} 
Dato $X' \in \mc A$ si possono restringere $\mc A$ e $\mu$ a $X'$ nel modo ovvio.

\textbf{Definizioni}.
\begin{itemize}
	\item $\mu$ si dice \textbf{completa} se $F \subset E, E \in \mc{A}$ e $\mu(E) = 0$, allora $F \in \mc{A}$ (e di conseguenza $\mu(F) = 0$).
	\item $\mu$ si dice \textbf{finita} se $\mu(X) < + \infty$.
	\item $\mu$ si dice \textbf{$\sigma$-finita} se esiste una successione $\{ E_n \}$ con $E_n \subset E_{n+1}$ tale che $\bigcup E_n = X$ con $\mu(E_n) < +\infty$ per ogni $n$.
\end{itemize}

\textbf{Notazione.}
Sia $P(X)$ un predicato che dipende da $x \in X$ allora si dice che \textbf{$P(X)$ vale $\mu$-quasi ogni $x \in X$} se l'insieme $\left\{ x \mid P(x) \text{ è falso}  \right\}$ è (contenuto in) un insieme di misura $\mu$ nulla.

D'ora in poi consideriamo solo misure complete.

\section{Esempi di misure}

\begin{itemize}
	
	\item \textbf{Misura che conta i punti.}
		$$
		X \text{ insieme}
		\qquad
		\mc A \coloneqq \mc P(X)
		\qquad
		\mu(E) \coloneqq \# E \in \N \cup \left\{ +\infty \right\}
		$$

	\item \textbf{Delta di Dirac in $x_0$.}
		$$
		X \text{ insieme, } x_0 \in X \text{ fissato}
		\qquad
		\mc A \coloneqq \mc P(X)
		\qquad
		\mu(E) \coloneqq \delta_{x_0}(E) = \One_E (x_0)
		$$

	\item \textbf{Misura di Lebesgue.}
		$$
		X = \R^n
		\qquad
		\mc{M}^n \text{ $\sigma$-algebra dei misurabili secondo Lebesgue}
		\qquad
		\mathscr L^n \text{ misura di Lebesgue}
		$$
		Dato $R$ parallelepipedo in $\R^n$, cioè $R = \prod_{k=1}^{n} I_k $ con $I_k$ intervalli in $\R$.
		Si pone
		$$
		\mathrm{vol}_n (R) \coloneqq \prod_{k=1}^{n}  \mathrm{lungh} (I_k)
		$$ 
		per ogni $E \subset \R^n$ (assumendo $\mathrm{lungh}([a, b]) = b - a$). Infine poniamo
		$$
		\mathscr L^n(E) \coloneqq \inf \left\{ \sum_{i} \mathrm{vol}_n (R_i) \mymid \left\{ R_i \right\} \text{tale che } E \subset \bigcup_i R_i  \right\}.
		$$
\end{itemize}
 
\textbf{Osservazioni.}
\begin{itemize}
	\item $\mathscr L^n(R) = \mathrm{vol}_n (R)$.

	\item $\mathscr L^n$ non è $\sigma $-addittiva su $\mc{P}(\R^n)$.
\end{itemize}

Il secondo punto giustifica l'introduzione della $\sigma $-algebra dei \textbf{misurabili secondo Lebesgue} che denotiamo con $\mc{M}^n$.

Dato $E \subset \R^n$ si dice che $E$ è misurabile (secondo Lebesgue) se
$$
\forall \epsilon > 0 \; \exists A \text{ aperto e } C \text{ chiuso, tali che }
C \subset E \subset A \; \text{e} \; \mathscr L^n (A \smallsetminus C) \leq \epsilon.
$$

\textbf{Osservazioni.}
\begin{itemize}
	\item Per ogni $E$ misurabile vale
$$
	\mathscr L^n(E) = \inf \left\{ \mathscr L^n \colon A \ \text{aperto}, A \supset E \right\} = \sup \left\{ \mathscr L^n \colon K \ \text{compatto}, K \subset E \right\}.
$$
	\item Notiamo che se $F \subset E$ con $E \subset \mc{M}^n$ e $\mathscr L^n(E) = 0$, allora $F \in \mc{M}^n$. Ovvero la misura di Lebesgue è completa!
\end{itemize}

\textbf{Notazione.} $\left| E \right| \coloneqq \mathscr L^n (E)$
