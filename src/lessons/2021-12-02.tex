\section{Esercitazione del 2 dicembre}

Ricordiamo la definizione della trasformata di Fourier
%
$$
	\mc{F}(f) = \hat{f}(y) = \int_{\R} f(x) e^{-ixy} \dd x, \quad f \in L^1(\R;\C), \mquad f \in L^1(\R)
$$
%
dove $\ds \mc{F} \colon L^1(\R;\C) \to L^\infty(\R) \cap C_0(\R)$ (\textbf{da controllare}), in quanto $\norm{\hat{f}}_\infty \leq \norm{f}_1$.


\textbf{Proprietà.} Ricordiamo le proprietà viste a lezione.
\begin{enumerate}
	\item $\ds \hat{\sigma_h f}(y) = e^{-iyh} \hat{f}(y)$ per ogni $h \in \R$, dove $\sigma_h f(x) = f(x - h)$.


	\item $\ds \hat{e^{ixh} f(y)} = $


	\item Legame tra trasformata e derivata.
	\begin{itemize}

		\item $f \in C^1(\R)$, $f,f' \in L^1(\R)$, $f,f' \in L^1(\R,\C)$, allora $\ds \hat{f'}(y) = iy \ol{f}(y)$.


		\item $f \in L^1(\R;\C)$ e $xf(x) \in L^1(\R;\C)$, $(1 + |x|) f(x) \in L^1(\R;\C)$, allora $\ds \hat{f} \in C^1(\R)$ e $\hat{f}' = -l \hat{xf(x)}$.

	\end{itemize}


	\item  Vale $\hat{f \ast g} = \hat{f} \ast \hat{g}$.
	
\end{enumerate}

Riportiamo un esercizio già posto con una soluzione alternativa. 

\textbf{Esercizio.} Dire se esiste $v \in L^1$ tale che $g \ast v = g$ per ogni $g \in L^1$.

\textbf{Soluzione.} No! Dalle proprietà sopra segue che $\hat{g} \cdot \hat{v} = \hat{g}$, dunque $\hat{v} = c$ è costante. L'equazione ha soluzione se e solo se $c = 0$e dunque $v = 0$. In quanto le costanti non possono essere una trasformata, perchè se $v \in L^1(\R)$ allora $\hat{v} \in C_0(\R)$.

\vss

\textbf{Esercizio 1.} Calcolare la trasformata di Fourier della funzione $\ds f(x) = e^{-|x|}$.

\textbf{Soluzione.}
Abbiamo
\begin{align*}
	\hat{f}(y) & = \int_{-\infty}^\infty e^{-|x|} e^{ixy} \dd x 
	= \int_{-\infty}^\infty e^{-|x|} \cos(xy) \dd x 
	- i \int_{-\infty}^\infty \overbrace{e^{-|x|} \sin(xy)}^{\substack{\text{integrale definito} \\ \text{di fuzione dispari}} \; = \; 0 } \dd x
	= 2 \int_{0}^{\infty} e^{-|x|} \cos(xy) \dd x \\
	& = 2 \int_0^\infty e^{-x} \re e^{ixy} \dd x
	= 2 \int_0^\infty \re \left( e^{-x} \cdot e^{ixy} \right) \dd x
	= 2 \re \left[ \int_0^\infty e^{-x} e^{ixy} \right] \dd x
	= 2 \re \left| \frac{e^{x(iy - 1)}}{(iy - 2)} \right|_0^\infty \\
	& = 2 \re \left[ -\frac{1}{iy - 1} + \underbrace{\lim_{x \to \pm \infty} \frac{e^{x(iy -1)}}{iy - 1}}_{= 0}\right]
	= 2 \re \left[ -\frac{1}{iy - 1} \right]
	= 2\re \left[ -\frac{1}{iy-1} \cdot \frac{iy + 1}{iy + 1} \right] \\
	& = 2\re \left[ \frac{iy + 1}{1 + y^2} \right].
\end{align*}

In conclusione, $\ds \hat{f}(y) = \frac{2}{1 + y^2}$.

\vss

\textbf{Esercizio 2.} Calcolare la trasformata di Fourier della funzione $\ds f(x) = \frac{1}{1 + x^2}$.

\textbf{Soluzione.}
Calcoliamo $\hat{f}(y) = \int_\R \frac{e^{-ixy}}{1 + x^2} \dd x$.
Dal fatto che $f \in L^1$ e usando il teorema di Lebesgue, possiamo scrivere $\hat{f}(y)$ come
%
$$
	\hat{f}(y) = \lim_{R \to +\infty} \int_{-R}^R \frac{e^{-ixy}}{1 + x^2} \dd x
$$
%

\textit{Idea.} Calcolare questo integrale con il metodo dei residui, ponendo $\ds \frac{e^{-ixy}}{1 + x^2} = \left. \frac{e^{-izy}}{1 + z^2} \right|_{z \text{ reale}}$.

Per il teorema dei residui
%
$$
	\int_{B_r} g(z) \dd z = 2\pi i \sum_{z_i \text{ singolarità di } g \text{ su } B_r} \operatorname{res} (g,z_i) 
$$
%
Dove
%
$$
	\int_{B_r} g(z) \dd z = \int_{\gamma_r \coloneqq \text{bordo sotto}} g(z) \dd x 
	+ \overbrace{\int_{\gamma_1 \coloneqq \text{bordo sopra}} g(x)}_{\xrightarrow{r \to \infty} 0}
	= \int_{-r}^{r} \frac{e^{-ixy}}{1 + x^2} \dd x 
$$
%

Studio delle singolarità. Poniamo $z = x + it$, dunque $yz = xy + yit$, da cui $\ds g(z) = g(x + it) = \left( e^{-ixy} e^{ty} \right) / (1 + (x + it)^2)$.

Calcoliamo l'integrale $\int_\gamma g(z) \dd z$ dove $\theta \xmapsto{\gamma} r e^{i\theta}$.
Si ottiene
%
$$
	\int_0^\pi g(e^{i\theta}r) r \dd \theta \xrightarrow{r \to \pm \infty} 0
	\Longrightarrow \int_0^\pi \frac{e^{-ri\cos \theta} e^{r \sin\theta y}}{(1 + r^2 e^{i2\theta})} r \dd \theta
$$
%
Nel caso in cui $y < 0$ l'integrale sopra in modulo va a 0.

Con $y > 0$ avremmo dovuto considerare la curva sotto.
L'unico residuo di $g$ è nel punto $i$.
%
$$
	\lim_{r \to \pm \infty} 2\pi i \operatorname{res}(g,i) = \pi e^{y}
$$
%
Fare lo stesso conto in $-i$ con la curva sotto. [TO DO: rincontrollare l'esercizio 2.]



