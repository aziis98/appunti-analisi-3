%
% Lezione del 29 settembre
%

\section{Funzioni misurabili}

\textbf{Definizione.}
Dato $(X, \mc{A}, \mu)$ e $f \colon X \to \R$ (o al posto di $\R$ in $Y$ spazio topologico), diciamo che $f$ è \textbf{misurabile} (più precisamente $\mc{A}$-misurabile), se
$$
\forall A \text{ aperto} \; f^{-1} (A) \in \mc{A}
$$ 


\textbf{Osservazioni.}
\nopagebreak
\begin{itemize}
	\item Dato $E \subset X$, vale $E \in \mc{A}$ se solo se $\One_E$ è misurabile.
	\item La classe delle funzioni misurabili è chiusa rispetto a molte operazioni
	\begin{itemize}
		\item \textit{Somma}, \textit{prodotto} (se hanno senso nello spazio immagine della funzione).
		\item \textit{Composizione con funzione continua}: Se $f \colon X \to Y$ continua e $g \colon  Y \to Y'$ continua, allora $g \circ f$ è misurabile.
		\item \textit{Convergenza puntuale}: data una successione di $f_n$ misurabili e $f_n \to f$ puntualmente, allora $f$ è misurabile.
		\item $\liminf$ e $\limsup$ (almeno nel caso $Y = \R$).
	\end{itemize}
\end{itemize}


\subsection{Funzioni semplici}

% Indico con $\mc{S}$ la classe delle funzioni $f \colon  X \to \R$ \textit{semplici}, cioè della forma $f = \sum_{i}^{n} \alpha_i \One_{E_i}$ con $\{E_i \}_{1 \leq i \leq n}$ misurabili e $\alpha_i \in \R$.
\textbf{Definizione.}
Definiamo la classe delle \textbf{funzione semplici} come
$$
\mc S := \left\{ f \colon X \to \R \mymid f = \sum_i \alpha_i \One_{E_i} \text{ con $E_i$ misurabili e $\{\alpha_i\}$ finito} \right\}
$$

\textbf{Osservazione.} La rappresentazione di una funzione semplice come combinazione lineare di indicatrici di insiemi \textit{non è unica}, però se necessario possiamo prendere gli $E_i$ disgiunti.

\section{Integrale}

\textbf{Definizione.}
Diamo la definizione di $\ds\int_X f \dd \mu$ per passi
\begin{enumerate}
	\item \label{item:def_int_1} 
		Se $f \in \mc{S}$ e $f \geq 0$ cioè $f = \sum_i \alpha_i \One_{E_i}$ con $\alpha_i \geq 0$ allora poniamo
$$
			\int_{X} f \dd \mu \coloneqq \sum_{i} \alpha_i \mu(E_i),
$$
		convenendo che $0 \cdot +\infty = 0$ in quanto la misura di un insieme non è necessariamente finita.
	
	\item \label{item:def_int_2} 
		Se $f \colon  X \to [0,+\infty]$ misurabile si pone
$$
			\int_{X} f \dd \mu \coloneqq \sup_{\substack{g \in \mc{S} \\ 0 \leq g \leq f}} \int_{X} g \dd \mu.
$$
		
	\item 
		$f \colon X \to \overline{\R}$ misurabile si dice \textbf{integrabile} se 
$$
			\int_{X} f^+ \dd \mu < + \infty \quad \text{oppure} \quad \int_{X} f^- \dd \mu < +\infty.
$$
		e per tali $f$ si pone
$$
			\int_{X} f \dd \mu \coloneqq  \int_{X} f^+ \dd \mu - \int_{X} f^- \dd \mu.
$$
	
	\item 
		$f \colon X \to \R^n$ si dice \textbf{sommabile} (o di \textbf{classe} $\mathscr L^1$) se $\int_X \left| f \right| \dd \mu < +\infty$. In tal caso, se $\int_X f_i^{\pm} \dd \mu < +\infty$ per ogni $f_i$ componente di $f$, allora $\int_X f \dd \mu$ esiste ed è finito.
\end{enumerate}

Per tali $f$ si pone
$$
	\int_{X} f \dd \mu \coloneqq  \left( \int_X f_1 \dd \mu, \ldots , \int_X f_n \dd \mu \right).
$$

\textbf{Notazione.}
Scriveremo spesso $\ds \int_E f(x) \dd x$ invece di $\ds \int_E f \dd \mathscr L^n$ .

\textbf{Osservazioni.}
\begin{itemize}
	\item L'integrale è lineare (sulle funzioni sommabili).
	
	\item I passaggi \ref{item:def_int_1} e \ref{item:def_int_2} danno lo stesso risultato per $f$ semplice $\geq 0$.
	
	\item La definizione in \ref{item:def_int_2} ha senso per ogni $f \colon X \to [0,+\infty]$ anche non misurabile. Ma in generale vale solo che
		$$
		\int_{X} f_1 + f_2 \dd \mu \geq \int_X f_1 \dd \mu + \int_X f_2 \dd \mu.
		$$
	
	\item Dato $E \in \mc{A}$, $f$ misurabile su $E$, notiamo che vale l'uguaglianza
		$$
		\int_E f \dd \mu \coloneqq \int_X f \cdot \One_E \dd \mu.
		$$ 
	
	\item Si può definire l'integrale anche per $f \colon X \to Y$ con $Y$ \textit{spazio vettoriale normato finito dimensionale}\footnote{È necessario avere uno spazio vettoriale, perché serve la linearità e la moltiplicazione per scalare} ed $f$ sommabile.
	
	\item Se $f_1 = f_2$ $\mu$-q.o. allora $\ds \int_X f_1 \dd \mu = \int_X f_2 \dd \mu$.
	
	\item Si definisce $\ds \int_X f \dd \mu$ anche se  $f$ è misurabile e definita su $X \setminus N$ con $\mu(N) = 0$.
	
	\item Se $f \colon [a,b] \to \R$ è integrabile secondo Riemann allora è misurabile secondo Lebesgue e le due nozioni di integrale coincidono. 
		
		\textbf{Nota.} Lo stesso vale per integrali impropri di funzioni positive. Ma nel caso più generale non vale: se consideriamo la funzione
		$$
		f \colon (0,+\infty) \to \R 
		\qquad
		f(x) \coloneqq \dfrac{\sin x}{x}
		$$
		allora l'integrale di $f$ definito su $(0,+\infty)$ esiste come integrale improprio ma non secondo Lebesgue, infatti
		$$
		\int_0^{+\infty} f^+ \dd x = \int_0^{+\infty} f^- \dd x = +\infty
		$$
	
	\item $\ds \int_X f \dd \delta_{x_0} = f(x_0)$
	
	\item Se $X = \N$ e $\mu$ è la misura che conta i punti l'integrale è 
		$$
		\int_X f \dd \mu = \sum_{n = 0}^{\infty} f(n) 
		$$ 
		per le $f$ positive o tali che $\sum f^+(n) < +\infty $ oppure $\sum f^-(n) < +\infty $.
		
		\textbf{Nota.} Come prima nel caso di funzioni non sempre positive ci sono casi in cui la serie solita non è definita come integrale di una misura, ad esempio
		$$
		\sum_{n=1}^{\infty} \frac{(-1)^n}{n}
		$$
		esiste come serie ma non come integrale.
		
	\item Dato $X$ qualunque, $\mu$ misura che conta i punti e $f \colon  X \to [0,+\infty] $ possiamo definire la somma di tutti i valori di $f$ 
		$$
		\sum_{x \in X} f(x) \coloneqq \int_{X} f \dd \mu.  
		$$ 
\end{itemize}

\section{Teoremi di convergenza}

Sia $(X, \mc{A}, \mu)$ come in precedenza.

\textbf{Teorema} (di convergenza monotona o Beppo-Levi).
Date $f_n \colon  X \to [0,+\infty]$ misurabili, tali che $f_n \uparrow f$ ovunque in $X$, allora
$$
\lim_{n \to +\infty} \int_{X} f_n \dd \mu = \int_X f \dd \mu,
$$
dove
$$
\lim_{n \to +\infty} \int_{X} f_n \dd \mu = \sup_n \int f_n \dd \mu.
$$


\textbf{Teorema} (lemma di Fatou).
Date $f_n \colon X \to [0,+\infty]$ misurabili, allora
$$
\liminf_{n \to +\infty} \int_X f \dd \mu \geq \int_{X} \left( \liminf_{n \to +\infty} f_n \right) \dd \mu.
$$ 

\textbf{Teorema} (di convergenza dominata o di Lebesgue).
Date $f_n \colon  X \to \R$ (o anche $\R^n$) con le seguenti proprietà
\begin{itemize}
	\item \textit{Convergenza puntuale:} $f_n (x) \to f(x)$ per ogni $x \in X$.
	\item \textit{Dominazione:} Esiste $g \colon X \to [0,+\infty]$ sommabile tale che $\left| f_n (x) \right| \leq g(x)$ per ogni $x \in X$ e per ogni $n \in \N$.
\end{itemize}
allora
$$
\lim_{n \to \infty} \int_{X} f_n \dd \mu = \int_X f \dd \mu. 
$$ 

\textbf{Nota.}
La seconda proprietà è essenziale; sostituirla con $\ds \int_X \left| f_n \right| \dd \mu \leq C < + \infty$ non basta!

\textbf{Definizione.}
Data una \textit{densità} $\rho \colon  \R^n \to [0,+\infty]$ misurabile, la \textbf{misura $\mu$ con densità $\rho$} è data da
$$
\forall E \in \mc A \quad \mu(E) \coloneqq \int_{E} \rho \dd x
$$ 

\textbf{Osservazioni.}
\begin{itemize}
	\item $\R^n$ e $\mathscr L^n$ possono essere sostituiti da $X$ e $\widetilde{\mu}$.
	\item il fatto che $\mu$ è una misura segue da Beppo Levi, in particolare serve per mostrare la subadditività.
\end{itemize}


\textbf{Teorema} (di cambio di variabile).
Siano $\Omega$ e $\Omega'$ aperti di $\R^n$, $\Phi \colon \Omega \to \Omega' $ un diffeomorfismo \footnote{funzione differenziabile con inversa differenziabile.} di classe $C^1$ e $f \colon \Omega' \to [0,+\infty]$ misurabile. Allora
$$
\int_{\Omega'} f(x') \dd x' = \int_{\Omega} f(\Phi(x)) \left| \det(\nabla \Phi(x)) \right| \dd x.
$$

La stessa formula vale per $f$ a valori in $\overline{\R}$ integrabile e per $f$ a valori in $\R^n$ sommabile.

\textbf{Osservazioni.}
\begin{itemize}
	\item Se $n = 1$, $\left| \det(\Lambda \Phi(x)) \right| = \left| \Phi'(x) \right|$ e non $\Phi'(x)$ come nella formula vista ad Analisi 1 (l'informazione del segno viene data dall'inversione degli estremi).
	
	\item Indebolire le ipotesi su $\Phi$ è delicato. Basta $\Phi$ di classe $C^1$ e $\foralmostall x' \in \Omega' \; \# \Phi^{-1}(x') = 1$ (supponendo $\Phi$ iniettiva la proprietà precedente segue immediatamente).
	Se $\Phi$ non è "quasi" iniettiva bisogna correggere la formula per tenere conto della molteplicità.

	\item Quest'ultima osservazione serve giusto per far funzionare il cambio in coordinate polari che non è iniettivo solo nell'origine.
\end{itemize}

\subsection{Fubini-Tonelli}

Di seguito riportiamo il teorema di Fubini-Tonelli per la misura di Lebesgue.

\textbf{Teorema} (di Fubini-Tonelli).
Sia $\R^{n_1} \times \R^{n_2} \simeq \R^n$ con $n = n_1 + n_2$, $ E \coloneqq E_1 \times E_2 $ dove $E_1, E_2$ sono misurabili e $f$ è una funzione misurabile definita su $E$.
Se $f$ ha valori in $[0,+\infty]$ allora
$$
\int_X f \dd \mu 
= \int_{E_2} \int_{E_1} f(x_1,x_2) \dd x_1 \dd x_2 
= \int_{E_1} \int_{E_2} f(x_1,x_2) \dd x_2 \dd x_1
$$ 

Vale lo stesso per $f$ a valori in $\R$ o in $\R^n$ sommabile.

\textbf{Osservazioni.}
Possiamo generalizzare il teorema di Fubini-Tonelli a misure generiche ed ottenere alcuni risultati utili che useremo ogni tanto.
\begin{itemize}
	\item Se $X_1, X_2$ sono spazi con misure $\mu_1,\mu_2$ (con opportune ipotesi) vale:
		$$
		\int_{E_2} \int_{E_1} f(x_1,x_2) \dd \mu_1(x_1)  \dd \mu_2(x_2) 
		= \int_{E_1} \int_{E_2} f(x_1,x_2) \dd \mu_2(x_2)  \dd \mu_1(x_1).
		$$ 
		se $f\geq 0$ oppure $\ds \int_{X_1} \int_{X_2} \left| f \right| \dd \mu_2(x_2)  \dd \mu_1(x_1) < + \infty $.
	
	\item \textbf{Teorema} (di scambio serie-integrale). Se $X_1 \subset \R$ (oppure $X_1 \subset \R^n$), $\mu_1 = \mathscr L^n$ e $X_2 = \N$, $\mu_2$ è la misura che conta i punti, allora la formula sopra diventa
		$$
		\sum_{n=0}^{\infty} \, \int_{X_1} f_n(x) \dd x  
		= \int_{X_1} \sum_{n=0}^{\infty} f_n(x)  \dd x.
		$$ 
		se $f_i \geq 0$ oppure $\ds \sum_{i} \int_{X_1} \left| f_i(x) \right| \dd x  < + \infty $.
	
	\item \textbf{Teorema} (di scambio di serie). Se $X_1 = X_2 = \N$ e $\mu_1 = \mu_2$ è la misura che conta i punti la formula sopra diventa
		$$
		\sum_{j=0}^{\infty} \sum_{i=0}^{\infty} a_{i,j}  
		= \sum_{i=0}^{\infty} \sum_{j=0}^{\infty} a_{i,j} 
		$$ 
		se $a_{i,j} \geq 0$ oppure $\ds \sum_{i} \sum_{j} \left| a_{i,j} \right| < +\infty $.
\end{itemize}

