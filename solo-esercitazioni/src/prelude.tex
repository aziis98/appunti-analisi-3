%%%
%%% Use a custom geometry for the page
%%%
\usepackage{geometry}
\geometry{
  a4paper,
  left=20mm,
  right=20mm,
  top=20mm,
  bottom=25mm
}

\usepackage{lipsum}

%%%
%%% File encoding
%%%
\usepackage[utf8]{inputenc}
\usepackage[italian]{babel}
\usepackage[T1]{fontenc}

%%%
%%% Font
%%%
\usepackage{lmodern}
\usepackage{textcomp}
\usepackage[font=itshape]{quoting}

%%%
%%% References
%%%
\usepackage[perpage]{footmisc}
\usepackage{hyperref}
\hypersetup{ 
  linkbordercolor={.77 .4 .20},
  citebordercolor={.31 .63 .31},
  pdftitle={Appunti}
}

%%%
%%% Pictures
%%%

% General
\usepackage{caption}
\usepackage{wrapfig}

%%% 
%%% Graphics
%%% 
\usepackage{graphicx}


% Path per le figure
\graphicspath{{./figures/.pdf_tex}}

% Magia arcana equivalente a "\graphicspath" ma per "\input"
\makeatletter
\def\input@path{{./figures/.pdf_tex}}
\makeatother

\newcommand{\inputfigure}[1]{\input{#1.pdf_tex}}

% TikZ
\usepackage{tikz}
\usetikzlibrary{cd}

% Inkscape figures
\usepackage{import}
\usepackage{xifthen}
\usepackage{pdfpages}
\usepackage{transparent}
\newcommand{\incfig}[1]{%
    \def\svgwidth{\columnwidth}
    \import{./figures}{#1.pdf_tex}
}

%%%
%%% Basic math and layout packages
%%%
\usepackage{amssymb}
\usepackage{amsmath}
\usepackage{amsthm}
\usepackage{mathrsfs}
% \usepackage{mathabx}
\usepackage{mathtools}
% \usepackage{xfrac}
\usepackage{array}
\usepackage[all]{xy}
\usepackage{stmaryrd}
\usepackage{multicol}
\usepackage{centernot}
\usepackage{esint}
\usepackage{cancel}

%%%
%%% Enumerations
%%%
\usepackage{enumitem}

% Custom settings for compact lists
\setitemize{topsep=5pt}
\setlist[itemize,1]{label=$\bullet$}
\setlist[itemize,2]{label=$\circ$}

\setlist[enumerate,1]{label=\roman*)}
\setenumerate{topsep=5pt}

%%%
%%% Sections and page borders styles
%%%

% \usepackage{titlesec}
% Adds horizontal rules under "\section" entries
% \titleformat{\section}{\Large\scshape\raggedright}{}{0em}{}[\titlerule]
% \usepackage{fancyhdr}
% \pagestyle{fancy}
% \pagestyle{fancyplain} % Makes all pages in the document conform to the custom headers and footers
% \fancyhead[L]{\leftmark}% Empty left header
% \fancyhead[C]{} % Page numbering for center header  
% \fancyhead[R]{\rightmark}% Empty right header
% \fancyfoot[L]{}% Empty left footer
% \fancyfoot[C]{\thepage}% Empty center footer
% \fancyfoot[R]{}% Empty left footer
% \fancyhf{}
% \chead{\footnotesize\rightmark}
% \rfoot{\thepage}

%%%
%%% Styles by @aziis98
%%%

\newcommand{\avint}{\mathbin{%
  \mathchoice%
    {\,-\hspace{-12.5pt}\int}%
    {\,-\hspace{-10.5pt}\int}%
    {\,-\hspace{0pt}\int}% TODO: se servono
    {\,-\hspace{0pt}\int}% TODO: se servono
}}

\newcommand{\supp}{\operatorname{supp}}

% Better "\setminus"
\newcommand{\mysetminusD}{%
  \hbox{\tikz{\draw[line width=0.6pt,line cap=round] (3pt,0) -- (0,6pt);}}
}
\newcommand{\mysetminusS}{%
  \hbox{\tikz{\draw[line width=0.45pt,line cap=round] (2pt,0) -- (0,4pt);}}
}
\newcommand{\mysetminusSS}{%
  \hbox{\tikz{\draw[line width=0.4pt,line cap=round] (1.5pt,0) -- (0,3pt);}}
}
\newcommand{\mysetminusT}{\mysetminusD}
\newcommand{\mysetminus}{\mathbin{\mathchoice{\mysetminusD}{\mysetminusT}{\mysetminusS}{\mysetminusSS}}}
\renewcommand{\setminus}{\mysetminus}

% Generali
% \renewcommand{\implies}{\Rightarrow}
% \renewcommand{\iff}{\Leftrightarrow}
\newcommand{\longiff}{\Longleftrightarrow}
\newcommand{\norm}[1]{\left\lVert #1 \right\rVert}
\newcommand{\sfrac}[2]{
    \raisebox{+0.55ex}{$#1$}
    /
    \raisebox{-0.3ex}{$#2$}
}

\renewcommand{\epsilon}{\varepsilon}
\newcommand{\myphi}{\varphi}
\newcommand{\compose}{\circ}
\newcommand{\mquad}{\;\;}
\newcommand{\mymid}{\;\middle|\;}
\newcommand{\curry}{\,\cdot\,}
\newcommand{\C}{\mathbb C}
\newcommand{\R}{\mathbb R}
\newcommand{\Q}{\mathbb Q}
\newcommand{\Z}{\mathbb Z}
\newcommand{\N}{\mathbb N}
\newcommand{\ds}{\displaystyle}
\newcommand{\dd}{\,\mathrm{d}}
\newcommand{\pd}{\partial}

% Fulmine dell'assurdo
\usepackage{stmaryrd}
\let\stmaryrdLightning\lightning
\renewcommand{\lightning}{\stmaryrdLightning}
\newcommand{\absurd}{$\lightning$}

% Funzione indicatrice
\newcommand{\bbOne}{\text{\usefont{U}{bbold}{m}{n}1}}
\MakeRobust{\bbOne}
\newcommand{\One}{\bbOne}

% Ambienti temporanei
% [Convergenza Monotona (Beppo Levi)]

\newenvironment{theorem}{\textbf{Teorema.}}{}
\newenvironment{named-theorem}[1]{\textbf{Teorema.} (\textit{#1})}{}

% \newcommand{\TdF}{\mathcal F}
\newcommand{\foralmostall}{\widetilde\forall}
\newcommand{\almosteverywhere}{\;\;\text{q.o.}}

\let\oldtilde\tilde
\renewcommand{\tilde}{\widetilde}

%%%
%%% Styles by @<arianna>
%%%

%%
%% Nuovi comandi
%%
% Utility

% Insiemi numerici
\newcommand{\restr}[2]{\left.#1\right|_{#2}}

% Dichiarazione lettera Chi maiuscola -> \Chi
\DeclareRobustCommand{\rchi}{{\mathpalette\irchi\relax}}
\newcommand{\irchi}[2]{\raisebox{\depth}{\mbox{\Large$#1\chi$}}} % inner command, used by \rchi%
\newcommand{\Chi}{\rchi}

% Altri comandi
% operatori matematici
\renewcommand{\dim}{\operatorname{dim}}
% \newcommand{\ker}{\operatorname{ker}}
\newcommand{\imm}{\operatorname{Im}}
\newcommand{\spn}{\operatorname{Span}}
\newcommand{\sgn}{\operatorname{sgn}}
\newcommand{\rk}{\operatorname{rk}}
% \newcommand{\ort}{\; \bot \;}

% utility and spacing
\renewcommand{\hat}[1]{\widehat{#1}}
\newcommand{\ol}[1]{\overline{#1}}
\newcommand{\ul}[1]{\underline{#1}}
\renewcommand{\vss}{\vspace{5mm}}
\newcommand{\vs}{\vspace{3mm}}

\newcommand{\mc}[1]{\mathcal{#1}}

\newcommand{\cper}{\mc{C}_{\operatorname{per}}}

\let\oldhat\hat
\renewcommand{\hat}{\widehat}

\newcommand{\re}{\operatorname{Re}}
