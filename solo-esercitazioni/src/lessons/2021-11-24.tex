\section{Appendice}

Studiamo alcune variazioni dell'equazione del calore.

\textbf{Nota.} Un problema del tipo $u_t = a(t) \cdot u_{xx}$ si può risolvere ripercorrendo i passaggi della risoluzione dell'equazione del calore.
Viceversa, il problema $u_t = a(x) \cdot u_{xx}$ non si può risolvere allo stesso modo, in quanto, non è vero che il prodotto di serie di Fourier ha come coefficienti il prodotto dei coefficienti.

Studiamo ora variazioni alle condizioni di bordo.

\textbf{Osservazione.} Quando proviamo a risolvere $u_t = u_{xx}$, passiamo alla serie di Fourier e deriviamo; per fare questo passaggio servono le condizioni al bordo\footnote{Anche se avevamo derivato le formule formalmente anche a posteriori l'ipotesi delle condizioni al bordo era necessaria.}; dunque, togliendo le condizioni di periodicità il sistema non funziona più molto bene.

Introduciamo delle varianti della serie di Fourier.

\begin{itemize}

	\item \textbf{Serie di Fourier reale}. Data $u \in L^2([-\pi,\pi])$, definiamo
	$$
	\begin{gathered}
		u(x) = \sum_{n=1}^\infty \left[ a_n \cos(nx) + b_n \sin(nx) \right] + a_0 \\
		\qquad
		a_n(u) = c_n(u) + c_{-n}(u)
		\qquad
		b_n(u) = i(c_n(u) - c_{-n}(u))
		\qquad
		a_0(u) = c_0(u) \\
		a_n = \frac{1}{\pi} \int_{-\pi}^\pi f(x) \cos(nx) \dd x
		\qquad
		b_n = \frac{1}{\pi} \int_{-\pi}^\pi f(x) \sin(nx) \dd x \\
		\text{e con base hilbertiana } \left\{ \frac{1}{\sqrt{2\pi}}, \frac{\cos(nx)}{\sqrt{\pi}}, \frac{\sin(nx)}{\sqrt{\pi}} \,\middle|\, n \geq 1 \right\}.
	\end{gathered}
	$$

	\item \textbf{Serie di Fourier su} \boldmath$[-\pi,\pi]^d$. Data \unboldmath$u \in L^2([-\pi,\pi]^2, \C)$, definiamo
	%
	$$
		u(x) = \sum_{\underline{n} \in \Z^d} c_{\ul{n}} e^{i \ul{n} x} \qquad 
		c_{\ul{n}} = c_{\ul{n}} (u) \coloneqq \frac{1}{(2\pi)^d} \int_{[-\pi,\pi]^d} u(x) e^{- i \ul{n} x} \dd x 
	$$
	%
	con base di Hilbert
	%
	$$
		\mc{F} = \left\{ \frac{e^{i \ul{n} x}}{(2\pi)^{d/2}} \colon n \in \Z^d \right\}
	$$
	%

	C'è da dimostrare che $\mc{F}$ è una base di Hilbert.

	\textbf{Dimostrazione} (idea). 
	\begin{itemize}

		\item Ortonormalità. È un conto [TO DO].


		\item Completezza. Si può dimostrare come per $d = 1$, oppure si usa il seguente lemma.

		\textbf{Lemma.} Sia $\mc{F}_1 \coloneqq \{ e_n^1 \}$ base di Hilbert di $L^2(X_1, \C)$ e $\mc{F}_2 \coloneqq \{ e_n^2 \}$ base di Hilbert di $L^2(X_2, \C)$. Allora, una base di Hilbert di $L^2(X_1 \times X_2, \C)$ è
		%
		$$
			\mc{F} = \left\{ e_{n_1,n_2}(x_1,x_2) \mymid e_{n_1}^1 (x_1) e_{n_2}^2(x_2) \right\}
		$$
		%

	\end{itemize}

	\textit{Formula chiave.}
	Se $u \in C_{\text{per}}^1 (\R^d) = \left\{ \text{funzioni $2\pi$-periodiche in tutte le variabili} \right\}$. Abbiamo che
	%
	$$
		c_{\ul{n}} (\nabla u) = i \ul{n} c_n(u), \qquad 
		c_{\ul{n}} (\Delta u) = - \left| \ul{n} \right|^2 c_{\ul{n}} (u) \mquad \text{se }  u \in \mc{C}_{\text{per}}^2
	$$
	%


	\item \textbf{Serie in seni.} Data $u \in L^2([0,\pi])$, allora 
	%
	$$
		u(x) = \sum_{n=1}^\infty b_n \sin(nx) \qquad 
		b_n = b_n(u) \coloneqq  \frac{2}{\pi} \int_0^\pi u(x) \sin(nx) \dd x
	$$
	%
	con base di Hilbert 
	%
	$$
		\mc{F} = \left\{ \sqrt{\frac{2}{\pi}} \sin(nx) \mymid n \geq 1 \right\}
	$$
	%

	\textbf{Dimostrazione.} Mostriamo l'ortonormalità e la completezza.

	\textit{Ortonormalità.} Sono conti. [TO DO]

	\textit{Completezza.} Data $u \in L^2([0,\pi])$. Sia $\tilde{u}$ l'estensione dispari a $[-\pi,\pi]$. Allora 
	%
	$$
		\tilde{u} = a_0 + \sum_{n=1}^{\infty} \tilde{a}_n \cos(nx) + \tilde{b}_n \sin(nx)
		\underset{\tilde{u} \text{ dispari}} = \sum_{n=1}^{\infty} \tilde{b}_n \sin(nx).
	$$
	%

	\textbf{Osservazione.} I coefficienti $\tilde{b}_n = b_n$. Si può vedere in diversi modi, un modo possibile è questo.
	%
	$$
		\tilde{b}_n = \frac{1}{\pi} \int_{-\pi}^{\pi} \tilde{u}(x) \sin(nx) \dd x 
		= \frac{2}{\pi} \int_{0}^{\pi} \tilde{u} \sin(nx) \dd x 
		= \frac{2}{\pi} \int_{0}^{\pi} u(x) \sin(nx) \dd x 
		= b_n.
	$$
	%

	\textbf{Formula chiave.} Data $u \in \mc{C}^2 ([0,\pi])$ con condizioni al bordo $u(\curry, 0) = u(\curry, \pi)$. 
	Allora
	%
	$$
		b_n(\ddot u) = -n^2 b_n(u)
	$$
	%
	dove 
	\begin{align*}
		b_n(\ddot u) & \coloneqq \frac{2}{\pi} \int_0^\pi \ddot u(x) \sin(nx) \dd x \\
		& = \frac{2}{\pi} \cancel{\left| \dot u(x) \sin(nx) \right|_0^\pi} - n\frac{2}{\pi} \int_{0}^{\pi} \dot u(x) \cos(nx) \dd x \\
		& = -n \frac{2}{\pi} \cancel{\left| u(x) \cos(nx) \right|_0^\pi} - n^2  \underbrace{\left( \frac{2}{\pi} \int_{0}^{\pi} u(x) \sin(nx) \dd x \right)}_{b_n(u)} \\
		& = -n^2 b_n(u).
	\end{align*}
	\qed

\end{itemize}


\textbf{Applicazione} (della serie in seni). Risoluzione di EDP su $[0,\pi]$ con condizioni di Dirichlet (omogenee) al bordo.

\textbf{Esempio.} Risolvere
%
\begin{equation}
\label{eq:24nov-problema-1} \tag{P}
	\begin{cases}
		u_t = u_{xx} \quad \text{su} \mquad [0,\pi] \\
		u(\curry, 0) = u(\curry,\pi) = 0 \\
		u(0,\curry) = u_0
	\end{cases} 
\end{equation}

\textbf{Soluzione.} Poniamo $b_n^0 \coloneqq b_n (u_0)$.
Scriviamo $\ds u(t,x) = \sum_{n=1}^{\infty} b_n(t) \sin(nx) $ serie di seni in $x$.

Formalmente,
%
$$
	u_t = \sum_{n=1}^\infty \dot b_n(t) \sin(nx) \qquad 
	u_{xx} = \sum_{n=1}^{\infty} -n^2 b_n(t) \sin(nx),
$$
%
dunque
%
$$
	u_t = u_{xx} \longiff \dot b_n(t) = -n^2 b_n(t) \qquad \forall t \forall n.
$$
%
Cioè $b_n(t)$ risolve il problema di Cauchy.
\begin{equation}
	\label{eq:24nov-problema-2} \tag{P'}
	\begin{cases}
		\dot y = -n^2 y \\
		y(0) = \dot b_n.
	\end{cases}
\end{equation}

Ovvero $b_n(t) = b_n^0 e^{-n^2 t}$, da cui  
\begin{equation}
\label{eq:24nov-soluzione-1} \tag{$\ast$}
	u(t,x) = \sum_{n=1}^\infty \underbrace{b_n^0 e^{-n^2t} \sin(nx)}_{u_n}.
\end{equation}


\textbf{Teorema 1} (di esistenza nel futuro).
Se $u_0 \colon [0,\pi] \to \R$ è continua è $\ds \sum_n |b_n^0| < +\infty$ (basta $u_0 \in \mc{C}^1$ e $u(0) = u(\pi) = 0$).
Allora la $u$ in \eqref{eq:24nov-soluzione-1} è ben definita e continua su $[0,+\infty) \times \R$ e risolve \eqref{eq:24nov-problema-1}.

\textbf{Dimostrazione.} Dimostriamo il teorema per passi.

\textit{Passo 1.}
Mostriamo che $u$ è ben definita e continua su $(0,+\infty) \times \R$: studiamo la norma del sup. Sia $R = [0, +\infty) \times \R$.
%
$$
	\norm{u_n}_{L^\infty(R)} \leq |b_n^0| \Longrightarrow u_n \mquad \text{converge totalmente su } \R.
$$
%

\textit{Passo 2.} Mostriamo che $\mc{C}^\infty$ su $(0,+\infty) \times \R$.
Sia $R_\delta = (\delta, +\infty) \times \R$.
Stimiamo le derivate.
\begin{align*}
	D_t^k D_x^h u_n = b_n^0 (-n^2)^k e^{-n^2t} \cdot n^h \cdot \underbrace{\ldots}_{\star} \\
	\Longrightarrow \norm{D_t^k D_x^h u_n}_{L^\infty(R_\delta)} 
	= |b_n^0| \underbrace{e^{-n^2\delta} \cdot |n|^{2k+h}}_{\substack{\text{limitato in $n$} \\ \text{ perché è infinitesimo in $n$ }}}
\end{align*}
Allora le norme delle derivate sono sommabili per ogni $n$, dunque $u \in \mc{C}^\infty(R_\delta)$ per ogni $\delta$, da cui $u \in \mc{C}^\infty( (0,+\infty),\R)$.

\textit{Passo 3.} Mostriamo che la $u(t,x)$ definita in \eqref{eq:24nov-soluzione-1} risolve \eqref{eq:24nov-problema-1}.
\begin{itemize}

	\item $u$ risolve $u_t = u_{xx}$ per $t > 0$.
	Infatti, l'equazione è lineare per quanto mostrato al punto sopra e dunque posso scambiare serie e derivata.


	\item $u$ soddisfa la condizione iniziale $u(0,\curry) = u_0$, perché hanno gli stessi coefficienti di Fourier.


	\item Sono soddisfatte anche le condizioni al bordo, infatti
	%
	$$
	u(\curry,0) = u(\curry,\pi) = 0
	$$
	%
\end{itemize}
\qed

\textbf{Domanda.} Quale ipotesi su $u_0$ garantisce $\sum_n |b_n^0| < +\infty$? Basta $u_0 \in \mc{C}^1$ e $u(0) = u(\pi) = 0$.

\textbf{Teorema 2} (non esistenza nel passato).
Esiste $u_0 \colon [0,\pi] \to \R$ $\mc{C}^\infty$, con $u_0(0) = u_0(\pi)$ tale che per ogni $\delta > 0$ \eqref{eq:24nov-problema-1} non ha alcuna soluzione $u \colon (-\delta,0] \times [0,\pi] \to \R$ continua e $\mc{C}^1$ in $t$ e $\mc{C}^2 $ in $x$.

\textbf{Teorema 3} (di unicità).
Sia $u \colon [0, T) \times [-\pi, \pi] \to \C$ continua, $C^1$ nel tempo e $C^2$ nello spazio per $t > 0$. Se $u$ risolve \eqref{eq:15nov2021_problem_1} su $t > 0$ allora $u$ è unica.
