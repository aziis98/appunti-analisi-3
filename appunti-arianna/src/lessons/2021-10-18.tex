%
% Lezione dell'18 Ottobre 2021
%

\chapter{Convoluzione}

\textbf{Definizione.} Date $f_1,f_2 \colon  \R^d \to \R$ misurabili, il \textbf{prodotto di convoluzione} $f_1 \ast f_2$ è la funzione (da $\R^d$ a $\R$) data da
%
$$
	f_1 \ast f_2(x) = \int_{\R^d} f_1(x-y) f_2(y) \dd y
$$
%
%
\textbf{Osservazioni.}
\begin{enumerate}
\item La definizione sopra è ben posta se $f_1,f_2 \geq 0$ ($f_1 \ast f_2(x)$ può essere anche $+\infty$).
In generale non è ben posta per funzioni a valori reali (non è detto che l'integrale esista). 

Ad esempio, se prendiamo $f_1 = 1$ e $f_2 = \sin x$ con $d = 1$, allora $f_1 \ast f_2(x)$ non è definito per alcun $x$.


\item Se $f_1 \ast f_2(x)$ esiste, allora $\ds f_1 \ast f_2(x) = f_2 \ast f_1(x)$, infatti
%
$$
	f_1 \ast f_2 (x) 
	= \int_{\R^d} f_1(x-y) f_2(y) \dd y 
	% \underbrace{=}_{} 
	= \left( 
	{\footnotesize \begin{gathered}
		t \coloneqq x - y \\ 
		\dd t = \dd y
	\end{gathered}}
	\right) =
	\int_{\R^d} f_1(t) f_2(x-t) \dd t 
	= f_2 \ast f_1(x).
$$
%

\item È importante che $f_1,f_2$ siano definite su $\R^d$ e che la misura sia quella di Lebesgue.

In realtà, si può generalizzare quanto sopra rimpiazzando $(\R^d, \mathscr L^d)$ con $(G,\mu)$, dove $G$ è un gruppo commutativo e $\mu$ una misura su $G$ invariante per traslazione. Per esempio, $\Z$ con la misura che conta i punti. Cioè $f_1,f_2 \colon \Z \to \R$, vale
%
$$
	f_1 \ast f_2(n) \coloneqq \sum_{n \in \Z} f_1(n - m) f_2(m).
$$
%

\item Data $f$ distribuzione di massa (continua) su $\R^3$, il potenziale gravitazionale generato è
%
$$
	v(x) = \int_{y \in \R^d} \frac{1}{\left| x - y \right|} \rho(y) \dd y
$$
%
cioè $v = g \ast \rho$, dove  $g (x) = 1 / \left| x \right|$ è il potenziale di una massa puntuale in $0$.

\item Se $X_1, X_2$ sono variabili aleatorie (reali) con distribuzione di probabilità continua $p_1,p_2$ e $X_1,X_2$ sono indipendenti, allora $X_1 + X_2$ ha distribuzione di probabilità $p_1 \ast p_2$. (Facile per $X_1,X_2$ in $\Z$).

\end{enumerate}
