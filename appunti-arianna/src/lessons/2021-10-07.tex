%
% Lezione del 07 ottobre
%

\section{Nozioni di convergenza per successioni di funzioni}
Fissiamo $X,\mc{A},\mu$ e prendiamo $f, f_n \colon X \to \R$ (o $\R^k$) misurabili.

\textbf{Definizione.}
Riportiamo le definizioni di alcune nozioni di convergenza.

\begin{itemize}

\item \textbf{Uniforme} : $\forall \epsilon \; \exists n_{\epsilon}$ tale che $\norm{f - f_n} < \epsilon \mquad \forall n > n_\epsilon$.

\item \textbf{Puntuale} : $f_n(x) \to f(x) \mquad \forall x \in X$.

\item \textbf{Puntuale} $\mu$\textbf{-quasi ovunque} : $f_n(x) \to f(x)$ per $\mu$-q.o. $x \in X$.

\item \textbf{In} $L^p$ : $\norm{f_n - f}_p \xrightarrow{n \to \infty} 0$.

\item \textbf{In misura} : $\ds \forall \epsilon > 0 \quad \mu \left( \left\{ x \mymid \left| f_n(x) - f(x) \right| \geq \epsilon \right\} \right) \xrightarrow{n \to +\infty} 0$.

\end{itemize}

\textbf{Osservazione.}
Abbiamo le seguenti implicazioni ovvie delle diverse nozioni di convergenza:

\begin{center}
	uniforme $\Rightarrow$ puntuale $\Rightarrow$ puntuale $\mu \almosteverywhere$
\end{center}


\textbf{Proposizione.}
Valgono le seguenti.
\begin{enumerate}

\item \label{item:convergenza_i} Data $f_n \to f$ q.o. e $\mu(X) < +\infty$, allora $f_n \to f$ in misura.

\item \label{item:convergenza_ii} (\textit{Severini-Egorov}): Data $f_n \to f$ q.o. e $\mu(X) < +\infty$, allora $\forall \delta > 0$ esiste  $E \in \mc{A}$ tale che $\mu(E) < \delta$ e $f_n \to f$ uniformemente su $X \setminus E$.

\item \label{item:convergenza_iii} $f_n \to f $ in $L^p$, $p < +\infty$, allora $f_n \to f$ in misura.

\item[iii')] \label{item:convergenza_iv} $f_n \to f \in L^\infty$, allora $\exists E $ tale che $\mu(E) = 0$ e $f_n \to f$ uniformemente su $X \setminus E$.

\item $f_n \to f$ in misura, allora $\exists n_k$ tale che $f_{n_k} \to f$ $\mu$-q.o.

\item $f_n \to f$ in $L^p$, allora $\exists n_k$ tale che $f_{n_k} \to f$ $\mu$-q.o.

\end{enumerate}

\textbf{Osservazione.}
In i) e ii) l'ipotesi $\mu(X) < +\infty$ è necessaria.
Infatti, preso $X = \R$ e $f_n = \One_{[n,+\infty)}$ si ha che $f_n \to 0$ ovunque ma $f_n$ non converge a $0$ in misura, e $f_n$ non converge a $0$ uniformemente in $\R \setminus E$ per ogni $E$ di misura finita.

\textbf{Lemma} (disuguaglianza di Markov).
Data $g \colon X \to [0,+\infty]$ misurabile e $m > 0$ si ha
%
$$
\mu \left( \left\{ x \in X \mymid g(x) \geq m \right\} \right) \leq \frac{1}{m} \int_X g \dd \mu
$$
%

\textbf{Dimostrazione.}
Poniamo $\ds E \coloneqq \left\{ x \in X \mymid g(x) \geq m \right\}$.
Osserviamo che $g \geq m \cdot \One_E$.
Dunque vale
%
$$
	\int_X g \dd \mu \geq \int_X m \cdot \One_E \dd \mu = m \cdot \mu \left( \left\{ x \in X \mymid g(x) \geq m \right\} \right).
$$
%
\qed

\textbf{Lemma} (Borel-Cantelli).
Dati $(E_n) \subset \mc{A}$ tali che $\sum \mu(E_n) \leq +\infty$, l'insieme
%
$$
	E \coloneqq \left\{ x \in X \mymid x \in E_n \; \text{frequentemente} \right\}
$$
%
ha misura nulla.
Cioè per $\mu$-q.o. $x$ vale $x \notin E_n$ definitivamente (in $n$.)

\textbf{Dimostrazione.}
Osserviamo che
%
$$
	E = \bigcap_{m=1}^\infty \Big( \underbrace{\bigcup_{n=m}^\infty E_n}_{F_m} \Big).
$$
%
Allora
%
$$
	\mu(E) \quad \underbrace{=}_{\mathclap{F_m \downarrow E \; \& \; \mu(F_1) < +\infty}} \quad  \lim_{m \to \infty} \mu(F_m) \leq \lim_{m \to \infty} \underbrace{\sum_{n=m}^{\infty} \mu(E_n)}_{\mathclap{\text{coda serie convergente}}} = 0.
$$
%

\textbf{Osservazione.}
L'ipotesi $\sum \mu(E_n) < +\infty$ non può essere sostituita con $\mu(E_n) \to 0$.

Ora dimostriamo la proposizione.

\textbf{Dimostrazione.} Definiamo gli insiemi
\begin{align*}
A_n^\epsilon & \coloneqq \left\{ x \in X \colon \left| f_n(x) - f(x) \right| \geq \epsilon \right\}, \\
B_m^\epsilon & \coloneqq \left\{ x \in X \colon \left| f_n(x) - f(x) \right| \geq \epsilon \; \text{per qualche} \; n \geq m\right\} = \bigcup_{n = m}^\infty A_n^\epsilon, \\
B^\epsilon & \coloneqq \left\{ x \in X \colon \left| f_n(x) - f(x) \right| \geq \epsilon \; \text{frequentemente}  \right\} = \left\{ x \in A_n^\epsilon \; \text{frequentemente}  \right\} = \bigcap_{m = 1}^\infty B_m^\epsilon.
\end{align*}

\begin{enumerate}

	\item Dobbiamo mostrare che $\lim_{m \to \infty} A_m^\epsilon = 0$.
	Per ipotesi, $f_n \to f$ quasi ovunque, cioè $\mu(B^\epsilon) = 0$ per ogni $\epsilon > 0$, ma $B_m^\epsilon \downarrow B^\epsilon$ e $\mu(X) < +\infty$.
	Allora
	%
	$$
		\lim_{m \to +\infty} \mu(B_m^\epsilon) = \mu(B^\epsilon) = 0 \Longrightarrow \lim_{m \to \infty} \mu(A_m^\epsilon) = 0.
	$$
	%
	dato che $A_m^\epsilon \subset B_m^\epsilon$.


	\item Dalla dimostrazione precedente abbiamo $\ds \lim_{m \to \infty} \mu(B_m^\epsilon) = 0$. 
	Allora per ogni $k$ esiste un $m_k$ tale che $\mu \left( B_{m_k}^{1/k} \right) \leq \delta / 2^k$.
	Poniamo $\ds E \coloneqq  \bigcup_{k} B_{m_k}^{1/k}$ per ogni $k$; allora $\mu(E) \leq \delta$.
	Inoltre,
	\begin{align*}
		x \in X \setminus E & \Longrightarrow x \notin B_{m_k}^{1/k} \; \forall k \iff x \notin A_n^{1/k} \mquad \forall k,n \geq m_ k \\
		& \Longrightarrow \left| f(x) - f_n(x) \right| < \frac{1}{k} \mquad \forall k,n \geq m_k \\
		& \Longrightarrow \sup_{x \in X \setminus E} \left| f(x) - f_n(x) \right| \leq \frac{1}{k} \mquad \forall k,n \geq m_k \\
		& \Longrightarrow f - f_m \; \text{uniformemente su} \; X \setminus E.
	\end{align*}


	\item Dobbiamo mostrare che per ogni $\epsilon > 0$ $\mu(A_n^\epsilon) \xrightarrow{n} 0$.
	Usando la disuguaglianza di Markov ottengo
	$$
		\mu \Big( A_n^\epsilon= \Big\{ x \Big| \overbrace{\left| f_n(x) - f(x) \right|^p}^{g} \geq \epsilon^p \Big\} \Big)
		\leq \frac{1}{m} \int_{X}^{} g \dd \mu = \frac{1}{\epsilon^p} \norm{f_n - f}_p^p \xrightarrow{n \to +\infty} 0.
	$$


	\item[iii')] Definiamo $\ds E_n \coloneqq  \left\{ x \mymid \left| f_n(x) - f(x) \right| > \norm{f_n - f}_\infty \right\}$ per ogni $n$, allora $\mu(E_n) = 0$.
	Poniamo $E = \bigcup_{n} E_n$ e $\mu(E) = 0$, dunque
	%
	$$
		\sup_{x \in X\setminus E} \left| f_n(x) - f(x) \right| \leq \norm{f_n - f}_\infty \xrightarrow{n \to \infty} 0.
	$$
	%


	\item Per ipotesi, $f_n \to f$ in misura, cioè
	\begin{align*}
		& \forall \epsilon > 0 \quad \mu \left( A_n^\epsilon \right) \xrightarrow{n \to +\infty} 0 \\
		& \Longrightarrow \forall k \; \exists n_k \colon \mu \left( A_{n_k}^{1/k} \right) \leq \frac{1}{2^k} \\
		& \Longrightarrow \sum_{k}^{} \mu \left( A_{n_k}^{1/k} \right) < +\infty. 
	\end{align*}
	Allora per Borel-Cantelli, si ha per $\mu$-quasi ogni $x$, $x \notin A_{n_k}^{1/k}$ definitivamente in $k$, cioè $\norm{f_{n_k}(x) - f(x)} < 1/k$ definitivamente in $k$, cioè $\ds f_{n_k}(x) \xrightarrow{k} f(x)$.


	\item[v)] Vogliamo mostrare che $f_n \to f$ in $L^p \Longrightarrow \exists n_k$ tale che $f_{n_k} \to f$ quasi ovunque. Consideriamo due casi
	\begin{itemize}

		\item se $p < +\infty$, allora $f_n \to f$ in $L^p \Longrightarrow f_n \to f$ in misura, da cui $\exists n_k$ tale che $f_{n_k} \to f$ quasi ovunque

		\item se $p=+\infty$, allora $f_n \to f$ uniformemente su $X \setminus E$ con $\mu(E) = 0 \Longrightarrow f_n \to f$ puntualmente su $X \setminus E \Longrightarrow f_n \to f$ quasi ovunque.

	\end{itemize}
	\qed

\end{enumerate}
