%
% Lezione del 6 Ottobre 2021
%

\section{Costruzione spazi $L^p$}

Fissiamo $(X, \mc A, \mu)$ come sempre.

\textbf{Definizione.}
Sia $\mathscr L^p$ l'insieme delle funzioni $f \colon X \to \R$ o $\R^d$ misurabili tali che $\norm{f}_p < +\infty$.

\textbf{Osservazioni.}
\begin{itemize}
	\item $\mathscr L^p$ è un sottospazio vettoriale dello spazio vettoriale dato da $\{ f \colon X \to \R \mid f \text{ misurabile} \}$ e $\norm{\curry}_p$ è una semi-norma.

		\textbf{Dimostrazione.}
		\begin{itemize}
			\item $\mathscr L^p$ è chiuso per somma e moltiplicazione per scalari.

			% \item $f_1, f_2 \in \mathscr L^p \implies f_1 + f_2 \in \mathscr L^p$

			\item Dalla definizione segue subito $\norm{\lambda f}_p = |\lambda| \cdot \norm{f}_p$ l'omogeneità della norma.

			\item Dalla disuguaglianza di Minkowski segue che $\norm{\curry}_p$ è una semi-norma.
		\end{itemize}

	\item In particolare non è una norma se $\{ 0 \} \subsetneq \{ f \mid \norm{f}_p = 0 \}$ ovvero se $\mc A$ contiene insiemi non vuoti di misura nulla.

	\item In generale dato $V$ spazio vettoriale e $\norm{\curry}$ semi-norma su $V$ possiamo introdurre $N \coloneqq \{ v \mid \norm{v} = 0 \}$. $N$ risulta essere un sottospazio di $V$ e la norma data da $\norm{[v]} \coloneqq \norm{v}$ per $[v] \in \sfrac{V}{N}$ è ben definita ed è proprio una norma su $\sfrac{V}{N}$.

	\item Nel caso della della norma $\norm{\curry}_p$ abbiamo che $[f_1] = [f_2] \iff [f_1 - f_2] = 0 \iff f_1 - f_2 = 0$ quasi ovunque. 
\end{itemize}

\textbf{Definizione.}
Poniamo $N \coloneqq \{ f \mid \norm{f}_p = 0 \}$ e definiamo gli \textbf{spazi $L^p$} come
$$
L^p := \sfrac{\mathscr L^p}{N} = \sfrac{\mathscr L^p}{\sim} 
\qquad
\norm{[f]}_p \coloneqq \norm{f}_p
$$

\textbf{Notazione.}
Ogni tanto serve precisare meglio l'insieme di partenza e di arrivo degli spazi $L^p$ ed in tal caso useremo le seguenti notazioni
$$
L^p = L^p(X) = L^p(X, \mu) = L^p(X, \mc A, \mu) = L^p(X, \mu; \R^d).
$$

\textbf{Nota.}
Nella pratica non si parla mai di ``classi di funzioni'' e si lavora direttamente parlando di ``funzioni in $L^p$''. Le ``operazioni'' comuni non creano problemi però in certi casi bisogna stare attenti di star lavorando con oggetti ben definiti. Ad esempio:
\begin{itemize}
	\item Preso $x_0 \in X$, consideriamo l'insieme $\{ f \in L^p \mid f(x_0) = 0 \}$. Notiamo che non è un sottoinsieme ben definito (a meno che $\mu(\{ x_0 \}) > 0$ ovvero che la misura sia atomica) di $L^p$, in quanto possiamo variare $f$ su un insieme di misura nulla.

	\item Invece ad esempio il seguente insieme è ben definito
		$$
		\left\{ f \in L^1 \;\middle|\; \int_X f \dd \mu = 0 \right\}
		$$
\end{itemize}

\subsection{Prodotto scalare su $L^2$}

Date $f_1,f_2 \in L^2(X)$ si pone
%
$$
\langle f_1, f_2 \rangle \coloneqq \int_{X}^{} f_1 \cdot f_2 \dd \mu.
$$
%
\textbf{Osservazioni.}

\begin{itemize}
\item La definizione di $\langle f_1, f_2 \rangle$ è ben posta, infatti basta far vedere che $\ds \int_X \left| f_1 f_2 \right| \dd \mu < +\infty$ ma per H\"older abbiamo
%
$$
\int_{X}^{} \left| f_1 f_2 \right| \dd \mu \leq \norm{f_1}_2 \norm{f_2}_2 < +\infty
$$
%

\item $\norm{f}_2^2 = \langle f,f \rangle$ per ogni $f \in L^2(X)$.

\item Inoltre, $\ds \left| \int_X f_1 f_2 \dd \mu \, \right| \leq \int_X \left| f_1 f_2 \right| \dd \mu$ quindi
%
$$
\left| \langle f_1, f_2 \rangle \right| \leq \norm{f_1}_2 \norm{f_2}_2 \quad \; \text{(\textit{Cauchy-Schwartz})}.
$$

\item L'operatore $\langle \curry, \curry \rangle$ è un prodotto scalare definito positivo.

\end{itemize}

\textbf{Osservazioni.}
\begin{itemize}
\item Dato $C$ spazio vettoriale reale con prodotto scalare $\langle \curry, \curry \rangle$, allora $\langle \curry, \curry \rangle$ si ricava dalla norma associata $\norm{\curry}$ tramite l'identità di polarizzazione:
%
$$
\langle v_1, v_2 \rangle = \frac{1}{4} \left( \norm{v_1 + v_2}^2 - \norm{v_1 - v_2}^2 \right).
$$
%

\item  Dato $V$ come sopra, vale l'identità del parallelogramma:
%
$$
\norm{v_1 + v_2}^2 + \norm{v_1 - v_2}^2 = 2 \norm{v_1}^2 + 2 \norm{v_2}^2 \quad \forall v_1,v_2 \in V.
$$
%
 Usando questa identità di dimostra che la norma di $L^p$ deriva da un prodotto scalare solo per $p=2$.

\end{itemize}

\textbf{Proprietà.}
Sia $V$ uno spazio vettoriale con norma $\norm{\curry}$. Allora vale l'identità del parallelogramma se solo se $\norm{\curry}$ deriva da un prodotto scalare.

\textbf{Esempio.}
La norma di $L^p \left( [-1,1] \right)$, deriva da un prodotto scalare solo per $p=2$.
Prendiamo $f_1 = \One_{[-1,0]}$ e $f_2 = \One_{[0,+1]}$.
Allora
\vspace{-5mm}
\begin{align*}
	\norm{f_1 + f_2}_p^p = \int_{-1}^{1} 1 \dd x = 2 \Rightarrow \norm{f_1 + f_2}_p = 2^{1/p} \\
	\norm{f_1 - f_2}_p = \norm{f_1 + f_2}_p = 2^{1/p}, \quad \norm{f_1}_p = \norm{f_2}_p = 1
\end{align*}
Se vale l'identità del parallelogramma allora
$$
	\norm{f_1 + f_2}_p^2 + \norm{f_1 - f_2}_p^2 = 2 \norm{f_1}_p^2 + 2 \norm{f_2}_p^2
$$
cioè
$$
	2^{2/p} + 2^{2/p} = 2 \cdot 1 + 2 \cdot 1 \iff p = 2.
$$

Dunque l'unica possibilità per verificare la regola del parallelogramma è che $p=2$.

\textbf{Domanda.} Per quali $X,\mc{A},\mu$ vale la stessa conclusione?


\section{Completezza degli spazi $L^p$}

Vediamo ora la proprietà più importante degli spazi $L^p$.

\textbf{Teorema.}
Lo spazio $L^p$ è completo per ogni $p \in [1,+\infty]$.

% @aziis98: Per ora mi pare non si possano usare direttamente \label e \ref in questi punti, cioè linka alla sezione.
\hypertarget{prop:completeness_lemma_1}{}
\textbf{Lemma 1.} 
Dato $(Y, d)$ spazio metrico, allora
\begin{enumerate}
	\item
		Ogni successione $(y_n)$ tale che
		$$
		\sum^{\infty}_{n=1} d(y_n, y_{n+1}) < +\infty
		$$
		è di Cauchy.

	\item \label{item:def_completeness_1}
		Se ogni $(y_n)$ tale che $\ds \sum^{\infty}_{n=1} d(y_n, y_{n+1}) < +\infty$ converge allora $Y$ è completo.
\end{enumerate}

\textbf{Osservazione.} Non tutte le successioni di Cauchy $(y_n)$ soddisfano quella condizione. Ad esempio la successione $(-1)^n / n$ definita su $\R$ è di Cauchy però
$$
\sum_{n=1}^\infty \left| \frac{(-1)^{n+1}}{n+1} - \frac{(-1)^n}{n} \right| 
= \sum_{n=1}^\infty \frac{2n + 1}{n^2 + n}
\approx \sum_{n=1}^\infty \frac{1}{n} \to \infty.
$$

\textbf{Nota.} Per mostrare la completezza degli spazi $L^p$ è sufficiente verificare la convergenza per una sottoclasse propria delle successioni di Cauchy.

\textbf{Dimostrazione.}
\begin{enumerate}
	\item Vorremmo vedere che $\forall \epsilon \, \exists N$ tale che $\forall m, n > N$ si ha $d(y_m, y_n) \leq \epsilon$. 
		Presi $n > m$ abbiamo che 
		$$
		d(y_m, y_n) \leq \sum_{k=m}^{n-1} d(y_k, y_{k+1}) \leq \sum_{k=m}^\infty d(y_k, y_{k+1}) \to 0
		$$
		in quanto \textit{coda di una serie convergente}, quindi 
		$$
		\forall \epsilon > 0 \mquad \exists m_\epsilon \text{ tale che } \sum_{k = m_\epsilon}^\infty d(y_k, y_{k+1})< \epsilon \implies \forall n > m \geq m_\epsilon \; d(y_m, y_n) \leq \epsilon
		$$ 

	\item
		Sia $(y_n)$ una successione di Cauchy, mostriamo che converge.
		Osserviamo che esiste una sottosuccessione $\left( y_{n_k} \right)$ tale che
		% Basta far vedere che data $(y_n)$ di Cauchy esiste una sottosuccessione $y_{n_k}$ tale che
		$$
			\sum_{k=1}^\infty d(y_{n_k}, y_{n_{k+1}}) < +\infty.
		$$
		Infatti, $\forall k \; \exists n_k$ tale che $\forall n, m \geq n_k \mquad d(y_m, y_n) \leq 1 /2 ^k$ e dunque $d(y_{n_k}, y_{n_{k+1}}) \leq 1 / 2^k$.
		Per ipotesi $\left( y_{n_k} \right)$ converge a un qualche $y \in Y$, da cui la tesi\footnote{Data una successione di Cauchy $x_n$, se una sottosuccessione $x_{n_k}$ converge, allora converge anche la successione.}.
		\qed

		% @aziis98: In realtà non so se era più chiaro come stava prima boh

		% Basta far vedere che data $(y_n)$ di Cauchy esiste una sottosuccessione $y_{n_k}$ tale che
		% $$
		% \sum_{k=1}^\infty d(y_{n_k}, y_{n_{k+1}}) < +\infty
		% $$
		% ma $\forall k \; \exists n_k$ tale che $\forall n, m \geq n_k \; d(y_m, y_n) \leq 1 /2 ^k$ dunque $d(y_{n_k}, y_{n_{k+1}}) \leq 1 / 2^k$.

		% Quindi per ipotesi $y_{n_k}$ converge a qualche $y \in Y$ ed anche $y_n \to y$.
		% \qed
\end{enumerate}


% @aziis98: Il latex mi vuole male e mi metteva solo Corollario e la prima frase alla fine della pagina e l'enumerate alla nuova, però se aggiungiamo cose prima poi lo togliamo.

\hypertarget{prop:completeness_lemma_2}{}
\textbf{Lemma 2.} 
Dato $Y$ spazio normato, i seguenti fatti sono equivalenti
\begin{enumerate}
	\item $Y$ è completo.

	% \item $\sum_{n=1}^\infty y_n$ converge in $Y$ per ogni $(y_n)$ tale che $\sum_{n=1}^\infty \norm{y_n} < +\infty$ ovvero 
	\item Per ogni successione $(y_n)$ tale che $\ds \sum_{n=1}^\infty \norm{y_n} < +\infty$, la serie $\ds \sum_{n=1}^\infty y_n$ converge \footnote{Nel senso che esiste $y$ tale che $\norm{y - \sum_{n=1}^N y_n} \xrightarrow{N \to \infty} 0$.}.
\end{enumerate}

\textbf{Dimostrazione.} 
i) $\Rightarrow$ ii). Dobbiamo mostrare che $\sum_{n=1}^\infty y_n$ è di Cauchy.
Per il Lemma~1 basta mostrare che la successione
%
$$
	z_n \coloneqq  \sum_{k=1}^n y_k \quad \text{soddisfa la proprietà} \quad
	\sum_{n=1}^\infty d(z_{n+1},z_n) < +\infty.
$$
%
Espandendo la formula sopra
%
$$
	\sum_{n=1}^\infty d(z_{n+1},z_n) = \sum_{n=1}^{\infty} \norm{\sum_k^{n+1} y_k - \sum_k^n y_k} 
	= \sum_{n=1}^{\infty} \norm{y_{n+1}}
$$
%
che è finito per ipotesi.

\vs

ii) $\Rightarrow$ i). Utilizziamo l'enunciato ii) del Lemma~2: mostriamo ogni $(y_n)$ che soddisfa la proprietà $\sum_n d(y_n,y_{n+1}) < +\infty$ converge.
Definiamo la successione $z_n \coloneqq y_{n+1} - y_n$. Per ipotesi, essendo che $\sum_n \norm{z_n} < +\infty$, la serie $\sum_n z_n$ converge.
Indicando con $L$ il limite della serie, abbiamo che $\lim_n y_n = L + y_1$.


\qed

\hypertarget{prop:completeness_lemma_3}{}
\textbf{Lemma 3} 
(Minkowski per somme infinite). 
Date delle funzioni $(g_n)$ positive su $X$ allora
$$
\norm{\sum_{n=1}^\infty g_n}_p \leq \sum_{n=1}^\infty \norm{g_n}_p
$$

% @aziis98: Si bisognerebbe scrivere qualche parola in più magari a questa dimostrazione.
\textbf{Dimostrazione.}
Per ogni $N$ abbiamo che
$$
\norm{\sum_{n=1}^N g_n}_p^p 
\leq \left( \,\sum_{n=1}^N \norm{g_n}_p \right)^p 
\leq \left( \,\sum_{n=1}^\infty \norm{g_n}_p \right)^p 
$$
e per convergenza monotona possiamo passare il termine di sinistra al limite
$$
\lim_N \norm{\sum_{n=1}^N g_n}_p^p 
= \lim_N \int_X \left( \sum_{n=1}^N g_n \right)^p \dd \mu
= \int_X \left( \lim_N \sum_{n=1}^N g_n \right)^p \dd \mu
= \norm{\sum_{n=1}^\infty g_n}_p^p
$$
\qed

\textbf{Dimostrazione} (Completezza spazi $L^p$).
\begin{itemize}

	\item Se $p = +\infty$: si tratta di vedere che data $(f_n)$ di Cauchy in $L^\infty(X)$ esiste $N$ con $\mu(N) = 0$ tale che $(f_n)$ è di Cauchy rispetto allora norma del sup in $X \setminus N$ che sappiamo essere completa per il punto ii) del \hyperlink{prop:completeness_lemma_2}{Lemma 2}\footnote{Dato che se $\sum \norm{y_n}_\infty < +\infty \Longrightarrow \sum_n y_n < +\infty$}.

	Determiniamo un tale $N$. Dato che $(f_n)$ è di Cauchy in $L^\infty(X)$ significa che
	$$
		\forall k \; \exists n_k \colon \forall n,m > n_k \colon \norm{f_n - f_m}_{L^\infty} < \frac{1}{2^k}, \qquad \norm{f_n - f_m}_{L^\infty} = \sup_{X \setminus N_{n,m}^k} |f_n - f_m|
	$$
	per un certo insieme $N_{n,m}^k$ di misura nulla. Definiamo $N^k = \bigcup_{n,m \in \N} N_{n,m}^k$ che ha misura nulla perché unione misurabile di insiemi di misura nulla.
	Infine definiamo $N \coloneqq \bigcup_{k \in \N} N^k$ anch'esso di misura nulla che è l'insieme cercato.


	\item Se $p < +\infty$: per il \hyperlink{prop:completeness_lemma_2}{Lemma 2}, basta far vedere che data $(f_n) \subset L^p(X)$ tale che $\sum_{n=1}^\infty \norm{f_n}_p < +\infty$ allora $\sum_n f_n$ converge a qualche $f \in L^p(X)$.

	La dimostrazione è suddivisa in tre passi, prima costruiamo $f$, poi mostriamo che $f_n$ converge a $f$ ed infine mostriamo $f \in L^p(X)$.

	\begin{itemize}

		\item \textit{Passo 1.} Per ipotesi abbiamo
		$$
			\infty 
			> \sum_{n=1}^\infty \norm{f_n}_p 
			= \sum_{n=1}^\infty \norm{|f_n|}_p 
			\geq \norm{\sum_{n=1}^\infty |f_n|}_p 
			= \left( \int \left( \sum_{n=1}^\infty |f_n(x)| \right)^p \dd \mu(x) \right)^{1/p}
		$$
		quindi $\sum_{n=1}^\infty |f_n(x)| < +\infty$ per ogni $x \in X \setminus E$ con $\mu(E) = 0$. Quindi $\sum_{n=1}^\infty f_n(x)$ converge a qualche $f(x)$ per ogni $x \in X \setminus E$ ed a questo punto ci basta estendere $f$ a zero in $E$ \footnote{Una costruzione alternativa degli spazi $L^p$ potrebbe anche partire da \textit{funzioni definite quasi ovunque}, questo ovvierebbe al problema di estendere a $0$ la funzione $f$ appena costruita. Però diventa più complicato mostrare di essere in uno spazio vettoriale poiché per esempio serve ridefinire $+$ per funzioni definite quasi ovunque.}.

		
		\item \textit{Passo 2.} Fissiamo $N$ ed osserviamo che $\forall x \in X \setminus E$ abbiamo
		$$
			\left| f(x) - \sum_{n=1}^N f_n(x) \right| 
			= \left| \sum_{n=N+1}^\infty f_n(x) \right| 
			\leq \sum_{n=N+1}^\infty |f_n(x)|
		$$
		da cui otteniamo
		$$
			\norm{f - \sum_{n=1}^N f_n}_p 
			\leq \norm{\sum_{n=N+1}^\infty |f_n|}_p
			\leq \sum_{n=N+1}^\infty \norm{f_n}_p
		$$
		dove l'ultimo termine è la coda di una serie convergente.
		

		\item \textit{Passo 3.} In particolare rileggendo il passo precedente per $N = 0$ otteniamo
		$$
			\norm{f}_p \leq \sum_{n=1}^\infty \norm{f_n}_p < +\infty \implies f \in L^p
		$$
		\qed

	\end{itemize}
\end{itemize}

\textbf{Esercizio.}\footnote{In questo corso non è strettamente necessario ricordarsi come si facciano tutti questi esercizietti di teoria della misura ma è bene saperli applicare in automatico quando serve.}
Sia $f \colon X \to [0, +\infty]$ allora $\ds \int_X f \dd \mu < +\infty \implies f(x) < +\infty$ per quasi ogni $x$.

\textbf{Dimostrazione.}
Sia $E := \{ x \mid f(x) = +\infty \}$, allora l'idea è che
$$
\infty > \int_X f \dd \mu \geq \int_E f \dd \mu = +\infty \cdot \mu(E).
$$
Oppure, osserviamo che $\forall m \in [0, +\infty)$ abbiamo $f \cdot \One_E \geq m \cdot \One_E$ per ogni $x \in E$ quindi integrando ricaviamo
$$
\underbrace{\int_E f \dd \mu}_{I} \geq m \cdot \mu(E) 
\implies \mu(E) \leq \frac{I}{m} \xrightarrow{m \to +\infty} 0
$$
\qed


\subsection{Alcune proprietà viste a esercitazione degli spazi $L^p$}

\textbf{Osservazioni.} Valgono le seguenti
\begin{itemize}

	\item Sia $X \in \R^n$, $\mu$ misura di Lebesgue e $1 \leq p_1 < p_2 \leq +\infty$. In generale non posso confrontare gli spazi $L^p$. Tuttavia, se $\mu(X) < +\infty$, allora
	%
	$$
		L^{p_2}(X) \subset L^{p_1}(X) \qquad \text{se } p_1 < p_2.
	$$
	%

	\item Dato $Y \in L^p(X)$ un sottoinsieme che eredita la norma $\norm{\cdot}_{L^p}$, allora $Y$ è completo $\longiff Y$ è chiuso.

	\item Negli spazi $L^p$ i sottoinsiemi chiusi e limitati non sono compatti. In particolare, le palle $Y = \{f \in L^p \mid \norm{f}_{L^p} \leq 1 \}$ non sono compatte.
	

\end{itemize}

\subsection{Separabilità degli spazi $L^p$}

\textbf{Proposizione.} Si ha che $L^p\left( \R^d, \mu \right)$ con $\mu $ la misura di Lebesgue, è separabile se solo se $p \neq +\infty$. Lo stesso risultato vale per $\ell^p$.

\textbf{Osservazione.} La proposizione è valida anche per $L^p(X,\mu)$ con $X \subset \R^d$ aperto.

Sia $1 \leq p < +\infty$, $L^p(\R^d, \mu)$ con $\mu$ la misura di Lebesgue.
Le funzioni semplici costituite da somme finite di insiemi di misura finita sono dense in $L^p(\R^d)$.

Prendiamo una base numerabile di $\R^d$ e la indichiamo con $\mc{B}$. L'insieme
%
$$
Y = \left\{ \sum_{i=1}^{n} \alpha_i \One_{B_i} \mymid B_i \in \mc{B}, \alpha_i \in \Q \right\}
$$
%
è numerabile. Vediamo che è denso in $L^p(\R^d)$.

\textit{Idea.} È sufficiente approssimare le funzioni semplici a somma finita $\sum_{i=1}^{N} \alpha_i \One_{E_i} $. In particolare, ci basta approssimare $\alpha \cdot \One_E$. Essendo $\alpha \in \R$ troviamo una successione di razionali $\alpha_j$ tali che $\alpha_j \xrightarrow{j \to \infty} \alpha$. Dunque, rimane da approssimare l'insieme $E$.

Fissiamo $E$ e supponiamo dapprima $E$ aperto. Possiamo scrivere $E$ come unione arbitraria di elementi della base $\mc{B}$
%
$$
E = \bigcup_{i = 1}^\infty B_i.
$$
%
Per approssimare $E$ consideriamo gli insiemi $E_N = \bigcup_{i = 1}^N B_i$.
Otteniamo $\ds \left| E \right| = \lim_N \left| E_N \right|$, da cui $\left| E \setminus E_N \right| \xrightarrow{N \to +\infty} 0$.
Concludiamo notando che il caso $E$ arbitrario si fa approssimandolo con una famiglia di aperti.

\vspace{3mm}

Per $\ell^p$ con $p < +\infty$ definiamo
%
$$
Y = \left\{ \left\{ x_n \right\} \mymid x_n = 0 \quad \text{definitivamente}, x_n \in \Q \right\}
$$
%
e verifico che è numerabile e separabile.

\textbf{Domanda.} Cosa succede per $p = +\infty$?

Consideriamo $L^\infty ([0,+\infty], \mu)$ con $\mu$ di Lebesgue e $E_n = [n,n+1]$.
Definiamo l'insieme
 %
$$
Z = \left\{ \forall J \subset \N \quad u = \sum_{j \in J} \One_{E_j} \right\}.
$$
%
$Z$ ha la cardinalità delle parti di $\N$ cioè è più che numerabile. Osserviamo che per ogni $u,v \in Z$, $u \neq v$ si ha che $\norm{u - v}_{L^\infty(\R)} = 1$.
Se per assurdo esistesse un insieme denso e numerabile $D$ in $\ell^\infty$, per definizione di insieme denso dovremmo trovare per ogni palla di raggio minore di 1 e centro in un qualsiasi elemento di $Z$, un elemento di $D$. Ma questo è impossibile in quanto $D$ ha cardinalità numerabile e $Z$ la cardinalità del continuo.

Vediamo in un altro modo che $l^\infty$ non è separabile. 
Se per assurdo $Y = \left\{ \underline{x}^k \right\}_{k \in \N}$ fosse denso in $L^\infty$, allora potremmo definire un elemento $z \in l^\infty$ tale che $\norm{\underline{x}^k - \underline{z}}_{l^\infty} \geq 1$ per ogni $k$.

Definiamo $z = \left\{ z_n \right\}$ come segue
%
$$
z_n = 
\begin{cases}
0 \quad \text{se} \quad | x_n^n | > 1 \\
2 \quad \text{se} \quad | x_n^n | \leq 1
\end{cases}. 
$$



