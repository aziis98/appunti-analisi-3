
\section{Proprietà della trasformata di Fourier}

Data $f \in L^1(\R; \C)$ abbiamo posto
$$
\forall y \in \R
\qquad 
\mathcal F(f)(y) = \hat f(y) \coloneqq \int_{-\infty}^{+\infty} f(x) e^{-iyx} \dd x
$$
ed abbiamo visto che

\textbf{Teorema 1.}
$\hat f \in C_0(\R; \C)$ e $\| \hat f \|_\infty \leq \norm{f}_1$.

\textbf{Proposizione 2.}
Data $f \in L^1(\R; \C)$ allora
\begin{enumerate}
	\item $\forall h \in \R$ vale $\hat{\tau_h f} = e^{-ihy} \hat f$
	\item $\forall h \in \R$ vale $\hat{e^{ihx} f} =  \tau_h \hat f$
	\item $\forall \delta \neq 0$ vale $\hat{\sigma_\delta f} = \hat f(\delta y)$
\end{enumerate}

\textit{Derivazione.}
Partendo dalla formula di inversione
$$
\begin{aligned}
	f(x) &= \frac{1}{2\pi} \int \hat f(y) e^{iyx} \dd y \\
	f(x - h) &= \frac{1}{2\pi} \int \underbrace{\hat f(y) e^{-ihy}}_{=\hat{f(x-h)}} e^{iyx} \dd y \\
\end{aligned}
$$

\textbf{Dimostrazione.}
Facciamo il calcolo diretto
$$
\begin{aligned}
	\hat{\tau_h f}
	&= \int_{-\infty}^{+\infty} f(x - h) e^{-ixy} \dd x =\\
	&= {\footnotesize \left(\begin{gathered} t = x - h \\ \dd t = \dd x \end{gathered}\right)} 
	= \int_{-\infty}^{+\infty} f(t) e^{-i(t + h)y} \dd t = \\
	&= e^{-ihy} \int_{-\infty}^{+\infty} f(t) e^{-ity} \dd t = e^{-ihy} \hat f(t).
\end{aligned}
$$
Analogamente seguono anche le altre
\qed

\textbf{Proposizione 3.}
Sia $f \in C^1(\R; \C)$ con $f, f' \in L^1$ allora $\hat{f'} = iy\hat f$ (da confrontare con $c_n(f') = in c_n(f)$ nel caso della serie di Fourier).

\textit{Derivazione.} Si deriva la formula di inversione
$$
f'(x) = \frac{1}{2\pi} \int_{-\infty}^{+\infty} \hat f(y) i y e^{ixy} \dd y
$$

\textbf{Dimostrazione.}
Vediamo prima una dimostrazione che non funziona e cerchiamo di aggiustarla. Abbiamo che
$$
\hat{f'}(y) 
= \int_{-\infty}^{+\infty} f'(x) e^{-iyx} \dd x
= \underbrace{\left[ f(x) e^{-iyx} \right]_{-\infty}^{+\infty}}_{=0} + i y \int_{-\infty}^{+\infty} f(x) e^{-iyx} \dd x = iy \hat f(y)
$$
serve che $f(x) \to 0$ per $|x| \to +\infty$ (ad esempio $f \in C \cap L^1$ lo implica), in realtà $f \in C^1$ e $f, f' \in L^1$ basta, ma la dimostrazione è più complicata.

Argomentiamo come segue, $f \in L^1 \implies \liminf_{|x| \to \infty} |f(x)| = 0$ (in quanto se $\liminf_{|x| \to \infty} |f(x)| = \delta > 0$ allora la funzione sarebbe $> \delta$ per $|x| \to +\infty$ ed avrebbe integrale $+\infty$) dunque esistono due successioni $a_n \to -\infty, b_n \to +\infty$ tali che $f(a_n) \to 0$ e $f(b_n) \to 0$ quindi come prima abbiamo
$$
\begin{aligned}
	\hat{f'}(y) 
	&= \lim_n \int_{a_n}^{b_n} f'(x) e^{iyx} \dd x = \\
	&= \lim_n \int \One_{[a_n, b_n]} f'(x) e^{iyx} \dd x = \\
	&= \lim_n \bigg( \underbrace{\left[ f(x) e^{-iyx} \right]_{a_n}^{b_n}}_{\to 0} + i y \int_{a_n}^{b_n} f(x) e^{-iyx} \dd x \bigg) = \\
	&= \lim_n iy \int_{a_n}^{b_n} f(x) e^{-iyx} \dd x = \\
	&= iy \hat f(y) \\
\end{aligned}
$$
\qed

\textbf{Proposizione 4.}
Sia $f \in L^1$ con $xf \in L^1$, allora $\hat{f} \in C^1(\R; \C)$ e $(\hat f)' = - \hat{i x f}$.

\textbf{Dimostrazione.}
$$
\hat f(y) = \int_{-\infty}^{+\infty} f(x) e^{-ixy} \dd x
\implies
(\hat{f})'(y) = \int_{-\infty}^{+\infty} f(x) (-ix) e^{-ixy} \dd x = \hat{-ix f}
$$

\textbf{Proposizione 5.} (Derivazione sotto segno di integrale)
Sia $I$ un intervallo di $\R$, $E$ misurabile in $\R^d$ e $g \colon I \times E \to \C$ tale che
\begin{enumerate}
	\item $g(\curry, x) \in C^1(I)$ per q.o. $x \in E$.
	\item $\exists h_0, h_1 \in L^1(E)$ tali che
		$$
		|g(t, x)| \leq h_0(x) 
		\quad
		\text{e}
		\quad
		\left|\frac{\pd g}{\pd t}(t, x)\right| \leq h_1(x)
		$$
\end{enumerate}
allora $G(t) \coloneqq \int_E g(t, x) \dd x$ è ben definita per ogni $t \in I$ e $G \in C^1(I)$ e
$$
G'(t) = \int_E \frac{\pd}{\pd t}g(t, x) \dd x
$$
\textbf{Traccia dimostrazione.}
\begin{itemize}
	\item \textit{Passo 1:} $G(t)$ e $\tilde G(t)$ sono ben definite $\forall t \in I$ (grazie alla dominazione) e continue in $t$ (usando convergenza dominata e le dominazioni)
	\item \textit{Passo 2:} Dobbiamo far vedere che $G$ è $C^1$ con derivata $\tilde G$, si usa la seguente forma del teorema fondamentale del calcolo integrale
		$$
		\forall t_0, t_1 \in I \text{ con } t_0 < t_1
		\qquad
		G(t_1) - G(t_0) \overset{(*)}{=} \int_{t_0}^{t_1} \tilde G(t) \dd t
		$$
		ed usando Fubini-Tonelli in ($*$).
\end{itemize}

\textbf{Proposizione 6.} (Prodotto di convoluzione e trasformata di Fourier).
Siano $f_1, f_2 \in L^1(\R; \C)$, allora $f_1 \ast f_2 \in L^1$ (già visto) e vale
$$
\mathcal F(f_1 \ast f_2) = (\mathcal F f_1) \cdot (\mathcal F f_2)
$$
\textbf{Dimostrazione.}
$$
\begin{aligned}
	\hat{f_1 \ast f_2}(y)
	&= \int f_1 \ast f_2 (x) e^{-ixy} \dd x = \\
	&= \iint f_1(x - t) f_2(t) \dd t \, e^{-ixy} \dd x = \\
	&= \int \left(\int f_1(x - t) e^{-i(x - t)y} \dd x \right) f_2(t) e^{-ity} \dd t = \\
	&= \int \hat f_1(y) f_2(t) e^{-ity} \dd t = \hat{f_1}(y) \cdot \hat{f_2}(y)
\end{aligned}
$$

\textbf{Definizione.}
Data $g \in L^1(\R; \C)$ definiamo l'\textbf{antitrasformata di Fourier} di $g$ la funzione
$$
	\check g(x) \coloneqq \int_{-\infty}^{+\infty} g(y) e^{ixy} \dd y.
$$
Cioè $\check g(x) = \hat g(-x)$ e scriviamo anche $\check g = \mathcal F^* g$. Effettivamente $\mathcal F^*$ è l'aggiunto di $\mathcal F$, almeno formalmente\footnote{In $L^1$ non è definito il prodotto scalare.} infatti abbiamo
$$
	\langle \mathcal F f, g \rangle
	= \iint f \overline{e^{ixy} g(y)} \dd x \dd y
	= \int f(x) \overline{\check g(x)} \dd x
	= \langle f, \mathcal F^* g \rangle.
$$

\textbf{Teorema 7.}
Data $f \in L^1(\R; \C)$ tale che $\hat f \in L^1(\R; \C)$ allora
$$
	\foralmostall x \in \R
	\qquad
	\mathcal F^* \mathcal F f = 2 \pi f
	\qquad
	\text{cioè }
	\int \hat f(x) e^{ixy} \dd y = 2\pi f(x)
$$

\textbf{Nota.} Una funzione continua e infinitesima non è, in generale, una funzione $L^1$; in particolare, l'ipotesi $\hat{f} \in L^1$ è necessaria e non deriva dalle proprietà già note.

\textbf{Dimostrazione.}
\textit{Dimostrazione diretta} (passando dalla Delta di Dirac dei fisici):
$$
\begin{aligned}
	\mathcal F^* \mathcal F f 
	&= \int_{-\infty}^{+\infty} \hat f(y) e^{iyx} \dd y = \\
	&= \iint f(t) e^{-iyt} \dd t e^{ixy} \dd y = \\
	&= \int f(t) \underbrace{\int e^{i(x-t)y} \dd y}_{\text{``$\delta(x-t)$''}} \dd t = f(x)
\end{aligned}
$$
\textit{Dimostrazione vera:} scegliamo una funzione ausiliaria $\varphi \colon \R \to \R$ tale che
\begin{enumerate}
	\item $\varphi(0) = 1$ continua in $0$ e $\varphi$ limitata
	\item $\varphi \in L^1$
	\item $\check\varphi \in L^1$
\end{enumerate}
e poniamo $g_\delta(x) \coloneqq \ds \int_{-\infty}^{+\infty} \hat f(y) \varphi(\delta y) e^{ixy} \dd y$.
\begin{itemize}
	\item \textit{Passo 1:}
		$g_\delta(x) \to \mathcal F^* \mathcal F f(x)$ per ogni $x \in \R$ per convergenza dominata
		$$
		\int \hat f(y) e^{iyx} \varphi(\delta y) \dd y \xrightarrow{\delta \to 0} 
		\int \hat f(y) e^{iyx} \dd y
		$$
		e come dominazione usiamo $|\hat f(y) e^{iyx} \varphi(\delta y)| \leq |\hat f(y)| \cdot \norm{\varphi}_\infty$

	\item \textit{Passo 2:}
		$\ds g_\delta(x) = \int \left( \int f(t) e^{-ity} \dd t \right) e^{ixy} \varphi(\delta y) \dd y$, per Fubini-Tonelli otteniamo
			$$
			\begin{aligned}
				&= \iint \varphi(\delta y) e^{i(x-t)y} \dd y f(t) \dd t = \\
				&= \int \sigma_\delta \check \varphi(x - t) f(t) \dd t = \sigma_\delta \check \varphi \ast f(x).
			\end{aligned}
			$$
	\item \textit{Passo 3:}
		$g_\delta \to m f$ in $L^1$ con $\ds m = \int_\R \check \varphi(x) \dd x$ (per il teorema di approssimazione e per ipotesi).

	\item \textit{Passo 4:}
		Usando il primo ed il terzo passo otteniamo $\mathcal F^* \mathcal F f = m f$ per quasi ogni $x$, in quanto la convergenza puntuale e quella in $L^1$ devono essere compatibili; in particolare, la convergenza in $L^1$ a meno di sottosuccessioni equivale alla convergenza puntuale e dunque coincidono.

	\item \textit{Passo 5:}
		$m = 2\pi$ ad esempio prendendo $\varphi(y) = e^{-|y|}$, segue che
		$$
		\check \varphi(y) = \frac{2}{1 + x^2}
		$$
		e dunque $m = 2\pi$. In realtà vale per ogni $\varphi$ che verifica le condizioni dell'ipotesi.

		\vs

		In conclusione, riportiamo la verifica di Fubini-Tonelli:
		$$
		\iint |f(t) e^{-ity} e^{ixy} \varphi(\delta y)| \dd t \dd y = \iint |f(t)| \cdot |\varphi(\delta y)| \dd t \dd y = \norm{f}_1 \cdot \norm{\varphi(\delta y)}_1 < +\infty
		$$
\end{itemize}
\qed

\textbf{Corollario 8.}
Date $f_1, f_2 \in L^1$ tali che $\hat f_1 = \hat f_2 \implies f_1 = f_2$ quasi ovunque cioè $\mathcal F$ è iniettiva, cioè $f$ è univocamente determinata da $\hat f$.

\textbf{Dimostrazione.} Per ipotesi, $\hat{f_1} - \hat{f_2} = \hat{f_1 - f_2} = 0$.
Applicando il Teorema 7 a $\hat{f_1 - f_2}$ (possiamo farlo perché $0 \in L^1$) otteniamo
%
$$
	0 = \int \hat{f_1 - f_2}(x) e^{ixy} \dd y = 2\pi (f_1(x) - f_2(x)) 
	\Rightarrow f_1(x) = f_2(x) \quad \foralmostall x \in \R.
$$
%
\qed


\textbf{Esercizio.}
Date $f_1, f_2 \in L^1([-\pi, \pi]; \C)$ e tali che per ogni $n \in \Z$ vale $c_n(f_1) = c_n(f_2)$ allora $f_1 = f_2$ quasi ovunque (e $c_n(f) = 0$ per ogni $n \implies f = 0$ q.o.).









