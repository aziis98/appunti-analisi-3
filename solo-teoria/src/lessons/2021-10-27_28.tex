%
% Lezioni del 27-28 Ottobre 2021
%

\chapter{Spazi di Hilbert}

Sia $H$ spazio vettoriale reale con prodotto scalare $\left<\cdot, \cdot \right>$ definito positivo e norma indotta $\norm{\curry}$ definita come $\norm{x} = \sqrt{\left<x,x \right>}$.

Si ricorda l'identità di polarizzazione
%
$$
\left<x_1,x_2 \right> = \frac{1}{4} \left( \norm{x_1 + x_2}^2 - \norm{x_1 - x_2}^2 \right).
$$
%

\textbf{Nota.} Siccome $\norm{\curry}$ è continua, dalla formula di polarizzazione segue che il prodotto scalare è continuo.

\textbf{Definizione.} $H$ si dice \textbf{spazio di Hilbert} se è completo.

\textbf{Esempi.} 
\begin{itemize}
\item Dato $(X, \mc{A}, \mu )$, gli spazi $L^2(X), L^2(X, \R^m)$ sono spazi di Hilbert.

\item Lo spazio $\ds \ell^2 = \left\{ (x_n) \mymid \sum_{n=0}^{\infty} x_n^2 < +\infty  \right\}$ è uno spazio di Hilbert.

\end{itemize}

\textbf{Definizione.} $\mc{F} \subset H$ è un \textbf{sistema ortonormale} se
%
$$
\norm{e} = 1 \mquad \forall e \in \mc{F}, \qquad  \left<e,e' \right> = 0 \mquad \forall e \neq e' \in \mc{F}.
$$
%


\textbf{Definizione.} $\mc{F}$ si dice \textbf{completo} se $\overline{\spn(\mc{F})} = H$\footnote{Lo span sono combinazioni lineari finite.}. In tal caso $\mc{F}$ si dice \textbf{base di Hilbert}.

\vs

\textbf{Osservazione.} In generale una base di Hilbert $\mc{F} \subset H$ non è anche una base algebrica di $H$.

L'esempio che segue spiega quanto appena detto.

\textbf{Esempio.}
In $\ell^2$ una base ortonormale è $\mc{F} = \left\{ e_n \mymid n \in \N \right\}$ con $e_n = (0,\ldots ,0,\overset{(n)}{1},0,\ldots )$. \\
Infatti, il fatto che siano ortonormali è banale; verifichiamo che sia una base. 

Studiamo $\spn(\mc{F}) = \left\{ x = (x_0,x_1,\ldots ) \mymid x_n \quad \text{è definitivamente nullo}  \right\}$: dato $x \in \ell^2$ e $m = \N$, definiamo
%
$$
P_mx \coloneqq (x_0,x_1,\ldots , x_m,0,\ldots ).
$$
%
Allora $\spn(\mc{F}) \supset P_m x \xrightarrow{m \to +\infty} x$ in $\ell^2$.
Infatti, 
%
$$
x - P_m x = (0,\ldots ,0, x_{m+1}, x_{m+2},\ldots ).
$$
%
Dunque
%
$$
\norm{x - P_m x} = \sum_{n = m+1}^{\infty} x_n^2 \xrightarrow{m \to +\infty} 0.
$$
%

\textbf{Teorema~1.} (della base di Hilbert.) Dato $H$ spazio di Hilbert, $\mc{F}$ sistema ortonormale al più numerabile, ovvero $\mc{F} = \left\{ e_n \mymid n \in \N \right\}$.
Definiamo per ogni $x \in H$, $n \in \N$ l'elemento $x_n = \left<x, e_n \right>$.
Allora
\begin{enumerate}
\item \label{item:27ott_thm1_1}
Vale $\ds \sum_{n} x_n^2 \leq \norm{x}^2$ (\textbf{Disuguaglianza di Bessel}).

\item \label{item:27ott_thm1_2}
La somma $\ds \sum_n x_n e_n$ converge a qualche $\overline{x} \in H$ e $\overline{x}_n = x_n$  per ogni $n$.

\item \label{item:27ott_thm1_3}
Vale $\ds \norm{\overline{x}}^2 = \sum_n x_n^2 \leq \norm{x}^2$.

\item \label{item:27ott_thm1_4}
Se $x - \overline{x} \perp \mc{F}$, allora $x - \overline{x} \perp \overline{\spn(\mc{F})}$, ovvero $\overline{x}$ è la proiezione di $x$ su $\overline{\spn(\mc{F})}$.

\item \label{item:27ott_thm1_5}
Se $\mc{F}$ è completo, allora $x = \overline{x}$ e in particolare
%
$$
x = \sum_{n=0}^{\infty} x_n e_n, \qquad \norm{x}^2 = \sum_{n=0}^{\infty} x_n^2  \qquad \text{(\textbf{Identità di Parseval})}.
$$
%
\end{enumerate}

%\textbf{Nota.} Il punto \ref{item:27ott_thm1_2} non segue dal fatto che la serie è assolutamente convergente. Infatti
%%
%$$
%\sum \norm{x_n e_n} = \sum \left| x_n \right|
%$$
%%
%può essere $+\infty$.


Alla dimostrazione del teorema premettiamo il seguente lemma.

\vs

\textbf{Lemma~2.} Siano $H$ e $\mc{F}$ come nel teorema.
Data $(a_n) \in \ell^2$, allora
\begin{enumerate}
	\item La serie $\ds \sum_n a_n e_n$ converge a qualche $\overline{x} \in H$. 

	\item $\overline{x}_n = a_n$ per ogni $n$.

	\item $\ds \norm{\overline{x}}^2 = \sum_n a_n^2$.
\end{enumerate}

\textbf{Dimostrazione.}
\begin{enumerate}
\item Dimostriamo che $\ds y_n = \sum_{n = 1}^{m} a_n e_n$ è di Cauchy in $H$.
Se $m' > m$, vale
%
$$
	y_{m'} - y_m = \sum_{n = m+1}^{m'} a_n e_n
	\Longrightarrow  \norm{y_{m'} - y_m}^2 = \norm{\sum_{n = m+1}^{m'} a_n e_n}^2 
	= \sum_{n = m+1}^{m'} a_n^2 \leq \sum_{n = m+1}^{\infty} a_n^2 < +\infty.
$$
%
Dunque, per ogni $\epsilon$ esiste $m_\epsilon$ tale che $\ds \sum_{m_\epsilon + 1}^{\infty} a_n^2 \leq \epsilon^2 $, per cui 
%
$$
	\norm{y_{m'} - y_m}^2 \leq \sum_{m+1}^{m'} a_n^2 \leq \sum_{m_\epsilon + 1}^{\infty} a_n^2 \leq \epsilon^2 \quad \forall m,m' \geq m_\epsilon.
$$
%

\item Se $m \geq n$, $\left<y_m, e_n \right> = a_n$, dunque, per continuità del prodotto scalare
%
$$
a_n = \left<y_m, e_n \right> \xrightarrow{m \to \infty} \left<\overline{x},e_n \right> = \ol{x}_n.
$$
%

\item Si ha l'uguaglianza $\ds \norm{y_m}^2 = \sum_{n=1}^{m} a_n^2$, per cui passando al limite per $m \to +\infty$ otteniamo 
\begin{align*}
& \norm{y_m}^2 \xrightarrow{m \to \infty} \norm{\ol{x}}^2 \\
& \quad \rotatebox[origin=c]{90}{=} \\
& \sum_{n=0}^{m} a_n^2 \xrightarrow{m \to \infty} \sum_{n=0}^{\infty} a_n^2 
\end{align*}
\qed

\end{enumerate}


\textbf{Dimostrazione Teorema~1.}

\begin{enumerate}
\item Studiamo la somma $\ds x = \sum_{n=0}^{m} x_n e_n + \overbrace{y}^{\text{resto}}$.

Notiamo che $x$ è somma di vettori ortogonali, infatti $y$ è ortogonale a $\ds \sum_{n=0}^{m} x_n e_n$ :
%
$$
\left<y,e_i \right> = \left<x - \sum_{n=0}^{m} x_n e_n, e_i \right>
= \left<x,e_i \right> - \sum_{n=0}^{m} x_n \underbrace{\left<e_n,e_i \right>}_{\delta_{i,n}} = x_i - x_i = 0.
$$
%
Essendo che $x$ è somma di vettori ortogonali abbiamo
%
$$
\norm{x}^2 = \sum_{n=1}^{m} x_n^2 + \norm{y}^2 \geq \sum_{n=1}^{m} x_n^2.
$$
%
Passando al limite per $m \to +\infty$ otteniamo
%
$$
\norm{x}^2 \geq \sum_{n=1}^{\infty} x_n^2. 
$$
%

\item Segue dal lemma notando che il punto precedente ci dice che la successione $(x_n)$ è a quadrato sommabile.

\item Analogamente al caso precedente.

\item Notiamo che $\left<x - \ol{x}, e_n \right> = x_n - \ol{x}_n \overset{\text{ii)}}{=} 0$ per ogni $n$. Cioè
%
$$
x - \ol{x} \perp e_n \Longrightarrow x - \ol{x} \perp \spn(\mc{F})
\Longrightarrow x - \ol{x} \underset{\substack{\text{continuità} \\ \text{pr. scalare}}}{\perp} \ol{\spn(\mc{F})}
$$
%

\item $x - \overline{x} \perp \overline{\spn(\mc{F})} \underbrace{=}_{\mathclap{\text{$\mc{F}$ è completo}}} H \Longrightarrow x - \overline{x} = 0$, cioè $x = \overline{x}$.
\end{enumerate}
\qed

\textbf{Corollario~3.} Siano $H$ spazio di Hilbert, $\mc{F} = \left\{ e_n \mymid n \in \N \right\}$ base di Hilbert, $x,x' \in H$. Valgono le seguenti.
\begin{enumerate}
\item $x_n = x_n' \mquad \forall n \in \N \longiff x = x'$ ($\Leftarrow$ è ovvia).

\item $\ds \left<x,x' \right> = \sum_{n=0}^{\infty} x_n  x_n'$ (\textbf{Identità di Parseval}).

\item L'applicazione $H \ni x \mapsto (x_n) \in \ell^2$ è un'isometria surgettiva\footnote{In particolare è bigettiva ma l'iniettività è ovvia.}.
\end{enumerate}

\vs 

\textbf{Dimostrazione.}

\begin{enumerate}
	\item Per l'enunciato \ref{item:27ott_thm1_5} se due vettori hanno la stessa rappresentazione rispetto a una base di Hilbert coincidono.

	\item La tesi segue usando l'identità di polarizzazione congiuntamente all'enunciato \ref{item:27ott_thm1_5} del teorema:
	%
	\begin{align*}
		\left<x,x' \right> & = \frac{1}{4} \left( \norm{x + x'}^2 - \norm{x - x'}^2 \right)
		= \frac{1}{4} \Big( \sum_n \overbrace{(x_n + x_n')^2}^{x_n^2 + x_n'^2 + 2x_n x_n'} - \sum_n \underbrace{(x_n - x_n')^2}_{x_n^2 + x_n'^2 - 2 x_n x_n'}  \Big) \\
		& = \frac{1}{4} \left( \cancel{\sum x_n^2} + \cancel{\sum x_n'^2} + 2 \sum x_n x_n' - \cancel{\sum x_n^2} - \cancel{\sum x_n'^2} + 2 \sum x_n x_n' \right).
	\end{align*}

	\item Il fatto che l'applicazione sia un'isometria segue da Parseval; che sia iniettiva dal fatto che $\mc{F}$ è una base di Hilbert e che sia surgettiva dai punti i) e ii) del Lemma~2.

	\qed
\end{enumerate}

\textbf{Osservazioni.}
\begin{itemize}
\item Gli enunciati \ref{item:27ott_thm1_1} e \ref{item:27ott_thm1_5} non richiedono $H$ completo, mentre \ref{item:27ott_thm1_2} non è vero se $H$ non è completo.

\item Se $H$ è uno spazio di Hilbert e $\mc{F}$ sistema ortonormale infinito, allora $\mc{F}$ non è mai una base algebrica\footnote{Per base algebrica s'intende un insieme di vettori di uno spazio vettoriale le cui combinazioni lineari generano tutto lo spazio.}. Dunque, combinazioni lineari finite di $\mc{F}$ non sono mai uguali ad $H$, ovvero $\spn(\mc{F}) \subsetneq H $.

\textbf{Dimostrazione.} Presi $(e_n) \subset \mc{F}$, consideriamo $\ds \ol{x} = \sum_{n=0}^{\infty} 2^{-n}e_n $. Allora  $\ds \ol{x} \in H \setminus \spn(\mc{F})$.
%
% \textbf{Nota.} I coefficienti sono univocamente determinati perché ottenuti tramite prodotto scalare. Notiamo che non si può usare il teorema di algebra lineare sull'unicità della rappresentazione poiché stiamo trattando combinazioni lineari infinite.

\item Siano $H$ uno spazio di Hilbert di dimensione infinita e $\mc{F}$ una base di Hilbert. Allora $\mc{F}$ è numerabile se solo se $H$ è separabile.

\textbf{Dimostrazione.}

\begin{itemize}

\item[$\boxed{\Rightarrow}$] Vale $ H = \overline{\spn(\mc{F})} = \overline{\spn_\Q (\mc{F})} $. Concludiamo notando che $\overline{\spn_\Q (\mc{F})}$ è numerabile se  $\mc{F}$ è numerabile.

\item[$\boxed{\Leftarrow}$] Se $\mc{F}$ non fosse numerabile, siccome $\norm{e - e'} = \sqrt{2}  \quad \forall e,e' \in \mc{F}$, potremmo definire per ogni elemento di $\mc{F}$ una palla di raggio $\sqrt{2}/2$, dunque potremmo definire un insieme di palle disgiunte. Dato un sottoinsieme denso di $H$, per definizione, deve intersecare ogni palla e dunque deve essere più che numerabile, dunque $H$ non sarebbe separabile.

\end{itemize}


\textbf{Esempio.} Lo spazio $H = L^2(X)$, con $X = \R^n$, $\mu $ misura di Lebesgue ha base di Hilbert numerabile.

\newpage

\item Dato $\mc{F}$ sistema ortonormale in $H$, allora $\mc{F}$ è completo se solo se $\mc{F}$ è massimale (nella classe dei sistemi ortonormali rispetto all'inclusione).

\textbf{Dimostrazione.}
\begin{itemize}

\item[$\boxed{\Rightarrow}$] Dato che $\mc{F}$ è completo segue che $\overline{\spn(\mc{F})} = X$, quindi
%
$$
\mc{F}^{\perp} = \left( \spn(\mc{F}) \right)^\perp
\underbrace{=}_{\mathclap{\substack{\text{continuità del} \\ \text{prodotto scalare}}}} \overline{\spn(\mc{F})}^\perp = H^\perp = \{ 0 \}.
$$
%
dunque $\mc{F}$ è massimale.

\item[$\boxed{\Leftarrow}$] Se  $\mc{F}$ non è completo, esiste $x \in H \setminus \spn(\mc{F})$.
Definiamo $\overline{x}$ come nel Teorema 1. Notiamo che $x - \overline{x} \perp \spn(\mc{F})$, dunque $x - \overline{x} \perp \mc{F}$ e $x - \overline{x} \neq \{ 0 \}$, da cui $\ds \mc{F} \; \cup \; \left\{ \frac{x - \overline{x}}{\norm{x - \overline{x}}} \right\}$ è un sistema ortonormale che include strettamente $\mc{F}$. \absurd

\end{itemize}


\textbf{Osservazione.} Nell'implicazione $\boxed{\Rightarrow}$ non abbiamo usato la completezza di $H$.

\item Ogni sistema ortonormale $\mc{F}$ si completa a $\tilde{\mc{F}}$ base di Hilbert di $H$.

\textbf{Dimostrazione.} Sia $X = \left\{ \mc{F} \; \text{sistema ortonormale} H \text{ tale che } \tilde{\mc{F}} \subset \mc{F} \right\}$.
Per Zorn, $X$ contiene un elemento massimale. Denotiamolo con $\tilde{\mc{F}}$. Allora $\tilde{\mc{F}}$ è una base di Hilbert.


\end{itemize}


\textbf{Teorema~4.} Dato $V$ sottospazio vettoriale chiuso di $H$. Allora
\begin{enumerate}
\item $H = V + V^\perp$, cioè per ogni $x \in H$ esiste $\overline{x} \in V$ e $\tilde{x} \in V^\perp$ tale che $x = \overline{x} + \tilde{x}$.

\item Gli elementi $\overline{x}$ e $\tilde{x}$ sono univocamente determinati (e indicati con $x_V$ e $x_V^\perp$).

\item $\overline{x}$ è caratterizzato come l'elemento di $V$ più vicino a $X$.
\end{enumerate}

\textbf{Dimostrazione.}
\begin{enumerate}
\item Dato che $V$ è chiuso, $V$ è completo, cioè $V$ è un sottospazio di $H$, dunque $V$ ammette base ortonormale $\mc{F} = \left\{ e_n \mymid n \in \N \right\}$.
Definiamo $\overline{x} \in \overline{\spn(\mc{F})}$ come nel Teorema 1 e $\tilde{x} \coloneqq x - \overline{x} \in \overline{\spn(\mc{F})} = V^\perp$ (per \ref{item:27ott_thm1_4}).

\item Se $x = \overline{x} + \tilde{x} = \overline{x}' + \tilde{x}'$, dove $\overline{x}, \overline{x}' \in V$ e $\tilde{x}, \tilde{x}' \in V^\perp$, allora
%
$$
\overline{x} - \overline{x}' = \tilde{x}' - \tilde{x} \underbrace{\Longrightarrow}_{V \cap V^\perp = \{0 \} }
\overline{x} - \overline{x}' = \tilde{x}' - \tilde{x} = 0.
$$
%

\item Per ogni $y \in V$ sia $f(y) = \norm{x - y}^2$. Mostriamo che $\overline{x}$ è l'unico minimo di $f$.
%
$$
f(y) = \norm{x - y}^2 = \lVert{\overbrace{x - \overline{x}}^{\in V^\perp} + \overbrace{\overline{x} - y}^{\in V}}\rVert^2 = \norm{x - \overline{x}}^2 + \norm{\overline{x} - y}^2
= f(\overline{x}) + \norm{\overline{x} - y}^2 \geq f(\overline{x}).
$$
%
\qed
\end{enumerate}

\vs

\textbf{Osservazione.}
Serve $V$ chiuso. Se per esempio $V$ è denso in $H$ ma $V \neq H$, allora
%
$$
\overline{V^\perp} = \overline{V}^\perp = H^\perp = \{0\} \Longrightarrow V \subseteq V + V^\perp \subseteq V + \ol{V^\perp} = V \subsetneq H.
$$
%
Un esempio di tale $V$ è $\spn(\mc{F})$ con $\mc{F}$ base di $H$ ($H$ di dimensione infinita).

\newpage

\hypertarget{thm:lez2728ott-teo3}{%
\textbf{Teorema~5} (di rappresentazione di Riesz.)} Dato $\Lambda \colon  H \to \R$ lineare e continuo, esiste $x_0 \in H$ tale che 
\begin{equation} \tag{$\ast$}
	\Lambda(x) = \left<x,x_0 \right> \quad \text{per ogni} \; x \in H.
\end{equation}

\textbf{Lemma~6.} Dato $\Lambda \colon X \to \R$ lineare, $\left( \ker \Lambda \right)^\perp$ ha dimensione 0 o 1.

\textbf{Dimostrazione} Se per assurdo $\dim (\ker \Lambda)^\perp \geq 2$, allora $(\ker \Lambda)^\perp$ conterrebbe un sottospazio $W$ di dimensione $2$.
Dunque, $\dim \left( \ker \restr{\Lambda}{W} \right) = \{1,2\}$, essendo che $\dim \R = 1$. Ma questo non è possibile, in quanto abbiamo definito $W = \ker^\perp$. 
\qed

\textbf{Dimostrazione Teorema~5.}
Sia $V \coloneqq \ker \Lambda$. Dato che $\Lambda$ è continuo segue che $V$ è chiuso.
Se $V = H \Longrightarrow \Lambda \cong 0$ e prendiamo $x_0 = 0$.

Se $V \neq H$, allora $V^\perp \neq \{0\}$ e definiamo $x_1 \in V^\perp$ con $\norm{x_1} = 1$.
Poniamo $x_0 \coloneqq  c x_1$ con $c \coloneqq  \Lambda x_1$ e $\tilde{\Lambda}(x) \coloneqq \left<x,x_0 \right>$. Abbiamo che
\begin{itemize}

	\item $x \in V \Longrightarrow x \perp x_1 \Longrightarrow x \perp x_0 \Longrightarrow \tilde{\Lambda} = 0 = \Lambda(x)$. Quindi $\tilde{\Lambda} = \Lambda$ su $V$.

	\item $\tilde{\Lambda}(x_1) = \left<x_1,x_0 \right> = \left<x_1, c x_1 \right> = c \norm{x_1}^2 = c = \Lambda(x_1)$. Quindi $\tilde{\Lambda} = \Lambda$ su $\spn(x_1) = V^\perp$.

	\item $\tilde{\Lambda} = \Lambda$ su $V + V^\perp = H$.

\qed
\end{itemize}

\textbf{Osservazione.}
Esistono funzioni $\Lambda \colon  H \to \R$ lineari ma non continue se $H$ ha dimensione infinita.

\textbf{Dimostrazione.} Prendo $\Lambda \colon H \to \R$ lineare definito come
%
$$
	\begin{cases}
	\Lambda(e_n) = n \quad \forall n \\
	\Lambda(e) = \text{qualsiasi } e \in \mc{G} \setminus \left\{ e_n \right\}.
	\end{cases} 
$$
%
Allora
%
$$
	+\infty = \sup_n \left| \Lambda(e_n) \right| \leq \sup_{\norm{x} \leq 1} \left| \Lambda(x) \right| 
$$
%
da cui segue che $\Lambda$ non è continuo.
