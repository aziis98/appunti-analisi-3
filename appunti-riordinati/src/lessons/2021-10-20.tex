%
% Lezione dell'20 Ottobre 2021
%

% \textbf{Definizione.}
% Date $f_1, f_2 \colon \R^d \to \R$ misurabile allora il \textbf{prodotto di convoluzione} è dato da
% $$
% f_1 \ast f_2 (x) \coloneqq \int_{\R^d} f_1(x - y) f_2(y) \dd y
% $$
% e se $f_1$ e $f_2$ sono positive allora $f_1 \ast f_2(x) \in [0, +\infty]$. Ma ad esempio se prendiamo $f_1 = 1$ e $f_2 = \sin x$ con $d = 1$ allora $f_1 \ast f_2(x)$ non è definito per alcun $x$.

\textbf{Proposizione 1.} Se $|f_1| \ast |f_2| (x) < +\infty$ allora $f_1 \ast f_2(x)$ è ben definito, in quanto $f_1 \ast f_2(x) \leq |f_1 \ast f_2(x)| \leq |f_1| \ast |f_2|(x)$.

\textbf{Dimostrazione.} Basta osservare che,
%
\begin{align*}
	f_1 \ast f_2(x) & = \int_{\R^d} f_1(x-y) \cdot f_2(y) \dd y 
	\leq \left| \int_{\R^d} f_1(x-y) \cdot f_2(y) \dd y \right| \\
	& \leq \int_{\R^d} |f_1(x-y) \cdot f_2(y)| \dd y
	= |f_1| \ast |f_2|(x) < +\infty.
\end{align*}
\qed

\textbf{Corollario 2.} Se $|f_1| \ast |f_2| \in L^p(\R^d)$ con $1 \leq p \leq +\infty$ allora $f_1 \ast f_2(x)$ è ben definito per quasi ogni $x \in \R^d$ e $\norm{f_1 \ast f_2}_p \leq \norm{|f_1| \ast |f_2|}_p$.

\textbf{Dimostrazione.} Dato che $|f_1| \ast |f_2| \in L^p$ segue che
$$
	\int_{\R^d} \left( |f_1| \ast |f_2|(x) \right)^p \dd x < +\infty
	\Longrightarrow f_1 \ast f_2(x) \leq |f_1| \ast |f_2|(x) < +\infty \quad \wtilde{\forall} x \in \R^d.
$$
Inoltre
\begin{multline*}
	\norm{f_1 \ast f_2}_p^p 
	= \int_{\R^d} \left| \left( \int_{\R^d} f_1(x-y) \cdot f_2(y) \dd y \right) \right|^p \dd x \\
	\leq \int_{\R^d} \left( \int_{\R^d} \left| f_1(x-y) \right| \cdot \left| f_2(y) \right| \dd y \right)^p \dd x
	= \norm{|f_1| \ast |f_2| }_p^p
\end{multline*}

\qed

\mybox{%
\textbf{Teorema 3} (disuguaglianza di Young per convoluzione.)
Se $f_1 \in L^{p_1}(\R^d)$ e $f_2 \in L^{p_2}(\R^d)$ e preso $r \geq 1$ tale che
\begin{equation}\label{eqn:conv_th3_cond}
	\frac{1}{r} = \frac{1}{p_1} + \frac{1}{p_2} - 1,
	\tag{$\star$}
\end{equation}
allora $f_1 \ast f_2$ è ben definito quasi ovunque e
\begin{equation}\label{eqn:conv_th3_thesis}
	\norm{f_1 \ast f_2}_r \leq \norm{f_1}_{p_1} \cdot \norm{f_2}_{p_2}
	\tag{$\star\star$}
\end{equation}
}

\textbf{Osservazioni.}
\begin{itemize}
	\item 
		Nel caso di prima $1$ e $\sin x$ sono solo in $L^\infty$ infatti viene $r = -1$ e la disuguaglianza non ha senso.


	% \item Data $f \colon \R^d \to \R$ e $\lambda > 0$ poniamo $R_{\lambda} f(x) \coloneqq f(x / \lambda)$ allora abbiamo $\norm{R_\lambda f}_p = \lambda^{d / p} \norm{f}_p$, dove l'esponente $d/p$ si può ricavare per \textit{analisi dimensionale}\footnote{Ovvero studiando le potenze delle unità di misura delle varie quantità.}.
	% % \footnote{In particolare ad Istituzioni di Analisi si vedono le disuguaglianze di Sobolev ed anche in quel caso tutte le condizioni sugli esponenti si riescono a ricavare per analisi dimensionale...}. 
	% Consideriamo l'espressione
	% $$
	% 	\norm{f}_p^p = \int_{\R^d} |f(x)|^p \dd x.
	% $$
	% Se $f(x)$ è una \textit{quantità adimensionale} allora $\int_{\R^d} f(x)^p \dd x$ ha dimensione di un \textit{volume} $\mathsf L^d$, da cui $\norm{f}_p$ ha dimensione di $\mathsf L^{d / p}$.

	% Similmente, per ottenere $\norm{R_\lambda(f_1 \ast f_2)}_r = \lambda^{d(1 + 1 / r)} \norm{f_1 \ast f_2}_r$, basta osservare che
	% $$
	% 	f_1 \ast f_2 (x) = \int_{\R^d} f_1(x - y) f_2(y) \dd y
	% $$
	% ha dimensione $\mathsf L^d$, da cui
	% $$
	% 	\norm{f_1 \ast f_2}_r = \bigg( \int_{\R^d} \underbrace{|f_1 \ast f_2|^r}_{\mathsf L^{dr}} \underbrace{\dd x}_{\mathsf L^d} \bigg)^{1 / r}
	% $$
	% ha dimensione di $\mathsf L^{d(1 + 1/r)}$.


	\item Supponiamo di avere $\norm{f_1 \ast f_2}_r \leq C \cdot \norm{f_1}_{p_1}^{\alpha_1} \cdot \norm{f_2}_{p_2}^{\alpha_2}$ allora vediamo che per ogni $f_1, f_2$ positive deve valere necessariamente $\alpha_1 = \alpha_2 = 1$ e $1/r = 1/p_1 + 1/p_2 - 1$.

	\textbf{Dimostrazione.} Vediamo prima $\alpha_1 = \alpha_2 = 1$ e successivamente $1/r = 1/p_1 + 1/p_2 - 1$.
	\begin{itemize}
	
		\item Per ogni $\lambda > 0$ consideriamo $\lambda f_1$ e $f_2$, allora 
		$$
			\norm{(\lambda f_1) \ast f_2}_r = \norm{\lambda (f_1 \ast f_2)}_r = \lambda \norm{f_1 \ast f_2}_r
		$$
		ma abbiamo anche
		$$
			\norm{(\lambda f_1) \ast f_2}_r \leq C \cdot \norm{\lambda f_1}_{p_1}^{\alpha_1} \cdot \norm{f_2}_{p_2}^{\alpha_2} = C \cdot \lambda^{\alpha_1} \norm{f_1}_{p_1}^{\alpha_1} \cdot \norm{f_2}_{p_2}^{\alpha_2},
		$$
		da cui necessariamente $\alpha_1 = 1$ e similmente si dimostra che $\alpha_2 = 1$.

		% A questo punto richiediamo anche che $f_1$ e $f_2$ siano tali che $\norm{f_1}_{p_1}, \norm{f_2}_{p_2} < +\infty$ e $\norm{f_1 \ast f_2} > 0$ (questo possiamo farlo in quanto basta prendere $f_1 = f_2 = \One_B$ con $B$ una palla, nel caso segue proprio che $f_1 \ast f_2 (x) > 0$ se $|x| < 1$).


		\item Data $f \colon \R^d \to \R$ e $\lambda > 0$ poniamo $R_{\lambda} f(x) \coloneqq f(x / \lambda)$ allora abbiamo
		\begin{align*}
			\norm{(R_\lambda f_1) \ast (R_\lambda f_2)}_r
			& = \norm{\int_{\R^d} f_1 \left( \frac{x-y}{\lambda} \right) \cdot f_2\left( \frac{y}{\lambda} \right) \dd y} 
			= 
			\begin{pmatrix}
				t = \frac{y}{\lambda} \\
				\lambda^d \dd t = \dd y
			\end{pmatrix} \\
			& = \lambda^d \cdot \norm{\int_{\R^d} f_1 \left( \frac{x}{\lambda} - t \right) \cdot f_2(t) \dd t} \\
			& = \lambda^d \cdot \norm{R_\lambda (f_1 \ast f_2)}_r.
		\end{align*}
		Similmente si dimostra che $\norm{R_\lambda(g)}_r = \lambda^{d/r} \norm{g}_r$, da cui otteniamo
		$$
			\norm{(R_\lambda f_1) \ast (R_\lambda f_2)}_r 
			= \lambda^{d \left( 1 + \frac{1}{r} \right)} \norm{f_1 \ast f_2}_r.
		$$
		Ma anche
		\begin{multline*}
			\lambda^{d(1 + \frac{1}{r})} \norm{f_1 \ast f_2}_r
			= \norm{(R_\lambda f_1) \ast (R_\lambda f_2)}_r 
			\overset{\eqref{eqn:conv_th3_thesis}}{\leq} C \cdot \norm{R_\lambda f_1}_{p_1} \cdot \norm{R_\lambda f_2}_{p_2} \\
			= C \cdot \lambda^{d \left( \frac{1}{p_1} + \frac{1}{p_2} \right)} \cdot \norm{f_1}_{p_1} \cdot \norm{f_2}_{p_2}.
		\end{multline*}
		Dunque abbiamo $\lambda^{d \left(1 + {1 / r} \right)} \leq C \cdot \lambda^{d\left( {1 / p_1} + {1 / p_2} \right)}$ per ogni $\lambda > 0$ e quindi $1 + {1 / r} = {1 / p_1} + {1 / p_2}$.
		\qed

	\end{itemize}

\end{itemize}

\vss

\textbf{Dimostrazione Teorema 3.}
Per via del Corollario 2. ci basta dimostrare (\ref{eqn:conv_th3_thesis}) se $f_1, f_2 \geq 0$.
\begin{itemize}
	\item
		\textit{Caso facile.} Se $p_1 = p_2 = 1$ e $r = 1$
		$$
		\begin{aligned}
			\norm{f_1 \ast f_2}_1
			&= \int f_1 \ast f_2 (x) \dd x 
			= \iint f_1(x - y) f_2(y) \dd y \dd x 
			= \int f_2(y) \int f_1(x - y) \dd x \dd y = \\
			&= \int \norm{f_1}_1 \cdot f_2(y) \dd y 
			= \norm{f_1}_1 \cdot \norm{f_2}_1
		\end{aligned}
		$$

	\item
		\textit{Caso leggermente meno facile.} Se $p_1 = p, p_2 = 1$ e $r = p$.
		Vogliamo vedere che
		$$
			\norm{f_1 \ast f_2}_p \leq \norm{f_1}_p \cdot \norm{f_2}_1
		$$
		allora
		$$
		\begin{aligned}
			\norm{f_1 \ast f_2}_p
			&= \int_{\R^d} (\underbrace{f_1 \ast f_2}_{h})^p \dd x
			= \int h \cdot h^{p-1} \dd x 
			= \iint f_1(x - y) f_2(y) h^{p-1}(x) \dd y \dd x = \\
			&= \iint f_1(y - x) h^{p-1}(x) \dd x f_2(y) \dd y 
			\overset{\text{H\"older}}{\leq} 
			\int \norm{f_1(y - \curry)}_p {\| h^{p-1} \|}_{p'} f_2(y) \dd y
		\end{aligned}
		$$
		con $p'$ esponente coniugato a $p$. Inoltre notiamo che $\norm{f_1(y - \curry)}_p = \norm{f_1}$ per invarianza di $\mathscr L^d$ per riflessioni e traslazioni. Infine otteniamo
		$$
			\norm{f_1}_p {\| h^{p-1} \|}_{p'} \norm{f_2}_1
			= \norm{f_1}_p \norm{h}_{p}^{p-1} \norm{f_2}_1.
		$$
		Dunque, $\norm{f_1 \ast f_2}_p^p \leq \norm{f_1 \ast f_2}_p^{p-1} \norm{f_1}_p \norm{f_2}_1 \implies \norm{f_1 \ast f_2}_p \leq \norm{f_1}_p \norm{f_2}_1$. Quest'ultima implicazione però è valida solo nel caso in cui $0 < \norm{f_1 \ast f_2}_p < +\infty$. Resterebbero da controllare i due casi in cui la norma è $0$ oppure $+\infty$. Il primo è ovvio; il secondo invece si fa per approssimazione e passando al limite.

		% https://math.stackexchange.com/questions/1344706/are-continuous-functions-with-compact-support-bounded
		Consideriamo $f_1, f_2$ e approssimiamole con $f_{1,n}, f_{2,n}$ a supporto compatto (e limitate\footnote{Le funzioni continue a supporto compatto sono limitate per Weierstrass.}), allora vale $\norm{f_{1,n} \ast f_{1,n}}_p \leq \norm{f_{1,n}}_p \cdot \norm{f_{2,n}}_1$ e passando al limite si ottiene la tesi. In particolare possiamo costruire le $f_n$ come
		$$
		f_n(x) \coloneqq (f(x) \cdot \One_{\mc B(0, n)}(x)) \land n
		$$

		\textbf{Osservazione.}
		Se $f_2 \geq 0$ e $\int f_2 \dd x = 1$ allora $\norm{f_1 \ast f_2}_p \leq \norm{f_1}_p$ è una versione semplificata della proposizione precedente, in particolare la dimostrazione si semplifica in quanto possiamo pensare a $f_2$ come distribuzione di probabilità e quindi $f_1 \ast f_2$ è una ``media pesata'' delle traslazioni di $f_1$ o più precisamente una combinazione convessa ``integrale''.

	\item
		\textit{Caso generale.} Non lo facciamo perché servono mille mila parametri e non è troppo interessante.
\end{itemize}
\qed

Nel caso $r = +\infty$ gli esponenti $p_1$ e $p_2$ sono proprio coniugati e possiamo rafforzare la tesi del teorema precedente.

\mybox{%
\textbf{Teorema 4} (caso $r = +\infty$ del Teorema 3).
Dati $p_1$ e $p_2$ esponenti coniugati e $f_1 \in L^{p_1}(\R^d), f_2 \in L^{p_2}(\R^d)$, allora
\begin{enumerate}
	\item \label{item:20ott_th4_1} 
		$f_1 \ast f_2(x)$ è ben definito per ogni $x \in \R^d$

	\item \label{item:20ott_th4_2}
		$|f_1 \ast f_2(x)| \leq \norm{f_1}_{p_1} \norm{f_2}_{p_2}$

	\item \label{item:20ott_th4_3} 
		$f_1 \ast f_2$ è uniformemente continua

	\item \label{item:20ott_th4_4}
		Se $1 < p_1, p_2 < +\infty$ allora $f_1 \ast f_2 \to 0$ per $|x| \to +\infty$
\end{enumerate}
}

Premettiamo i seguenti risultati.

\textbf{Proposizione 5.}
Data $f \in L^p(\R^d)$ con $p < +\infty$ la mappa
$$
\begin{array}{cccc}
	\tau_h f : & \R^d & \to & L^p(\R^d) \\
	& h & \mapsto & f(\curry - h)
\end{array}
$$
è continua.

\textbf{Dimostrazione.}
Per prima cosa notiamo che basta vedere solo la continuità in $0$ in quanto
$$
\tau_{h'} f - \tau_h f = \tau_{h} (\tau_{h' - h} f - f) 
\implies \norm{\tau_{h'} f - \tau_{h} f}_p = \norm{\tau_{h' - h} f - f}_p.
$$
Dimostriamo ora la proposizione in due passi.
\begin{itemize}
	\item
		\textit{Caso 1:} $f \in C_C(\R^d)$
		$$
		\norm{\tau_h f - f}_p^p 
		= \int_{\R^d} |f(x - h) - f(x)|^p \dd x \xrightarrow{|h| \to 0} 0
		$$
		per convergenza dominata, verifichiamo però che siano rispettate le ipotesi
		\begin{enumerate}
			\item
				La convergenza puntuale, ovvero $|f(x - h) - f(x)|^p \xrightarrow{|h| \to 0} 0$ segue direttamente dalla continuità di $f$.
			\item
				Come dominazione invece usiamo $|f(x - h) - f(x)|^p \leq (2 \norm{f}_\infty)^p \cdot \One_{\mc B(0, R + 1)}$ usando che $f \in C_C \implies \operatorname{supp}(f) \subset \overline{B(0, R)}$ e poi che 
				$$
				\operatorname{supp}(f(\curry - h) - f(\curry)) \subset \overline{\mc B(0, R + |h|)}
				$$
				infine se $|h| < 1$ come raggio ci basta prendere $R + 1$.
		\end{enumerate}
	\item 
		\textit{Caso 2:} $f$ qualunque
		Dato $\epsilon > 0$ prendiamo $g \in C_C(\R^d)$ tale che $\norm{g - f} \leq \epsilon$ allora aggiungiamo a sottraiamo $g + \tau_h g$ e raggruppiamo in modo da ottenere
		$$
		\begin{gathered}
			\tau_h f - f = \tau_h(f - g) + (\tau_h g - g) + (g - f) \\
			\implies \norm{\tau_h f - f}_p 
			\leq \underbrace{\norm{\tau_h(f - g)}_p}_{\leq \epsilon} 
			+ \norm{\tau_h g - g}_p
			+ \underbrace{\norm{g - f}_p}_{\leq \epsilon} 
			\leq 2 \epsilon + \underbrace{\norm{\tau_h g - g}_p}_{\to 0 \text{ per \textit{Caso 1}}}
		\end{gathered}
		$$
		dunque $\limsup_{|h| \to 0} \norm{\tau_h f - f}_p \leq 2\epsilon$ ma per arbitrarietà di $\epsilon$ otteniamo anche che $\norm{\tau_h f - f}_p \to 0$ per $|h| \to 0$.
\end{itemize}
\qed

\textbf{Lemma 6.}
Lo spazio $C_0(\R^d) = \{ f \colon \R^d \to \R \text{ continue con } f(x) \to 0 \text{ per } |x| \to \infty \}$ è chiuso rispetto alla convergenza uniforme.


\textbf{Dimostrazione Teorema 4.}
\begin{enumerate}
\item Osserviamo che
$$
|f_1| \ast |f_2|(x) = \int_{\R^d} |f_1(x-y)| \cdot |f_2(y)| \dd y
\overset{\text{Hölder}}{\leq} \norm{f_1(x - \curry)}_{p_1} \norm{f_2}_{p_2}
= \norm{f_1}_{p_1} \norm{f_2}_{p_2}
$$
e concludiamo per la Proposizione 1.

\item Dal punto precedente abbiamo che $|f_1| \ast |f_2|(x) \leq \norm{f_1}_{p_1} \norm{f_2}_{p_2}$, da cui si conclude banalmente.

\item Uno tra $p_1$ e $p_2$ è finito; supponiamo lo sia $p_1$.
Fissiamo $x,h \in \R^d$
%
$$
	f_1 \ast f_2(x+h) - f_1 \ast f_2(x) = \int_{\R^d} \left( f_1(x+h-y) - f_1(x-y) \right) f_2(y) \dd y,
$$
%
quindi
\begin{align*}
	\left| f_1 \ast f_2(x+h) - f_1 \ast f_2(x) \right|
	& \leq \int \left| f_1(x+h - y) - f_1(x - y) \right| | f_2| \dd y  \\
	& \underset{\text{Holder}}{\leq} \norm{f_1(x+h - \curry) - f_1(x - \curry)}_{p_1} \norm{f_2}_{p_2} \\
	& = \norm{f_1(\curry - h) - f_1(\curry)}_{p_1} \norm{f_2}_{p_2} \\
	& = \underbrace{\norm{\tau_h f_1 - f_1}_{p_1}}_{\xrightarrow[\text{Proposizione 5}]{h \to 0} 0} \norm{f_2}_{p_2}
\end{align*}
da cui segue la tesi\footnote{\textit{Nota.} In generale, quanto appena mostrato ci direbbe che la funzione è continua, ma essendo che stiamo maggiorando con una quantità indipendente da $x$ segue l'uniforme continuità.}.

\item Approssimiamo $f_1$  e $f_2$ con $f_{1,n}$ e $f_{2,n} \in \mc{C}_C(\R^d)$ in $L^{p_1}$ e $L^{p_2}$ rispettivamente.

Osserviamo che $f_{1,n} \ast f_{2,n} \in \mc{C}_C(\R^d) \subset \mc{C}_0(\R^d)$.
Per il Lemma 6 basta dimostrare che $f_{1,n} \ast f_{2,n} \longrightarrow f_1 \ast f_2$ uniformemente
%
\begin{align*}
	\norm{f_{1,n} \ast f_{2,n} - f_1 \ast f_2}_\infty
	& = \norm{\left( f_{1,n} \ast f_{2,n} - f_{1,n} \ast f_2 \right) + \left( f_{1,n} \ast f_2 - f_1 \ast f_2 \right)}_\infty \\
	& \leq \norm{f_{1,n} \ast \left( f_{2,n} - f_2 \right)}_\infty + \norm{\left( f_{1,n} - f_1 \right) \ast f_2}_\infty \\
	& \underset{ii)}{\leq} \underbrace{\norm{f_{1,n}}_{p_1}}_{\to \norm{f_1}_{p_1}} \underbrace{\norm{f_{2,n} - f_2}_{p_2}}_{\to 0} + \underbrace{\norm{f_{1,n} - f_1}_{p_1}}_{\to 0} \norm{f_2}_{p_2}.
\end{align*}

Quindi $\norm{f_{1,n} \ast f_{2,n} - f_1 \ast f_2}_\infty \to 0$.
\qed
\end{enumerate}

% \textbf{Dimostrazione \ref{item:20ott_th4_1} e \ref{item:20ott_th4_2}.}
% Seguono subito da (\ref{eqn:conv_th3_cond}) per $f_1, f_2 \geq 0$ (con il Corollario 2.), se $f_1, f_2 \geq 0$ allora
% $$
% f_1 \ast f_2 (x) 
% = \int_{\R^d} f_1(x - y) f_2(y) \dd y 
% \leq \norm{f_1(x - \curry)}_{p_1} \norm{f_2}_{p_2} 
% = \norm{f_1}_{p_1} \norm{f_2}_{p_2}
% $$

% \textbf{Proposizione 5.}
% Data $f \in L^p(\R^d)$ con $p < +\infty$ la mappa
% $$
% \begin{array}{cccc}
% 	\tau_h f : & \R^d & \to & L^p(\R^d) \\
% 	& h & \mapsto & f(\curry - h)
% \end{array}
% $$
% è continua.

% \textbf{Lemma 6.}
% Lo spazio $C_0(\R^d) = \{ f \colon \R^d \to \R \text{ continue con } f(x) \to 0 \text{ per } |x| \to 0 \}$ è chiuso rispetto alla convergenza uniforme.

% \textbf{Dimostrazione \ref{item:20ott_th4_3}.} 
% Supponiamo $p_1 < +\infty$ allora
% $$
% \begin{aligned}
% 	|f_1 \ast f_2(x + h) - f_1 \ast f_2(x)|
% 	&\leq \int |f_1(x + h - y) - f_1(x - y)| \cdot |f_2(y)| \dd y \\
% 	&\leq \norm{f_1(x + h - \curry) - f(x - \curry)}_{p_1} \norm{f_2}_{p_2} \\
% 	&= \norm{\tau_h f_1 - f_1}_{p_2} \norm{f_2}_{p_2}
% \end{aligned}
% $$
% \qed

% \section{Rimanenze dalla lezione precedente}

% [TO DO: il teorema sotto non è già stato dimostrato?]

% \textbf{Teorema.}
% Siano $f_1 \in L^{p_1}(\R^d)$ e $f_2 \in L^{p_2}(\R^d)$ con $p_1$ e $p_2$ esponenti coniugati, allora $f_1 \ast f_2$ è definita per ogni $x$ e uniformemente continua
% $$
% 	|f_1 \ast f_2(x)| \leq \norm{f_1}_{p_1} \cdot \norm{f_2}_{p_2} \quad \forall x.
% $$

% \textbf{Dimostrazione.}
% Prendiamo $f_{1,n}, f_{2, n} \in C_C(\R^d)$ tali che $f_{1, n} \to f_1$ in $L^{p_1}$ e $f_{2, n} \to f_2$ in $L^{p_2}$.
% \begin{itemize}
% 	\item 
% 		Per prima cosa verifichiamo che $f \ast g$ è ben definita. Notiamo che $f_{1,n} \ast f_{2,n}$ ha supporto limitato, infatti se $\supp(f_{i,n}) \subset \overline{\mc B(0, r_{i,n})}$ per $i = 1, 2$ allora
% 		$$
% 		\supp(f_{1,n} \ast f_{2,n}) \subset \overline{\mc B(0, r_{1,n} + r_{2,n})}
% 		$$
% 		e basta notare che l'espressione
% 		$$
% 		f_1 \ast f_2(x) = \int_{\R^d} f_1(x - y) f_2(y) \dd y
% 		$$
% 		ha integranda nulla per ogni $y$ se $|x| \geq r_{1,n} + r_{2,n}$.
	
% 	\item
% 		Vediamo che $f_{1,n} \ast f_{2,n} \to f_1 \ast f_2$ uniformemente
% 		$$
% 		f_{1,n} \ast f_{2,n} - f_1 \ast f_2 
% 		= (f_{1,n} - f_1) \ast f_{2,n} - f_1 \ast (f_{2,n} - f_2)
% 		$$
% 		$$
% 		\begin{aligned}
% 			\norm{f_{1,n} \ast f_{2,n} - f_1 \ast f_2}_p
% 			&\leq \norm{(f_{1,n} - f_1) \ast f_{2,n}}_p + \norm{f_1 \ast (f_{2,n} - f_2)}_p \\
% 			&\leq 
% 			\underbrace{\norm{f_{1,n} - f_1}_{p_1}}_{\to 0}
% 			\cdot \underbrace{\norm{f_{2,n}}_{p_2}}_{\to \norm{f_2}_{p_2}}
% 			+ \underbrace{\norm{f_1}_{p_1}}_{\text{cost.}}
% 			\cdot \underbrace{\norm{f_{2,n} - f_2}_{p_2}}_{\to 0}
% 			\to 0
% 		\end{aligned}
% 		$$

% 	\item 
% 		$C_0(\R^d)$ è chiuso per convergenza uniforme [TODO: da fare per esercizio]
% \end{itemize}
