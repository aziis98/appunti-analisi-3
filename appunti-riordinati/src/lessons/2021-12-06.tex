
\section{Trasformata di Fourier su $L^2$}

Abbiamo visto che la \textit{serie di Fourier} si definisce naturalmente su $L^2$ (uno spazio di Hilbert) mentre la \textit{trasformata di Fourier} ha bisogno di $L^1$ che non è uno spazio di Hilbert. Vedremo ora come estendere la trasformata di Fourier ad $L^2$ e come poter fare i conti.

\mybox{%
\textbf{Proposizione 1.}
Data $f \in L^1(\R; \C) \cap L^2(\R; \C)$ vale $\| \hat f\, \|_2 = \sqrt{2\pi} \norm{f}_2$.
}

\mybox{%
\textbf{Teorema 2.} Valgono i seguenti
\begin{itemize}

	\item $\mathcal F$ si estende per continuità da $L^1 \cap L^2$ a tutto $L^2$

	\item $\mathcal F / \sqrt{2\pi}$ risulta essere un'isometria\footnotemark (come operatore a valori in $L^2$); ovvero, per ogni $f_1, f_2 \in L^2(\R; \C)$ vale
	$$
		\langle \hat{f_1}, \hat{f_2} \rangle = 2\pi \langle f_1, f_2 \rangle
		\qquad \text{identità di Plancherel}.
	$$

\end{itemize}
}
\footnotetext{Dunque è uniformemente continua.}

\textbf{Dimostrazione.} Il primo enunciato si ottiene dal seguente lemma di estensione.

\textbf{Lemma.} Dati $X$ e $Y$ spazi metrici, $Y$ completo e $D$ denso in $X$, $g \colon D \to Y$ uniformemente continua allora $g$ ammette un'unica estensione $G \colon X \to Y$ continua. (Inoltre se $X$ e $Y$ sono spazi normati e $g$ è lineare allora anche $G$ è lineare).
\qed

Il secondo enunciato si ottiene dal primo enunciato del Teorema~2, dalla Proposizione~1 e  dalla formula di polarizzazione:
\begin{align*}
	\left<\hat{f_1}, \hat{f_2} \right>
	& =
	\frac{1}{4} \left( \norm{\hat{f_1} + \hat{f_2}}^2 + \norm{\hat{f_1} - \hat{f_1}}^2 \right) \\
	& = \frac{2\pi}{4} \left( \norm{f_1 + f_2}^2 + \norm{f_1 - f_1}^2 \right) \\
	& = 2\pi \left<f_1,f_2 \right>.
\end{align*}
\qed

% \textbf{Teorema 2.}
% $\mathcal F$ si estende per continuità da $L^1 \cap L^2$ a tutto $L^2$ e $\mathcal F / \sqrt{2\pi}$ risulta essere un'isometria\footnote{dunque è uniformemente continua.} (come operatore a valori in $L^2$).

% \textbf{Dimostrazione.} Segue dalla Proposizione~1 e dal seguente lemma di estensione.

% \textbf{Lemma.} Dati $X$ e $Y$ spazi metrici, $Y$ completo e $D$ denso in $X$, $g \colon D \to Y$ uniformemente continua allora $g$ ammette un'unica estensione $G \colon X \to Y$ continua. (Inoltre se $X$ e $Y$ sono spazi normati e $g$ è lineare allora anche $G$ è lineare).
% \qed

% \textbf{Corollario 3.} (Identità di Plancherel).
% Per ogni $f_1, f_2 \in L^2(\R; \C)$ vale $\langle \hat{f_1}, \hat{f_2} \rangle = 2\pi \langle f_1, f_2 \rangle$.

% \textbf{Dimostrazione.} Segue dal Teorema~2 e dall'identità di polarizzazione. 
% Infatti, 
% \begin{align*}
% 	\left<\hat{f_1}, \hat{f_2} \right>  & \overset{\text{identità pol}}{=} 
% 	\frac{1}{4} \left( \norm{\hat{f_1} + \hat{f_2}}^2 + \norm{\hat{f_1} - \hat{f_1}}^2 \right) \\
% 	\substack{\text{Teo~2}} \to & = \frac{2\pi}{4} \left( \norm{f_1 + f_2}^2 + \norm{f_1 - f_1}^2 \right) \\
% 	& = 2\pi \left<f_1,f_2 \right>.
% \end{align*}
% \qed

\vss

\textbf{Dimostrazione Proposizione 1.}

\textit{Dimostrazione che non funziona.} Proviamo a svolgere il calcolo diretto:
$$
\begin{aligned}
	\|\hat f\,\|_2^2 
	& = \int_{-\infty}^{+\infty} \hat f(y) \overline{\hat f(y)} \dd y \\
	& = \iiint f(x) e^{-ixy} \overline{f(t) e^{-ity}} \dd t \dd x \dd y \\
	\substack{\text{Non posso usare} \\ \text{Fubini}} \to & = \iint f(x) \overline{f(t)} \bigg( \underbrace{\int_{-\infty}^{+\infty} e^{-iy(t-x)} \dd y}_{\delta(x-t)} \bigg) \dd t \dd x \\
	& = \int \left( \int f(x) \delta(x - t) \dd x \right) \overline{f(t)} \dd t \\
	& = 2\pi \int f(t) \overline{f(t)} \dd t = 2\pi \norm{f}_2^2
\end{aligned}
$$
% vediamo però che compare l'integrale $\int_{-\infty}^{+\infty} e^{-iy(t-x)} \dd y$ e serve assumere che corrisponda a $\delta(x - t)$ dove $\delta$ è la ``funzione Delta di Dirac'', vediamo ora la dimostrazione formale usando una funzione ausiliaria.

\textit{Dimostrazione formale.} Prendiamo $\varphi \colon \R \to [0,+\infty]$ tale che
\begin{enumerate}
	\item $\varphi$ continua in $0$, crescente per $y < 0$ e decrescente per $y > 0$ e $\varphi(0) = 1$.
	\item $\varphi \in L^1$ e $\check\varphi \in L^1$.
\end{enumerate}

\vss

\begin{itemize}

	\item \fbox{Passo 1} 
	Poniamo per ogni $\delta$
	$$
		I_\delta = \int_{-\infty}^{+\infty} |\hat f(y)|^2 \varphi(\delta y) \dd y
		\xrightarrow{\delta \to 0}
		\int_{-\infty}^{+\infty} |f(y)|^2 \qquad \text{per Beppo Levi.}
	$$

	\item \fbox{Passo 2}
		$$
		\begin{aligned}
			I_\delta 
			&= \int \hat f(y) \overline{\hat f(y)} \varphi(\delta y) \dd y = \\
			&= \int \left( \int f(x) e^{-ixy} \dd x \right) \left( \int \overline{f(t)} e^{ity} \dd t \right) \varphi(\delta y) \dd y = \\
			&\overset{\mathclap{\text{FT}}}{=} 
			\iint f(x) \overline{f(t)} 
			\bigg( \underbrace{\int \varphi(\delta y) e^{i(t-x)y} \dd y}_{\sigma_\delta \check \varphi(t - x)} \bigg)
			\dd x \dd t = \\
			&= \int \left( \int f(x) \sigma_\delta \check\varphi(t - x) \dd x \right) \overline{f(t)} \dd t =\\
			&= \int f \ast \sigma_\delta \check \varphi(t) \cdot \overline{f(t)} \dd t = \\
			&= \langle f \ast \sigma_\delta \check \varphi; f \rangle \\
		\end{aligned}
		$$
		e possiamo applicare il teorema di Fubini-Tonelli in quanto le ipotesi sono verificate, infatti
		$$
		\begin{aligned}
			&\iiint |f(x) \overline{f(t)} e^{i(t - x)y} \varphi(\delta y)| \dd x \dd t \dd y = \\
			= &\iiint |f(x)| \cdot |f(t)| \cdot |\varphi(\delta y)| \dd x \dd t \dd y = \\
			= &\norm{f}_1^2 \norm{\varphi(\delta y)}_1 < +\infty
		\end{aligned}
		$$
		e $\norm{\varphi(\delta y)}_1 < +\infty$ poiché $\varphi \in L^1$.

	\item \fbox{Passo 3}
		$I_\delta \xrightarrow{\delta \to 0} m \norm{f}_2^2 $. Infatti $I_\delta = \langle f \ast \sigma_\delta \check \varphi; f \rangle$ e
		$$
		\sigma_\delta \check \varphi \xrightarrow{\text{in $L^2$}} m f
		\qquad
		\text{con }
		m = \int \check \varphi(x) \dd x
		$$

	\item \fbox{Passo 4}
		Infine $m = 2\pi$ ad esempio prendendo $\varphi(y) = 1/(1+y^2)$.
\end{itemize}
\qed

\newpage

\hypertarget{oss-trasformata-su-ldue}{\textbf{Calcolo TdF su $\mathbf{L^2 \setminus L^1}$.}}
% Come si può calcolare $\hat f$ per $f \in L^2 \setminus L^1$? Se per quasi ogni $y \in \R$ esiste il limite
% $$
% 	\lim_n \underbrace{\int_{-n}^{n} f(x) e^{-ixy} \dd x}_{\hat{f_n}(y)}
% $$
% allora coincide con $\hat f(y)$.

% Infatti, per ogni $n$ posto $f_n \coloneqq f \cdot \One_{[-n, n]}$ abbiamo che $\lim_n \int_{-n}^n f(x) e^{-ixy} \dd x = \hat{f_n}(x)$.
% A questo punto, osserviamo che $f_n \to f$ in $L^2$ (da controllare per esercizio) e quindi $\hat f_n \to \hat f$ in $L^2$ (segue dalla continuità della trasformata). Siccome per ipotesi $\hat f_n$ converge puntualmente quasi ovunque allora $\hat f_n \to \hat f$ puntualmente quasi ovunque. [TO DO: da modificare.]

Non si può applicare la formula per $f \in L^1$.
Tuttavia osserviamo che 
$$
	f_n \coloneqq f \cdot \One_{[-n,n]} \xrightarrow{n \to +\infty} f \quad \text{in } L^2
$$
e $f_n \in L^1$.
Quindi $\hat{f_n}$ si calcola con la solita formula e converge a $\hat{f} \in L^2$.
\textit{Se esiste il limite puntuale q.o. di}
$$
	\hat{f_n}(y) = \int_{-n}^n f(x) e^{-ixy} \dd x
$$
allora deve coincidere con $\hat{f}$ cioè vale
$$
	\hat{f}(y) = \int_{-\infty}^\infty f(x) e^{-ixy} \dd x \quad \text{per q.o.} x
$$
con l'integrale inteso come intgrale improprio (nel senso di Analisi 1) e non nel senso di Lebesgue.




\subsection{Proprietà della trasformata di Fourier in $L^2$}

\textbf{Proposizione 4.}
\begin{itemize}
	\item $\hat{\tau_h f} = e^{-ihy} \hat f$
	\item $\hat{e^{ihx} f} =  \tau_h \hat f$
	\item $\hat{\sigma_h f} = \hat f(\delta y)$
\end{itemize}

\textbf{Dimostrazione.}
Le identità valgono in $L^1 \cap L^2$ che è denso in $L^2$ e dunque si estendono per continuità ad $L^2$.

\textbf{Proposizione 5.}
Se $f \in C^1(\R; \C)$ e $f \in L^1 \cup L^2$ e $f' \in L^1 \cup L^2 \implies \hat{f'} = iy\hat{f}$.

\textbf{Dimostrazione.} 
Come per il caso di $f, f' \in L^1$ si definiscono le successioni $a_n, b_n$ tali che $a_n \to -\infty$ e $b_n \to +\infty$ con $f(a_n) \to 0$ e $f(b_n) \to 0$.

Definiamo $f_n \coloneqq f \cdot \One_{[a_n,b_n]}$; segue subito che $f_n \xrightarrow{L^2} f$, dunque $\hat{f_n} \xrightarrow{L^2} \hat{f}$ e a meno di sottosuccessioni $\hat{f_n}(y) \to \hat{f}(y)$ per quasi ogni $y$. Perciò
$$
	\wtilde{\forall} y \;\exists \lim_n \int_{a_n}^{b_n} f(x) e^{-ixy} \dd x \quad \text{ed è uguale a} \quad \int_{-\infty}^{+\infty} f(x) e^{-ixy} \dd x = \hat{f}(y)
$$
Analogamente
$$
	\wtilde{\forall} y \;\exists \lim_n \int_{a_n}^{b_n} f'(x) e^{-ixy} \dd x \quad \text{ed è uguale a} \quad \int_{-\infty}^{+\infty} f'(x) e^{-ixy} \dd x = \hat{f'}(y)
$$
Si integra per parti
$$
\begin{aligned}
	\int_{a_n}^{b_n} f'(x) e^{-ixy} \dd x = \left[f(x) e^{-ixy}\right]_{a_n}^{b_n} + iy \int_{a_n}^{b_n} f(x) e^{-iyx} \dd x
\end{aligned}
$$
passando al limite in $n$
$$
	\hat{f'}(y) = 0 + iy \hat{f}(y) \quad \wtilde{\forall} y.
$$
\qed


%\textbf{Dimostrazione.}
%La stessa fatta per $f, f' \in L^1$. Si parte da $a_n, b_n$ tali che $a_n \to -\infty$ e $b_n \to +\infty$ con $f(a_n) \to 0$ e $f(b_n) \to 0$ e si integra per parti
%$$
%\begin{aligned}
%	\mathcal F(f' \cdot \One_{[a_n, b_n]})
%	&= \int_{a_n}^{b_n} f'(x) e^{-ixy} \dd x \\
%	&= \underbrace{\left[f(x) e^{-ixy}\right]_{a_n}^{b_n}}_{\to 0} + iy \int_{a_n}^{b_n} f(x) e^{-iyx} \dd x = iy \mathcal F(f \cdot \One_{[a_n, b_n]}).
%\end{aligned}
%$$
%Per concludere si dimostra che 
%%
%\begin{gather*}
%	\mc{F}(f' \cdot \One_{[a_n,b_n]}) \xrightarrow{n \to \infty} \mc{F}(f') \text{ in } L^2 \\
%	\mc{F}(f \cdot \One_{[a_n,b_n]}) \xrightarrow{n \to \infty} \mc{F}(f) \text{ in } L^2
%\end{gather*}

%Ovvero si dimostra che
%\begin{gather*}
%	\int_{b_n}^{+\infty} |f(x) e^{-ixy}|^2 \dd x + \int_{-\infty}^{a_n} |f(x) e^{-ixy}|^2 \dd x \xrightarrow{n \to \infty} 0 \\
%	\int_{b_n}^{+\infty} |f'(x) e^{-ixy}|^2 \dd x + \int_{-\infty}^{a_n} |f'(x) e^{-ixy}|^2 \dd x \xrightarrow{n \to \infty} 0 \\
%\end{gather*}
%Ma questo è vero in quanto $f,f' \in L^2$.

\newpage

\textbf{Proposizione 6.}
Se $f \in C^1, f \in L^1, f' \in L^2 \implies \hat f \in L^1$ e soddisfa le ipotesi del teorema di inversione.

\textbf{Dimostrazione.}
Sappiamo che $iy \hat f = \hat{f'} \in L^2 \implies y \hat f \in L^2$.
$$
\begin{aligned}
	\int_{\R} |\hat f(y)| \dd y 
	& = \int_{|y| \leq 1} |\hat f(y)| \dd y + \int_{|y| \geq 1} |\hat f(y)| \dd y \\
	& \leq 2 \| \hat f\, \|_\infty + \int_{|y| \geq 1} |\hat f(y) y| \frac{1}{|y|} \dd y \\
	& \leq 2 \norm{f}_1 + \| \hat f y \|_2 \left( \int_{|y| \geq 1} \frac{1}{|y|^2} \dd y \right)^{1/2} \\
	& \leq 2 \norm{f}_1 + 2 \norm{\hat{f'}}_2
\end{aligned}
$$

\textbf{Corollario.}
$f \in C_C^1 \implies f, \hat f \in L^1$

\textbf{Proposizione 7.}
Se $f_1, f_2 \in L^2(\R; \C)$ (e dunque $f_1 f_2 \in L^1(\R; \C)$ per H\"older) allora
$$
2\pi \hat{f_1 f_2} = \hat{f_1} \ast \hat{f_2}
$$

\textbf{Dimostrazione.} Dimostriamo l'identità per casi.

Per prima cosa osserviamo che $f_1, f_2 \in L^2 \implies f_1 f_2 \in L^1$ per H\"older. 

\fbox{Caso 1.} $f_1,f_2 \in C_C^1$. In tal caso
$$
	f_1,f_2,\hat{f_1},\hat{f_2},f_1f_2,\hat{f_1f_2} \in L^1
$$
quindi
$$
	\mc{F}^* \left( \frac{1}{2\pi} \hat{f_1} \ast \hat{f_2} \right) = \frac{1}{2\pi} \mc{F}^*(\mc{F}(f_1))\mc{F}^*(\mc{F}(f_2)) = 2\pi f_1 f_2 = \mc{F}^* \left( \hat{f_1f_2} \right)
$$
e per l'iniettività di $\mc{F}^*$ segue
$$
	\frac{1}{2\pi} \hat{f_1} \ast \hat{f_2} = \hat{f_1f_2} \qquad \text{ovunque}.
$$

\fbox{Caso 2.} Prendiamo $(f_{1,n}), (f_{2,n}) \subset C_C^1$ tali che 
$$
	f_{1,n} \xrightarrow{L^2} f_1 \qquad f_{2,n} \xrightarrow{L^2} f_2
$$
allora
\begin{itemize}

	\item $\mc{F}(f_{1,n} f_{2,n}) = \frac{1}{2\pi} \hat{f_{1,n}} \ast \hat{f_{2,n}}$

	\item $f_{1,n} f_{2,n} \xrightarrow{L^1} f_1 f_2 \Longrightarrow \mc{F}\left( f_{1,n} \cdot f_{2,n} \right) \xrightarrow{C_0} \hat{f_1 f_2}$

	\item $\hat{f_{1,n}} \xrightarrow{L^2} \hat{f_1}$; $\hat{f_{2,n}} \xrightarrow{L^2} \hat{f_2} \Longrightarrow \hat{f_{1,n}} \ast \hat{f_{2,n}} \xrightarrow{L^\infty} \hat{f_1} \ast \hat{f_2}$

\end{itemize}
Dove abbiamo usato che gli operatori
$$
	\cdot \colon L^2 \times L^2 \longrightarrow L^1 \qquad \ast \colon L^2 \times L^2 \longrightarrow L^\infty
$$
sono continui (è da verificare). \qed

% Dimostriamo la proposizione per $f_1, f_2 \in C_C^1 \implies f_1, f_2, f_1 f_2 \in C_C^1 \implies$ tutte in $L^1$ e con trasformate in $L^1$.
% $$
% \mathcal F^* \left( \frac{1}{2\pi} \hat{f_1} \ast \hat{f_2} \right)
% = \frac{1}{2\pi} \mathcal F^*(\hat{f_1}) \mathcal F^*(\hat{f_2})
% = \frac{1}{2\pi} (2\pi f_1) \cdot (2\pi f_2)
% = 2\pi f_1 \cdot f_2 = \mathcal F^*(\hat{f_1 f_2})
% $$
% ed usando che $\mathcal F^*$ è iniettiva otteniamo che $2 \pi \hat{f_1 f_2} = \hat{f_1} \ast \hat{f_2} $. 

% Per $f_1, f_2 \in L^2$ si procede per continuità e si approssimano $f_1$ ed $f_2$ con $f_{1,n}$ e $f_{2,n}$ in $C_C^1$.
% \qed
