% 
% Lezione del 14 Ottobre 2021
% 

\section{Complementi su approssimazioni di funzioni in $L^p$}

Sia $X$ misurabile in $\R^n$ con $\mu = \mathscr L^n$ su $X$. In precedenza abbiamo visto che

\textbf{Proposizione 3.} Le funzioni in $C_C(\R^n)$ \textit{ristrette a $X$} sono dense in $L^p$ se $p < +\infty$.

\begin{wrapfigure}{r}{130pt}
	\centering
	\vspace{-1.7\baselineskip}
	\inputfigure{supporto-compatto}
	\vspace{-3.5\baselineskip}
\end{wrapfigure}

\textbf{Osservazione.}
Si vede facilmente che $C_C(\R^n) \subset L^p(\R^n)$.

\textbf{Domanda.} Vale un risultato analogo per le funzioni $C_C(X)$?

Le funzioni continue a supporto compatto in $X$ sono continue anche su $\R^n$ solo se $X$ è aperto, in quanto il supporto ha veramente distanza non nulla dal bordo e possiamo estendere con continuità la funzione a $0$ fuori da $X$.

\mybox{%
\textbf{Proposizione 4.} 
Sia $X$ aperto di $\R^n, \mu = \mathscr L^n$ allora $C_C(X)$ è denso in $L^p$ per ogni $p < +\infty$.
}

\textbf{Dimostrazione.}
\begin{itemize}
	\item
		$\mathscr S_C := \{ \text{funzioni semplici con supporto compatto in $X$} \}$ è denso in $L^p(X)$ per ogni $p < +\infty$.

	\item
		Dato $E$ relativamente compatto\footnote{Un sottospazio relativamente compatto di uno spazio topologico è un sottoinsieme dello spazio topologico la cui chiusura è compatta.} in $X$ esiste $f_n \in C_C(X)$ tale che $f_n \to \One_E$ in $L^p$ per ogni $p < +\infty$.
\qed
\end{itemize}

La Proposizione 3 non vale per $p = +\infty$, intuitivamente in quanto data $f \in L^\infty(X)$ discontinua, se trovassimo $f_n \to f$ in $L^\infty(X)$ con $f_n$ continue avremmo $f_n \to f$ \textit{uniformemente} e dunque $f$ continua.  Per spiegare meglio il perché la Proposizione 3 non si estende al caso $p = +\infty$ consideriamo quanto segue.

\textbf{Fatto.} 
In generale vale che data $f \colon X \to \R$ misurabile, $\norm{f}_\infty \leq \sup_{x \in X} |f(x)|$ (detta anche \textit{norma del sup})

\textbf{Esercizio.} 
Se $X$ è aperto\footnote{In particolare possiamo anche estenderci a $X \subseteq \R^n$ tali che ogni $A$ aperto relativamente a $X$ abbia misura positiva.} in $\R^n$ e $\mu = \mathscr L^n$ e $f \colon X \to \R$ continua, allora $\norm{f}_\infty = \sup_{x \in X} |f(x)|$.

\textbf{Soluzione.}
Se per assurdo $\exists x \in X$ tale che $\norm{f}_\infty < |f(x)|$ allora la continuità di $f$ implica che esiste un intorno di $x$ in cui $\norm{f}_\infty < |f(x)|$; ma un intorno contiene una palla aperta di misura positiva. \absurd

\begin{wrapfigure}{r}{125pt}
	\centering
	\vspace{-2.5\baselineskip}
	\inputfigure{aperto-a-misura-positiva}
	\vspace{-3.5\baselineskip}
\end{wrapfigure}


Consideriamo la funzione
$$
f(x) =
\begin{cases}
	1 & x \geq 0 \\
	0 & x < 0
\end{cases}
$$
e vediamo che $\nexists f_n \colon \R \to \R$ tale che $f_n \to f$ in $L^\infty$ e $(f_n) \subset C_c(\R)$. 

Se esistesse $(f_n)_n$, allora sarebbe di Cauchy rispetto alla norma $\norm{\curry}_\infty$ allora per continuità $(f_n)_n$ è di Cauchy anche rispetto alla norma del sup $\implies f_n \to \tilde f$ uniformemente con $\tilde f$ continua, quindi $\tilde f = f$ quasi ovunque ma questo non è possibile per la $f$ definita sopra.

(In particolare dato $E = \{ x \mid f(x) = \tilde f(x) \}$, prendiamo $x_n, y_n \in E$ tali che $x_n \uparrow 0$ e $y_n \downarrow 0$ ma i limiti di $f$ sono $0$ e $1$ \absurd)

\vss

\mybox{%
\textbf{Teorema} (di Lusin).
Dato $X \subset \R^d, \mu = \mathscr L^d$ e data $f \colon X \to \R$ o $\R^m$ misurabile e $\epsilon > 0$, esiste $E$ aperto in $X$ con $|E| \leq \epsilon$ tale che $f$ è continua su $X \setminus E$ (la restrizione di $f$ a $X \setminus E$ è continua)
}

\textbf{Osservazione.} 
In generale $f$ può essere non continua in tutti i punti di $X$, infatti $E$ può essere denso e $X \setminus E$ avere parte interna vuota.

\textbf{Dimostrazione.}
Basta trovare $E$ misurabile (per ottenere $E$ aperto si usa la regolarità della misura)
\begin{itemize}
	\item \textit{Caso 1}:
		$f \in L^1(X)$ e $|X| < +\infty$

		Abbiamo che $f \in L^1 \implies \exists f_n$ continue tali che $f_n \to f$ in $L^1 \implies f_n \to f$ in misura e per Severini-Egorov esiste $E$ tale che $|E| \leq \epsilon$ e $f_n \to f$ uniformemente su $X \setminus E \implies f$ è continua su $X \setminus E$.

	\item \textit{Caso 2}:
		$f$ qualunque misurabile e $|X| < +\infty$

		\textbf{Lemma.}
		Dati $X, \mc A, \mu$ con $\mu(X) < +\infty$ e data $f \colon X \to \R$ misurabile e $\epsilon > 0$ esiste $F$ misurabile con $\mu(F) \leq \epsilon$ tale che $f$ è limitata su $X \setminus F$.
		
		\textbf{Dimostrazione.}
		$\forall m > 0$ sia $F_m := \{ x \mid |f(x)| > m \}$ allora $F_m \downarrow \varnothing \implies \mu(F_m) \downarrow 0$ e quindi esiste $m$ tale che $\mu(F_m) \leq \epsilon$.

		Quindi data $f$ qualunque misurabile e $|X| < +\infty$ esiste $F$ misurabile tale che $|F| \leq \epsilon / 2$ e con $f$ limitata su $X \setminus F \implies f \in L^\infty(X \setminus F) \subset L^1(X \setminus F)$, dunque per il \textit{Caso 1} esiste $E$ misurabile tale che $|E| \leq \epsilon / 2$ e $f$ è continua su $X \setminus (E \cup F)$ e $\mu(E \cup F) \leq \epsilon$

	\item \textit{Caso 3}:
		$f$ qualunque misurabile

		Per ogni $n$ poniamo $X_n \coloneqq X \cap B(0, n)$ per il \textit{Caso 2} esistono $E_n$ misurabili con $|E_n| \leq \epsilon / 2^n$ tali che $f$ è continua su $X_n \setminus E_n$, infine prendo $E \coloneqq \bigcup_{n=1}^\infty E_n$ con $\mu(E) \leq \epsilon \implies f$ è continua su $X_n \setminus E$ per ogni $n \implies f$ è continua su $X \setminus E$. 

\end{itemize}
\qed

\textbf{Lemma} (di estensione di Tietze). Dato $X$ spazio metrico e $C \subset X$ chiuso, $f \colon C \to \R$ continua allora $f$ si estende a una funzione continua su $X$.

Usando questo lemma possiamo enunciare nuovamente il teorema precedente come segue

\mybox{%
\textbf{Teorema} (di Lusin$'$).
Data $f \colon X \to \R$ misurabile e $\epsilon > 0$, esiste $E$ aperto con $|E| \leq \epsilon$ e $g \colon X \to \R$ continua tale che $f = g$ su $X \setminus E$.

Inoltre se $f \in L^p(X)$ e $p < +\infty$ si può anche chiedere che $\norm{f - g}_p \leq \epsilon$.
}

\newpage

\section{Appendice}

\textbf{Proposizione.} Siano $V,W$ spazi normati, $T \colon V \to W$ lineare.
Sono fatti equivalenti
\begin{enumerate}
\item $T$ è continua in $0$.

\item $T$ è continua.

\item $T$ è lipschitziana, cioè esiste una costante $c < +\infty$ tale che $\norm{Tv - Tv'}_W \leq c \norm{v - v'}_V$.

\item Esiste una costante $c$ tale che $\norm{Tv}_W \leq c \norm{v}_V$ per ogni $v \in V$.

\item Esiste una costante $c$ tale che $\norm{Tv}_W \leq c$ per ogni $v \in V$, $\norm{v}_V = 1$.
\end{enumerate}

\textbf{Dimostrazione.}
v) $\Rightarrow$ iv). Vale la seguente
%
$$
	\norm{Tv}_W \underbrace{=}_{v = \lambda \tilde{v}, \norm{\tilde{v}}_V = 1} \left| \lambda \right| \norm{T \tilde{v}}_W \leq c \lambda = c \norm{v}_V.
$$
%
iv) $\Rightarrow$ iii). Vale la seguente
%
$$
	\norm{Tv - Tv'}_W = \norm{T(v - v')}_W \leq c \norm{v - v'}_W.
$$
%
iii) $\Rightarrow$ ii) e ii) $\Rightarrow$ i) sono ovvie. \\
i) $\Rightarrow$ v). $T$ continua in $0$, dunque esiste $\delta > 0$ tale che
%
$$
	\norm{Tv - T0}_W \leq 1 \quad \text{se} \quad \norm{v - 0}_V \leq \delta,
$$
%
cioè
%
$$
	\norm{Tv} \leq 1 \quad \text{se} \quad \norm{v} \leq \delta,
$$
%
da cui segue che $\norm{Tv} \leq 1/ \delta$ se $\norm{v} \leq 1$.
\qed

\textbf{Osservazione.} Le costanti ottimali iii), iv), v) sono uguali e valgono
%
$$
c = \sup_{\norm{v}_V \leq 1} \norm{Tv}_W.
$$
%

\textbf{Esempi.}
\begin{enumerate}

	\item Sia $(X, \mc{A}, \mu)$ coma al solito, con $\mu(X) < +\infty$.
	Allora, dati $1 \leq p_1 < p_2 \leq +\infty$, vale
	\begin{equation} \tag{$\star$} \label{eq:star_1}
		L^{p_2}(X) \subset L^{p_1}(X).
	\end{equation}
	Inoltre, l'inclusione $i \colon L^{p_2}(X) \to L^{p_1}(X)$ è continua.

	\textbf{Dimostrazione.} La dimostrazione di \eqref{eq:star_1} segue dalla stima
	%
	$$
		\norm{u}_{p_1} \underbrace{\leq}_{\mathclap{\text{Hölder generalizzato}}} \norm{\One_X}_q \norm{u}_{p_2} \quad \text{dove} \quad q = \frac{p_1 p_2}{p_2 - p_1}.
	$$
	%
	Dove
	%
	$$
		\norm{\One_X}_{\frac{p_1 p_2}{p_2 - p_1}} \norm{u}_{p_2} = \left( \mu(X) \right)^{\frac{1}{p_1} - \frac{1}{p_2}} \norm{u}_{p_2}.
	$$
	%
	Quanto sopra soddisfa la condizione al punto iv).
	\qed


	\item L'applicazione $\ds L^1(X) \ni u \mapsto \int u \dd \mu \in \R$ è continua.

	\textbf{Dimostrazione.} Infatti, vale
	%
	$$
		\left| \int_{X} u \dd \mu \right| \leq \int_{X} \left| u \right| \dd \mu = \norm{u}_1.
	$$
	%
	Quanto sopra soddisfa la condizione al punto iv).
	\qed


	\item Cosa possiamo dire invece dell'applicazione $\ds L^p(X) \ni u \mapsto \int u \dd \mu \in \R$?
	Se $\mu(X) < +\infty$ la continuità segue dagli esempi i) e ii) sopra.
	Se invece $\mu(X) = +\infty$? Per esempio $L^2(\R)$? [TO DO].

\end{enumerate}
