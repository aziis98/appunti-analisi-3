%
% Lezione del 18 Novembre 2021
%

\section{Esercizi}

\subsection{Calcolo dei coefficienti di Fourier}

Data $f \colon [-\pi, \pi] \to \R$ o $\C$ uno degli esercizi più comuni è doverne calcolare lo sviluppo di Fourier complesso o reale.

\textbf{Osservazione.}
Ricordiamo che $c_n(f)$ può essere calcolato anche solo se $f \in L^1$ inoltre
$$
\begin{gathered}
	\operatorname{SF}_\C(f) = \sum_{n \in \Z} c_n(f) e^{inx}
	\qquad
	c_n(f) = \frac{1}{2\pi} \int_{-\pi}^\pi f(x) e^{-inx} \dd x \\
	\text{e con base hilbertiana } \left\{ \frac{e^{inx}}{\sqrt{2\pi}} \,\middle|\, n \in \Z \right\}
\end{gathered}
$$
invece nel caso reale abbiamo visto
$$
\begin{gathered}
	\operatorname{SF}_\R(f) = \sum_{n=1}^\infty \left[ a_n \cos(nx) + b_n \sin(nx) \right] + a_0 \\
	\qquad
	a_n(f) = c_n(f) + c_{-n}(f)
	\qquad
	b_n(f) = i(c_n(f) - c_{-n}(f))
	\qquad
	a_0(f) = c_0(f) \\
	a_n = \frac{1}{\pi} \int_{-\pi}^\pi f(x) \cos(nx) \dd x
	\qquad
	b_n = \frac{1}{\pi} \int_{-\pi}^\pi f(x) \sin(nx) \dd x \\
	\text{e con base hilbertiana } \left\{ \frac{1}{\sqrt{2\pi}}, \frac{\cos(nx)}{\sqrt{\pi}}, \frac{\sin(nx)}{\sqrt{\pi}} \,\middle|\, n \geq 1 \right\}
\end{gathered}
$$

\textbf{Esercizio.}
Sia $f(x) = \cos^2(x) \sin(3x)$, calcolare i coefficienti di Fourier\footnote{Con funzioni ottenute come combinazioni di prodotti di potenze di funzioni trigonometriche (anche con argomento moltiplicato per un naturale) conviene calcolare lo sviluppo complesso e poi passare a quello reale.}.

\textbf{Svolgimento.}
Usiamo lo sviluppo complesso
$$
\cos(x) = \frac{e^{ix} + e^{-ix}}{2}
\qquad
\sin(x) = \frac{e^{ix} - e^{-ix}}{2i}
$$
Dunque possiamo riscrivere $f(x)$ come
$$
\begin{aligned}
	&= \frac{i}{8} \left( e^{i2x} + e^{-i2x} + 2 \right) \left( e^{-i3x} - e^{i3x} \right) = \\
	&= \frac{i}{8} \left( e^{-ix} + e^{-i5x} + 2e^{-i3x} - e^{i5x} - e^{ix} - 2e^{i3x} \right) = \\
	&= \frac{i}{8} e^{-ix} + \frac{i}{8} e^{-i5x} + \frac{i}{4} e^{-i3x} - \frac{i}{8} e^{i5x} - \frac{i}{8} e^{ix} - \frac{i}{4} e^{i3x}. 
\end{aligned}
$$
Dunque possiamo già scrivere i coefficienti di Fourier complessi di $f(x)$
$$
\begin{gathered}
	c_n(f) \neq 0 \iff n = \pm 1, \pm 3, \pm 5 \\
	c_{\pm 1}(f) = \mp\frac{i}{8}  
	\quad
	c_{\pm 3}(f) = \mp\frac{i}{4}
	\quad
	c_{\pm 5}(f) = \mp\frac{i}{8}.
\end{gathered}
$$
Continuiamo ora il conto precedente e ricostruiamo la serie di Fourier reale ricomponendo i termini
$$
\begin{aligned}
	&= -\frac{i^2}{4}\left( \frac{e^{ix} - e^{-ix}}{2i} \right) -\frac{i^2}{2}\left( \frac{e^{i3x} - e^{-i3x}}{2i} \right) -\frac{i^2}{4}\left( \frac{e^{i5x} - e^{-i5x}}{2i} \right) = \\
	&= \frac{1}{4} \sin(x) + \frac{1}{2} \sin(3x) + \frac{1}{4} \sin(5x),
\end{aligned}
$$
in particolare possiamo notare che $f(\pi) = f(-\pi) = 0$ dunque potevamo già dedurre che la serie di Fourier reale sarebbe stata composta solo da seni.

\textbf{Esercizio.}
Caratterizzare i coefficienti $c_n(f)$ di una $f \colon [-\pi, \pi] \to \C$ in $L^2$ tale che $\operatorname{Im}(f) \subseteq \R$.

\textit{Suggerimento.} Si usa che per $z \in \C$ vale $z \in \R \iff z = \overline z$.


\subsection{Applicazione serie di Fourier: risoluzione di PDE}

\textbf{Esercizio.}
Determinare la soluzione di (P) e stabilire unicità e regolarità della soluzione $u \colon [0, T) \times [-\pi, \pi] \to \R$ o $\C$.
\begin{equation}
	\tag{P}
	\left\{
	\begin{aligned}
		& u_t = 4 u_{xx} \\
		& u(\curry, -\pi) = u(\curry, \pi) \\
		& u_x(\curry, -\pi) = u_x(\curry, \pi) \\
		& u(0, x) = \cos^2(x) \sin(3x)
	\end{aligned}
	\right.
\end{equation}

\textbf{Svolgimento.}
Per prima cosa troviamo formalmente una soluzione in serie di Fourier $u(t, x) = \sum_{n \in \Z} c_n(t) e^{inx}$ dove $c_n(t)$ è il coefficiente di $u(t, \curry)$. 

Per il teorema di derivazione sotto il segno di integrale abbiamo
$$
u_t(t, x) = \sum_{n \in \Z} \dot c_n(t) e^{inx}
\qquad
\text{con }
c_n(t) = \frac{1}{2\pi} \int_{-\pi}^\pi u(t, x) e^{-inx} \dd x
$$
Le condizioni al bordo assicurano che $c_n(u_{xx}(t, \curry)) = - 4 n^2 c_n(u(t, \curry))$ da cui otteniamo il seguente problema di Cauchy sui coefficienti
$$
\left\{
\begin{aligned}
	& \dot c_n(t) = -4 n^2 c_n(t) \\
	& c_n(0) = c_n(u(0, \curry)) = c_n(\cos^2(x) \sin(3x))
\end{aligned}
\right.
\qquad
\forall n \in \Z
$$
Inoltre dato che $c_n^0 = 0$ se $n \neq \pm 1, \pm 3, \pm 5 \implies c_n(t) = 0$ per questi $n$, dunque complessivamente i sistemi sono
$$
\newcommand{\uffaA}[3]{%
	\left\{
	\begin{aligned}
		& \dot c_{#1}(t) = - #2 c_{#1}(t) \\
		& c_{#1}(0) = -\frac{i}{#3}
	\end{aligned}
	\right.
}
\uffaA{1}{4}{8}
\quad
\uffaA{3}{36}{4}
\quad
\uffaA{5}{100}{8}
$$
con la condizione $c_{-n}(t) = \overline{c_n(t)}$, così otteniamo
$$
\newcommand{\uffaB}[3]{%
	c_{#1}(t) = #3 e^{-#2t}
}
\begin{aligned}
	& \uffaB{1}{4}{-\frac{i}{8}}
	& \quad \uffaB{3}{36}{-\frac{i}{4}}
	& \quad\;\; \uffaB{5}{100}{-\frac{i}{8}} \\
	& \uffaB{-1}{4}{\frac{i}{8}}
	& \quad \uffaB{-3}{36}{\frac{i}{4}}
	& \quad\;\; \uffaB{-5}{100}{\frac{i}{8}} \\
\end{aligned}
$$
ed infine fattorizzando
$$
\begin{aligned}
	u(t, x) &= 
	\frac{e^{-4t}}{4} \left( -\frac{i}{2}e^{ix} + \frac{i}{2}e^{-ix} \right)
	+ \frac{e^{-36t}}{2} \left( -\frac{i}{2}e^{i3x} + \frac{i}{2}e^{-i3x} \right)
	+ \frac{e^{-100t}}{4} \left( -\frac{i}{2}e^{i5x} + \frac{i}{2}e^{-i5x} \right)
	= \\
	&= \frac{1}{4} e^{-4t} \sin(x)
	+ \frac{1}{2} e^{-36t} \sin(3x)
	+ \frac{1}{2} e^{-100t} \sin(5x)
\end{aligned}
$$

\textbf{Esercizio.}
Consideriamo il problema (P) dato da
\begin{equation}
	\tag{P}
	\left\{
	\begin{aligned}
		& u_t = u_{xx} + u \\
		& u(\curry, -\pi) = u(\curry, \pi) \\
		& u_x(\curry, -\pi) = u_x(\curry, \pi) \\
		& u(0, \curry) = u_0
	\end{aligned}
	\right.
\end{equation}
dove $u_0(x)$ è $\cos^2(x) \sin(3x)$ oppure $\ds \sum_{n \in \Z} \frac{1}{2^{|n|}} e^{inx}$.

\textbf{Svolgimento.}
Per ora lavoriamo con $u_0(x) = \cos^2(x) \sin(3x)$, notiamo subito che i coefficienti soddisfano l'equazione
$$
\left\{
\begin{aligned}
	& \dot c_n(t) = -n^2 c_n(t) + c_n(t) = (1 - n^2) c_n(t) \\
	& c_n(0) = c_n(\cos^2(x) \sin(3x))
\end{aligned}
\right.
$$
da cui $\dot c_n(t) = (1 - n^2)c_n$ con soluzione $c_n(t) = \gamma e^{(1 - n^2) t}$, quindi ad esempio abbiamo
$$
c_{\pm 1}(t) = \mp \frac{i}{8}
\qquad
c_{\pm 3}(t) = \mp \frac{i}{4} e^{-8t}
\qquad
c_{\pm 5}(t) = \mp \frac{i}{8} e^{-24t}
$$
Dunque la soluzione finale è
$$
\begin{aligned}
	u(t, x) 
	&= \frac{i}{8} e^{-ix} -\frac{i}{8} e^{ix} + \frac{i}{4}e^{3t} e^{-i3x} + \frac{i}{4}e^{3t} e^{i3x} + \frac{i}{8} e^{-24t} e^{-i5x} + \frac{i}{8} e^{-24 t} e^{i5x} = \\
	&= 
	-\frac{i}{4} \left( \frac{e^{ix} - e^{-ix}}{2} \right)
	-\frac{i}{2} e^{-3t} \left( \frac{e^{i3x} - e^{-i3x}}{2} \right)
	-\frac{i}{4} e^{-24t} \left( \frac{e^{i5x} - e^{-i5x}}{2} \right) = \\
	&= 
	\frac{1}{4} \sin(x) 
	- \frac{1}{2} e^{-3t} \sin(3x)
	- \frac{1}{4} e^{-24t} \sin(5x)
\end{aligned}
$$

Invece considerando la condizione iniziale $u_0(x) = \sum_{n \in \Z} e^{inx} / 2^{|n|}$ abbiamo che $c_n(u_0) = 1 / 2^{|n|}$, notiamo che i coefficienti sono sommabili
$$
\sum_{n \in \Z} \frac{1}{2^{|n|}} 
= 2 \sum_{n=1}^\infty \frac{1}{2^n} + 1 < +\infty
\qquad
u(t, x) \coloneqq \sum_{n \in \Z} \frac{1}{2^{|n|}} e^{(1 - n)^2 t} e^{inx}
$$
in particolare formalmente possiamo scriverla meglio come
$$
= \sum_{n \in \Z} \frac{1}{2^{|n|}} e^{(1 - n)^2 t} e^{inx}
= e^t \left(1 + \sum_{n > 0} \frac{1}{2^{|n|}} e^{(1 - n)^2} \cos(nx) \cdots\cdots \right.
$$
[TODO: Finire meglio questo conto]

\textbf{Esercizio.} (della volta scorsa)
Consideriamo la funzione
$$
f(x) = \sum_{n \neq 0} \frac{\cos(n)}{|n|^{3/2}} e^{inx}
$$
\begin{itemize}
	\item Dire se $f$ è ben definita e continua.
	\item Dire se $f$ è derivabile.
\end{itemize}

\textbf{Svolgimento.}
$$
\sum_{n \neq 0} |c_n| 
= 2 \sum_{n=1}^\infty \frac{|\cos(n)|}{|n|^{3/2}}
\leq 2 \sum \frac{1}{n^{3/2}} < +\infty
$$
dunque la serie di Fourier converge uniformemente a $f \implies$ è continua e periodica.

Se $\sum |n| \cdot |c_n| < +\infty$ si potrebbe dire che $f$ è derivabile però
$$
\sum_{n \neq 0} |n| \cdot |c_n| = 2 \sum_{n=1}^\infty \frac{|\cos(n)|}{\sqrt{n}} = +\infty
\text{ non converge assolutamente}
$$
Ma la candidata derivata ha coefficienti $i n c_n$ e non starebbe in $L^2$ ovvero
$$
\sum n^2 |c_n|^2 = +\infty \implies \sum i n c_n e^{inx} \notin L^2
$$
