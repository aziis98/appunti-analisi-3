% \section{Superfici $k$-dimensionali in $\R^d$ di classe $C^1$}

\vss

\textbf{Definizione.} Un insieme $E \subset \Sigma$ è \textbf{misurabile} (secondo Lebesgue) se $\forall \phi \colon D \to \Sigma \cap U$ parametrizzazione regolare e $D \subset \R^k$, l'insieme $\phi^{-1}(E \cap U) \subset \R^k$ è misurabile secondo Lebesgue.

\textbf{Notazione.} $\mc{M}(\Sigma) \coloneqq  \{ E \subset \Sigma \text{ misurabili} \}$.

\mybox{%
\textbf{Proposizione 1.} Esiste un'unica misura $\sigma_k$ su $\mc{M}(\Sigma)$ tale che per ogni $E \subset \Sigma \cap U$ misurabile e per ogni $\phi \colon D \to \Sigma \cap U$ parametrizzazione regolare
%
\begin{equation}
	\label{eq:16dic_formula1} \tag{1}
	\sigma_k (E \cap U) = \int\limits_{\mathclap{\phi^{-1}(E \cap U)}} \overbrace{|\det (\dd_s \phi)|}^{J\phi (s)} \dd s.
\end{equation}
}

\textbf{Commenti.} 
\begin{itemize}

	\item $\sigma_k$ misura di volume $k$-dimensionale su $\Sigma$.

	\item $\sigma_k$ coincide con la misura di Hausdorff $\mc{H}^k$ ristretta a $\Sigma$.

	\item $\ds J\phi (s) \coloneqq |\det \dd_s \phi| = \sqrt{\det (\nabla^t \phi(s) \nabla \phi(s))} = \sqrt{\sum_Q (\det Q)^2} $ dove $Q$ sono i minori $k \times k$ di $\nabla \phi(s)$.

	\item Se $k = 1$, vale $J\phi(s) = \sqrt{\det(\phi'(s))^t \phi'(s)} = |\phi'(s)| $.

\end{itemize}


\textbf{Dimostrazione.} 

\fbox{Passo 1: costruzione di $\sigma_k$.}
Prendiamo $\phi_i \colon  D_i \to \Sigma \cap U_i$ famiglia numerabile di parametrizzazioni regolari; inoltre sia $\left\{ \Sigma_i \right\}$ partizione di $\Sigma$ tale che $\{\Sigma_i\} \subset \bigcup_i U_i$ e $\Sigma_i \subset \Sigma \cap U_i$ e poniamo

% tali che $\bigcup_{i}  U_i$ dove $\{D_i\}$ è una famiglia numerabile, tale che $\Sigma \subset \bigcup U_i$.
% Prendiamo $\Sigma_i$ misurabili e disgiunti tali che $\bigcup \Sigma_i = \Sigma$ e $\Sigma_i \subset U_i$.

% Per ogni $E \in \mc{M}(\Sigma)$ poniamo 
%
$$
	\sigma_k (E) = \sum_i \quad  \int\limits_{\mathclap{\phi_i^{-1}(E \cap \Sigma_i)}} J \phi_i(s) \dd s \qquad \forall E \; \text{misurabile}
$$
%
Evitiamo di verificare che sia una misura $\sigma$-addittiva.

\newpage

\fbox{Passo 2: dimostrazione di \eqref{eq:16dic_formula1}.}
Il punto chiave è dimostrare il seguente lemma.

\textbf{Lemma.} Date $\phi \colon D \to \Sigma \cap U$ e $\wtilde{\phi} \colon \wtilde{D} \to \Sigma \cap \wtilde{U}$ ed $E \subset \Sigma \cap U \cap \wtilde{U}$ misurabile, allora
%
\begin{equation}
	\label{eq:16dic_formula2} \tag{2}
	\int_{\phi^{-1}(E)} J\phi(s) \dd s = \int_{\wtilde{\phi}(E)} J\wtilde{\phi}(\wtilde{s}) \dd \wtilde{s}
\end{equation}
%

Infatti, dando per buono il lemma, osserviamo quanto segue.
Fissato $i$
$$
	\phi \colon D \to \Sigma \cap U \qquad \phi_i \colon D \to \Sigma \cap U_i
	\qquad E_i \subset \Sigma \cap U \cap U_i
$$
dunque
$$
	\int_{\phi_i^{-1}(E \cap \Sigma_i)} J \phi_i(s) \dd s
	= \int_{\phi^{-1}(E \cap \Sigma)} J \phi(s) \dd s.
$$
Infine sommando in $i$
$$
	\sigma_k(E)
	= \sum_i \int_{\phi_i^{-1}(E \cap \Sigma_i)} J \phi_i(s) \dd s 
	= \int_{\phi^{-1}(E \cap \Sigma)} J \phi(s) \dd s,
$$
da cui la tesi.

\textbf{Dimostrazione lemma.} Usiamo il cambio di variabile $s = \phi^{-1}(\wtilde{\phi}(\wtilde{s})) =: g(\wtilde{s})$.
\begin{align*}
	\int_F J\phi(s) \dd s & = \int_{g^{-1}(F) = \tilde{F}} J\phi(s) Jg(\tilde{s}) \dd \tilde{s}
	= \int_{\tilde{F}} |\det(\dd_s \phi)| \cdot | \det(\dd_{\tilde{s}}g)| \dd \tilde{s} \\
	& = \int_{\tilde{F}} \left| \det (\dd_{\tilde{s}}(\phi \circ g)) \right| \dd \tilde{s}
	= \int_{\tilde{F}} J \tilde{\phi}(\tilde{s}) \dd \tilde{s}
\end{align*}
%
\qed

\mybox{%
\textbf{Corollario 2.} Data $\phi \colon D \to \Sigma \cap U$ $C^1$ parametrizzazione bigettiva (non necessariamente regolare) , $f \colon \Sigma \cap U \to \ol{\R}$ misurabile e integrabile rispetto a $\sigma_k$.
%
\begin{equation}
	\label{eq:16dic_formula3} \tag{3}
	\int_{\Sigma \cap U} f(x) \dd \sigma_k(x) = \int_D f(\phi(s)) J\phi(s) \dd s
\end{equation}
}
\textbf{Dimostrazione.} La formula \eqref{eq:16dic_formula3} vale se $f$ è semplice (per la formula \eqref{eq:16dic_formula1}) e poi si estende per approssimazione a tutte le $f$ positive...
\qed

Per calcolare $\int_\Sigma f \dd \sigma_k$ e $\sigma_k(E)$ conviene usare mappe $\phi$ che non sono esattamente parametrizzazioni; ci chidiamo dunque come vanno corrette \eqref{eq:16dic_formula1} e \eqref{eq:16dic_formula3} se $\phi \colon D \to \Sigma \cap U$ è solo $C^1$?

\textbf{Proposizione} (formula dell'area). Valgono le seguenti.
\begin{equation}
	\tag{1'}
	\int_{E \cap U} \# \phi^{-1}(x) \dd \sigma_k(x) = \int_{\phi^{-1}(E \cap U)} J\phi(s) \dd s
\end{equation}
e
\begin{equation}
	\tag{3'}
	\int_{E \cap U} f(x) \# \phi^{-1}(x) \dd \sigma_k (x) = \int_D f(\phi(s)) J\phi(s) \dd s
\end{equation}

\textbf{Nota.} Le formule sopra giustificano il fatto che parametrizzazione non esattamente bigettive possono essere usate lo stesso per il calcolo dei volumi.

\vss

\textbf{Esempio.} Parametrizzazione di $\mathbb{S}^d \subset \R^{d+1}$ con coordinate sferiche.

Consideriamo $\phi_d \colon \R^d \to \mathbb{S}^d$ definita come 
%
\begin{align*}
	\phi_d(\alpha_1,\ldots,\alpha_d) = & ( \cos \alpha_1, \sin \alpha_1 \cos \alpha_2, \sin \alpha_1 \sin \alpha_2 \cos \alpha_3,\ldots, \\
	& \sin \alpha_1 \cdots \sin \alpha_{d-1} \cos \alpha_d, \sin \alpha_1 \cdots \sin \alpha_d ) 
\end{align*}

Definizione ricorsiva
%
$$
\phi_1 (\alpha_1) = (\cos \alpha_1, \sin \alpha_1), \quad 
\phi_d(\alpha) = (\cos \alpha_1, \sin \alpha_1 \cdot \phi_{d-1}(\alpha_2,\ldots,\alpha_d))
$$
%
Dunque 
\begin{itemize}

	\item $\phi_d \left( [0,\pi]^{d-1} \times [0,2\pi] \right) = \mathbb{S}^d$ è una parametrizzazione in coordinate sferiche ed è iniettiva.

	\item $J\phi_d(\alpha) = \sin(\alpha_1)^{d-1} \sin(\alpha_2)^{d-2} \cdots \sin(\alpha_{d-1})^1$

\end{itemize}

\textit{Nota.} In $J_{\phi_d}(\alpha)$ non c'è il modulo perché i primi $d-1$ angoli sono in $[0,\pi]$ e il seno è positivo in questo intervallo.

\textbf{Memo.} Data $f(s) \colon \R^d \to \R$ il grafico $\Gamma_f$ di $f$ è una superficie di dimensione $d$ parametrizzata da $\phi \colon s \mapsto (s,f(s))$ e avente Jacobiano
$$
	J_\phi(s) = \sqrt{1 + |\nabla f(s)|}.
$$

\vss

\textbf{Proposizione 3.} Sia $\Sigma$ superficie come al solito. Allora esiste un'unica misura $\mu$ sui $\mc{M}(\Sigma)$ tale che per ogni $\epsilon > 0$ esiste $\delta > 0$ tale che data $f \colon \Sigma \cap U \to \R^k \in C^1$ che è $\delta$-isometria, cioè
%
\begin{equation}
	\label{eq:16dic_prop3} \tag{P}
	\frac{1}{1 + \delta} |x-x'| \leq |f(x) - f(x')| \leq (1+\delta)|x-x'| \quad \forall x,x' \in \Sigma \cap U
\end{equation}
Allora
%
$$
	\frac{1}{1+\epsilon} \sigma_k(E) \leq |f(E)| \leq (1+\epsilon)\sigma_k(E) \quad \forall E \text{ mis } \subset \Sigma \cap U
$$
%

\textbf{Corollario 4} (coincidenza con la misura di Hausdorff.) Poiché $\sigma_k$ e la restrizione di $\mc{H}^k$ a $\Sigma$ hanno la proprietà \eqref{eq:16dic_prop3}, coincidono.

\textbf{Dimostrazione} (Unicità). 
Prendiamo $\mu,\mu'$ che soddisfano \eqref{eq:16dic_prop3}.
Fissiamo $E,\epsilon > 0$ e $\delta$ di conseguenza usando \eqref{eq:16dic_prop3}. Allora
\begin{itemize}

	\item Per ogni $x \in \Sigma$ esiste $\phi_x \colon U_x \cap \Sigma \to \R^k$ tale che $\dd_x \phi_x \colon T_x \Sigma \to \R^k$ è un'isometria.


	\item Per ogni $x$ esiste $V_x \subset U_x$ tale che $\phi_x \colon \Sigma \cap V_x \to \R^k$ è $\delta$-isometria.


	\item Ricopriamo $\Sigma$ con una successione $V_i \coloneqq V_{x_i}$.


	\item Scriviamo $\ds E = \bigsqcup_i E_i$ con $E_i \subset V_i$.

\end{itemize}
 
Per \eqref{eq:16dic_prop3} abbiamo che
%
\begin{gather*}
	\frac{1}{1+\epsilon} \mu(E_i) \leq |f(E_i)| \leq (1+\epsilon) \mu(E_i) \\
	\frac{1}{1+\epsilon} \wtilde{\mu}(E_i) \leq |f(E_i)| \leq (1+\epsilon) \wtilde{\mu}(E_i)
\end{gather*}
da cui incrociando le disuguaglianze otteniamo
$$
\Longrightarrow \;
\begin{gathered}
	\frac{1}{(1+\epsilon)^2} \mu(E_i) \leq \wtilde \mu(E_i) \leq (1+\epsilon)^2 \mu(E_i) \\
	\frac{1}{(1+\epsilon)^2} \wtilde \mu(E_i) \leq \mu(E_i) \leq (1+\epsilon)^2 \wtilde \mu(E_i)
\end{gathered}
$$
e per arbitrarietà di $\epsilon$ ricaviamo $\mu(E) = \wtilde{\mu}(E)$. Per arbitrarietà di $E$ otteniamo $\mu = \wtilde{\mu}$.
\qed


\section{$k$-covettori}

Dato $V$ spazio vettoriale su $\R$ e $k=1,2,\ldots$, l'applicazione $\alpha \colon  V^k \to \R$ si dice \textbf{$k$-covettore o $k$-lineare e alternante} se
\begin{itemize}

	\item è lineare in ogni variabile

	\item per ogni $\sigma$ permutazione in $S_k$, $\alpha(v_{\sigma(1)},\ldots,v_{\sigma(k)}) = \sgn(\sigma) \alpha(v_1,\ldots,v_k)$ (equivalentemente, $\alpha$ cambia segno scambiando due variabili).

\end{itemize}

\textbf{Notazione.} $\Lambda^k (V) \coloneqq  \{\alpha \text{ $k$-covettori su } V \}$. Formalmente $\Lambda^0(V) \coloneqq \{0\}$.

\textbf{Osservazione.} 
\begin{itemize}

	\item $\Lambda^k(V)$ è uno spazio vettoriale. 

	\item $\Lambda^1(V) = V^*$ duale di $V$.

	\item $\det$ è $n$-lineare alternante nelle colonne (o righe).

	\item Se $v_1,\ldots,v_k$ sono linearmente dipendenti, allora $\alpha(v_1,\ldots,v_k) = 0$.

	\item Se $k > \dim V$, allora $\Lambda^k(V) = \{0\}$.

\end{itemize}

\textbf{Definizione.} Dati $V,V'$ spazi vettoriali, $T \colon V \to V'$ lineare, $\alpha \in \Lambda^k (V')$, il \textbf{pull-back} di $\alpha$ secondo $T$ è
%
$$
T^\# \alpha \in \Lambda^k(V) \quad \text{dato da} \quad T^\# \alpha(v_1,\ldots,v_k) = \alpha(Tv_1,\ldots,Tv_n)
$$
%
Inoltre, dati $\alpha \in \Lambda^k(V), \beta \in \Lambda^h(V)$ si definisce \textbf{prodotto esterno} e si indica con $\alpha \wedge \beta \in \Lambda^{k+h}(V)$ quanto segue
%
$$
	\alpha \wedge \beta (v_1,\ldots,v_{k+h}) = \frac{1}{k!h!} \sum_{\sigma \in S_{k+h}} \sgn (\sigma) \alpha(v_{\sigma(1)},\ldots,v_{\sigma(k)}) \beta(v_{\sigma(k+1)},\ldots,v_{\sigma(k+h)})
$$
%
