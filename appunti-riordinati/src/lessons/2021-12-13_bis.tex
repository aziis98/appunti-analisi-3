\subsection{Operatori autoaggiunti}

\textbf{Esercizi.}
\begin{enumerate}
	\item[1)] Esempio classico di $H = L^2([-\pi, \pi]; \C)$ e $D = \{ u \in C^1([-\pi, \pi]; \C) \mid u(-\pi) = u(\pi) \}$ e $T u = i \dot u$ allora $T$ è un operatore autoaggiunto ed ha autovalori $\lambda = n \in \Z$.
	
	\item[2)] $H = L^2([0, \pi]; \R)$ e $D = \{ u \in C^2([0, \pi]) \mid u(0) = 0 \text{ e } u(\pi) = 0 \}$ sono dette condizioni di Dirichlet con $T u = - \ddot u$.
\end{enumerate}

Ora usiamo sempre $T u = -\ddot u$ ma su domini differenti.

\begin{enumerate}
	\item[3)] $D_3 = \{ u \in C^2([0, \pi]) \mid \dot u(0) = 0 \text{ e } \dot u(\pi) = 0 \}$ sono dette condizioni di Neumann.
	
	\item[4)] $D_4 = \{ u \in C^2([0, \pi]) \mid u(0) = 0 \text{ e } \dot u(\pi) = 0 \}$ sono dette condizioni di Robin.
\end{enumerate}
Dire per 2), 3) e 4) rispondere alle seguenti
\begin{itemize}
	\item L'operatore $T$ è autoaggiunto e controllare se il relativo $D$ è denso in $L^2$
	\item Controllare se esistono autovalori ed eventualmente dire chi sono gli autovettori.
	\item Stabilire se esiste una base Hilbertiana di autovettori.
\end{itemize}

\textbf{Risoluzione.}
\begin{enumerate}
	\item[2)] $D_2$ è denso. Vediamo l'operatore è autoaggiunto
		$$
		\begin{aligned}
			\langle Tu, v \rangle 
			&= \int_0^\pi (-\ddot u(x)) v(x) \dd x 
			= \underbrace{-\dot u(x) v(x) \bigg|_0^\pi}_{v(0) = v(\pi) = 0} - \int_0^\pi (-\dot u(x)) \dot v(x) \dd x 
			= \int_0^\pi \dot u(x) \dot v(x) \dd x \\
			\langle u, Tv \rangle 
			&= \int_0^\pi u(x) (-\ddot v(x)) \dd x 
			= \underbrace{u(x) (-\dot v(x)) \bigg|_0^\pi}_{u(0) = u(\pi) = 0} - \int_0^\pi \dot u(x)) (-\dot v(x)) \dd x 
			= \int_0^\pi \dot u(x) \dot v(x) \dd x \\
		\end{aligned}
		$$
		dunque $\langle Tu, v \rangle = \langle \dot u, \dot v \rangle = \langle u, Tv \rangle$.

		Inoltre $T$ è anche definito positivo infatti $\langle T u, u \rangle = \langle \dot u, \dot u \rangle = \| \dot u \|_{L^2} \geq 0$.

		Cerchiamo gli autovalori quindi poniamo $-\ddot u = Tu = \lambda u$ con $\lambda \geq 0$ e $u \in D_2$. Segue $p(t) = t^2 + \lambda \implies t = \pm i \sqrt{\lambda}$ se $\lambda \neq 0$. 

		Se $\lambda = 0$ invece otteniamo $\ddot u = 0 \implies u(x) = a x + b$ ma per le condizioni al bordo segue $a, b = 0$ e dunque $u = 0 \implies \lambda = 0$ non è autovalore.

		Invece se $\lambda > 0$ abbiamo $u(x) = A \cos(\sqrt{\lambda} x) + B \sin(\sqrt{\lambda} x)$ e segue $A = 0$ e $\lambda = n^2$ per $n \in \N \setminus \{ 0 \}$.

	\item[3)] $D_2$ è denso e similmente si vede che anche in questo caso $T$ è autoaggiunto. Anche in questo caso $T$ è definito positivo perché vale sempre $\langle Tu, v \rangle = \langle \dot u, \dot v \rangle$.

		Per cercare gli autovalori risolviamo il seguente sistema
		$$
		\begin{cases}
			-\ddot u = \lambda u \\
			\dot u(0) = 0 \\
			\dot u(\pi) = 0
		\end{cases}
		$$
		Se $\lambda = 0$ allora $u(x) = \text{cost.}$ è un autovettore per l'autovalore $0$.

		Se invece $\lambda \neq 0$ allora $u(x) = A \cos(\sqrt{\lambda} x) + B \sin(\sqrt{\lambda} x) \implies \cos(nx)$ è un autovettore e $\lambda = n^2$ per $n = 1, 2, \dots$.

	\item[4)]
		In questo caso vediamo che vale sempre $\langle Tu, v \rangle = \langle \dot u, \dot v \rangle$ ma per motivi diversi infatti
		$$
		\begin{aligned}
			\langle Tu, v \rangle 
			&= \int_0^\pi (-\ddot u(x)) v(x) \dd x 
			= \underbrace{-\dot u(x) v(x) \bigg|_0^\pi}_{\dot u(\pi) = 0, \; v(0) = 0} - \int_0^\pi (-\dot u(x)) \dot v(x) \dd x 
			= \int_0^\pi \dot u(x) \dot v(x) \dd x \\
			\langle u, Tv \rangle 
			&= \int_0^\pi u(x) (-\ddot v(x)) \dd x 
			= \underbrace{u(x) (-\dot v(x)) \bigg|_0^\pi}_{\dot v(\pi) = 0, \; u(0) = 0} - \int_0^\pi \dot u(x)) (-\dot v(x)) \dd x 
			= \int_0^\pi \dot u(x) \dot v(x) \dd x \\
		\end{aligned}
		$$
		Considerando il sistema $-\ddot u = \lambda u$ con le condizioni al bordo di Robin caso $\lambda = 0$ non è un autovalore mentre se $\lambda \neq 0$ abbiamo che $\dot u(x) = \sqrt{\lambda} B \cos(\sqrt{\lambda} x) = 0$ per $x = \pi$ dunque $\sqrt{\lambda} = n + 1/2$ per $n = 0, 1, 2, \dots$ e gli autovettori sono
		$$
		u_n(x) = \sin\left(\left( n + \frac{1}{2} \right)x\right)
		$$
\end{enumerate}

\textbf{Osservazione.}
$T \colon D \to L^2$ operatore lineare e continuo $\iff \exists M > 0$ tale che $\norm{Tu}_2 \leq M\norm{u}$ per ogni $u \in D$.

Vediamo ad esempio che $D_1$ non è continuo infatti gli autovalori sono $\lambda_n = n^2 \implies n^2 \norm{u_n}_2 \leq M \norm{u_n}_2 \implies M \geq n^2$ per ogni $n$. Dunque $M$ è illimitato e l'operatore non può essere continuo.

\textbf{Esempio.} Se ad esempio abbiamo $T u = -\ddot u$ con $\tilde D = \{ u \in C^2 \mid u(0) = u(\pi) = 1 \}$ allora $T$ non è autoaggiunto e basta trovare $u, v$ tali che
$$
\langle T u, v \rangle \neq \langle u, T v \rangle
$$

\textbf{Esercizio.}
\begin{enumerate}
	\item Sia $T_1 \colon \ell^2 \to \ell^2$ dato da
		$$
		T_1((x_n)_{n > 0}) = (0, x_1, x_2, \dots)
		$$
	\item Sia $T_2 \colon \ell^2 \to \ell^2$ dato da
		$$
		T_2((x_n)_{n > 0}) = (x_2, x_3, \dots)
		$$
\end{enumerate}
Dire se sono autoaggiunti ed eventualmente chi sono gli autovalori.

\textit{Esercizi più da compito sono invece cose del tipo...}

\textbf{Esercizio.}
Sia $H = L^2([0, \pi] \times [0, \pi]; \R)$ e $T u = -\Delta u = -u_{xx} -u_{yy}$
\begin{itemize}
	\item $D_1 = \{ u \in C^2([0, \pi] \times [0, \pi]; \R) \mid u|_{\pd Q} = 0 \}$
	\item $D_2 = \{ u \in C^2([0, \pi] \times [0, \pi]; \R) \mid \nabla u|_{\pd Q} = 0 \}$
	\item $D_3 = \{ u \in C^2([0, \pi] \times [0, \pi]; \R) \mid \text{$u = 0$ su due lati paralleli e $\nabla u = 0$ sugli altri due} \}$
\end{itemize}
e dire se l'operatore è autoaggiunto ed eventualmente trovare gli autovalori.
