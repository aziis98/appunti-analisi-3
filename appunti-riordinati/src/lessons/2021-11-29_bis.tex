Data $f \colon \R \to \C$ poniamo
%
$$
	f(x) \overset{(\ast)}{=} \frac{1}{2\pi} \int_{-\infty}^\infty \hat{f}(y) e^{iyx} \dd y 
	\qquad \hat{f}(y) \coloneqq \int_{-\infty}^\infty f(x) e^{-iyx} \dd x.
$$
%
Dove $\hat{u}$ si chiama \textit{trasformata di Fourier}\footnote{Sostituisce la serie di Fourier quando si passa da funzioni su $\R$ $2\pi$-periodiche a funzioni su $\R$.} di $u$ e la formula $(\ast)$ si dice \textit{formula di inversione}.

\textit{Derivazione formale} (della formula di inversione).
Prendiamo $f \in \mc{C}_C^1(\R,\C)$ e $\delta > 0$ tale che $\supp(f) \subset [-\pi / \delta, \pi / \delta]$. 

Scriviamo $f$ in serie di Fourier su $[-\pi / \delta, \pi / \delta]$ (serve un cambio di variabile per ricondursi alla serie di Fourier su $[-\pi,\pi]$).
%
\begin{gather*}
	f(x) = \sum_{n \in \Z} c_n^\delta (f) e^{in \delta x} \\
	c_n^\delta (f) \coloneqq \frac{\delta}{2\pi} \int_{-\pi / \delta}^{\pi / \delta} f(x) e^{-in \delta x} \dd x
	= \frac{\delta}{2\pi} \int_{-\infty}^\infty f(x) e^{-in \delta x} \dd x
	= \frac{\delta}{2\pi} \hat{f}(n\delta).
\end{gather*}
Dunque,
%
$$
	f(x) = \frac{1}{2\pi} \sum_{n \in \Z} \frac{\delta}{2\pi} \underbrace{\hat{f}(n\delta) e^{i(n\delta)x}}_{\hat{f}(y) e^{iyx} \; \text{calcolata in } y = n\delta}
$$
%
dove $\ds \sum_{n \in \Z} \frac{\delta}{2\pi}\hat{f}(n\delta) e^{i(n\delta)x} $ è la somma di Riemann di $\ds \int_{-\infty}^{\infty} \hat{f} e^{iyx} \dd y$.
Dunque
%
$$
	f(x) = \frac{1}{2\pi} \sum_{n \in \Z} \frac{\delta}{2\pi} \hat{f}(n\delta) e^{i(n\delta)x}
	\xrightarrow{\delta \to 0}
	\frac{1}{2\pi} \int_{-\infty}^{\infty} \hat{f}(y) e^{iyx} \dd y.
$$
%
Quest'ultimo passaggio non è giustificato rigorosamente ma si può rendere rigoroso per $f \in \mc{C}_C^1(\R)$.

\vss

\textbf{Definizione.} Data $f \in L^1(\R; \C)$ la \textbf{trasformata di Fourier} $\hat{f}$ è definita da
%
$$
	\hat{f}(y) = \int_{-\infty}^\infty f(x) e^{-ixy} \dd x \quad \forall y \in \R.
$$
%

\newpage

\mybox{%
\textbf{Teorema.} Data $f \in L^1(\R;\C)$, allora
\begin{enumerate}

	\item $\hat{f}$ è ben definita in ogni punto di $\R$.

	\item Vale $\norm{\hat{f}}_\infty \leq \norm{f}_1$.

	\item $\hat{f}$ è continua

	\item $\hat{f}$ è infinitesima.

\end{enumerate}
}

\textbf{Dimostrazione.} 
\begin{enumerate}

	\item $\hat{f}(y)$ è ben definita per ogni $y \in \R$. Infatti, $f(x) e^{-iyx} \in L^1$ dato che
	%
	$$
		\norm{\hat{f}}_1 \leq \int_{-\infty}^\infty \left| f(x) e^{-iyx} \right| \dd x 
		= \int_{-\infty}^\infty \left| f(x) \right| \dd x
		= \norm{f}_1.
	$$
	%


	\item $\norm{\hat{f}}_\infty \leq \norm{f}_1$. Infatti,
	%
	$$
		|\hat{f}| \leq \int \left| f(x) e^{-iyx} \right| \dd x = \norm{f}_1
	$$
	%


	\item $\hat{f}$ è continua. Se $y_n \to y$, allora
	%
	$$
		\hat{f}(y_n) = \int_{-\infty}^\infty f(x) e^{-ixy_n} \dd x \xrightarrow{n \to \infty}
		\int_{-\infty}^\infty f(x) e^{-iny} \dd x = \hat{f}(y)
	$$
	%
	per convergenza dominata. Infatti, la convergenza puntuale segue dalla continuità dell'esponenziale; mentre la dominazione è data da $|f(x) e^{-iyx}| = |f(x)|$.


	\item $\hat{f}(y) \xrightarrow{y \to \pm \infty} 0$ per il lemma di Riemann-Lebesgue.

\end{enumerate}
\qed

