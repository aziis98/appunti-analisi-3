
\section{Risoluzione dell'equazione delle onde}

Consideriamo il problema
\begin{equation}
	\tag{P}
	\left\{
	\begin{aligned}
			& u_{tt} = v^2 u_{xx} \\
			& u(\curry, \pi) = u(\curry, -\pi) \\
			& u_x(\curry, \pi) = u_x(\curry, -\pi) \\
			& u(0,\curry) = u_0 \\
			& u_t(0,\curry) = u_1
	\end{aligned}
	\right.
\end{equation}
ed abbiamo visto che ha soluzione
\begin{equation}
	\tag{$*$}
	u(t,x) = c_0^0 + c_0^1 t + \sum_{n \neq  0} (\alpha_n^+ e^{in(x + vt)} + \alpha_n^- e^{in(x - vt)})
\end{equation}
$$
	\alpha^{\pm}_n = \frac{1}{2}\left( c_n^0 \pm \frac{c^1_n}{i n v} \right).
$$

Inoltre, possiamo scrivere l'equazione $(*)$ come
\begin{equation}
	\tag{$**$}
	u(t,x) = c_0^0 + c_0^1 t + \myphi^+(x + vt) + \myphi^-(x - vt)
\end{equation}
dove $\varphi^+, \varphi^-$ sono funzioni $2\pi$-periodiche.

Vedremo i seguenti risultati

\begin{itemize}
	\item Esistenza usando la forma ($**$), specifico per equazione delle onde.
	\item Esistenza usando la forma ($*$), che però richiede maggiore regolarità su $u_0$ e $u_1$.
	\item Unicità.
\end{itemize}

\textbf{Teorema 1.}
Dati $u_0 \in C^1_{\text{per}}$ allora esistono $c_0^0, c_0^1$ e $\varphi^+, \varphi^- \in C^2_{\text{per}}$ tali che la $u$ in ($**$) è di classe $C^2$ su $\R \times \R$, $2\pi$-periodica in $x$ e risolve (P).

\textbf{Lemma 4.}
Date $h, g \in C^1(\R)$ con $g$ primitiva di $h$ e $T > 0$ allora $g$ è $T$-periodica $\iff h$ è $T$-periodica e $\int_0^T h(x) \dd x = 0$.

\textbf{Dimostrazione.}
Notiamo che $h$ è $T$-periodica se e solo se $\forall x \; \int_{x}^{T+x} h(x) \dd x = \text{cost.}$
$$
\int_{x}^{T+x} h(x) \dd x = g \Big|_x^{T+x} = g(T + x) - g(x) = 0 \iff \text{$g$ è $T$-periodica}
$$

\textbf{Dimostrazione Teorema 1.}

\textit{Parte 1.}
Se $c_0^0, c_0^1 \in \R$ e $\varphi^+, \varphi^- \in C^2_\text{per}$ allora la $u$ data da ($**$) è $C^2$ su $\R \times \R$ e $2\pi$-periodica in $x$ e risolve $u_{tt} = v^2 u_{xx}$.
$$
\begin{aligned}
	u_{tt} &= [\ddot \varphi^+(x + vt) + \ddot \varphi^-(x - vt)] v^2 \\
	u_{xx} &= \ddot \varphi^+(x + vt) + \ddot \varphi^-(x - vt)
\end{aligned}
\implies u_{tt} = v^2 u_{xx}
$$

\textit{Parte 2.}
$\exists c_0^0, c_0^1 \in \R$ e $\varphi^+, \varphi^- \in C^2_\text{per}$ tali che la $u$ data da ($**$) soddisfa la condizione iniziale in (P), per $t = 0$, poste $\varphi^\pm = \varphi^\pm(x \pm v0)$
$$
\begin{cases}
	c_0^0 + \varphi^+ + \varphi^- = u_0 \\
	c_0^1 + v(\dot \varphi^+ - \dot \varphi^-) = u_1
\end{cases}
\implies
\begin{cases}
	\varphi^+ + \varphi^- = u_0 - c_0^0 \\
	(\varphi^+ - \varphi^-)' = (u_1 - c_0^1) / v \\
\end{cases}
$$
ed ora fissiamo $c_0^0 = \avint_{-\pi}^\pi u_0 \dd x$ e $c_0^1 = \avint_{-\pi}^\pi u_1 \dd x$. In questo modo possiamo applicare il lemma precedente ed ottenere
$$
\begin{cases}
	\varphi^+ + \varphi^- = g_0 \\
	(\varphi^+ - \varphi^-)' = g_1'
\end{cases}
\implies
\varphi^+ = \frac{1}{2}(g_0 + g_1)
\qquad
\varphi^- = \frac{1}{2}(g_0 - g_1)
$$
\qed

% ??
% \textbf{Osservazione.}
% Il termine ``$c_0^1 t$'' rappresenta il caso in cui è l'anello di base a ruotare.

\textbf{Teorema 2.}
Siano $u_0, u_1 \in C^0_\text{per}$ tali che $\sum n^2 |c_n^0| < +\infty$ e $\sum |n| \cdot |c_n^1| < +\infty$. Allora ($*$) definisce una funzione $u \colon \R \times \R \to \C$ di classe $C^2$, $2\pi$-periodica in $x$ che risolve (P).

\textbf{Dimostrazione.}
$$
u(t,x) = c_0^0 + c_0^1 t + \sum_{n \neq  0} 
(
\underbrace{\alpha^+ e^{in(x + vt)}}_{v^+_n} 
+ 
\underbrace{\alpha^- e^{in(x - vt)}}_{v^-_n}
)
$$

\textit{Passo 1.}
Dimostriamo che $u \in C^0(\R \times \R)$ e $2\pi$-periodica in $x$. 

La funzione $u$ soddisfa le condizioni di periodicità. Per mostrare la continuità è sufficiente mostrare che la serie converga totalmente su $\R \times \R$.
$$
\norm{v^\pm_n}_{L^\infty(\R \times \R)} = |\alpha^\pm| = O\left( |c_n^0| + \frac{|c_n^1|}{n} \right)
$$
che sono sommabili in $n$.

\textit{Passo 2.}
Mostriamo che $u \in C^2(\R \times \R)$. 

Abbiamo 
%
\begin{align*}
	& D_t^h D_x^k v_n^\pm = \alpha^\pm_n e^{in(x \pm vt)} (in)^k (ivn)^h \\
	\Longrightarrow & \norm{D_t^h D_x^k v^\pm_n}_{L^\infty(\R \times \R)} = |\alpha^\pm_n| \cdot |v|^h \cdot |n|^{k+h}
	= O(|c_n^0| \cdot |n|^{k+h} + |c_n^1| \cdot |n|^{k + h - 1}) 
\end{align*}
%
che è sommabile se $k + h \leq 2$ in $n$. La serie in ($*$) converge totalmente su $\R \times \R$ con tutte le derivate di ordine $\leq 2 \implies u$ è $C^2$.

\textit{Passo 3.}
Dimostriamo che $u$ risolve l'equazione $u_{tt} = v^2 u_{xx}$.

% Infatti $u$ la risolve $c_0^0 + c_0^1 t$ e $v^\pm_n(x, t)$ e derivate e serie commutano. 
$u$ risolve l'equazione perché derivata e serie commutano e per come abbiamo impostato (P$'$) $c_n(u(0, \curry)) = c_n(u_0) \implies u(0, \curry) = u_0$. $c_n(u_t(0, \curry)) = c_n(u_1) \implies u_t(0, \curry) = u_1$.

\textbf{Teorema 3.} (Unicità)
Se $u \colon I \times [-\pi, \pi] \to \C$ è $C^2$ in $x$ e $t$ e risolve (P) allora è unica.

\textbf{Dimostrazione.}
Si ripercorre la stessa dell'equazione del calore. Dimostriamo che i coefficienti $c_n(t) = c_n(u(t, \curry))$ definiti per $t \in I$ risolvono (P$'$)...

\section{Altre applicazioni della serie di Fourier}

\subsection{Disuguaglianza isoperimetrica}

Sia $D$ un aperto limitato con frontiera $C^1$ parametrizzata da un unico cammino $\gamma$ (quindi niente buchi o più di una componente connessa). Allora $L^2 \geq 4 \pi A$ dove $L$ è la lunghezza di $\pd D$ e $A$ è l'area di $D$. Inoltre vale l'uguale se e solo se $D$ è un disco.

\textbf{Dimostrazione.}

Possiamo scegliere $\gamma \colon [-\pi, \pi] \to \R^2 \simeq \C$ e $\gamma$ parametrizzazione di $\pd D$ in senso antiorario ed a velocità costante (da cui $|\dot\gamma(t)| = L / 2\pi$)

\textit{Passo 1.}
$$
L^2 = 2\pi \int_{-\pi}^\pi |\dot \gamma|^2 \dd t = 2\pi \norm{\dot \gamma}_2 = 4 \pi^2 \sum |c_n(\dot \gamma)|^2 = 4 \pi^2 \sum n^2 |c_n|^2
$$

\textit{Passo 2.}
$$
A \overset{(*)}{=} \frac{1}{2}\langle -i\dot\gamma, \gamma \rangle = \frac{1}{2} 2\pi \sum (-i (inc_n))c_n = \pi \sum n |c_n|^2
$$
Vediamo che vale questa formula per l'area usata in ($*$), poniamo $\gamma = \gamma_x + i \gamma_y$ allora
$$
\begin{aligned}
	\langle \dot\gamma, \gamma \rangle 
	&= \int_{-\pi}^\pi \dot\gamma \; \overline\gamma \dd t \\
	&= \int_{-\pi}^\pi (\gamma_x - i \gamma_y) (\dot\gamma_x + i \dot\gamma_y) \dd t = \\
	&= \int_\gamma (x - iy) (\dd x + i \dd y) = \\
	&= \int_D 2i \dd x \mathrm d y = 2i A
\end{aligned}
$$
\textit{Passo 3.}
Infine $L^2 = 4 \pi \sum n^2 |c_n|^2$ e $4\pi A = 4\pi \sum n |c_n|^2$, dunque segue subito che $L^2 \geq 4 \pi A$ e vale l'uguale se e solo se $n^2 = n$ o se $c_n = 0$ per ogni $n \implies \gamma(t) = c_0 + c_1 e^{it}$ che è una circonferenza di centro $c_0$ e raggio $|c_1|$.




